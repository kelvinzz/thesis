\documentclass[]{beamer}

%\usepackage{beamerthemesplit}
\usepackage{multimedia}
\usepackage{epsfig}
\usepackage{graphicx}
\graphicspath{ {images/} }

\usepackage{amssymb}
\usepackage{color}

\usepackage{amsmath}
\usepackage{amsfonts}
\usepackage{amsthm}
\usepackage{amsxtra}


\usetheme{Frankfurt}
%\usetheme{Madrid}
%\usetheme{Preview}
%\usetheme{Marburg}
%\usetheme{Berkeley}
%\usetheme{Copenhagen}
%\usetheme{AnnArbor}
\usecolortheme{orchid}
%\newcommand{\sech}{\mathrm{sech}}
%\newtheorem{theorem}{Theorem}[section]
%\newtheorem{corollary}{Corollary}
%\newtheorem*{main}{Main Theorem}
%\newtheorem{lemma}[theorem]{Lemma}
%\newtheorem{proposition}{Proposition}
%\newtheorem{conjecture}{Conjecture}
%\newtheorem*{problem}{Problem}
%\theoremstyle{definition}
%\newtheorem{definition}[theorem]{Definition}
%\newtheorem{remark}{Remark}
%\newtheorem*{notation}{Notation}

\setbeamercolor{section in head/foot}{fg=white, bg=black}
%\setbeamercolor{upper separation line head}{bg=red}
%\useoutertheme[subsection=false]{smoothbars}
%\useinnertheme[shadow=true]{rounded}
\colorlet{bullet}{white}
\defbeamertemplate*{mini frame}{Frankfurt}
{%
	\begin{pgfpicture}{0pt}{0pt}{0.1cm}{0.1cm}
		%    \pgfsetcolor{bullet}% draw and fill in red
		\pgfsetfillcolor{bullet}% only fill in red
		\pgfpathcircle{\pgfpoint{0.05cm}{0.05cm}}{0.05cm}
		\pgfusepath{fill,stroke}
	\end{pgfpicture}%
}
[action]
{ \setbeamersize{mini frame size=.14cm,mini frame offset=.03cm} }

\makeatother
\setbeamertemplate{footline}
{
	\leavevmode%
	\hbox{%
		\begin{beamercolorbox}[wd=.37\paperwidth,ht=2.25ex,dp=1ex,center]{author in head/foot}%
			\usebeamerfont{author in head/foot}\insertshortauthor
		\end{beamercolorbox}%
		\begin{beamercolorbox}[wd=.43\paperwidth,ht=2.25ex,dp=1ex,center]{title in head/foot}%
				\usebeamerfont{title in head/foot}\insertshorttitle\hspace*{3em}
		\end{beamercolorbox}%
		\begin{beamercolorbox}[wd=.2\paperwidth,ht=2.25ex,dp=1ex,center]{title in head/foot}%
			\usebeamerfont{institute in head/foot}\insertshortinstitute 
%			\insertframenumber{} / \inserttotalframenumber\hspace*{1ex}
		\end{beamercolorbox}}%
		\vskip0pt%
	}
	\makeatletter
	\setbeamertemplate{navigation symbols}{}


%\useoutertheme[subsection=false]{miniframes}
%\makeatletter
%\let\beamer@writeslidentry@miniframeson=\beamer@writeslidentry
%\def\beamer@writeslidentry@miniframesoff{%
%	\expandafter\beamer@ifempty\expandafter{\beamer@framestartpage}{}% does not happen normally
%	{%else
%		 removed \addtocontents commands
%		\clearpage\beamer@notesactions%
%	}
%}
%\newcommand*{\miniframeson}{\let\beamer@writeslidentry=\beamer@writeslidentry@miniframeson}
%\newcommand*{\miniframesoff}{\let\beamer@writeslidentry=\beamer@writeslidentry@miniframesoff}
%\makeatother

\theoremstyle{plain}
%\newtheorem{theorem}{Theorem}[section]
%\newtheorem{corollary}[theorem]{Corollary}
%\newtheorem{lemma}[theorem]{Lemma}
\newtheorem{proposition}[theorem]{Proposition}
\newtheorem{claim}[theorem]{Claim}
\newtheorem{remark}[theorem]{Remark}
%\newtheorem{proof}[theorem]{Proof}
\newtheorem{assumption}[theorem]{Assumption}
\newtheorem{question}[theorem]{Question}
\newtheorem{conjecture}[theorem]{Conjecture}
\newtheorem{hypotheses}[theorem]{Hypotheses}


\newcommand{\argmax}{\operatornamewithlimits{argmax}}

%\usepackage{lipsum}
%\newcommand\Fontvi{\fontsize{6}{7.2}\selectfont}

% for specifying a name
\theoremstyle{plain} % just in case the style had changed
\newcommand{\thistheoremname}{}
\newtheorem{genericthm}[theorem]{\thistheoremname}
\newenvironment{namedtheorem}[1]
{\renewcommand{\thistheoremname}{#1}%
	\begin{genericthm}}
	{\end{genericthm}}

\newenvironment{namedremark}[1]
{\renewcommand{\thistheoremname}{#1}%
	\begin{genericthm}}
	{\end{genericthm}}

\newenvironment{namedcorollary}[1]
{\renewcommand{\thistheoremname}{#1}%
	\begin{genericthm}}
	{\end{genericthm}}

\newenvironment{namedproposition}[1]
{\renewcommand{\thistheoremname}{#1}%
	\begin{genericthm}}
	{\end{genericthm}}

\theoremstyle{definition}
%\newtheorem{definition}[theorem]{Definition}
%\newtheorem{remark}[theorem]{Remark}
%\newtheorem{example}[theorem]{Example}

\theoremstyle{remark}
\newtheorem*{notation}{Notation}
\newtheorem{exercise}[theorem]{Exercise}

% Common mathematical symbols
\newcommand{\CC}{\mathbb{C}}
\newcommand{\R}{\mathbf{R}}
\newcommand{\QQ}{\mathbb{Q}}
\newcommand{\ZZ}{\mathbb{Z}}
\newcommand{\NN}{\mathbb{N}}
\newcommand\dom{{\mathop{\rm dom}}}
\newcommand{\Gzero}{{\rm (G0)}}
\newcommand{\Gone}{{\rm (G1)}}
\newcommand{\Gtwo}{{\rm (G2)}}
\newcommand{\Gthree}{{\rm (G3)}}
\newcommand{\Gfour}{{\rm (G4)}}
\newcommand{\Gfive}{{\rm (G5)}}
\newcommand{\Gsix}{{\rm (G6)}}
\newcommand{\Gseven}{{\rm (G7)}}
\newcommand{\Geight}{\rm (G8)}
\newcommand{\nul}{{\emptyset}}
\newcommand{\Lip}{\operatornamewithlimits{Lip}}
\newcommand{\argmin}{\operatornamewithlimits{argmin}}
\newcommand{\yG}{ y_G}
\newcommand{\barG}{\bar{G}}
\newcommand{\barx}{\bar{x}}
\newcommand{\bary}{\bar{y}}



\title[Concavity of Principal-Agent Maximization Problem]{\textbf{On concavity of the principal's profit maximization facing agents who respond nonliearly to prices %On concavity of the monopolist's problem facing consumers with nonlinear price preferences%ON CONCAVITY OF THE PRINCIPAL'S PROFIT MAXIMIZATION FACING
		%AGENTS WHO RESPOND NONLINEARLY TO PRICES
		} }

\author[Kelvin Shuangjian Zhang with Robert McCann]{Kelvin Shuangjian Zhang\\This is joint work with my supervisor Robert J. McCann}
\institute{University of Toronto}
\date{
 May 15, 2018}


\begin{document}

\frame{\titlepage}

%\section[Outline]{}
%\frame{\tableofcontents}

\section[Introduction]{}
\subsection{}
\frame{
	\frametitle{Consumers' problem}
	\includegraphics[height=7cm]{carslides1}
	
	}
	
\frame{
	\frametitle{Monopolist's problem}	
		\includegraphics[height=7cm]{carslides2}
		
	}
\frame{
	\frametitle{Questions for Monopolist:}
	\begin{enumerate}[\topsep=0ex]
		\item 1. Can monopolist's total profit achieve its maximum by manipulating the price menus?
		\item 2. If the maximizers exist, under what conditions will that be unique?
		\item 3. What is the structure of profit functional?
	\end{enumerate}
	
	}
	
\frame{
	\frametitle{Main Result:}
	
	{\it Identified certain hypotheses} under which this maximization problem is {\bf strictly concave} on a {\bf convex} function space, where the maximizer is unique.
	}	
	
	
\frame
{
  \frametitle{Mathematical framework of principal-agent problem on nonlinear pricing}
  \begin{enumerate}[\topsep=0ex]
    \item monopolist(principal): produces and sells products  $y \in cl(Y)$, at price $v(y)\in \R$, to be designed, where $Y \subset {\R^n}$. 
    \item  consumers(agents): $x\in X$, where $X \subset \R^m$, buys one of those products which maximize his benefit 
    \begin{equation}\label{1}
    u (x) := \max_{y \in cl(Y)} G(x, y, v(y))
    \end{equation}
    given benefit function $G(x,y,z): X\times cl(Y) \times cl(Z) \longrightarrow \R$, denotes the benefit to consumer $x$ when he chooses product $y$ at price $z$, where $Z=(\underbar z,\bar z) $ with $-\infty <\underbar z < \bar z \le +\infty$.
    \begin{enumerate}[\topsep=0ex]
    	\item  e.g. $G(x,y,z) :=  \langle x, y \rangle - \sqrt{z}$, where $X,Y \subset \R^n$, $Z\subset [0, \infty]$ and $\langle \ ,\ \rangle$ is the Euclidean inner product in $\R^n$.
    \end{enumerate}
  \end{enumerate}
}

\frame
{
	\frametitle{Principal-agent framework on nonlinear pricing(cont'd)}
	\begin{enumerate}[\topsep=0ex]
		\item distribution of consumers: $d\mu (x)$ on $X$.
		\item profit gained  by monopolist: 
		\begin{equation}
			\Pi(v,y)=\int_{X} \pi(x, y(x), v(y(x))) d\mu(x),
		\end{equation}
		given profit function $\pi(x,y,z): X\times cl(Y) \times cl(Z) \longrightarrow \R$, which represents profit to the monopolist who sells product $\color{red} y$ to consumer $\color{red} x$ at price $\color{red} z$. 
		\begin{enumerate}[\topsep=0ex]
			\item where $\color{red} y(x)$ denotes that product $\color{red} y$ which consumer $\color{red} x$ chooses to buy, while the function $\color{red} v$ represents a price menu.
			\item e.g. $\pi(x,y,z) = z - a(y)$, where $\color{red} a(y)$ denotes the monopolist's cost of manufacturing $\color{red} y$.
		\end{enumerate}
		\item Question of monopolist: How to maximize her profit among all {\color{red}feasible price menus}?
	\end{enumerate}
}


%
%\frame
%{
%  \frametitle{Previous Existence Results by Others (2)}
% \begin{enumerate}[\topsep=0ex]
%  \item<1-> N$\ddot{o}$ldeke and Samuelson (2015, \cite{noldeke2015implementation}): provided a general existence result for $X,Y$ being COMPACT and benefit function $G$ being decreasing with respect to its third variable, by duality argument and Galois Connection. 
% \end{enumerate}
%
%}


\frame
{
\frametitle{Constraints}
\begin{definition}[incentive compatible]
A measurable map $x \in X \longmapsto (y(x),z(x)) \in cl(Y) \times %[v_{-1}, +\infty)
cl(Z)$ of consumers to (product, price) pairs
is called {\bf incentive compatible} if and only if $G(x,y(x),z(x)) \ge G(x, y(x'), z(x'))$ for all $x,x'\in X$.
\end{definition}
Such a map offers consumer $x$ no incentive to pretend to be another consumer type $x'$.

\begin{definition}[individually rational]
	A measurable map $x \in X \longmapsto (y(x),z(x)) \in cl(Y) \times %[v_{-1}, +\infty)
	cl(Z)$ of consumers to (product, price) pairs
	is called {\em \bf individually rational} if and only if $G(x,y(x),z(x)) \ge G(x,y_\nul,z_\nul)$ for all $x \in X$ and some fixed $(y_\nul, z_\nul)\in cl(Y\times Z)$,
	meaning no individual $x$ strictly prefers the outside option $(y_\nul, z_\nul)$ to his assignment $(y(x),z(x))$.
\end{definition}
}
%\begin{definition}[participation constraint]
%	There exists a function $u_{\emptyset}: X\longrightarrow \R$ such that the consumers' utility $u$ is bounded below by $u_{\emptyset}$, i.e. $u(x)\ge u_{\emptyset}(x)$, for all $x\in X$.
%\end{definition}
%	This constraint provides an outside option for each consumer so that he can choose not to participate if the maximum utility gained from buying activity is less than %his a priori expectation 
%	$u_{\emptyset}(x)$. %It also prevents monopolist to raise prices arbitrarily high.
%}

\frame{
	
	\frametitle{Monopolist's program}
		\begin{proposition}
		The monopolist's program can be described as follows:
		\begin{equation*}
		(P_0)
		\begin{cases}
		\sup \Pi(v,y)=\int_{X} \pi(x,y(x), v(y(x)) )d\mu(x)\\
		s.t.\\
%		y(x)\in \argmax\limits_{y\in cl(Y)} G(x,y,v(y)) \text{ for all } x \in X;\\
		x\in X \longmapsto (y(x),v(y(x))) \text{ incentive compatible};\\
		x\in X \longmapsto (y(x),v(y(x))) \text{ individually rational};\\
		{  v:cl(Y)\longrightarrow cl (Z)\  \text{lower semicontinuous with}\ v(y_\nul) \le z_\nul}. \\
		%u(x):=G(x,y(x),v(y(x)))\ge u_{\emptyset}(x) \text{ for all } x \in X;\\
		%\pi(x,y(x), v(y(x))) \text{ is measurable}.\\
		\end{cases}
		\end{equation*}
	\end{proposition}	
	}
	
	\frame
	{
		\frametitle{Previous Results by Others}
		\begin{enumerate}[\topsep=0ex]
			\item Existence: Mirrlees(1971), Spence(1974), ...,  Rochet(1987), ..., Rochet-Chon\'{e} (1998, \cite{rochet1998ironing}), Monteiro-Page(1998), ...,  Carlier(2001, \cite{carlier2001}), ..., N$\ddot{o}$ldeke $\&$ Samuelson (2015, \cite{noldeke2015implementation}), Zhang (2017), etc.
			\begin{enumerate}[\topsep=0ex]
				\item<1-> Mirrlees, Spence:  one-dimensional 
				\item<1-> Rochet-Chon\'e, Monteiro-Page, Carlier: quasi-linear pricing models, multi-dimensional
				\item<1-> N$\ddot{o}$ldeke $\&$ Samuelson, Zhang: nonlinear pricing model, multi-dimensional
			\end{enumerate}
%		\end{enumerate}
%
%		\begin{enumerate}[\topsep=0ex]
			\item Concavity and Stability: Figalli-Kim-McCann (FKM, 2011), etc.
			\begin{enumerate}[\topsep=0ex]
				\item<1-> FKM: quasilinear pricing model, total social welfare.
			\end{enumerate}
			\item Economic Interpretation: Mirrlees(1971), Spence(1974),..., Rochet-Chon\'{e} (1998, \cite{rochet1998ironing}), etc.
			\begin{enumerate}[\topsep=0ex]
				\item<1-> Nobel Economics Prizes: Mirrlees(1996), Spence(2001) %demonstrated that under certain conditions, well-informed consumers can improve their market outcome by signaling their private information to poorly informed consumers.
			\end{enumerate}
		\end{enumerate}
	}
	
	\frame{
		\frametitle{Why not quasi-linear?}
		\begin{enumerate}[$\blacksquare$]
			\item $G(x,y,z) = b(x,y) -z$, where utility $G$ linearly depends on prices $z$.
			\item $G(x,y,z) = b(x,y) -\sqrt{z}$, non-constant marginal utilities.
					\includegraphics[height=5cm,origin={2,0}]{f1}
			\item $G(x,y,z) = b(x,y)-f(x,z)$, where consumers have different sensitivities to the same price.
		\end{enumerate}
		
	}
	
%\begin{comment}
%\section{$b$-convexity and results of FKM}
%\subsection{}
%\frame{
%	\frametitle{$b$-convexity}
%	\begin{enumerate}[\topsep=0ex]
%		\item<1->[] \begin{definition}[$b$-convexity]
%			A function $u: cl(X) \longrightarrow \R$ is called $b$-convex if $u=(u^{b^*})^b$ and $v: cl(Y) \longrightarrow \R$ is called $b^*$-convex if $v=(v^b)^{b^*}$, where
%			\begin{equation}
%			v^b(x) = \sup\limits_{y\in cl(Y)} b(x,y)-v(y) \text{  and  } u^{b^*}(y)= \sup\limits_{x\in cl(X)} b(x,y)-u(x)
%			\end{equation}
%		\end{definition}
%		{\footnotesize Taking $b(x,y) = \langle x,y\rangle$, then $b$-convexity coincides with convexity.}
%		\item<2->[] \begin{definition}[$b$-concavity]
%			A function $u: cl(X) \longrightarrow \R$ is called $b$-concave if $(-u)$ is $(-b)$-convex, i.e., $u=-((-u)^{(-b)^*})^{(-b)}$. And $v: cl(Y) \longrightarrow \R$ is called $b^*$-concave if $(-v)$ is $(-b)^*$-convex.
%		\end{definition}
%	\end{enumerate}
%	}
%
%
%
%\frame{
%	\frametitle{Result of FKM}
%	\begin{theorem}[Figalli-Kim-McCann, 2011]
%		Suppose $m=n$, $G(x,y,z) = b(x,y) -z$, $\pi(x,y,z) = z-a(y)$,  $b$ satisfies $\bf (B0-B3)$ and the manufacturing cost $a: cl(Y)\longrightarrow \R$ is $b^*$-convex, then the monopolist's problem becomes a concave maximization over a convex set.
%	\end{theorem}
%}		
%
%
%\frame{
%	\frametitle{Hypotheses of FKM}
%	\begin{enumerate}[\topsep=0ex]
%	 \item<1->[]	\begin{hypotheses}[$G(x,y,z) = b(x,y) -z$, $m=n$]
%		 	\begin{enumerate}[\topsep=0ex]
%		 			\item [] {\bf (B0)} $b \in C^4(cl(X\times Y))$, where $X, Y\in \R^n$ are open and bounded;
%		 			\item [] {\bf (B1)} (bi-twist) Both $y\in cl(Y) \longmapsto D_x b(x_0, y)$ and  $x\in cl(X) \longmapsto D_yb(x, y_0)$ are diffeomorphisms onto their ranges, for each  $x_0 \in X$ and $y_0 \in Y$, respectively;
%		 			\item []  {\bf (B2)} (bi-convexity) Both ranges $D_x b(x_0, Y)$ and $D_yb(X, y_0)$ are convex subsets of $\R^n$, for each  $x_0 \in X$ and $y_0 \in Y$, respectively;
%		 		\item []  	{\bf (B3)} (non-negative cross-curvature)  
%		 		\begin{equation}
%		 		\frac{\partial^4}{\partial s^2 \partial t^2}\Bigg|_{(s,t) = (0,0)} b(x(s),y(t)) \ge 0
%		 		\end{equation}
%		 		whenever either $s\in [-1,1] \longmapsto D_yb(x(s), y(0))$ or $t\in [-1, 1] \longmapsto$ $ D_x b(x(0), y(t))$ forms an affinely parameterized line segment.
%		 	\end{enumerate}
%		 \end{hypotheses}	
%		\end{enumerate}
%	}
%\end{comment}

		
		
		
	
\section[G-convexity and Hypotheses%, G-subdifferential
]{}
\subsection{}


\frame{
		\frametitle{Our Hypotheses}

	\begin{enumerate}[\topsep=0ex]
%		\setlength\itemsep{0em}
%		\itemsep0em 
		\item<1->[]  \begin{hypotheses}
			\begin{enumerate}
				\item[\Gzero] $G \in C^{1}(cl(X\times Y \times Z))$, where $X\subset \R^m, Y \subset \R^n$ are open and bounded and $Z=(\underbar z,\bar z)$ with $-\infty <\underbar z < \bar z \le +\infty$.\\
				
				\item[\Gone] For each $x \in X$, the map $(y,z) \in cl( Y \times Z) \longmapsto (G_x, G)(x,y,z)$ is a homeomorphism onto its range;\\
				\item[\Gtwo] its range $(cl( Y \times Z))_x := (G_x,G)(x,cl(Y \times Z)) \subset \R^{m+1}$ is convex.\\
			%\item [] 	{ {\bf (C0)} (Strictly Monotonicity) $G(x,y,z)$ is strictly decreasing in $z$, for any $ (x,y)\in X\times  cl(Y)$, $z\in cl(Z)$;}
			\end{enumerate}
		\end{hypotheses}
%		\item<1-> [] {{\bf (C0)} is automatically satisfied when $G(x,y,z)=b(x,y)-z$.}
%{\small For all $x\in X, y\in Y,u\in G(x,y, cl(Z))$, define $H(x,y,u) := z$ by solving $z$ from $G(x,y,z) = u$, i.e. $H(x, y, \cdot)= G^{-1}(x,y,\cdot)$, for each $ x\in X, y\in Y$.}
%		\item<3-> []  \begin{hypotheses}
%			\begin{enumerate}[\topsep=0ex]
%			
%			\end{enumerate}
%		\end{hypotheses}
	\end{enumerate}
}
	


\frame{
	\frametitle{$G$-segment}
	\begin{enumerate}[\topsep=0ex]
		%		\setlength\itemsep{0em}
		%		\itemsep0em 
		\item<1-> [] \begin{definition}[$G$-segment]
			For each $x_0\in X$ and $(y_0, z_0),(y_1,z_1) \in cl(Y\times Z)$, 
			define $(y_t, z_t)$ $\in$ $ cl(Y\times Z)$ such that the following equation holds for each $t\in [0,1]$:
			{\small \begin{equation}\label{$G$-segment}
				(G_x, G)(x_0,y_t,z_t) = (1-t)(G_x, G)(x_0,y_0,z_0)+t (G_x, G)(x_0,y_1,z_1) 
				\end{equation}}
			By \Gone\ and \Gtwo, $(y_t, z_t)$ is uniquely determined by (\ref{$G$-segment}). 
			We call $t \in [0,1] \longmapsto (x_0,y_t,z_t)$ the  $G$-segment connecting $(x_0, y_0, z_0)$ and $(x_0, y_1, z_1)$.
		\end{definition}
	\end{enumerate}
}	

\frame{
	\frametitle{Our Hypotheses(cont'd)}
	\begin{enumerate}[\topsep=0ex]
		\item<1->[]  \begin{hypotheses}[cont'd]
			\begin{enumerate} 
				\item[\Gthree] For each $x,x_0 \in X$, assume $t \in [0,1] \longmapsto G(x, y_t, z_t)$ is convex along all $G$-segments ($\ref{$G$-segment}$).\\
				%\marginpar{The stronger hypothesis $G_z<0$ is often inferred and required; should we not build it into \Gfour?}
				\item[\Gfour]  For each $(x,y,z) \in X \times cl(Y)\times cl(Z)$, assume $G_{z}(x,y,z)<0$.  \\ 
%			\end{enumerate}
%		\end{hypotheses}
%		\item<1->[]  \begin{hypotheses}[cont'd]
%			\begin{enumerate} 
				\item[\Gfive] $\pi\in C^0(cl(X\times Y \times Z))$ and $u_\nul (x) := G(x,y_\nul,z_\nul)$ for some fixed
				$(y_\nul,z_\nul) \in cl(Y \times Z)$  satisfying
				$$G(x,y,\bar z)  := \lim_{z \to \bar z}G(x,y,z) \le G(x,y_\nul,z_\nul) \text{ for all}\ (x,y) \in X \times cl (Y).$$
				When $\bar z = +\infty$ assume this inequality is strict, and moreover that $z$ sufficiently large implies
				$$G(x,y,z) < G(x,y_\nul,z_\nul) \text{ for all}\ (x,y) \in X \times cl(Y).$$
			\end{enumerate}
		\end{hypotheses}
	\end{enumerate}
}	



%\frame{
%	\frametitle{}
%	\begin{enumerate}[\topsep=0ex]
%		\item<1->[]\begin{proposition}[1]\label{convex-subdiff}
%			A function $u: X \rightarrow \R$ is $G$-convex if and only if it is $G$-subdifferentiable everywhere.
%		\end{proposition}
%		\item<2->[]\begin{proposition}[2]\label{incen/convex}
%			Let $(y,z)$ be a pair of functions from $X$ to $cl(Y) \times cl(Z)$, then it represents an incentive compatible contract if and only if $u(\cdot):=G(\cdot,y(\cdot),z(\cdot))$ is $G$-convex on X and $y(x)\in \partial^G u(x)$ for each $ x \in X$.
%		\end{proposition}
%	
%	\end{enumerate}
%}



%
%\frame{
%	\frametitle{Hypotheses(cont'd)}
%\begin{enumerate}[\topsep=0ex]
%	\setlength\itemsep{0em}
%	\item<1->[]\begin{hypotheses}[cont'd]
%		\begin{enumerate}[\topsep=0ex]
%			\item [] {\bf (C1)} $G\in C^1(cl(X\times Y \times Z))$, where $X\in \R^m$, $Y\in \R^n$ are open and bounded and $Z=(\underline{z}, \bar{z})$ with $-\infty <\underline{z}<\bar{z}\le +\infty$;
%			\item [] {\color{gray} {\bf (B0)} $b \in C^4(cl(X\times Y))$, where $X, Y\in \R^n$ are open and bounded;}
%			\item [] {\bf (C2)} (twist) The map $(y,z)\in cl(Y\times Z) \longmapsto (G_x,G)(x_0,y,z)$ is homeomorphism onto its range, for each $x_0 \in X$;
%			\item [] {\color{gray} {\bf (B1)} (bi-twist) Both $y\in cl(Y) \longmapsto D_x b(x_0, y)$ and $x\in cl(X) \longmapsto$ $ D_yb(x, y_0)$ are diffeomorphisms onto their ranges, for each  $x_0 \in X$ and $y_0 \in Y$;}
%		\end{enumerate}
%	\end{hypotheses}
%	\item<1->[] The first part of {\bf (B1)} implies {\bf (C2)}, in the quasilinear case.
%\end{enumerate}
%}


\frame{
	\frametitle{$G$-convexity%, $G$-subdifferentiability
		}
	\begin{enumerate}[\topsep=0ex]
		\item<1->[]\begin{definition}[$G$-convexity: Trudinger, 2014; etc.]
			A function $u\in C^0(X)$ is called $G$-convex in $X$, if for each $x_0 \in X$, there exists $y_0 \in cl(Y)$, and $z_0 \in cl(Z)$ such that $u(x_0)=G(x_0, y_0, z_0)$, and $u(x)\ge G(x, y_0, z_0), \text{  for all  } x\in X$.
		\end{definition}
		{\footnotesize For $G(x,y,z) = \langle x,y \rangle -z$, $G$-convexity coincides with the ordinary convexity.}
	\end{enumerate}
	
	%	From the definition, we know if $u$ is a $G$-convex function, for any $x\in X$, there exists $y\in Y$, and $z\in I(x,y)$, such that
	%	\begin{equation}\label{2}
	%	u(x)= G(x, y, z),\ \ \   Du(x) = D_x G(x, y, z)
	%	\end{equation}
	%	
	%	Given $u(x), Du(x)$, one can solve $y, z$, according to Condition (C1).
	
}
%
%\frame{
%	\frametitle{ $G$-subdifferentiability
%	}
%	\begin{enumerate}[\topsep=0ex]
%		\item[]  \begin{definition}[$H(x, y, \cdot)= G^{-1}(x,y,\cdot)$]
%				{ For all $x\in X, y\in cl(Y),u\in G(x,y, cl(Z))$, define $H(x,y,u) := z$ where $z$ satisfies $G(x,y,z) = u$.}
%			\end{definition}
%		\item[]	\begin{definition}[$G$-subdifferential]
%			The $G$-subdifferential of a 
%			%$G$-convex % 
%			function $u(x)$ is defined by
%			\begin{equation*}
%			\partial^G u(x):= \{ y\in Y | u(x')\ge G(x',y, H(x,y,u(x))), \forall x'\in X\}
%			\end{equation*}	
%			A function $u$ is said to be $G$-subdifferentiable at $x$ if and only if $\partial^G u(x) \neq \emptyset$.
%		\end{definition}
%		{\footnotesize For $G(x,y,z) = \langle x,y \rangle -z$, $G$-subdifferential coincides with subdifferential.}		
%		
%	\end{enumerate}
%	
%	%	From the definition, we know if $u$ is a $G$-convex function, for any $x\in X$, there exists $y\in Y$, and $z\in I(x,y)$, such that
%	%	\begin{equation}\label{2}
%	%	u(x)= G(x, y, z),\ \ \   Du(x) = D_x G(x, y, z)
%	%	\end{equation}
%	%	
%	%	Given $u(x), Du(x)$, one can solve $y, z$, according to Condition (C1).
%	
%}
%
%\frame{
%	\frametitle{$G$-convexity, $G$-subdifferentiability, incentive compatibility}
%	\begin{itemize}
%		\item<1->[]
%%		\begin{lemma}[$G$-subdifferentiability characterizes $G$-convexity]\label{convex-subdiff}
%%			A function $u: X \rightarrow \R$ is $G$-convex if and only if it is $G$-subdifferentiable everywhere.
%%		\end{lemma}
%%		\item<2->[]
%		\begin{lemma}[$G$-convex utilities characterize incentive compatibility]\label{incen/convex}
%			Let $(y,z)$ be a pair of functions from $X$ to $cl(Y) \times cl(Z)$, then it represents an incentive compatible contract if and only if $u(\cdot):=G(\cdot,y(\cdot),z(\cdot))$ is $G$-convex on X and $y(x)\in \partial^G u(x)$ for each $ x \in X$.
%		\end{lemma}
%	\end{itemize}
%}
%%
%
%\frame{
%	\frametitle{Duality between prices and indirect utilities}
%	\begin{itemize}
%		\item[]\begin{proposition}[Duality between prices and indirect utilities]\label{Prop:Gtransform}
%			Assume \Gzero\ and \Gfour. \\
%			{\bf (a)} If for each $(x,y)\in X\times cl(Y)$, assume
%			\begin{equation}\label{repulsed}
%			G(x,y,\bar{z}) := \lim_{z \to \bar z}G(x,y,z) = \inf_{(\tilde{y}, \tilde{z})\in cl(Y\times Z)} G(x, \tilde{y}, \tilde{z}) ,
%			\end{equation} 
%			%and if these limits converge uniformly in $y$,  
%			then a function $u \in C^0(X)$ is $G$-convex if and only if there exist a lower semicontinuous 
%			$v: cl(Y) \longrightarrow cl(Z)$ such that $u(x) = \max_{y\in cl(Y)} G(x,y,v(y))$.\\
%			{\bf (b)} If instead of \eqref{repulsed} we assume \Gfive,
%			%\begin{equation}\label{repulsed}
%			%G(x,y,\bar{z}) = \inf_{(\tilde{y}, \tilde{z})\in Y\times Z} G(x, \tilde{y}, \tilde{z}) 
%			%\qquad \forall (x,y)\in X\times cl(Y).
%			%\end{equation} 
%			then $u_\nul := G(x, y_\nul, z_\nul) \le$ $ u \in C^0(X)$ is $G$-convex if and only if there exists a lower semicontinuous function $v: cl(Y) \longrightarrow cl(Z)$ with $v(y_\nul) \le z_\nul$ such that $u(x) = \max_{y\in cl(Y)} G(x,y,v(y))$.
%		\end{proposition}
%	\end{itemize}
%}
%
%
%
%\frame{
%	\frametitle{Restate Monopolist's problem}
%	\begin{enumerate}[\topsep=0ex]
%           \item [] \begin{theorem}[Reformulation of monopolist's problem using the consumers' indirect utilities]
%			Assume hypotheses \Gzero-\Gone and \Gfour--\Gfive, 
%			%$\pi \in C^0(cl(X\times Y \times Z))$, 
%			$\bar z < +\infty$  and $\mu 
%			\ll \mathcal{L}^m$. Setting \\ $\tilde{\Pi}(u,y)=\int_{X} \pi(x, y(x), H(x,y(x), u(x))) d\mu(x)$,
%			the monopolist's problem $(P_0)$ is equivalent to
%			\begin{equation*}
%			(P)
%			\begin{cases}
%			\max \tilde{\Pi}(u,y) \\%=\int_{X} \pi(x, y(x), H(x,y(x), u(x))) d\mu(x)\\
%			\text{\rm among } \\
%			$G$\text{\rm-convex  $u(x) \ge u_\nul(x)$ with }
%			%u(x)\ge u_{\emptyset}(x) \text{ \rm and 
%			y(x) \in \partial^G u(x)\ \text{\rm for all } x \in X.\\
%			\end{cases}
%			\end{equation*}
%			This maximum is attained. Moreover, $u$ determines $y(x)$ uniquely for a.e. $x \in X$.
%			
%		\end{theorem}
%	\end{enumerate}
%}

%
%\frame{
%	\frametitle{Further reformulation of Monopolist's maximization functional}
%	
%	By (C2), the optimal choice $y(x)$ of   Lebesgue almost every consumer $x\in X$
%	is uniquely determined by $u$. For $x\in \dom Du$, let $y(x, u(x), Du(x))$ be the unique solution $y$ of the system
%	\begin{equation}
%	u(x)= G(x, {\color{red}y}, {\color{red}z}),\ \ \   Du(x) = D_x G(x, {\color{red}y}, {\color{red}z}).
%	\end{equation}
%	Then the monopolist's problem (P) can be rewritten as maximizing a functional 
%	depending only on the consumers' indirect utility $u$:
%	\begin{equation*}
%	\pmb\Pi(u) := \int_X \pi(x, y(x, u(x), Du(x)), H(x,y(x, u(x), Du(x)),u(x)))  d \mu(x)
%	\end{equation*}
%	
%	on the space $\mathcal{U}_{\emptyset}:=\{ u: X\longrightarrow \R| u \text{ is $G$-convex and } u\ge u_{\emptyset}\}$.
%	
%}
%


\frame{
	\frametitle{Reformulation of Monopolist's  functional}
		\begin{enumerate}[\topsep=0ex]
			\item []
		By \Gone, the optimal choice $y(x)$ of   Lebesgue almost every consumer $x\in X$
		is uniquely determined by $u$. For $x\in \dom Du$, denote $\bar{y}_G(x, u(x), Du(x))$  the unique solution $(y,z)$ of the system
		\begin{equation}
		u(x)= G(x, {\color{red}y}, {\color{red}z}),\ \ \   Du(x) = D_x G(x, {\color{red}y}, {\color{red}z}).
%		u(x)= G(x, y, z),\ \ \   Du(x) = D_x G(x, y, z).
		\end{equation}
		\item [] \begin{proposition}[Reformulation of Monopolist's problem]
			Assume hypotheses \Gzero-\Gone and \Gfour--\Gfive, 
			%$\pi \in C^0(cl(X\times Y \times Z))$, 
			$\bar z < +\infty$  and $\mu 
			\ll \mathcal{L}^m$. Then the monopolist's problem $(P_0)$ can be rewritten as maximizing a functional 
			depending only on the consumers' utility $u$:
			{\small \begin{equation*}
			\pmb\Pi(u) := \int_X \pi(x, \bar{y}_G(x, u(x), Du(x)))  d \mu(x)
			\end{equation*}}
			on the space 
			\begin{flalign*}
				\mathcal{U}_{\emptyset}:=\{ u: X\longrightarrow \R| u \text{ is $G$-convex and } u\ge u_{\emptyset}\}.
			\end{flalign*}
		\end{proposition}
	\end{enumerate}
	
}


%	\setlength{\abovedisplayskip}{0pt}
%	\setlength{\belowdisplayskip}{0pt}


\section[Main result]{}
\subsection{}




\frame{
	\frametitle{Convexity of underlying space/Concavity of the functional}
	\begin{enumerate}
		\item<1-> [] \begin{namedtheorem}{Theorem 1}[McCann-Z., 2017]\label{maintheorem1}
			If $G: cl(X\times Y\times Z) \longrightarrow \R$ satisfies \Gzero-\Gtwo, then \Gthree\ becomes necessary and sufficient for the convexity of the set $\mathcal{U}_\nul$.
		\end{namedtheorem}
		\item<2-> [] \begin{namedtheorem}{Theorem 2}[McCann-Z., 2017]\label{maintheorem2}
			If $G$ and $\pi: cl(X\times Y\times Z) \longrightarrow \R$ satisfy \Gzero-\Gfive, the following statements are equivalent:\\
			\begin{enumerate}
				\item[(i)]  $t\in[0,1] \longmapsto \pi(x, y_t(x) ,z_t(x))$ is concave  along all G-segments $(x, y_t(x), z_t(x))$ whose 
				endpoints satisfy $\min\{G(x, y_0(x), z_0(x)),G(x,y_1(x),z_1(x))\} \ge u_{\emptyset}(x)$;\\
				\item[(ii)] $\pmb \Pi(u)$ is concave in $\mathcal{U}_{\emptyset}$  for all $\mu\ll \mathcal{L}^m$. 
			\end{enumerate}
		\end{namedtheorem}
	\end{enumerate}
	 
	 	
	

}


\frame{
	\frametitle{Remarks}
		\begin{enumerate}[\topsep=0ex]
			\item<1->[]\begin{namedremark}{Remark 1}[Strictly Concavity of Monopolist's functional]
					Let $\pi$ and $G$ satisfy \Gzero--\Gfive. If  $t\in[0,1] \longmapsto \pi(x, y_t(x) ,z_t(x))$ is strictly concave along all G-segments $(x, y_t(x), z_t(x))$ whose endpoints satisfy $\min\{G(x, y_0(x), z_0(x)),G(x,y_1(x),z_1(x))\} \ge u_{\emptyset}(x)$, then 
					$\pmb \Pi(u)$ is strictly concave in $\mathcal{U}_{\emptyset} \subset W^{1,2}(X,d\mu)$ for all $\mu\ll \mathcal{L}^m$. 
				
				%Under the same hypotheses as in Theorem 2,  (i) remains equivalent to (ii) when both occurences of concavity are replaced by convexity; (i) implies to (ii) when both occurences of concavity are replaced by strictly concavity or strictly convexity, respectively.
			\end{namedremark}
			
		
			\item<2->[] 	\begin{namedremark}{Remark 2}[Uniqueness]
				Theorem 1 and Theorem 2 [strict version] together imply uniqueness of monopolist's maximization problem.
			\end{namedremark} 
			
		\end{enumerate}
	}
	

\frame{
	\frametitle{Our Hypotheses(cont'd)}
	\begin{enumerate}[\topsep=0ex]
		\setlength\itemsep{0em}
		\item[] Define $\bar{G}(\bar{x}, \bar{y})=\bar{G}(x,w, y,z) := w G(x, y,z)$, where $\bar{x}=(x, w)$, $\bar{y}=(y,z)$ and $w\in (0, \infty)$. \\
		
		
		\item[]\begin{hypotheses}[cont'd]
			\begin{enumerate}[\topsep=0ex]
				\item [\Gsix]%\footnote{\footnotesize When $G(x,y,z) = b(x,y)-z$, $\bf (B1)$ implies \Gsix.} % []{\Gsix}
				(non-degeneracy) $G\in C^2(cl(X\times Y \times Z)%\RR
				)$, and $D_{\bar{x},\bar{y}}(\bar{G})(x,1,y,z)$ has full rank, for each $(x,y,z)\in cl(X\times Y\times Z)$. \\
			\end{enumerate}
		\end{hypotheses}
		Assuming \Gsix,  we denote $(D^2_{\bar{x},\bar{y}}\bar{G})^{-1}$ the left inverse of $D_{\bar{x},\bar{y}}(\bar{G})(x,-1,y,z)$.
		
			
	\end{enumerate}
}	
	
\frame{
	\frametitle{Uniformly concavity of the monopolist's objective}
	\begin{enumerate}%[\topsep=0ex]
	\item<1-> [] \begin{namedtheorem}{Theorem 3}[McCann-Z., 2017]\label{maintheorem3}
		Assume $G\in  C^{2}(cl(X\times Y\times Z))$ satisfies \Gzero-{\Gsix}. In case $\bar{z}=+\infty$, assume the homeomorphisms of \Gone\ are uniformly bi-Lipschitz. Then the following statements are equivalent:\\
			\begin{enumerate}
			\item[(i)]  Uniformly concavity of $\pi$ along G-segments $(x, y_t(x), z_t(x))$ whose 
			endpoints satisfy $\min\{G(x, y_0(x), z_0(x)),G(x,y_1(x),z_1(x))\} \ge u_{\emptyset}(x)$, i.e., there exists $\lambda>0$, for any such $G$-segment $(x, y_t, z_t)$, and any $t\in [0,1]$, 
			\begin{flalign*}\label{uniformconvavity}
			\begin{split}
			&\pi(x, y_t(x), z_t(x)) - (1-t)\pi(x, y_0(x), z_0(x)) - t\pi(x, y_1(x), z_1(x)) \\
			\ge & t(1-t)\lambda||(y_1(x)-y_0(x),z_1(x)-z_0(x))||^2_{\R^{n+1}}
			\end{split}
			\end{flalign*} \\
			\item[(ii)] $\pmb \Pi(u)$ is uniformly concave in $\mathcal{U}_\nul \subset W^{1,2}(X,d\mu)$,  uniformly for all $\mu\ll \mathcal{L}^m$. 
		\end{enumerate}
	\end{namedtheorem}
	
\end{enumerate}

	}



\frame{
	\frametitle{Characterizing concavity of functional in the smooth case}
\begin{itemize}
	\item<1->[] \begin{lemma}[Characterizing concavity of monopolist's profit in the smooth case]\label{LemmaProfitConcavity}
			When $G \in C^3(cl(X\times Y \times Z))$ satisfies \Gzero-\Gsix \ and  $\pi \in C^2(cl(X\times Y \times Z))$, then differentiating $\pi$ along an arbitrary $G$-segment $t \in[0,1] \longrightarrow (x,y_t,z_t)$ yields
			\begin{equation}\label{pi second}
			\frac{d^2}{dt^2} \pi(x, y_t, z_t) = (\pi_{,\bar{k}\bar{j}}- \pi_{,\bar{l}} \bar{G}^{\bar i,\bar l}\bar{G}_{\bar{i},\bar{k}\bar{j}}) \dot {\bar y}^{\bar k} \dot {\bar y}^{\bar j}
			%\cdot (\partial_t y_t, \partial_t z_t)^{\bar{k}}(\partial_t y_t, \partial_t z_t)^{\bar{j}}
			\end{equation}
			where $\bar{G}^{\bar i,\bar l}$ denotes the left inverse of the matrix $\bar{G}_{\bar{i}, \bar{k} }$
			and $\dot {\bar y}^{\bar k} = (\frac{d}{dt})\bar y^{\bar k}_t$.
			Thus $(i)$ in Theorem 2  is equivalent to non-positive  definiteness of the quadratic form 
			$\pi_{,\bar{k}\bar{j}}- \pi_{,\bar{l}} \bar{G}^{\bar i,\bar l}\bar{G}_{\bar{i},\bar{k}\bar{j}}$
			%\big|_{x_0=-1}$ 
			% $\pi_{,\bar{y}\bar{y}} - \pi_{,\bar{y}}(\bar{G}_{\bar{x},\bar{y}})^{-1}\bar{G}_{\bar{x},\bar{y}\bar{y}}$
			%	(as a matrix with indices $\bar{k},\bar{j}$)  
			on $\R^{n+1}$, { for each  $(x, \bar y) \in X \times Y \times Z$.} Similarly, Theorem 3 $(i)$ is equivalent to uniform negative definiteness of the same form.
		\end{lemma}	
\end{itemize}
}

%	
%\frame{
%	\frametitle{Comparison}
%	\begin{enumerate}[\topsep=0ex]
%		\item<1->[]	\begin{namedcorollary}{Corollary 1}[Concavity of monopolist's objective with her utility not depending on consumers' types]\label{J^*-Concavity}
%			If $G \in C^3(cl(X\times Y \times Z)%\RR
%			)$ satisfies \Gzero-\Gsix,  $\pi \in C^2( cl( Y \times Z)%\R
%			)$ is $\bar{G}^*$-concave and $\mu\ll \mathcal{L}^m$, then $\pmb \Pi$ is concave. 
%		\end{namedcorollary}
%%		{\color{red} Here recall $(\bar{G})^*-concave$ again!}
%		\item<2->[]  \begin{theorem}[Figalli-Kim-McCann, 2011; $m=n$, $G(x,y,z)=b(x,y)$ $-z$, $\pi(x,y,z) =z-a(y)$]
%				If $b$ satisfies $\bf (B0-B3)$ and the manufacturing cost $a: cl(Y)\longrightarrow \R$ is $b^*$-convex, then the monopolist's problem becomes a concave maximization over a convex set.
%				\end{theorem}
%	\end{enumerate}
%
%	}
%	

%	\frame{
%	\frametitle{A shaper result}
%	
%		\begin{definition}[$(-G)$-concavity]\label{-Gconcavity}
%			{\color{red} 		A function $\pi : cl( Y \times Z) \longrightarrow \R\cup \{+\infty\}$, not identically $+\infty$, is said to be $(-G)$-concave if there exists  $J\subset cl(X)$, such that $\pi(\bar{y}) = \inf\limits_{x\in cl(J)} -G(x, \bar{y})$, for all $\bar{y} \in cl( Y \times Z)$.}
%			\end{definition}
%			
%			\begin{namedcorollary}{Corollary 2}
%				{\color{red} 		Suppose $G \in C^3(cl(X\times Y \times Z)%\RR
%				)$ satisfies $(C0-C5)$,  $\pi \in C^2( cl( Y \times Z)%\R
%				)$, $\mu\ll \mathcal{L}^m$, if $\pi$ is $(-G)$-concave, i.e., there exists $J\in cl(X)$  such that $\pi(\bar{y}) = \min\limits_{x\in cl(J)} -G(x, \bar{y})$ for each $\bar{y} \in cl( Y \times Z)$, and the equation $ (\pi+G)_{\bar{y}}(x, \bar{y})=0$ has unique solution $x\in cl(J)$ for each $\bar{y} \in  Y \times Z$, then $\pmb \Pi$ is concave.}
%				%			In addition, if %either 
%				%			$G$ satisfies $(C4)_s$, (and when $m=n$, $(\pi+G)$ satisfies a (C4)-like condition,) %{\color{red}  or $\pi$ is strictly $(-G)$-concave[fails in general cases]}
%				%			then $\pmb \Pi$ is strictly concave.
%				\end{namedcorollary}
%	}
%\frame{
%	\frametitle{Necessary Condition}
%		\begin{remark}
%			1. If $m=n+2$ and $G \in C^3(cl(X\times Y \times Z)%\RR
%			)$ satisfies $(C0-C5)$,  $\pi \in C^2( cl( Y \times Z)%\R
%			)$ and $\mu\ll \mathcal{L}^m$, there exists $J\in cl(X)$  such that the equation $ (\pi+G)_{\bar{y}}(x, \bar{y})=0$ has unique solution $x\in cl(J)$ for each $\bar{y} \in  Y \times Z$, then under certain hypotheses stated in  [Trudinger14, Lemma 2.1], $\pi(\bar{y}) = \min\limits_{x\in cl(J)} -G(x, \bar{y})$ for each $\bar{y} \in cl( Y \times Z)$ if and only if $\pmb \Pi$ is concave.\\
%			2. If $m=n$, $G(x,y,z)= b(x,y)-z \in C^3(cl(X\times Y \times Z)%\RR
%			)$ satisfies (C0-C5),  $\pi(y,z)=z-a(y) \in C^2( cl( Y \times Z)%\R
%			)$ and $\mu\ll \mathcal{L}^m$, under certain hypotheses stated in  [Kim-McCann10, Theorem A.1],  $a$ is $b^*$-convex if and only if $a$ is locally $b^*$-convex, hence if and only if $\pmb \Pi$ is concave.
%		\end{remark}
%		}
\frame{
	\frametitle{Example 1}
	\begin{example}[Nonlinear yet homogeneous sensitivity of consumers to prices]
		\label{general example1}
	{\small Take $\pi(x, y, z) =z- a(y)$, $G(x, y, z) = b(x,y)-f(z)$, satisfying \Gzero-\Gsix,  $G \in C^3(cl(X\times Y \times Z)%\RR
	)$, $\pi \in C^2(cl(X\times Y \times Z)%\RR
	)$, and assume $\bar{z}<+\infty$.\\}
		\begin{enumerate}[\topsep=0ex]
			\item<1-> {\small 1. If $f(z)$ is convex [respectively concave] in $cl(Z)$, then $\pmb \Pi(u)$ is concave [respectively convex] for all $\mu\ll \mathcal{L}^m$ if and only if there exists $\varepsilon \ge 0$ such that each $(x,y,z) \in X \times Y\times Z$ and $\xi \in \R^{n}$ satisfy }
			{\scriptsize \begin{equation}\label{robert1}
			\pm \Bigg\{a_{kj}(y)-\frac{b_{,kj}(x,y)}{f'(z)}+\Big(\frac{b_{,l}(x,y)}{f'(z)}- a_l(y)\Big)b^{i,l}(x,y) b_{i,kj}(x,y)\Bigg\} \xi^{k}\xi^{j} \ge  \varepsilon \mid \xi\mid ^2.%(\le - \varepsilon \mid \xi\mid ^2).
			\end{equation}}
			\item<1-> {\small 2. In addition, $\pmb \Pi(u)$ is uniformly concave [respectively uniformly convex] on $W^{1,2}(X,d\mu)$  uniformly for all $\mu\ll \mathcal{L}^m$ if and only if $\pm f''> 0$ and  \eqref{robert1} holds with $\varepsilon >0$.}
		\end{enumerate}
	\end{example}
}
	

%\frame{
%	\frametitle{Example 2}
%	\begin{Example}[Inhomogeneous sensitivity of consumers to prices, zero cost]
%		Take $\pi(x, y, z) =z$,  $G(x,y,z)$ $= b(x,y)-f(x,z)$,  satisfying $\bf (C0-C5)$,  $G \in C^3(cl(X\times Y\times Z)%\RR
%		)$, $\pi \in C^2(cl(X\times Y\times Z)%\RR
%		)$ and $\mu\ll \mathcal{L}^m$. Suppose $D_{x,y}b(x,y)$ has full rank for each $(x,y) \in X\times Y$.\\
%		\begin{enumerate}[\topsep=0ex]
%			\item <1-> 1.	If {\small $ (x,y,z)\longmapsto h(x,y,z):=f(x,z)-b_{,l}(x,y)(b_{i,l}(x,y))^{-1}$ $f_{i,}(x,z)$} is strictly increasing and convex with respect to $z$, then $\pmb \Pi(u)$ is concave if and only if there exist $\varepsilon \ge 0$ such that each $(x,y) \in X \times Y$ and $\xi \in \R^{n}$ satisfy 
%			$$ \Big\{-b_{,kj}(x,y)+b_{,l}(x,y)[ b_{i,l}(x,y)]^{-1} b_{i,kj}(x,y)\Big\} \xi^{k}\xi^{j} \ge  \varepsilon  |\xi| ^2.%(\le - \varepsilon \mid \xi\mid ^2).
%			$$
%			\item<1-> 2. If in addition, $ h_{zz}> 0$ and $\varepsilon >0$,
%			then  $\pmb \Pi(u)$ is strictly concave.
%		\end{enumerate}
%
%	\end{Example}
%	
%	
%}
%	
	\frame{
	
	\frametitle{Example 2}
	\begin{example}[Inhomogeneous sensitivity of consumers to prices]\label{general example3}
		{\small	Take $\pi(x, y, z) =z-a(y)$,  $G(x,y,z)= b(x,y)-f(x,z)$,  satisfying \Gzero-\Gsix,  $G \in C^3(cl(X\times Y \times Z)%\RR
		)$, $\pi \in C^2(cl(X\times Y \times Z)%\RR
		)$, and assume $\bar{z}<+\infty$. Suppose $D_{x,y}b(x,y)$ has full rank for each $(x,y) \in$ $ X\times Y$, and $1- (f_{z})^{-1}b_{,\beta}b^{\alpha,\beta}f_{\alpha,z} \ne 0$, for all $(x, y,z) \in X\times Y\times Z$.}\\
		\begin{enumerate}[\topsep=0ex]
			\item<1->  {\small 1. If {\scriptsize $(x,y,z)\longmapsto h(x,y,z):=a_{l}b^{i,l}f_{i,zz}+\frac{[a_{\beta}b^{\alpha,\beta}f_{\alpha,z}-1][b_{,l}b^{i,l}f_{i,zz}-f_{zz}]}{f_{z} -b_{,\beta}b^{\alpha,\beta}f_{\alpha,z}} \ge 0 [\le 0]$}, then $\pmb \Pi(u)$ is concave  [respectively convex] for all $\mu\ll \mathcal{L}^m$ if and only if there exists $\varepsilon \ge 0$ such that each $(x,y,z) \in X \times Y\times Z$ and $\xi \in \R^{n}$ satisfy}
		{\tiny \begin{equation}\label{robert3}
		\pm \Bigg\{a_{kj} -a_{l}b^{i,l}b_{i,kj}+\frac{1-a_{\beta}b^{\alpha,\beta}f_{\alpha,z}}
		{1- (f_{z})^{-1}b_{,\beta}b^{\alpha,\beta}f_{\alpha,z}}
		\Big[-\frac{b_{,kj}}{f_z}+\frac{b_{,l}}{f_z}b^{i,l} b_{i,kj}\Big] \Bigg\}\xi^{k}\xi^{j} \ge  \varepsilon \mid \xi\mid ^2.%(\le - \varepsilon \mid \xi\mid ^2).
		\end{equation}}
		\item<1-> {\small 2. If in addition, $\pmb \Pi(u)$ is uniformly concave [respectively uniformly convex] on $W^{1,2}(X,d\mu)$ uniformly for all $\mu\ll \mathcal{L}^m$ if and only if $\pm h>0$ and  \eqref{robert3} holds with $\varepsilon>0$.}
		\end{enumerate}
	\end{example}
}
	
	
	\frame{
		\frametitle{Example 3}
		\begin{enumerate}[\topsep=0ex]
			\item<1->[] \begin{example}[Zero sum transactions]\label{general example 4}
				Take $\pi(x, y, z) = -G(x,y,z)$, satisfying \Gzero-\Gfive\ and $\mu\ll \mathcal{L}^m$, which means the monopolist's profit in each transaction coincides exactly with the 
				consumer's loss. Then $\pmb \Pi(u)$ is linear.
			\end{example}	
		\end{enumerate}
}
	
	
%\section[Equivalence of $\bf (B3)$ and $\bf (C5)$]{}	
	
\frame{
	\frametitle{Analytic representation of $\bf \Gthree$}%Our Hypotheses(cont'd)}
	\begin{hypotheses}[cont'd]
		\begin{enumerate}[\topsep=0ex]
			\item[\Gseven] For each $(y, z)\in cl( Y \times Z)$, the map $x \in X \longmapsto \frac{G_y}{G_z}(\cdot, y,z)$ is one-to-one;\\
			\item[\Geight] its range $X_{(y,z)} := \frac{G_y}{G_z}(X,y,z) \subset \R^n$ is convex.\\
		\end{enumerate}
	\end{hypotheses}
}

\frame{
	\frametitle{Analytic representation of $\bf \Gthree$}
	\begin{namedproposition}{Proposition 1}[$\bf \Gthree$: a 4-th order hypothesis]
		Assuming $m=n$,  $G\in C^4(cl(X\times Y \times Z)%\RR
		)$ satisfying $\bf \Gzero-\Gtwo,$ $\bf \Gfour-\Geight$,  then $\bf \Gthree$ is equivalent to:\\
		For any given curve $x_s\in X$ connecting $x_0$ and $x_1$,  and any curve $(y_t, z_t) \in cl(Y\times Z)$ connecting $(y_0,z_0)$ and $(y_1, z_1)$, we have 
		\begin{equation}
		\frac{\partial^2}{\partial s^2 }\Biggl(\frac{1}{G_z(x_s, y_t, z_t)}\frac{\partial^2}{\partial t^2} G(x_s,y_t,z_t) \Biggr)\Bigg|_{(s,t) = (s_0,t_0)}\le 0,
		\end{equation}
		whenever either of the two curves $t\in [0,1] \longmapsto (G_x, G)(x_{s_0}, y_t, z_t)$ and  $s\in [0,1] \longmapsto \frac{G_y}{G_z}(x_s, y_{t_0}, z_{t_0})$ forms an affinely parametrized line segment.\\
	\end{namedproposition}
	}	
	
\frame{
		\frametitle{Geometric representation of $\bf \Gthree$}
		For any fixed  $\Xi_0=(\barx_0, \bary_0)$ on  $M = X \times (0, \infty) \times Y \times \R$, define $$\delta(\barx, \bary, \barx_{0}, \bary_{0}) = -\barG (\barx, \bary) -\barG(\barx_0, \bary_0) +\barG(\barx, \bary_0) + \barG(\barx_0, \bary)$$%\vspace{-0.6cm}
		
		as a function of variable $\Xi =(\barx, \bary)$ on  $M$. Let $T_{\Xi_0}M$ denote the tangent spaces to $M$ at ${\Xi_0}$. Define the pseudo-Riemannian metric $ g_{\Xi_0}: T_{\Xi_0}M\times T_{\Xi_0}M \rightarrow \R$ at $\Xi_0$ on $M$ as the $2(n+1) \times 2(n+1)$  symmetric matrix: $\nabla_{ij} \delta (\Xi, \Xi_0){\big|}_{\Xi=\Xi_0 }$. Denote $R$ to be the corresponding pseudo-Riemannian curvature tensor.
		
		\begin{namedproposition}{Proposition 2}[$\bf \Gthree$: a non-negative curvature hypothesis]
			Assuming $m=n$,  $G\in C^4(cl(X\times Y \times Z)%\RR
			)$ satisfying $\bf \Gzero-\Gtwo,$ $\bf \Gfour-\Geight$,  then $\bf \Gthree$ is equivalent to:\\
			For any vector $P = p \oplus 0$, $Q = 0 \oplus q \in \R^{2n+2}$, and any point $\Xi_{\nul} = (\barx_{\nul}, \bar{y}_{\nul})\in M$, the sectional curvature satisfies
			\begin{equation}\label{Curvature_sectional2}
			\sec_{\Xi_{\nul}}^{(M, g)} P \wedge Q := R_{ijkl}(\Xi_{\nul})\cdot P^{i}\cdot P^{l} \cdot Q^{j} \cdot Q^{k} \ge 0.
			\end{equation}
		\end{namedproposition}
	}	
	

%\frame
%{
% \frametitle{Remarks}
% \begin{enumerate}[\topsep=0ex]
%
% \end{enumerate}
%
%}
%
%
%
%
%
%
%\section[Idea of the proof]{}
%\frame{
%\frametitle{Idea of the proof}
% For a convex curve $\mathcal{C}$ (parametrized by gauss map) evolving under the generalized curve shortening flow, we have the following monotone quantity
% \begin{equation}
%\mathcal{L}^m=\begin{cases}
% \sqrt{\frac{A(\mathcal{C})}{\pi}}(\frac{1}{2\pi}\int_{S^1}\kappa^{\alpha-1}d\theta)^{\frac{1}{\alpha-1}}\ \text{if}\ \alpha\ne 1 \\
% \sqrt{\frac{A(\mathcal{C})}{\pi}}\left(\exp\{\frac{1}{2\pi}\int_{S^1}-\log(\kappa) d\theta\}\right)^{-1} \ \text{if}\ \alpha=1,
% \end{cases}
% \end{equation}
% where $A(\mathcal{C})$ is the area enclosed by the curve, and $\kappa$ is the curvature.
%
%When $\alpha>\frac{1}{3}$ ($\alpha=\frac{1}{3}$),
%  $\mathcal{L}^m$ is decreasing strictly under the flow unless $\mathcal{C}$ is a(an) circle (ellipse).
%}
%\frame{
%\begin{enumerate}[\topsep=0ex]
%   \item<1->Given a convex compact ancient solution $X:\ell\times(-\infty,0)\rightarrow\mathbb{R}^{2}$ to \eqref{E1}, when $\alpha>\frac{1}{3}$, we know that $\mathcal{C}_t:=(-(1+\alpha)t)^{-\frac{1}{1+\alpha}}X(\ell, t)\rightarrow S^1$ smoothly as $t\rightarrow 0.$ Notice that by definition of $\mathcal{L}^m$, we have $\mathcal{L}^m(\mathcal{C}_{t})\rightarrow \mathcal{L}^m(S^1)$, as $t\rightarrow 0.$
%   \item<2->We will prove similar property in another direction, namely one can find a sequence $t_{m}\rightarrow-\infty$ as $m\rightarrow\infty$, such that
%  $\mathcal{C}_{t_m}\rightarrow S^1$ smoothly. Then $\mathcal{L}^m(\mathcal{C}_{t_m})\rightarrow \mathcal{L}^m(S^1),$ as $m\rightarrow \infty$.
%    \item<3->By the monotonicity of $\mathcal{L}^m$, we have $\mathcal{L}^m(\mathcal{C}_{t_m})>\mathcal{L}^m(\mathcal{C}_{-1})>\mathcal{L}^m(\mathcal{C}_{t}),$ when
%    $m$ is large, and $t$ is close to 0. By taking limits $m\rightarrow \infty$ and $t\rightarrow 0$ we see that $\mathcal{L}^m(\mathcal{C}_{-1})=\mathcal{L}^m(S^1)$, which implies $\mathcal{C}_{-1}$ must be a circle.
%\end{enumerate}
%}
%
%
%
%
%
%
%\section[Outline of the proof]{}
%\frame{
%\frametitle{Outline of the proof}
%
% Step 1.  Representing the generalized curve shortening flow as the level sets of a function, namely we can define a
%  function $u: \mathbb{R}^{2}\rightarrow \mathbb{R}$, such that $\{x|u(x)=-t\}=X(\ell,t)$ evolves under the generalized curve shortening flow. A simple computation shows that
%  $u$ satisfies the following equation
%  \begin{equation}
%  \label{E2}
%  \text{div}\frac{Du}{|Du|}=\frac{1}{|Du|^{\frac{1}{\alpha}}}.
%  \end{equation}
%  By using maximum principle, we can prove a very useful property which says that when the solution to the flow is convex and ancient,  the corresponding $u$ not only has convex level sets, but is a convex function.
%}
%
%\frame{
%\frametitle{Outline of the proof}
%Step 2. Power growth estimate for entire convex solutions of equation \eqref{E2}.  We have the following lemma
%\begin{lemma}\label{L1}
%Let $u$ be an entire convex solution of \eqref{E2} in $\mathbb{R}^{2}$ with $\alpha>\frac{1}{2}$, then
%\begin{eqnarray}
%\label{E3}
%u(x)\leq C(1+|x|^{1+\alpha}),
%\end{eqnarray}
%where the constant $C$ depends only on the upper bound for $u(\textbf{0})$ and $|Du(\textbf{0})|$.
%\end{lemma}
%
%
%
%}
%\frame{
%\frametitle{Outline of the proof}
%
%
% \begin{enumerate}[\topsep=0ex]
% \item<1-> This estimate implies that if we have a family of entire convex solutions to \eqref{E2} with bounded  $u(\textbf{0})$ and $|Du(\textbf{0})|$, then this family of solutions is compact.
% \item<2-> The proof of the above estimate leads to an interesting corollary which says that if a complete convex solution to \eqref{E2} is not entire, then it must be defined in a strip region.
% \end{enumerate}
%
%}
%
%\frame{
%\frametitle{Outline of the proof}
%Step 3. The Theorem is proved by combining the above lemma, rescaling invariance and the fact that the quantity $\mathcal{L}^m$ is decreasing under the flow. We will assume $\textbf{0}$ is the minimum point of $u$ and $u(\textbf{0})=0$.
%\begin{enumerate}[\topsep=0ex]
% \item<1-> Notice that if $u$ is a solution of \eqref{E2}, then $u_{h}(x):=\frac{1}{h}u(h^{\frac{1}{1+\alpha}}x)$ is also a solution of \eqref{E2}.
% \item<2-> Since $u(\textbf{0})=0$ and $|Du_{h}(\textbf{0})|=h^{-\frac{\alpha}{1+\alpha}}|Du(\textbf{0})|=0$ for all $h$, the power growth estimate guarantees that we can take a subsequence $u_{h_{m}}$ converging to a solution $u_{\infty}$ of \eqref{E2} locally uniformly. Note that, by the power growth estimate, $u_{\infty}$ has compact level sets.
%\end{enumerate}
%}
%
%
%
%
% \frame{
%\frametitle{}
%
%Simple computation shows that $$\{u_{h_{m}}=\frac{1}{1+\alpha}\}=\mathcal{C}_{t_m}=(-(1+\alpha)t_m)^{-\frac{1}{1+\alpha}}\tilde{X}(\ell, t_m),$$ with $t_m=-\frac{1}{1+\alpha}h_m\rightarrow -\infty$, as $m\rightarrow \infty.$
%}
%
%
%
%
%
%
%
\begin{thebibliography}{03}
	
		
	\bibitem{rochet1998ironing}Rochet, Jean-Charles and Chon{\'e}, Philippe,
	\emph{Ironing, sweeping, and multidimensional screening.}
	Econometrica  (1998), pp. 783--826.
	
	\bibitem{carlier2001}Carlier, Guillaume,
	\emph{A general existence result for the principal-agent problem with adverse selection.}
	Journal of Mathematical Economics {\bf 35} (2001), no.~1, pp. 129--150.
	
	\bibitem{noldeke2015implementation}Noldeke, Georg and Samuelson, Larry,
	\emph{The Implementation Duality.}
	Cowles Foundation Discussion Paper (2015).
	
	\bibitem{trudinger2013}Trudinger, Neil S,
	\emph{On the local theory of prescribed Jacobian equations.}
	Discrete Contin. Dyn. Syst. {\bf 34} (2014), no.~4, pp. 1663-1681.
	
	\bibitem{fkm2011}Figalli, Alessio and Kim, Young-Heon and McCann, Robert J,
	\emph{When is multidimensional screening a convex program?}
	Journal of Economic Theory {\bf 146} (2011), no.~2, pp. 454--478.
	
	
\end{thebibliography}


%\section{Thank you}
\frame
{
   \LARGE
   \begin{center}
  {\color{blue}{Thank you!}}
   \end{center}

}

	\frame{	
		\frametitle{A shaper result}
		\begin{itemize}
			\item<1->[]\begin{namedproposition}{Proposition 1}[Concavity of monopolist's objective with her utility not depending on consumers' types 2]\label{bar{G}^*-Concavity2}
				Suppose $G \in C^3(cl(X\times Y \times Z)%\RR
				)$ satisfies \Gzero-\Gsix,  $\pi \in C^2( cl( Y \times Z)%\R
				)$, and assume there exists a set $J\subset cl(X)$ such that for each $\bar{y}\in Y\times Z$, $ 0\in ( \pi_{\bar{y}}+G_{\bar{y}})(cl(J), \bar{y})$.  Then the following statements are equivalent:
				\begin{enumerate}[(i)]
					\item \text{\rm local ${\bar G}^*$-concavity of} $\pi$: i.e. $\pi_{\bar{y}\bar{y}}(\bar{y}) + G_{\bar{y}\bar{y}}(x, \bar{y})$ is non-positive definite whenever
					$(x, \bar{y}) \in cl(J)\times Y \times Z$ satisfies $\pi_{\bar{y}}(\bar{y})+ G_{\bar{y}}(x, \bar{y})=0$;
					\item $\pmb \Pi$ is concave on $\mathcal{U}$ for all $\mu\ll \mathcal{L}^m$.
				\end{enumerate}
			\end{namedproposition}
			
		\end{itemize}
	}
	
	\frame{	
		\frametitle{A shaper result}
		\begin{itemize}
			\item<1->[]
			\begin{corollary}
				Suppose $G(x,y,z) = b(x,y)- z \in C^3(cl(X\times Y \times Z))$ satisfies $\bf (B0-B3)$ and 
				$\pi(x,y,z) = z -a(y) \in C^2( cl( Y \times Z))$, then $a(y)$ is $b^*$-convex if  and only if $\pmb \Pi$ is concave on $\mathcal U$ and for every $y \in Y$, there exists $x \in cl(X)$ such that $Da(y) = D_y b(x,y)$.
			\end{corollary}
			
		\end{itemize}
	}
	


\end{document}
