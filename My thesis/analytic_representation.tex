\chapter{Analytic representation of condition \Gthree}\label{chapter:analytic_representation}



\section{A fourth-order differential re-expression of condition \Gthree}\label{section:4thorder}


%Lastly, We introduce a theorem expressing our crucial hypothesis \Gthree \  in a fourth-order differential form, which is %the main result in Appendix \ref{A:4thorder}.

In our convexity argument, hypothesis \Gthree\ plays a crucial role. In this chapter, we 
localize this hypothesis using differential calculus. Inspired by and strongly connected with Trudinger's theory of generalized prescribed Jacobian equations, this form is analogous to the non-negative cross-curvature condition (B3) 
of Figalli-Kim-McCann \cite{FigalliKimMcCann11}, a fourth order condition in the spirit of the Ma-Trudinger-Wang \cite{MaTrudingerWang05}. For another formulation, see e.g.\ \cite{GuillenKitagawa15}.
\medskip

%The resulting expression
%shows it to be a direct analog of the non-negative cross-curvature (B3) from \cite{FigalliKimMcCann11},
%which in turn was inspired by \cite{MaTrudingerWang05}. 

Apart from the assumptions of Section \ref{section:Hypotheses}, we shall need the non-degeneracy
condition assumed in Section \ref{section:concavity}:
\begin{itemize}
	\item[\Gsix] $G\in C^2(cl(X\times Y \times Z)
	)$, and $D_{\bar{x},\bar{y}}(\bar{G})(x,-1,y,z)$ has full rank, for each $(x,y,z)\in cl(X\times Y\times Z)$. 
\end{itemize}
For this and the next chapter only, we assume the dimensions of spaces $X$ and $Y$ are equal, i.e. $m=n$,
so that the matrix mentioned in \Gsix\ is square.
We shall also need to extend the twist and convex range hypotheses \Gone\ and \Gtwo\  
to the function $H$ in place of $G$. This is equivalent to assuming:

\begin{itemize}
	\item[\Gseven] For each $(y, z)\in cl( Y \times Z)$ the map $x \in X \longmapsto \frac{G_y}{G_z}(\cdot, y,z)$ is one-to-one;
	\item[\Geight] its range $X_{(y,z)} := \frac{G_y}{G_z}(X,y,z) \subset \R^n$ is convex.
\end{itemize}

We can now state:
\medskip

\begin{theorem}\label{theorem:4thorder}
	Assume \Gzero-\Gtwo\ and \Gfour-\Geight. If, in addition, $G\in C^4(cl(X\times Y \times Z)
	)$,  then the following statements are equivalent:
	\begin{enumerate}[(i)]
		\item \Gthree.
		
		\item 	 For any given  $x_0, x_1\in X$, any curve $(y_t, z_t) \in cl( Y \times Z)$ connecting $(y_0,z_0)$ and $(y_1, z_1)$, we have 
		\begin{equation*}
		\frac{\partial^2}{\partial s^2 }\Biggl(\frac1{G_z(x_s, y_t, z_t)}\frac{\partial^2}{\partial t^2} G(x_s,y_t,z_t) \Biggr)\Bigg|_{t=t_0}\le 0,
		\end{equation*}
		whenever  { $s\in [0,1] \longmapsto \frac{G_y}{G_z}(x_s, y_{t_0}, z_{t_0})$} forms an affinely parametrized line segment for some $t_0 \in [0,1]$.
		
		
		\item For any given curve $x_s\in X$ connecting $x_0$ and $x_1$,  any $(y_0, z_0),  (y_1, z_1) \in cl( Y \times Z) $, we have 
		\begin{equation*}
		\frac{\partial^2}{\partial s^2 }\Biggl(\frac{1}{G_z(x_s, y_t, z_t)}\frac{\partial^2}{\partial t^2} G(x_s,y_t,z_t) \Biggr)\Bigg|_{s=s_0}\le 0,
		\end{equation*}
		whenever $t\in [0,1] \longmapsto (G_x, G)(x_{s_0}, y_t, z_t)$  forms an affinely parametrized line segment for some $s_0\in [0,1]$.
		
	\end{enumerate} 
\end{theorem}

The proof of this theorem and its embellishments are represented in the following section:



\section{Proof and variations on Theorem \ref{theorem:4thorder}}\label{A:4thorder}




\begin{proof}[Proof of Theorem \ref{theorem:4thorder}]
	$\mathrm{(i)}\Rightarrow \mathrm{(ii)}.$ Suppose for some $t_0 \in [0,1]$, $s \in [0,1] \longmapsto \frac{G_y}{G_z}(x_s, y_{t_0}, z_{t_0})$ forms an affinely parametrized line segment.
	
	For any fixed $s_0\in [0,1]$, consider $x_{s_0}\in X$, there is a $G$-segment $(x_{s_0}, y_t^{s_0},$ $ z_t^{s_0})$ passing through $(x_{s_0}, y_{t_0}, z_{t_0})$ at $t=t_0$ with the same tangent vector as $(x_{s_0}, y_t, z_t)$ at $t=t_0$, i.e., there exists
	another curve $(y_t^{s_0}, z_t^{s_0}) \in cl( Y \times Z)$, such that $(y_t^{s_0},z_t^{s_0})\mid _{t=t_0} = (y_t,z_t)\mid _{t=t_0}$,  $(\dot{y_t}^{s_0},\dot{z_t}^{s_0})\mid _{t=t_0} \parallel (\dot{y}_t,\dot{z}_t)\mid _{t=t_0}$, and $(G_x, G)(x_{s_0},y_t^{s_0},z_t^{s_0}) = (1-t)(G_x, G)(x_{s_0},y_0^{s_0},z_0^{s_0})+t (G_x, G)(x_{s_0},y_1^{s_0},z_1^{s_0})$. 
	
	Computing the fourth mixed derivative yields
	\begin{flalign*}
	&\frac{\partial^2}{\partial s^2 }\Biggl(\frac{1}{G_z(x_s, y_t, z_t)}\frac{\partial^2}{\partial t^2} G(x_s,y_t,z_t) \Biggr)\\
	=~&\frac{\partial^2}{\partial s^2}\Biggl(\frac{1}{G_z}\Biggr) \frac{\partial^2}{\partial t^2} G + 2 \frac{\partial}{\partial s} \Biggl(\frac{1}{G_z}\Biggr) \frac{\partial^3}{\partial s \partial t^2} G + \frac{1}{G_z} \frac{\partial^4}{\partial s^2 \partial t^2} G  \\
	=~& [-(G_z)^{-2}G_{i,z} \ddot{x_s}^{i} - (G_z)^{-2} G_{ij,z}\dot{x_s}^i \dot{x_s}^j+ 2(G_z)^{-3} G_{i,z} G_{j,z} \dot{x_s}^i \dot{x_s}^j]\cdot [G_{,k}\ddot{y_t}^{k}+G_z\ddot{z}_t+G_{,kl}\dot{y_t}^k \dot{y_t}^l\\
	&\hspace{0.5cm} + 2G_{,kz} \dot{y_t}^k \dot{z}_t +G_{zz} (\dot{z}_t)^2]\\
	&+ 2[-(G_z)^{-2}G_{i,z}\dot{x_s}^i]\cdot[G_{j,k}\dot{x_s}^j\ddot{y_t}^{k}+G_{j,z}\dot{x_s}^j\ddot{z}_t+G_{j,kl}\dot{x_s}^j\dot{y_t}^k \dot{y_t}^l + 2G_{j,kz}\dot{x_s}^j \dot{y_t}^k \dot{z}_t +G_{j,zz} \dot{x_s}^j(\dot{z}_t)^2]\\
	&+(G_z)^{-1}[G_{i,k}\ddot{x_s}^i\ddot{y_t}^{k} +G_{ij,k}\dot{x_s}^i\dot{x_s}^j\ddot{y_t}^{k} +G_{i,z}\ddot{x_s}^i\ddot{z}_t +G_{ij,z}\dot{x_s}^i\dot{x_s}^j\ddot{z}_t  +G_{i,kl}\ddot{x_s}^i\dot{y_t}^k \dot{y_t}^l +G_{ij,kl}\dot{x_s}^i\dot{x_s}^j\dot{y_t}^k \dot{y_t}^l \\
	&\hspace{0.5cm} +2G_{i,kz}\ddot{x_s}^i\dot{y_t}^k\dot{z}_t 
	 +2G_{ij,kz}\dot{x_s}^i\dot{x_s}^j\dot{y_t}^k\dot{z}_t +G_{i,zz}\ddot{x_s}^i(\dot{z}_t)^2 +G_{ij,zz}\dot{x_s}^i\dot{x_s}^j(\dot{z}_t)^2]\\ 
	=~& [((G_z)^{-1}G_{i,k}-(G_z)^{-2}G_{i,z}G_{,k})\ddot{x_s}^i +((G_z)^{-1}G_{ij,k}- (G_z)^{-2} G_{,k}G_{ij,z} -2(G_z)^{-2}G_{i,z}G_{j,k} \\
		& \hspace{0.5cm}   +2(G_z)^{-3}G_{,k}G_{i,z} G_{j,z}) \dot{x_s}^i \dot{x_s}^j]\ddot{y_t}^k \\
		&+ [(G_z)^{-1}G_{i,kl}-(G_z)^{-2}G_{i,z}G_{,kl}]\ddot{x_s}^i\dot{y_t}^k \dot{y_t}^l \\
		&+ [(G_z)^{-1}G_{ij,kl}-(G_z)^{-2} G_{ij,z}G_{,kl}+ 2(G_z)^{-3} G_{i,z} G_{j,z}G_{,kl} -2(G_z)^{-2}G_{i,z}G_{j,kl}]\dot{x_s}^i \dot{x_s}^j\dot{y_t}^k \dot{y_t}^l\\
		&+[2(G_z)^{-1}G_{i,kz}-2(G_z)^{-2}G_{i,z}G_{,kz}] \ddot{x_s}^{i}\dot{y_t}^k\dot{z}_t\\
		&+[2(G_z)^{-1}G_{ij,kz}-2(G_z)^{-2} G_{ij,z}G_{,kz}+ 4(G_z)^{-3}G_{i,z}G_{j,z}G_{,kz}-4(G_z)^{-2}G_{i,z}G_{j,kz}]\dot{x_s}^i\dot{x_s}^j\dot{y_t}^k\dot{z}_t\\
			&+[(G_z)^{-1}G_{i,zz}-(G_z)^{-2}G_{i,z}G_{zz}] \ddot{x_s}^{i}(\dot{z}_t)^2 \\
			&+[(G_z)^{-1}G_{ij,zz} -(G_z)^{-2} G_{ij,z}G_{zz} +2(G_z)^{-3} G_{i,z} G_{j,z}G_{zz} -2(G_z)^{-2}G_{i,z}G_{j,zz}]\dot{x_s}^i \dot{x_s}^j(\dot{z}_t)^2.
	\end{flalign*}

	
	The coefficient of $\ddot{y_t}^k$ vanishes when this expression is evaluated at $t =  t_0$, due to the assumption that $s \in [0,1] \longmapsto \frac{G_y}{G_z}(x_s, y_{t_0}, z_{t_0})$ forms an affinely parametrized line segment, which implies
	\begin{flalign*}
	0=~& \frac{\partial^2}{\partial s^2} \frac{G_y}{G_z}(x_s, y_{t_0}, z_{t_0})\\
	=~&[((G_z)^{-1}G_{i,y}-(G_z)^{-2}G_{i,z}G_{,y})\ddot{x}^i +((G_z)^{-1}G_{ij,y}- (G_z)^{-2} G_{,y}G_{ij,z} \\
	&\hspace{0.5cm}-2(G_z)^{-2}G_{i,z}G_{j,y}  +2(G_z)^{-3}G_{,y}G_{i,z} G_{j,z}) \dot{x}^i \dot{x}^j]
	\end{flalign*}
	for all  $s\in [0,1]$.
	
	Since $(\dot{y_t}^{s_0},\dot{z_t}^{s_0})|_{t=t_0} \parallel (\dot{y}_t,\dot{z}_t)|_{t=t_0}$, there exists some constant $C_1>0$, such that $(\dot{y}_t,\dot{z}_t)|_{t=t_0} =C_1(\dot{y_t}^{s_0},\dot{z_t}^{s_0})|_{t=t_0} $. Moreover, since  $ (y_t,z_t)\mid _{t=t_0}=(y_t^{s_0},z_t^{s_0})\mid _{t=t_0} $, we have
	\begin{flalign*}
	&\frac{\partial^2}{\partial s^2 }\Biggl(\frac{1}{G_z(x_s, y_t, z_t)}\frac{\partial^2}{\partial t^2} G(x_s,y_t,z_t) \Biggr) \Bigg|_{t=t_0} \\
	=~& C_1^2	\frac{\partial^2}{\partial s^2 }\Biggl(\frac{1}{G_z(x_s, y_t^{s_0}, z_t^{s_0})}\frac{\partial^2}{\partial t^2} G(x_s,y_t^{s_0},z_t^{s_0}) \Biggr) \Bigg|_{t=t_0} .
	\end{flalign*}
	Denote $g(s):=\frac{\partial^2}{\partial t^2}|_{t=t_0} G(x_s,y_t^{s_0},z_t^{s_0})$, for $s\in[0,1]$. Since $(x_{s_0},y_t^{s_0},z_t^{s_0})$ is a $G$-segment, by \Gthree, we have $g(s) \ge 0$, for all $s \in [0,1]$.
	By definition of $(y_t^{s_0},z_t^{s_0})$, it is clear that $g(s_0) =0$.  The first- and second-order
	conditions for an interior minimum then give $g'(s_0) =0 \le g''(s_0)$; (in fact $g'(s_0)=0$ also follows
	directly from the definition of a $G$-segment).
	
	By the assumption \Gfour,  we have $G_z <0$, thus, 
	\begin{flalign*}
	~&\frac{\partial^2}{\partial s^2 }\Biggl(\frac{1}{G_z(x_s, y_t, z_t)}\frac{\partial^2}{\partial t^2} G(x_s,y_t,z_t) \Biggr) \Bigg|_{(s,t)=(s_0,t_0)}\\
	=~& C_1^2 	\frac{\partial^2}{\partial s^2 }\Biggl(\frac{1}{G_z(x_s, y_t^{s_0}, z_t^{s_0})}\frac{\partial^2}{\partial t^2} G(x_s,y_t^{s_0},z_t^{s_0}) \Biggr) \Bigg|_{(s,t)=(s_0,t_0)} \\
	=~&C_1^2\frac{\partial^2}{\partial s^2}\Bigg|_{(s,t)=(s_0,t_0)} \Biggl(\frac{1}{G_z}(x_s,y_t^{s_0},z_t^{s_0})\Biggr) g(s_0)\\
	~&+ 2C_1^2 \frac{\partial}{\partial s}\Bigg|_{(s,t)=(s_0,t_0)} \Biggl(\frac{1}{G_z(x_s,y_t^{s_0},z_t^{s_0})}\Biggr) g'(s_0) + \frac{C_1^2}{G_z(x_s,y_t^{s_0},z_t^{s_0})} g''(s_0)  \\
	\le ~& 0.
	\end{flalign*}
	
	$\mathrm{(ii)}\Rightarrow \mathrm{(i)}.$ For any fixed $x_0 \in X$ and $G$-segment $(x_0, y_t, z_t)$,  we need to show $\frac{\partial^2}{\partial t^2}G(x_1, y_t, z_t) \ge 0$,  for all $t \in [0,1]$ and $x_1 \in X$.
	
	For any fixed $t_0 \in [0,1]$ and $x_1 \in X$, define $x_s$ as the solution $\hat{x}$ to the equation
	\begin{equation}
	\frac{G_y}{G_z}(\hat{x}, y_{t_0},z_{t_0}) = (1-s) \frac{G_y}{G_z}(x_0, y_{t_0},z_{t_0}) +s \frac{G_y}{G_z}(x_1, y_{t_0},z_{t_0}).
	\end{equation}
	
	By \Gseven\ and \Geight, $x_s$ is uniquely determined for each $s\in (0,1)$. In addition, $x_0$ and $x_1$ satisfy the above equation for $s =0$ and $s=1$, respectively.  
	
	Define $g(s):=
	\frac{1}{G_z(x_s,y_t,z_t)}\frac{\partial^2}{\partial t^2}G(x_s, y_t, z_t)
	\Big|_{t=t_0}$ for $s\in [0,1]$.
	
	Then $g(0) =0 = g'(0)$ from the two conditions defining a $G$-segment. 
	
	In our setting,  $s\in [0,1] \longmapsto \frac{G_y}{G_z}(x_s, y_{t_0}, z_{t_0})$ forms an affinely parametrized line segment, thus
	$0\ge \frac{\partial^2}{\partial s^2 }\Bigl(\frac{1}{G_z(x_s, y_t, z_t)}\frac{\partial^2}{\partial t^2} G(x_s,y_t,z_t) \Bigr)\Big|_{t=t_0} = g''(s)$ for all $ s \in [0,1]$ by hypothesis $\mathrm{(ii)}$.
	
	Hence $g$ is concave in $[0,1]$, and $g(0)=0$ is a critical point, thus $g(1)\le 0$. Since $G_z<0$ this implies $\frac{\partial^2}{\partial t^2}\Big|_{t=t_0}G(x_1, y_t, z_t) \ge 0$ for any $t_0 \in [0,1]$ and $x_1 \in X$,
	as desired.
	\medskip
	
	$\mathrm{(i)}\Rightarrow \mathrm{(iii)}.$
	For any fixed $s_0\in [0,1]$, suppose $t\in [0,1] \longmapsto (G_x, G)(x_{s_0}, y_t,$ $ z_t)$  forms an affinely parametrized line segment. For any fixed $t_0 \in [0,1]$, define $g(s):=\Big(\frac{1}{G_z(x_s,y_t,z_t)}\frac{\partial^2}{\partial t^2}G(x_s, y_t, z_t)\Big)\Big|_{t=t_0}$, for all $s \in [0,1]$. By \Gthree-\Gfour, we know $g(s)\le 0$, for all $s\in [0,1]$. By the definition of $(y_t, z_t)$, we have $g(s_0)=g'(s_0) =0$. Thus $g''(s_0)\le 0$.
	\medskip
	
	$\mathrm{(iii)}\Rightarrow \mathrm{(i)}.$ For any fixed $x_0 \in X$, suppose $(x_0, y_t^{0}, z_t^{0})$ is a $G$-segment, then we need to show $\frac{\partial^2}{\partial t^2}G(x_1, y_t^{0}, z_t^{0}) \ge 0$,  for all $t \in [0,1], x_1 \in X$.
	
	For any fixed $t_0 \in [0,1]$, $x_1 \in X$, define $x_s$ as the solution $\hat{x}$ of equation
	\begin{equation*}
	\frac{G_y}{G_z}(\hat{x}, y_{t_0}^{0},z_{t_0}^{0}) = (1-s) \frac{G_y}{G_z}(x_0, y_{t_0}^{0},z_{t_0}^{0}) +s \frac{G_y}{G_z}(x_1, y_{t_0}^{0},z_{t_0}^{0}).
	\end{equation*}
	
	By \Gseven\ and \Geight, $x_s$ is uniquely determined for each $s\in (0,1)$. In addition, $x_0$ and $x_1$ satisfy the above equation for $s =0$ and $s=1$ respectively. 
	
	Define $g(s):=\frac{1}{G_z(x_s,y_t^{0},z_t^{0})}\frac{\partial^2}{\partial t^2}G(x_s, y_t^{0}, z_t^{0})\Big|_{t=t_0}$, for $s\in [0,1]$.
	
	Then $g(0) =g'(0)  =0 $ by the two conditions defining a $G$-segment. 
	
	
	For any fixed $s_0 \in [0,1]$, there is a $G$-segment $(x_{s_0}, y_t^{s_0}, z_t^{s_0})$ passing through $(x_{s_0}, y_{t_0}^{0}, z_{t_0}^{0})$ at $t=t_0$ with the same tangent vector as $(x_{s_0}, y_t^{0}, z_t^{0})$ at $t=t_0$, i.e., there exists 
	another curve $(y_t^{s_0}, z_t^{s_0}) \in cl( Y \times Z)$ and some constant $C_2>0$, such that $(y_t^{s_0},z_t^{s_0})\mid _{t=t_0} = (y_t^{0},z_t^{0})\mid _{t=t_0}$,  $(\dot{y_t}^{s_0},\dot{z_t}^{s_0})\mid _{t=t_0} = \frac{1}{C_2} (\dot{y}_t^{0},\dot{z}_t^{0})\mid _{t=t_0}  $, and $(G_x, G)(x_{s_0},y_t^{s_0},z_t^{s_0}) = (1-t)(G_x, G)(x_{s_0},y_0^{s_0},z_0^{s_0})+t (G_x, G)(x_{s_0},y_1^{s_0},z_1^{s_0})$. 
	
	Computing the mixed fourth derivative yields
	\begin{flalign*}
	&\frac{\partial^2}{\partial s^2 }\Biggl(\frac{1}{G_z(x_s, y_t^{0}, z_t^{0})}\frac{\partial^2}{\partial t^2} G(x_s,y_t^{0},z_t^{0}) \Biggr) \Bigg|_{(s,t)=(s_0, t_0)} \\
	=~& C_2^2	\frac{\partial^2}{\partial s^2 }\Biggl(\frac{1}{G_z(x_s, y_t^{s_0}, z_t^{s_0})}\frac{\partial^2}{\partial t^2} G(x_s,y_t^{s_0},z_t^{s_0}) \Biggr) \Bigg|_{(s,t)=(s_0, t_0)} ,
	\end{flalign*}
	where the equality is derived from the condition that $s \in [0,1] \longmapsto \frac{G_y}{G_z}(x_s, y_{t_0},$ $ z_{t_0})$ forms an affinely parametrized line segment,  $(y_t^{s_0},z_t^{s_0})\mid _{t=t_0} = (y_t,z_t)\mid _{t=t_0}$ and $ (\dot{y}_t^{0},\dot{z}_t^{0})\mid _{t=t_0} = C_2(\dot{y_t}^{s_0},\dot{z_t}^{s_0})\mid _{t=t_0} $. Moreover, the latter expression is non-positive by assumption $\mathrm{(iii)}$.
	
	Thus $g''(s_0)\le 0$ for all $s_0 \in [0,1]$. Since $g$ is concave in $[0,1]$, and $g(0)=0$ is a critical point, we have $g(1)\le 0$. Thus $G_z<0$ implies $\frac{\partial^2}{\partial t^2}\Big|_{t=t_0}G(x_1, y_t^{0}, z_t^{0}) \ge 0$ for all $t_0 \in [0,1]$ and $x_1 \in X$, as desired. 
\end{proof}


For strictly concavity of the profit functional, one might need a strict version of hypothesis \Gthree:
\begin{itemize}
\item[\Gthree$_{s}$] For each $x,x_0 \in X$ and $x\ne x_0$, assume $t \in [0,1] \longmapsto G(x, y_t, z_t)$ is strictly convex along all $G$-segments $(x_0, y_t, z_t)$ defined in (\ref{$G$-segment}).
\end{itemize}


\begin{remark}\label{(C5)_s and (C5)_u}
	Strict inequality in $\mathrm{(ii)}$ [or $\mathrm{(iii)}$] implies \Gthree$_{s}$ but the reverse is not necessarily true, i.e. \Gthree$_{s}$ is intermediate in strength between \Gthree\ and strict inequality version of $\mathrm{(ii)}$ [or $\mathrm{(iii)}$]. Besides, strict inequality versions of $\mathrm{(ii)}$ and $\mathrm{(iii)}$ are equivalent, and denoted by $\Gthree_{u}$.
	
	Note inequality \eqref{foc} below and its strict and uniform versions \Gthree$_{s}$ and \Gthree$_{u}$
	precisely generalize of the analogous hypotheses $(B3)$, $(B3)_{s}$ and $(B3)_{u}$ from the quasilinear case in
	\cite{FigalliKimMcCann11}.
\end{remark}
\begin{proof}
	We only show strict inequality of $\mathrm{(ii)}$ implies that of $\mathrm{(iii)}$ here since the other direction is similar.
	
	For any fixed $s_0\in [0,1]$, suppose $t\in [0,1] \longmapsto (G_x, G)(x_{s_0}, y_t, z_t)$  forms an affinely parametrized line segment. For any fixed $t_0 \in [0,1]$, define ${x}_s^{t_0}$ as a solution to the equation
	$\frac{G_y}{G_z}({x}_{s}^{t_0}, y_{t_0},z_{t_0})$ $= (1-s) \frac{G_y}{G_z}({x}_{0}^{t_0}, y_{t_0},z_{t_0}) +s \frac{G_y}{G_z}({x}_{1}^{t_0}, y_{t_0},z_{t_0})$, with initial conditions ${x}_s^{t_0} |_{s=s_0}= x_{s_0}$ and $\dot{x}_s^{t_0}|_{s=s_0} = C_1 $ $\cdot \dot{x}_s|_{s=s_0}$, for some constant $C_1 >0$.	
	Thus, by strict inequality of $\mathrm{(ii)}$, we have 
	\begin{flalign*}
	0>~&\frac{\partial^2}{\partial s^2 }\Big(\frac1{G_z(x_s^{t_0}, y_t, z_t)}\frac{\partial^2}{\partial t^2} G(x_s^{t_0},y_t,z_t) \Big)\Big|_{(s,t)=(s_0, t_0)}\hspace{1.38cm}\\
	=~& \frac{\partial^2}{\partial s^2 }\Big(\frac1{G_z(x_s^{t_0}, y_t, z_t)}\Big)\frac{\partial^2}{\partial t^2} G(x_s^{t_0},y_t,z_t) \Big|_{(s,t)=(s_0, t_0)}\\
	~&+ \frac{\partial}{\partial s }\Big(\frac1{G_z(x_s^{t_0}, y_t, z_t)}\Big)\frac{\partial^3}{\partial s \partial t^2} G(x_s^{t_0},y_t,z_t) \Big|_{(s,t)=(s_0, t_0)}\\
	~&+ \frac1{G_z(x_s^{t_0}, y_t, z_t)}\frac{\partial^4}{\partial s^2 \partial t^2} G(x_s^{t_0},y_t,z_t) \Big|_{(s,t)=(s_0, t_0)}\\
	=~& 
	-\frac{G_{x,z}(x_s^{t_0},y_t,z_t)}{G_z^2(x_s^{t_0}, y_t, z_t)}\frac{\partial^2}{\partial t^2} G_x(x_s^{t_0},y_t,z_t) (\dot{x}_s^{t_0})^2 \Big|_{(s,t)=(s_0, t_0)}\\
	~&+ \frac1{G_z(x_s^{t_0}, y_t, z_t)}\frac{\partial^2}{\partial t^2} G_{xx}(x_s^{t_0},y_t,z_t)(\dot{x}_s^{t_0})^2  \Big|_{(s,t)=(s_0, t_0)}\\
	=~&C_1^2\Big[-\frac{G_{x,z}(x_{s},y_t,z_t)}{G_z^2(x_{s}, y_t, z_t)}\frac{\partial^2}{\partial t^2} G_x(x_{s},y_t,z_t) (\dot{x}_s)^2 \Big|_{(s,t)=(s_0, t_0)}\\
	~&+ \frac1{G_z(x_{s}, y_t, z_t)}\frac{\partial^2}{\partial t^2} G_{xx}(x_{s},y_t,z_t)(\dot{x}_s)^2  \Big|_{(s,t)=(s_0, t_0)}\Big]\\
	=~&C_1^2\frac{\partial^2}{\partial s^2 }\Big(\frac1{G_z(x_s, y_t, z_t)}\frac{\partial^2}{\partial t^2} G(x_s,y_t,z_t) \Big)\Big|_{(s,t)=(s_0, t_0)}.
	\end{flalign*}
	Here we use the initial condition ${x}_s^{t_0} |_{s=s_0}= x_{s_0}$ and $\dot{x}_s^{t_0}|_{s=s_0} = C_1 \dot{x}_s|_{s=s_0}$.  Besides, since $(x_{s_0}, y_t, z_t)$ forms a $G$-segment, therefore we have 
	\begin{flalign*}
	\frac{\partial^2}{\partial t^2} G(x_s^{t_0},y_t,z_t) \Big|_{(s,t)=(s_0, t_0)}=\frac{\partial^2}{\partial t^2} G(x_s,y_t,z_t) \Big|_{(s,t)=(s_0, t_0)}=0,
	\end{flalign*}
	\begin{flalign*}
	\text{ and }\ \  \frac{\partial^2}{\partial t^2} G_x(x_s^{t_0},y_t,z_t) \Big|_{(s,t)=(s_0, t_0)}=\frac{\partial^2}{\partial t^2} G_x(x_s,y_t,z_t) \Big|_{(s,t)=(s_0, t_0)}=0.
	\end{flalign*}
	
	From the above inequality and $C_1 >0$, one has 
	\begin{equation*}
	\frac{\partial^2}{\partial s^2 }\Big(\frac1{G_z(x_s, y_t, z_t)}\frac{\partial^2}{\partial t^2} G(x_s,y_t,z_t) \Big)\Big|_{(s,t)=(s_0, t_0)}<0,
	\end{equation*} 
	whenever $\dot{x_s}|_{s=s_0}$ and $(\dot{y_t},\dot{z}_t)|_{t=t_0}$ are nonzero. Since this inequality holds for each fixed $t_0 \in [0,1]$, the strict version of  $\mathrm{(iii)}$ is proved.
\end{proof}

Combining $\mathrm{(ii)}$ and $\mathrm{(iii)}$, one can conclude they are also equivalent to the following statement $\mathrm{(iv)}$.


\begin{corollary}\label{FourthOrder4}
	Assuming \Gzero-\Gtwo, \Gfour-\Geight\ and $G\in C^4(cl(X\times Y \times Z)
	)$,  then \Gthree\ is equivalent to the following statement:\\
	\begin{enumerate}[(i)]
	\item[$\mathrm{(iv)}$] For any given curve $x_s\in X$ connecting $x_0$ and $x_1$,  and any curve $(y_t, z_t) \in cl( Y \times Z)$ connecting $(y_0,z_0)$ and $(y_1, z_1)$, we have 
	\begin{equation}\label{foc}
	\frac{\partial^2}{\partial s^2 }\Biggl(\frac{1}{G_z(x_s, y_t, z_t)}\frac{\partial^2}{\partial t^2} G(x_s,y_t,z_t) \Biggr)\Bigg|_{(s,t) = (s_0,t_0)}\le 0,
	\end{equation}
	whenever either of the two curves $t\in [0,1] \longmapsto (G_x, G)(x_{s_0}, y_t, z_t)$ and  $s\in [0,1] \longmapsto \frac{G_y}{G_z}(x_s, y_{t_0}, z_{t_0})$ forms an affinely parametrized line segment.
	\end{enumerate}
\end{corollary}	




