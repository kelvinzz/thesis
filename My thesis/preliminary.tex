\chapter{Preliminaries and $G$-convexity}\label{chapter:preliminaries}

\section{Preliminaries}


Let $X$ be a subset of $\R^m$. %and $S$ be any set of functions on $X$, i.e., $S\subset \{f: X \longrightarrow \R \}$.
\medskip
%%\begin{definition}[Convex Sets]
%%	A set $A \subset \R^m$ or $A \subset S$ is called convex if and only if for any $x , y \in A$, and any $t \in [0,1]$,  $ t x + (1-t)y \in A$.
%%\end{definition}

%\begin{lemma}
%	Let $A$ be a subset contained in $\R^m$ or $S$, the following statements are equivalent:
%	\begin{enumerate}
%		\item $A$ is convex;
%		\item For any $x, y \in A$, $\frac{1}{2}x + \frac{1}{2}y$ also belongs to $A$;
%		\item For any $r\in \N$, $r>1$, any element $x_1, ..., x_r \in A$, and for any nonnegative numbers $t_1, ..., t_r$ such that $t_1 + ... + t_r =1$, then $\sum_{k = 1}^{r} t_k x_k \in A$.
%	\end{enumerate}
%\end{lemma}

%%\begin{definition}[(Strictly) Convex Functions]
%%	Let $X$ be a convex set in $\R^m$ and let $f: X \longrightarrow \R$ be a function. Then $f$ is called (strictly) convex if for any $x_1, x_2 \in X$ and any $t\in (0,1)$, the following inequality holds.
%%	\begin{flalign}
%%		f(tx_1 +(1-t) x_2) (<) \le tf(x_1) +(1-t) f(x_2).
%%	\end{flalign} 
%%\end{definition}

%%A function $f: X \longrightarrow \R$ is said to be (strictly) concave is $-f$ is (strictly) convex.
%%\medskip

\begin{definition}[Subdifferential]
	Recall that the subdifferential of a function $u: X \longrightarrow \R$ at $x_0 \in X$ is defined as the set:
	\begin{equation}\label{defn:subdifferential}
	\partial u(x_0) := \left\{ y \in \R^m~|~ u(x) - u(x_0) \ge \langle  x- x_0,  y \rangle, \text{ for all } x \in X  \right\}.
	\end{equation}
	Here  $ \langle\ ,\ \rangle$ denotes the Euclidean inner product. 
\end{definition}

\begin{lemma}\label{lemma:subdiff}
	The set defined in \eqref{defn:subdifferential} is nonempty for every $x_0 \in X$ if and only if $u$ is convex.
\end{lemma}

We give a proof below for a generalized version of this lemma.\medskip

For any vectors $p, w \in \R^n$, we denote $p\parallel w$ if $p$ and $w$ are parallel.\medskip

We use $\mathcal{L}^m$ to denote Lebesgue measure on $\R^m$, which characterizes the $m$-dimensional volume. A non-negative measure $\mu$ is said to be absolutely continuous with respect to  $\mathcal{L}^m$ if for every measurable set $A$, $\mathcal{L}^m(A) =0$ implies $\mu(A) =0$. This is written as $\mu \ll \mathcal{L}^m$. \medskip

Let $f$ be a function on $X$. We say $f \in L^1(X)$ if $\int_X |f(x)| d\mathcal{L}^m(x) < \infty$. Denote by $W^{1,1}(X)$ the Sobolev space of $L^1(X)$ functions whose first derivatives exist in the weak sense and belong to $L^1(X)$. For more properties of Sobolev spaces and weak derivatives, see Evans~\cite[Chapter 5]{Evans98}. If $\omega$ is a subset of $X$, the notation $\omega \subset \subset X$ means that the closure of $\omega$ is also included in $X$.\medskip

Here we use $G_x=\big(\frac{\partial G}{\partial x^1}, \frac{\partial G}{\partial x^2}, ..., \frac{\partial G}{\partial x^m}\big)$, $G_y=\big(\frac{\partial G}{\partial y^1}, \frac{\partial G}{\partial y^2}, ..., \frac{\partial G}{\partial y^n}\big)$, $G_z = \frac{\partial G}{\partial z}$ to denote derivatives with respect to $x\in X\subset \R^m$ , $y \in Y \subset \R^n$, and $z\in \R$, respectively. Also, for second partial derivatives, we adopt the following notation
\begin{flalign*}
G_{x,y}=
\begin{bmatrix}
\frac{\partial^2 G}{\partial x^1 \partial y^1} & \frac{\partial^2 G}{\partial x^1 \partial y^2} & ... & \frac{\partial^2 G}{\partial x^1 \partial y^n}  \\
\frac{\partial^2 G}{\partial x^2 \partial y^1} & \frac{\partial^2 G}{\partial x^2 \partial y^2} & ... & \frac{\partial^2 G}{\partial x^2 \partial y^n}  \\	
\vdots & \vdots & \ddots & \vdots \\
\frac{\partial^2 G}{\partial x^m \partial y^1} & \frac{\partial^2 G}{\partial x^m \partial y^2} & ... & \frac{\partial^2 G}{\partial x^m \partial y^n}  
\end{bmatrix}
\end{flalign*}
and $G_{x,z}=\big(\frac{\partial^2 G}{\partial x^1 \partial z}, \frac{\partial^2 G}{\partial x^2 \partial z}, ..., \frac{\partial^2 G}{\partial x^m \partial z}\big)$. 
\medskip

A function is said to be $C^0$ if it is continuous on its domain. We say $G \in C^{1}(cl(X\times Y \times Z))$, if all the partial derivatives $\frac{\partial G}{\partial x^1}$, ...,$\frac{\partial G}{\partial x^m}$, $\frac{\partial G}{\partial y^1}$, ..., $\frac{\partial G}{\partial y^n}$, $\frac{\partial G}{\partial z}$ exist and are continuous. Also, we say $G \in C^{2}(cl(X\times Y \times Z))$, if all the partial derivatives up to second order (i.e. $\frac{\partial^2 G}{\partial \alpha \partial \beta}$, where $\alpha, \beta = x^1, ... , x^m, y^1, ..., y^n, z$) exist and are continuous. Any bijective continuous function whose inverse is also continuous, is called a homeomorphism (a.k.a.\ bicontinuous).
\medskip

We will use Einstein notation for simplifying expressions including summations of vectors, matrices, and general tensors for higher order derivatives. There are essentially three rules of Einstein summation notation, namely: 1. repeated indices are implicitly summed over; 2. each index can appear at most twice in any term; 3.~both sides of an equation must contain the same non-repeated indices. For example, $a_{ij}v_i =\sum_{i}a_{ij}v_i$, $a_{ij}b^{kj}v_k=\sum_{j}\sum_{k}a_{ij}b^{kj}v_k$. We also use a comma to separate subscripts: the subscripts before the comma represent derivatives with respect to first variable and those after the comma represent derivatives with respect to the second variable. For instance, for $b=b(x,y)$, $b_{,kl}$ represents second derivatives with respect of $y$ only. And for $G=G(x,y,z)$, where $z\in \R$, $G_{i,jz}$ denotes third order derivatives with respect of $x$, $y$ and $z$, instead of using another comma to separate subscripts corresponding to $y$ and $z$.\medskip


\section{$G$-convex, $G$-subdifferentiability}\label{section:G-convexity}
In this section, we introduce some tools from convex analysis and the notion of $G$-convexity  (c.f. \cite{Trudinger14,Balder77,Singer97}), which is a generalization of ordinary convexity. \medskip

Let $X$, $Y$, and $Z$ be subsets of $\R^m$, $\R^n$, and $\R$ respectively. Assume $G: X \times Y \times Z \longrightarrow \R$ is any function which is strictly decreasing in its last variable. For each $(x, y) \in X\times cl(Y)$ and $u\in G(x,y, cl(Z))$, define $H(x,y,u) := z$ whenever $G(x,y,z) = u$, i.e., $H(x, y, \cdot)= G^{-1}(x,y,\cdot)$. In the context of nonlinear pricing, $G(x,y,z)$ represents the utility that consumer $x$ obtains for purchasing product $y$ at price $z$, while $H(x,y,u)$ denotes the price paid by agent $x$ for product $y$ when receiving value $u$. \medskip


From Lemma \ref{lemma:subdiff}, for any convex function $u$ on $X$ and any fixed point $x_0 \in X$, there exists $y_0 \in \partial u(x_0)$, satisfying% {$u(x) - u(x_0) \ge \langle x - x_0, y_0\rangle$}, i.e., 
\begin{equation}\label{convex_function}
	u(x) \ge  \langle x , y_0\rangle -( \langle x_0, y_0\rangle -  u(x_0)),	\text{  for all $x \in X$},
\end{equation} 
where equality holds at $x = x_0$. On the other hand, if for any $x_0\in X$, there exists $y_0$, such that \eqref{convex_function} holds for all $x\in X$, then $u$ is convex. The following definition is analogous to this property, which is a special case of $G$-convexity, when $G(x,y,z) = \langle x, y \rangle -z$. In this case, we have $H(x,y,u) = \langle x, y \rangle -u$. \medskip
	
\begin{definition}[$G$-convexity]\label{defn:GConvexity}
 A function $u\in C^0(X)$ is called {\it $G$-convex} if for each $x_0 \in X$, there exist $y_0 \in   cl(Y)$, and $z_0 \in  cl(Z)$ such that $u(x_0)=G(x_0, y_0, z_0)$, and $u(x)\ge G(x, y_0, z_0)$, for all $x\in X$.
\end{definition}
		
		%%\begin{comment}
		%%From the definition, we know if $u$ is a $G$-convex function, for any $x\in X$, there exist $y\in cl(Y)$ and $z \in cl(Z)$, such that
		%%\begin{equation}\label{2}
		%%u(x)= G(x, y, z),\ \ \   Du(x) = D_x G(x, y, z)
		%%\end{equation}
		%%
		%%Given $u(x), Du(x)$, one can solve $y, z$, according to Assumption 1. \medskip
		%%\end{comment}
		
Similarly, one can also generalize the definition of subdifferential from \eqref{convex_function}.\medskip
		
\begin{definition}[$G$-subdifferentiability]\label{defn:GSubdifferential}
	The $G$-subdifferential of a 
			%$G$-convex % 
	function $u: X \longrightarrow \R$ is a point-to-set mapping defined by
	\begin{equation*}
			\partial^G u(x):= \{ y\in  cl(Y)| u(x')\ge G(x',y, H(x,y,u(x))), \text{ for all } x'\in X\}.
	\end{equation*}
	A function $u$ is said to be {\it $G$-subdifferentiable} at $x$ if and only if $\partial^G u(x) \neq \emptyset$.\footnote{In Trudinger \cite{Trudinger14}, this point-to-set mapping $\partial^G u$ is also called $G$-$normal$ mapping; see this paper for more properties related to $G$-convexity.}		
\end{definition}
			
			
			
			
In particular, if $G(x,y,z) = \langle x, y \rangle - z$, then the $G$-subdifferential coincides with the subdifferential. There are other generalizations of convexity and subdifferentiability. For instance, $h$-convexity in Carlier \cite{Carlier01}, or equivalently, $b$-convexity in Figalli-Kim-McCann \cite{FigalliKimMcCann11}, or $c$-convexity in  Gangbo-McCann \cite{GangboMcCann96}, is a special form of $G$-convexity, where $G(x,y,z)=  h(x,y) -z$, which plays an important role in the quasilinear case. For more references of convexity generalizations, see Kutateladze-Rubinov \cite{KutateladzeRubinov72}, Elster-Nehse \cite{ElsterNehse74}, Balder \cite{Balder77}, Dolecki-Kurcyusz \cite{DoleckiKurcyusz78},  Singer \cite{Singer97},  Rubinov \cite{Rubinov00a}, and Martínez-Legaz \cite{MartinezLegaz05}.\medskip
			
			
As mentioned above, a well-known result in convex analysis is that a function is convex if and only if it is subdifferentiable everywhere. The following lemma adapts this to $G$-convexity. \medskip
			
\begin{lemma}\label{convex-subdiff0}
A function $u: X \rightarrow \R$ is $G$-convex if and only if it is $G$-subdifferentiable everywhere.
\end{lemma}
\begin{proof}[Proof]
		Assume $u$ is $G$-convex, we want to show that $u$ is $G$-subdifferentiable everywhere, i.e., we need to prove  $\partial^G u(x_0)\neq \emptyset$ for all $x_0\in X$.
		
		Since $u$ is $G$-convex, by definition, for each $x_0$, there exists $ y_0, z_0$, such that $u(x_0) = G(x_0,y_0,z_0)$, and for all  $ x \in X$, 
		\begin{equation*}
		u(x)\ge G(x, y_0, z_0) = G(x, y_0, H(x_0,y_0,u(x_0))).
		\end{equation*}
		By the definition of $G$-subdifferentiability, $y_0 \in \partial^G u(x_0)$, i.e. $\partial^G u(x_0) \neq \emptyset$.
		
		On the other hand, assume $u$ is $G$-subdifferentiable everywhere, then for each $ x_0 \in X$,  there exists $ y_0 \in \partial^G u(x_0)$. Set $z_0:=H(x_0,y_0,u(x_0))$ so that $u(x_0) = G(x_0, y_0, z_0)$.
		
		Since $y_0\in \partial^G u(x_0)$, for all $x\in X$, we have 
		\begin{equation*}
		u(x)\ge G(x,y_0,H(x_0,y_0,u(x_0))) = G(x,y_0,z_0).
		\end{equation*}
		By definition, $u$ is $G$-convex.
\end{proof}
%\begin{proof}%[Proof of Lemma \ref{convex-subdiff0}]
%	$"\Rightarrow"$. Assume $u$ is $G$-convex, we want to show $u$ is $G$-subdifferentiable everywhere, i.e., the statement $\partial^G u(x_0)\neq \emptyset$ is true for any $ x_0\in X$.
%	
%	Since $u$ is $G$-convex, by definition, for any fixed $x_0 \in X$, there exist $y_0$ and $z_0$, such that $u(x_0) = G(x_0,y_0,z_0)$, and for any $x \in X$, $u(x)\ge G(x, y_0, z_0)$. By definition of function $H$, we know $z_0 = H(x_0,y_0,u(x_0))$. Therefore, combining the above inequality, we have $u(x) \ge G(x, y_0, H(x_0,y_0,u(x_0)))$, for any $x \in X$. By definition of $G$-subdifferentiability, it implies $y_0 \in \partial^G u(x_0)$, that is, $\partial^G u(x_0) \neq \emptyset$.
%	
%	
%	$"\Leftarrow"$. Assume $u$ is $G$-subdifferentiable everywhere. Then for any fixed $x_0 \in X$, there exists $y_0 \in \partial^G u(x_0)$. Define $z_0:=H(x_0,y_0,u(x_0))$, then by definition of function $H$, one has $u(x_0) = G(x_0, y_0, z_0)$.
%	%\sloppy
%	
%	Since $y_0\in \partial^G u(x_0)$, by definitions of $G$-subdifferentiability, we have $u(x)\ge G(x,y_0,H(x_0,y_0,u(x_0)))$\ $ = G(x,y_0,z_0)$, for any $x\in X$, where the last equality comes from the definition of $z_0$.
%	
%	By definition (of $G$-convexity), $u$ is $G$-convex.
%\end{proof}				
				
				
Using Lemma \ref{convex-subdiff0}, one can show the following result,  %Proposition \ref{incen/convex} 
which connects incentive compatibility in the economic context with $G$-convexity and $G$-subdifferentiability in mathematical analysis, generalizing the results of Rochet \cite{Rochet87} and Carlier \cite{Carlier01}.
\medskip




\begin{proposition}[$G$-convex utilities characterize incentive compatibility]\label{incen/convex}
	Let $(y,z)$ be a pair of mappings from $X$ to $cl(Y) \times cl(Z)$.  This (product, price) pair is incentive compatible 
	if and only if $u(\cdot):=G(\cdot,y(\cdot),z(\cdot))$ is $G$-convex and $y(x)\in \partial^G u(x)$ for each $x \in X$.
\end{proposition}
					
\begin{proof}%[Proof of Proposition \ref{incen/convex}]
	$``\Rightarrow"$. Suppose $(y,z)$ is incentive compatible. For any fixed $x_0 \in X$, let $y_0 = y(x_0)$ and $z_0 = z(x_0)$. Then $u(x_0) = G(x_0, y(x_0), z(x_0)) = G(x_0, y_0, z_0)$. By incentive compatibility of the contract $(y,z)$, for any $x\in X$, one has $G(x, y(x), z(x)) \ge G(x, y(x_0), z(x_0))$. This implies $u(x) \ge G(x,y_0,z_0)$, for any $x\in X$, since $u(x)= G(x, y(x), z(x))$,  $y_0 = y(x_0)$ and $z_0 = z(x_0)$. By definition, $u$ is $G$-convex. 
	
	Since $u(x_0)=G(x_0, y_0, z_0)$, by definition of function $H$, one has $z_0 = H(x_0, y_0, u(x_0))$.  Combining with $u(x) \ge G(x, y_0, z_0)$,  for any $x\in X$, which is concluded from above, we have $u(x)\ge G(x, y_0, H(x_0, y_0, u(x_0)))$, for any $x\in X$. By definition of  $G$-subdifferentiability, one has $y_0 \in \partial^G u(x_0)$, and thus $y(x_0) = y_0 \in \partial^G u(x_0)$.
	
	$``\Leftarrow"$. Assume that $u = G(x, y(x),z(x))$ is $G$-convex, and $y(x)\in \partial^G u(x)$, for any $x\in X$. For any fixed $x \in X$, since $y(x)\in \partial^G u(x)$, for any $x'\in X$, one has 
	\begin{equation}\label{eqn_prop3.4}
	u(x')\ge G(x', y(x), H(x, y(x), u(x))).
	\end{equation} 
	Since $u(x) = G(x, y(x),z(x))$, by definition of function $H$, one has $z(x) = H(x,y(x), u(x))$. Combining with the inequality \eqref{eqn_prop3.4}, we have $u(x')\ge G(x', y(x), z(x))$. Noticing $u(x') = G(x',y(x'),$ $z(x')) $, one has $G(x',y(x'),z(x')) = u(x') \ge G(x', y(x), z(x))$.
	By definition, $(y,z)$ is incentive compatible.
\end{proof}












