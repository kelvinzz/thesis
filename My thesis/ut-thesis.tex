%% ut-thesis.tex -- document template for graduate theses at UofT
%%
%% Copyright (c) 1998-2013 Francois Pitt <fpitt@cs.utoronto.ca>
%% last updated at 16:20 (EDT) on Wed 25 Sep 2013
%%
%% This work may be distributed and/or modified under the conditions of
%% the LaTeX Project Public License, either version 1.3c of this license
%% or (at your option) any later version.
%% The latest version of this license is in
%%     http://www.latex-project.org/lppl.txt
%% and version 1.3c or later is part of all distributions of LaTeX
%% version 2005/12/01 or later.
%%
%% This work has the LPPL maintenance status "maintained".
%%
%% The Current Maintainer of this work is
%% Francois Pitt <fpitt@cs.utoronto.ca>.
%%
%% This work consists of the files listed in the accompanying README.

%% SUMMARY OF FEATURES:
%%
%% All environments, commands, and options provided by the `ut-thesis'
%% class will be described below, at the point where they should appear
%% in the document.  See the file `ut-thesis.cls' for more details.
%%
%% To explicitly set the pagestyle of any blank page inserted with
%% \cleardoublepage, use one of \clearemptydoublepage,
%% \clearplaindoublepage, \clearthesisdoublepage, or
%% \clearstandarddoublepage (to use the style currently in effect).
%%
%% For single-spaced quotes or quotations, use the `longquote' and
%% `longquotation' environments.


%%%%%%%%%%%%         PREAMBLE         %%%%%%%%%%%%

%%  - Default settings format a final copy (single-sided, normal
%%    margins, one-and-a-half-spaced with single-spaced notes).
%%  - For a rough copy (double-sided, normal margins, double-spaced,
%%    with the word "DRAFT" printed at each corner of every page), use
%%    the `draft' option.
%%  - The default global line spacing can be changed with one of the
%%    options `singlespaced', `onehalfspaced', or `doublespaced'.
%%  - Footnotes and marginal notes are all single-spaced by default, but
%%    can be made to have the same spacing as the rest of the document
%%    by using the option `standardspacednotes'.
%%  - The size of the margins can be changed with one of the options:
%%     . `narrowmargins' (1 1/4" left, 3/4" others),
%%     . `normalmargins' (1 1/4" left, 1" others),
%%     . `widemargins' (1 1/4" all),
%%     . `extrawidemargins' (1 1/2" all).
%%  - The pagestyle of "cleared" pages (empty pages inserted in
%%    two-sided documents to put the next page on the right-hand side)
%%    can be set with one of the options `cleardoublepagestyleempty',
%%    `cleardoublepagestyleplain', or `cleardoublepagestylestandard'.
%%  - Any other standard option for the `report' document class can be
%%    used to override the default or draft settings (such as `10pt',
%%    `11pt', `12pt'), and standard LaTeX packages can be used to
%%    further customize the layout and/or formatting of the document.

%% *** Add any desired options. ***
\documentclass{ut-thesis}[10pt]


%% *** Add \usepackage declarations here. ***
%% The standard packages `geometry' and `setspace' are already loaded by
%% `ut-thesis' -- see their documentation for details of the features
%% they provide.  In particular, you may use the \geometry command here
%% to adjust the margins if none of the ut-thesis options are suitable
%% (see the `geometry' package for details).  You may also use the
%% \setstretch command to set the line spacing to a value other than
%% single, one-and-a-half, or double spaced (see the `setspace' package
%% for details).

\usepackage[utf8]{inputenc}
\usepackage{amsfonts}
\usepackage{amsthm}
\usepackage{amssymb}
\usepackage{amsmath}
%%\usepackage{amscd}
%%\usepackage[all]{xy}
%%\usepackage{fancyhdr}
\usepackage{enumerate}
\usepackage{accents}
\usepackage{xcolor}
%\usepackage{color}
\usepackage{upgreek}
%%\usepackage{verbatim}
\usepackage{mathrsfs}
%%\usepackage[numbers]{natbib}
%%\usepackage[fit]{truncate}
\usepackage{soul}
\usepackage{mathtools}

\usepackage{tabularx}
\usepackage{ragged2e}
\usepackage{booktabs}
\usepackage{caption}
%\usepackage{tabularx,ragged2e,booktabs,caption}



%%%%%%%%%%%%%%%%%%%%%%%%%%%%%%%%%%%%%%%%%%%%%%%%%%%%%%%%%%%%%%%%%%%%%%%%
%%                                                                    %%
%%                   ***   I M P O R T A N T   ***                    %%
%%                                                                    %%
%%  Fill in the following fields with the required information:       %%
%%   - \degree{...}       name of the degree obtained                 %%
%%   - \department{...}   name of the graduate department             %%
%%   - \gradyear{...}     year of graduation                          %%
%%   - \author{...}       name of the author                          %%
%%   - \title{...}        title of the thesis                         %%
%%%%%%%%%%%%%%%%%%%%%%%%%%%%%%%%%%%%%%%%%%%%%%%%%%%%%%%%%%%%%%%%%%%%%%%%

%% *** Change this example to appropriate values. ***
\degree{Doctor of Philosophy}
\department{Mathematics}
\gradyear{2018}
\author{Kelvin Shuangjian Zhang}
\title{Existence, Uniqueness, Concavity and Geometry of the Monopolist's Problem Facing Consumers with Nonlinear Price Preferences}

%% *** NOTE ***
%% Put here all other formatting commands that belong in the preamble.
%% In particular, you should put all of your \newcommand's,
%% \newenvironment's, \newtheorem's, etc. (in other words, all the
%% global definitions that you will need throughout your thesis) in a
%% separate file and use "\input{filename}" to input it here.

%
%\newtheorem{theorem}{Theorem}[section]
%\newtheorem{lemma}[theorem]{Lemma}
%\newtheorem{corollary}[theorem]{Corollary}
%\newtheorem{proposition}[theorem]{Proposition}
%\newtheorem{example}[theorem]{Example}
%% Do not remove the following line
%\theoremstyle{definition}
%\newtheorem{definition}[theorem]{Definition}
%% Do not remove the following line
%\theoremstyle{remark}
%\newtheorem{remark}[theorem]{Remark}
%%\newtheorem*{notation}{Notation}


% newtheorem

\theoremstyle{plain}
\newtheorem{theorem}{Theorem}[section]
\newtheorem{corollary}[theorem]{Corollary}
\newtheorem{lemma}[theorem]{Lemma}
\newtheorem{proposition}[theorem]{Proposition}
\newtheorem{claim}[theorem]{Claim}
%\newtheorem{proof}[theorem]{Proof}
\newtheorem{assumption}{Assumption}
\newtheorem{question}[theorem]{Question}
\newtheorem{conjecture}[theorem]{Conjecture}

\theoremstyle{definition}
\newtheorem{definition}[theorem]{Definition}
%\newtheorem{remark}[theorem]{Remark}
\newtheorem{example}[theorem]{Example}

\theoremstyle{remark}
\newtheorem{remark}[theorem]{Remark}
\newtheorem*{notation}{Notation}
\newtheorem{exercise}[theorem]{Exercise}





%%      ---------------------------------------------------------------------
%%      -------------------- AUTHORS' MACROS ---------------------
%%      ---------------------------------------------------------------------
\newcommand{\R}{{\mathbf R}}
\newcommand{\NN}{\mathbb{N}}
\newcommand{\N}{\mathbf{N}}
\newcommand{\HH}{{\mathbf H}}
%\newcommand{\H}{\mathbf{H}}
\newcommand\dom{{\mathop{\rm dom}}}
%\newcommand\Dom{{\mathop{\rm Dom}}}
\newcommand{\Gzero}{{\rm (G0)}}
\newcommand{\Gone}{{\rm (G1)}}
\newcommand{\Gtwo}{{\rm (G2)}}
\newcommand{\Gthree}{{\rm (G3)}}
\newcommand{\Gfour}{{\rm (G4)}}
\newcommand{\Gfive}{{\rm (G5)}}
\newcommand{\Gsix}{{\rm (G6)}}
\newcommand{\Gseven}{{\rm (G7)}}
\newcommand{\Geight}{\rm (G8)}
\newcommand{\nul}{{\emptyset}}
\newcommand{\Lip}{\operatornamewithlimits{Lip}}
%\newcommand{\Lip}{\operatorname{Lip}}
\newcommand{\argmin}{\operatornamewithlimits{argmin}}
\newcommand{\argmax}{\operatornamewithlimits{argmax}}
\newcommand{\yG}{ y_G}
\newcommand{\barG}{\bar{G}}
\newcommand{\barx}{\bar{x}}
\newcommand{\bary}{\bar{y}}
\newcommand{\sign}{\textrm{sign}}
\newcolumntype{C}[1]{>{\Centering}m{#1}}
\renewcommand\tabularxcolumn[1]{C{#1}}
\newcommand{\kel}[1]{\textbf{\textcolor{purple}{#1}}} 
\makeatletter
\def\munderbar#1{\underline{\sbox\tw@{$#1$}\dp\tw@\z@\box\tw@}}
\makeatother





\numberwithin{equation}{section}

%\allowbreak
\allowdisplaybreaks
%\numberwithin{equation}{section}

%% *** Adjust the following settings as desired. ***

%% List only down to subsections in the table of contents;
%% 0=chapter, 1=section, 2=subsection, 3=subsubsection, etc.
\setcounter{tocdepth}{2}

%% Make each page fill up the entire page.
\flushbottom


%%%%%%%%%%%%      MAIN  DOCUMENT      %%%%%%%%%%%%

\begin{document}

%% This sets the page style and numbering for preliminary sections.
\begin{preliminary}

%% This generates the title page from the information given above.
\maketitle

%% There should be NOTHING between the title page and abstract.
%% However, if your document is two-sided and you want the abstract
%% _not_ to appear on the back of the title page, then uncomment the
%% following line.
\cleardoublepage

%% This generates the abstract page, with the line spacing adjusted
%% according to SGS guidelines.
\begin{abstract}
%% *** Put your Abstract here. ***

	A monopolist wishes to maximize her profits by finding an optimal price menu. After she announces a menu of products and prices, each agent will choose to buy that product which maximizes his utility, if positive. 
	The principal's profits are the sum of the net earnings produced by each product sold.  
	These are determined by the costs of production and the distribution of products sold, which in turn are based on the distribution of anonymous agents and
	the choices they make in response to the principal's price menu. \medskip 
	
	In this thesis, two existence results will be provided, assuming each agent's disutility is a strictly increasing but not necessarily affine (i.e.,\ quasilinear) function of the price paid. This has been an open problem for several decades before the first multi-dimensional result obtained here and independently by N\"oldeke and Samuelson in 2015.\medskip
	

	Additionally, a necessary and sufficient condition for the convexity or concavity of this principal's (bilevel) optimization problem is investigated. 
	Concavity when present, makes the problem more amenable to 
	computational and theoretical analysis;  it is key to obtaining uniqueness and stability results for the principal's strategy in particular.  Even in the quasilinear case, our analysis goes beyond previous work by addressing convexity as well as concavity,  by establishing conditions which are not only sufficient but necessary,  and by requiring fewer hypotheses on the agents' preferences.	Moreover, the analytic and geometric interpretations of a specific condition relevant to the concavity of the problem have been explored.\medskip
	
	Finally, various examples are given to explain the interaction between preferences of agents' utility and monopolist's profit which ensure   statements equivalent to the concavity of the principal-agent problem. In particular, an example with quasilinear preferences on $n$-dimensional hyperbolic spaces is given with explicit solutions to show uniqueness without concavity. Similar results on spherical and Euclidean spaces are also provided. Additionally, the solutions of hyperbolic and spherical cases converge to those of Euclidean spaces as curvature goes to 0.\medskip
	%by showing (in exact form) the unique solutions of special examples with quasilinear preferences where domains are symmetric disks on $n$-dimensional hyperbolic,  spherical, or Euclidean spaces. What is more, the solutions of hyperbolic and spherical converges to that of Euclidean space as curvature goes to 0.
	
%	 Oriented by Loeper's work, %\cite{Loeper09}, 
%	my second solo work proves uniqueness by showing (in exact form) the unique solutions of special examples with quasilinear preferences where domains are symmetric disks on $n$-dimensional hyperbolic,  spherical, or Euclidean spaces. What is more, the solutions of hyperbolic and spherical converges to that of Euclidean space as curvature goes to 0.

%% (At most 150 words for M.Sc. or 350 words for Ph.D.)
\end{abstract}
\cleardoublepage
%% Anything placed between the abstract and table of contents will
%% appear on a separate page since the abstract ends with \newpage and
%% the table of contents starts with \clearpage.  Use \cleardoublepage
%% for anything that you want to appear on a right-hand page.

%% This generates a "dedication" section, if needed -- just a paragraph
%% formatted flush right (uncomment to have it appear in the document).
\begin{dedication}
	To my parents 
\\	Shuying Yu and Muli Zhang\\
	and my sister \\
	Qiuyu Zhang
%% *** Put your Dedication here. ***
\end{dedication}

%% The `dedication' and `acknowledgements' sections do not create new
%% pages so if you want the two sections to appear on separate pages,
%% uncomment the following line.
\newpage  % separate pages for dedication and acknowledgements

%% Alternatively, if you leave both on the same page, it is probably a
%% good idea to add a bit of extra vertical space in between the two --
%% for example, as follows (adjust as desired).
%\vspace{.5in}  % vertical space between dedication and acknowledgements

%% This generates an "acknowledgements" section, if needed
%% (uncomment to have it appear in the document).
\begin{acknowledgements}
%% *** Put your Acknowledgements here. ***

First and foremost, I would like to express my most profound gratitude to my advisor, Robert J. McCann, for his generous support of inspiration, encouragement, guidance, time, and patience, as well as financial support during my graduate study. Robert led me to the principal-agent problem and kept encouraging me on the research and writing processes. Our meetings and discussions on this topic have been the primary source of the ideas in this dissertation, most of which come from joint work with him.\medskip

I would like to thank my committee members, Xianwen Shi, for his guidance and encouragement in writing economics papers and fruitful discussions in economics topics, as well as his financial support during the summer of 2016, and Almut Burchard, for her encouragement and massive support in the early stage of my graduate career.\medskip

One of the most enjoyable periods of my graduate study must be the summer and fall semester spent at MSRI, Berkeley, in 2013. I would like to thank Robert for making it happen. In MSRI, I met a lot of renowned researchers in Optimal Transport and PDEs, %including %Cedric Villani, Luigi Ambrosio, Yann Brenier, Wilfrid Gangbo, Felix Otto, Alessio Figalli, and 
most notably Neil Trudinger, whose talk ``On the local theory of prescribed Jacobian equations" inspired our work on the generalization of nonlinear pricing problems.    \medskip

I also benefited much from the Fall 2014 program of the Fields Institute for the Mathematical Sciences. During this program, I met Guillaume Carlier, whose work influenced me. I would like to thank Guillaume for stimulating conversations. I would also like to thank Alfred Galichon and Robert McCann for inviting me to the NYU Workshop on Optimal Transportation, Equilibrium, and Applications to Economics, in April 2016. I would like to thank Alfred for his generous support, encouragement, and discussions.\medskip

I am grateful to Ivar Ekeland for stimulating conversations, Georg N\" oldeke and Larry Samuelson for sharing vital remarks.
I also got much help from the Optimal transport community, including  Afiny Akdemir, Jean-David Benamou, Yann Brenier, Shibing Chen, Marco Cuturi, Alessio Figalli, Nassif Ghoussoub, Nestor Guillen, Young-Heon Kim, Jun Kitagawa, Rosemonde Lareau-Dussault, Justin Martel, Evan Miller, Brendan Pass, Gabriel Peyré, Filippo Santambrogio and Cédric Villani. I much appreciate help from professors and researchers at the University of Toronto and other institutions, including John Bland, Guan Bo, James Colliander, Alfonso Gracia-Saz, Roger~Grosse, Marco Gualtieri, Robert Haslhofer, Ulrich Horst, Robert L. Jerrard, Jonathan Korman, Adrian Nachman, Mary Pugh, Joe Repka, Luis Seco,  Israel Michael Sigal, Adam Stinchcombe, and Catherine Sulem.
I would like to thank some graduate/undergraduate students and postdocs at the University of Toronto, from whom I got numerous encouragements, including Xinliang An,  Francis Bischoff, Li Chen, the late Yuri Cher,  Andrew Colinet, Payman Eskandari, Adam Gardner, Aidan Gomez, Mary He,  Jia Ji, Tomas Kojar, Chia-Cheng Liu, Xiao Liu,  James Lucas, Kevin Luk,  Amber Ma,  Mykola Matviichuk,  Xin Shen,  Kaixuan Wang,  Ming Xiao, Bin Xu,   Cheng Yang,  Qingwan Yin, Yuanyuan Zheng, and Zhifei Zhu.\medskip
 


I would also like to thank the administrative staff at the math department, in particular, the late Ida Bulat, Jemima Merisca, Patrina Seepersaud, and Sonja Injac, for all their endless help on making the graduate study more smooth and less stressful.  \medskip

Last but not the least, I would like to thank my parents and sister for their unconditional love, encouragement, and financial support.\medskip


\end{acknowledgements}

%% This generates the Table of Contents (on a separate page).
\tableofcontents

%% This generates the List of Tables (on a separate page), if needed
%% (uncomment to have it appear in the document).
%\listoftables

%% This generates the List of Figures (on a separate page), if needed
%% (uncomment to have it appear in the document).
%\listoffigures

%% You can add commands here to generate any other material that belongs
%% in the head matter (for example, List of Plates, Index of Symbols, or
%% List of Appendices).

%% End of the preliminary sections: reset page style and numbering.
\end{preliminary}


%%%%%%%%%%%%%%%%%%%%%%%%%%%%%%%%%%%%%%%%%%%%%%%%%%%%%%%%%%%%%%%%%%%%%%%%
%%  Put your Chapters here; the easiest way to do this is to keep     %%
%%  each chapter in a separate file and `\include' all the files.     %%
%%  Each chapter file should start with "\chapter{ChapterName}".      %%
%%  Note that using `\include' instead of `\input' will make each     %%
%%  chapter start on a new page, and allow you to format only parts   %%
%%  of your thesis at a time by using `\includeonly'.                 %%
%%%%%%%%%%%%%%%%%%%%%%%%%%%%%%%%%%%%%%%%%%%%%%%%%%%%%%%%%%%%%%%%%%%%%%%%

%% *** Include chapter files here. ***

\chapter{Introduciton}\label{chapter: introduction}

\section{Problem Formulation}

As one of the central problems in microeconomic theory, the {\em principal-agent framework} characterizes the type of non-competitive decision-making problems which involve aligning incentives so that one set of parties (the agents) finds it beneficial to act in the interests of another (the principal) despite holding private information.
It arises in a variety of different contexts. 
{ Besides nonlinear pricing \cite{Armstrong96,MussaRosen78,Spence80,Wilson93}, economists also use this framework to model many different types of transactions, including tax policy \cite{GuesnerieLaffont78,Mirrlees71,Rochet85}, contract theory \cite{QuinziiRochet85}, regulation of monopolies \cite{BaronMyerson82}, product line design \cite{RochetChone98}, labour market signaling \cite{Spence74}, public utilities \cite{Roberts79}, and mechanism design \cite{
KadanRenySwinkels11, MaskinRiley84, McAfeeMcMillan88, MonteiroPage98, Myerson81, Vohra11}. Many of these share the same mathematical model. }
In this thesis, we use nonlinear pricing to motivate the discussion,  in spite of the fact that our conclusions may be equally pertinent to many other areas of application. Besides, we only consider the case where both agent types and product attributes are continuous.\medskip

Consider the problem for a multiproduct monopolist who sells indivisible products to a population of consumers, who each buy at most one unit. Assume there is neither cooperation, nor competition between agents. Additionally, assume the monopolist is able to produce enough of each product such that there are neither product supply shortages {nor economies of scale}. Taking into account participation constraints and incentive compatibility, the monopolist would like to find the optimal menu of prices to maximize its total profit.\medskip

Suppose the monopolist wants to maximize her profits by selecting the dependence of the price $v(y)$ on each type  $y \in cl(Y)$ of product sold. An agent of type $x \in X$ will choose to buy that product which maximizes his benefit 
\begin{equation}\label{1}
u (x) := \max_{y \in cl(Y)} G(x, y, v(y)),
\end{equation}
where $(x, y, z)\in X \times cl(Y)\times \R \longmapsto G(x,y,z) \in \R$, is the given direct utility function for agent type $x$ to choose product type $y$ at price $z$, and $X,Y$ are open and bounded subsets in $\R^m$ and $\R^n$ ($m \ge n$), respectively,
with closures $cl(X)$ and $cl(Y)$.\medskip

After agents, whose distribution $d \mu(x)$ is known to the monopolist, have chosen their favorite items to buy, the monopolist calculates her profit to be given by the functional 
\begin{equation}\label{profit}
\Pi(v, y):=\int_{X} \pi(x,y(x),v(y(x))) d\mu(x),
\end{equation}
where $y(x)$ denotes the product type $y$ which agent type $x$ chooses to buy (and which maximizes \eqref{1}), $v(y(x))$ denotes the selling price of type $y(x)$ and $\pi \in C^0(cl(X\times Y)\times \R)$ denotes the principal's net profit of selling product type $y \in cl(Y)$ to agent type $x \in X$ at price $z \in \R$. 
The monopolist wants to maximize her net profit among all lower semicontinuous pricing policies. \medskip




The following is a table of notation:

\begin{minipage}{\linewidth}
	\bigskip
	\RaggedRight 
	\captionof{table}{Notation} \label{tab:title} 
	\begin{tabularx}{6in}{ C{0.88in} l l}\toprule[1.5pt]
		Mathematical Expression & {Economic Meaning}  & \\ 
		\midrule[0.5pt]
		$x$  & {agent type}  &\\
		$y$  & {product type} &\\
		$X \subset \R^m$  & {(open, bounded) domain of agent types} &\\
		$cl(Y) \subset \R^n$  & {domain of product types, closure of $Y$} &\\
		$v(y)$  & {selling price of product type $y$ (we use $p(y)$ in Chapter \ref{chapter:existence} instead)}&\\
		$v(y_\nul) \le z_\nul$ & {price normalization of the outside option $y_\nul \in cl(Y)$}&\\ 
		$u(x)$  & {indirect utility of agent type $x$}&\\
		$\dom Du$ & {points in $X$ where $u$ is differentiable} &\\ 
		$G(x,y,z)$ & {direct utility of buying product $y$ at price $z$ for agent $x$ }&\\
		$H(x,y,u)$ & \multicolumn{2}{p{12.1cm}}{\raggedright {price at which $y$ brings $x$ value $u$, so that $H(x,y,G(x,y,z))=z$}} \\
		$\pi(x,y,z)$  & \multicolumn{2}{p{12.1cm}}{the principal's profit for selling product $y$ to agent $x$ at price $z$ }\\
		$d\mu(x)$ & \multicolumn{2}{p{12.1cm}}{Borel probability measure giving the distribution of agent types on $X$}\\
		$\mu  \ll {\mathcal L}^m$ & \multicolumn{2}{p{12.1cm}}{$\mu$ vanishes on each subset of $\R^m$ having zero Lebesgue volume ${\mathcal L}^m$} \\
		$\Pi(v,y)$ & \multicolumn{2}{p{12.1cm}}{monopolist's profit facing agents' responses $y(\cdot)$ to her chosen price policy $v(\cdot)$} \\	
		${\pmb\Pi}(u)$ & \multicolumn{2}{p{12.9cm}}{monopolist's profit, viewed instead as a function of agents' indirect utilities $u(\cdot)$ } \\
		\bottomrule[1.25pt]
		\end {tabularx}\par
		\bigskip
	\end{minipage}
\medskip

%One may consider this model as a bilevel optimization. After a monopolist publishes its price menu, each consumer maximizes his utility through the purchase of at most one product. The monopolist maximizes aggregate profits based on consumers' choices, knowing only the statistical distribution of consumer types.\medskip

In economic models, 
incentive compatibility is  needed to ensure that all the agents report their preferences truthfully. 
According to the revelation principle (such as \cite{Myerson79}), this costs no generality.
Decisions made by monopolist according to the information collected from agents then lead 
to the expected market reaction (as in \cite{Carlier01,RochetChone98}).  
Individual rationality
is required to ensure full participation, so that each agent will choose to play,  possibly by accepting the outside option. And individual agents accept to contract only if the benefits they earn are no less than their outside option. We model this by assuming the existence of a distinguished point $y_\nul \in cl(Y)$ which represents the outside option, and whose price cannot exceed some fixed value $z_\nul \in \R$ beyond the monopolist's control. This removes any incentive for the monopolist to raise the prices of other options too high. (We can choose normalizations such as $\pi(x,y_\nul,z_\nul)=0=G(x,y_\nul,z_\nul)$ and $(y_\nul,z_\nul)=(0,0)$, or not, as we wish.)\medskip

\begin{definition}[Incentive compatible and individually rational]
	A measurable map $x \in X \longmapsto (y(x),z(x)) \in cl(Y \times Z)$ of agents to (product, price) pairs
	is called {\em incentive compatible} if and only if $G(x,y(x),z(x)) \ge G(x, y(x'), z(x'))$ for all $(x,x')\in X^2$.
	Such a map offers agent $x$ no incentive to pretend to be $x'$.
	It is called {\em individually rational} if and only if $G(x,y(x),z(x)) \ge G(x,y_\nul,z_\nul)$ for all $x \in X$,
	meaning no individual $x$ strictly prefers the outside option $(y_\nul,z_\nul)$ to his assignment $(y(x),z(x))$.
\end{definition}

\begin{proposition}
	Then principal's program can be described as follows:
	\begin{equation*}
	(P_0)
	\begin{cases}
	\sup \Pi(v,y)=\int_{X} \pi(x, y(x), v(y(x))) d\mu(x)\quad \text{among}\\ 
	x \in X \longmapsto (y(x),v(y(x))) \text{  incentive compatible,  individually rational,}\\ 
	{ \text{and}\ v:cl(Y)\longrightarrow cl (Z)\  \text{lower semicontinuous with}\ v(y_\nul) \le z_\nul}. 
	\end{cases}
	\end{equation*}
\end{proposition}

\medskip


\section{Background}
The thesis study a general version of a multidimensional nonlinear pricing model,  which is a natural extension of the models studied by Mussa-Rosen \cite{MussaRosen78}, %Roberts \cite{Roberts79}, 
Mirrlees \cite{Mirrlees71},	Spence \cite{Spence74, Spence80}, Myerson \cite{Myerson81}, Baron-Myerson \cite{BaronMyerson82}, Maskin-Riley \cite{MaskinRiley84}, Wilson \cite{Wilson93}, Rochet-Chon$\acute{e}$ \cite{RochetChone98}, Monteiro-Page \cite{MonteiroPage98} and  Carlier~\cite{Carlier01}. A major distinction lies in whether the private type is one-dimensional (such as \cite{MussaRosen78, MaskinRiley84}), or multidimensional (such as \cite{QuinziiRochet85, RochetChone98,MonteiroPage98, Carlier01}). Another distinction is whether preferences are quasilinear on price (such as \cite{Armstrong96, Carlier01}) or fully nonlinear (such as \cite{NoldekeSamuelson15p, McCannZhang17}), especially for multidimensional models.  
	\medskip



	For the quasilinear case, where the utility $G(x,y,z)$ depends linearly on its third variable, and net profit $\pi(x,y,z)=z-a(y)$ represents difference of selling price $z$ and manufacturing cost $a$ of product type $y$, theories of existence \cite{Basov05,RochetStole03,Carlier01,MonteiroPage98}, uniqueness 
	\cite{CarlierLachand-Robert01,FigalliKimMcCann11,MussaRosen78,RochetChone98} 
	and robustness \cite{Basov05,FigalliKimMcCann11} have been well studied.
\medskip

When parameterization of preferences is linear in agent types and price, where $ cl(X) = cl(Y) = [0,\infty)^n$, $G(x,y,z) = \langle x,y \rangle -z$,  %$a(y) = \langle y,y\rangle /2$, 
and $(y_\nul,z_\nul)=(0,0)$, Rochet and Chon$\acute{e}$ (1998, \cite{RochetChone98}) not only obtain existence results but also partially characterize optimal solutions and expound their economic interpretations, %the solution for optimality with economic interpretations, 
given that monopolist profits can be characterized by {the aggregate difference between selling prices and quadratic manufacturing costs.} Here $\langle\ ,\ \rangle$ denotes the Euclidean inner product.\medskip

More generally, Carlier (\cite{Carlier01}) proves existence results for general quasilinear utility  $G(x,y,z) = b(x,y)- z$ , where agent type and product type are not necessarily of the same dimension and monopolist profit equals selling price minus some linear manufacturing cost.\medskip

Figalli-Kim-McCann \cite{FigalliKimMcCann11} reveals the equivalence of function space convexity to the non-negative fourth order cross-curvature condition, and conditions of functional concavity, where uniqueness and stability of the monopolist's maximizing strategy follow from strict concavity.\medskip

%{\color{red} 
%Warning:	Add more reference in this background section.
%\color{blue}	
% see also Mirrlees \cite{Mirrlees71}, Spence \cite{Spence74}, Monteiro and Page \cite{MonteiroPage98}. }


\section{Motivation}

Starting from celebrated work of Nobel Laureates Mirrlees \cite{Mirrlees71} and Spence \cite{Spence74}, there are two main types of generalizations. One generalization is in terms of dimension, from 1-dimensional to multi-dimensional. The other generalization is in utility functional form, from quasilinear to non-quasilinear.\medskip

The generalization of quasilinear to nonlinear preferences has many potential applications. For example, the benefit function $G(x,y,v(y))=b(x,y) -v^2(y)$ models agents who are more sensitive to higher prices, while another function $G(x,y,v(y))=b(x,y)-v^{\frac{1}{2}}(y)$ models agents who are less sensitive to higher prices, and utility $G(x,y,v(y))=b(x,y) -f(x, v(y))$ describes the scenario when different agents might have different sensitivities to the same price. See Wilson~\cite[Chapter 7]{Wilson93} for the importance of taking income effects into account. 
Very few results are known for such nonlinearities,  due to the complications which they entail. \medskip


	In 2013, Trudinger's lecture at the optimal transport program at MSRI inspired us to try generalizing Carlier \cite{Carlier01} and Figalli-Kim-McCann \cite{FigalliKimMcCann11} to the non-quasilinear case. With the tools developed by Trudinger \cite{Trudinger14} and others  \cite{Balder77,Singer97}, we are able to provide existence, convexity and concavity theorems for general utility and net profit functions. \medskip

	The generalized existence problem was also mentioned as a conjecture by Basov \cite[Chapter 8]{Basov05}. 
	Recently, N\"oldeke and Samuelson (2015, \cite{NoldekeSamuelson15p}) provided a general existence result for $cl(X)$, $cl(Y)$ being compact and the utility $G$ being decreasing with respect to its third variable, by implementing a duality argument based on Galois Connections. \medskip



	The equivalence of concavity to the corresponding non-negative cross-curvature condition revealed by Figalli-Kim-McCann \cite{FigalliKimMcCann11} directly inspires our work. In addition to the quaslinearity of
	$G(x,y,z) = b(x,y) - z$ essential to their model,  they require additional restrictions such as $m=n$ and $b \in C^4(cl(X\times Y))$ which are not economically motivated
	and which we shall relax or remove. However,  we shall eventually show that under certain conditions the concavity or convexity of $G$ and $\pi$ (or their derivatives)
	with respect to $v$ tends to be reflected by concavity or convexity of $\Pi$, not with respect to 
	$v$ or $y$,  but rather with respect to the agents indirect utility $u$, in terms of 
	which the principal's maximization is reformulated below. Moreover, our results allow for the monopolist's profit $\pi$ to depend in a general
	way both on monetary transfers and on the agents' types $x$,  revealed after contracting.  Such dependence plays an important role in applications such as insurance marketing.\medskip
	
	Inspired by Kim-McCann \cite{KimMcCann10}, which expressed the fourth-order Ma-Trudinger-Wang condition in optimal transportation theory via non-negativity of the sectional curvature in some pseudo-Riemannian geometry,
	we would like to explore the geometric interpretations of some hypothesis to the concavity results.\medskip


Figalli-Kim-McCann \cite{FigalliKimMcCann11} provides a non-negative definiteness condition of some fourth order differential expression (B3), which not only is equivalent to the convexity of function space,  but also implies concavity of the maximization functional, and thus uniqueness follows from a strict version of (B3). One may wonder what happens if this curvature condition (B3) is dissatisfied. Inspired by Loeper \cite{Loeper09}, which claims that, for quasilinear Riemannian quadratic utility, (B3) is satisfied if and only if the Riemannian sectional curvature is non-negative, some part of the thesis aims to investigate uniqueness  
without concavity on the hyperbolic spaces with constant negative curvatures. Besides, previously there are few explicit results on spaces with dimensionality greater than two.\medskip

It is worth mentioning that given the technical arguments exploited in this thesis, it may be very fruitful to study possible generalizations of other known results for convex functions to $G$-convex functions.
\medskip


\section{Outline of the Thesis}


	Chapter 2 provides some preliminaries and, in particular, a generalized notion of convex functions: the $G$-convex function (c.f. \cite{Trudinger14,Balder77,Singer97}).
	We'll also see that the incentive compatibility is conveniently encoded via the  $G$-convexity
	of the agents' indirect utility $u$, which is an analog of Carlier \cite{Carlier01}.
	\medskip

	Initialed independently of \cite{NoldekeSamuelson15p}, chapter \ref{chapter:existence}  provides a general existence result for the multidimensional monopolist model with general nonlinear preferences with less restriction on boundedness of the product domain, {by extending} %in an extension of 
	Carlier \cite{Carlier01} to fully nonlinear preferences. Due to lack of natural compactness, the proof of this work is quite different from that of N$\ddot{o}$ldeke-Samuelson. Furthermore, $G$-convex analysis, which is strongly tied to Trudinger's theory on regularity of nonlinear PDEs \cite{Trudinger14}  developed for vastly different purposes, is employed to deal with the difficulty of non-quasilinear preferences.	
	\medskip


%My joint work with Robert J. McCann

Chapter \ref{chapter:existence_bounded} presents another general existence result given the generalized single-crossing condition and boundedness of the consumer-type and product-type spaces. This result is also shown using $G$-convex analysis, but the proof is different from chapter \ref{chapter:existence}, since most assumptions are different.\medskip

%Moreover, together with McCann, 

We will show convexity results in chapter \ref{chapter:convexity}. In Chapter \ref{chapter:convexity}, we generalize uniqueness and concavity results of Figalli-Kim-McCann to the non-quasilinear case. In this work, we first give a necessary and sufficient condition \Gthree~under which the function space $\mathcal{U}_\nul$ is convex. \medskip

We then provide the equivalent conditions, respectively, to the concavity, convexity, uniform concavity, and uniform convexity of the functional $\pmb\Pi$. We also give sufficient conditions for strict concavity, which implies uniqueness for this problem. Besides, the maximizers of $\pmb\Pi$ may not be unique under convexity, but are attained at extreme points of the function space $\mathcal{U}_{\emptyset}$.\medskip 

We also show that the concavity (uniform concavity) condition is equivalent to non-positive (uniform negative) definiteness of some quadratic form on $\R^{n+1}$. \medskip

The condition \Gthree~is so crucial to the concavity result that we want to investigate it a bit more. Chapter \ref{chapter:analytic_representation} shows that \Gthree~is equivalent to the non-positive definiteness of some fourth order differential expression along affinely parametrized line segments, which is an analog of the non-negative definiteness of the fourth order condition adopted in Trudinger \cite{Trudinger14} for regularity of prescribed Jacobian equations. It also coincides in the quasilinear case with the fourth order condition provided in Figalli-Kim-McCann \cite{FigalliKimMcCann11}, which corresponds to the Ma-Trudinger-Wang condition \cite{MaTrudingerWang05} in regularity theory of Optimal Transport.\medskip

Motived by Kim-McCann \cite{KimMcCann10}, in chapter \ref{chapter:geometry}, we will show that \Gthree~is equivalent to non-negativity of the sectional curvature in some natural pseudo-Riemannian geometry.\medskip




Oriented by Loeper's work \cite{Loeper09}, chapter \ref{chapter:examples} proves uniqueness by showing (in exact form) the unique solutions of special examples with quasilinear preferences where domains are symmetric disks on $n$-dimensional hyperbolic $\mathbf{H}^n$, and the utility is a quasilinear quadratic hyperbolic distance. It also shows solutions on spherical $\mathbf{S}^n$ and Euclidean spaces $\R^n$, where the utility is a quasilinear quadratic spherical or Euclidean distance. Moreover, the solutions on $\mathbf{S}^n$ and $\mathbf{H}^n$ converge to those on $\R^n$, as curvature goes to $0$.
\medskip



%since convexity imply uniqueness but uniqueness does not necessarily implies convexity. 
%This work shows that we may still have uniqueness, via uniqueness examples on some negative curvature spaces. \medskip \medskip



For non-quasilinear preferences, we specialize the form obtained from chapter \ref{chapter:convexity} into various examples  and give the equivalent conditions to the concavity/convexity of the maximization problem.\medskip


\begin{remark}
	Chapter \ref{chapter:existence_bounded}, \ref{chapter:convexity}, \ref{chapter:analytic_representation}, \ref{chapter:geometry} and second part of chapter \ref{chapter:examples} are joint work with my advisor Robert J. McCann. It should be mentioned here that neither the convexity work, nor the earlier two existence results, require the monopolist profit to take on a special form, which is a generalization from much of the literature. And the $G$-convexity method in this thesis is potentially applicable to other problems under the same principal-agent framework, such as the study of tax policy (\cite{Mirrlees71}) and other regulatory policies (\cite{BaronMyerson82}). For an application of $G$-convexity to geometric optics, see~\cite{GuillenKitagawaCPAM}. \medskip
	
\end{remark}








\chapter{Preliminaries and $G$-convexity}\label{chapter:preliminaries}

\section{preliminaries}


Let $X$ be a subset of $\R^m$ and $S$ be any set of functions on $X$, i.e., $S\subset \{f: X \longrightarrow \R \}$.
\medskip
\begin{definition}[Convex Sets]
	A set $A \subset \R^m$ or $A \subset S$ is called convex if and only if for any $x , y \in A$, and any $t \in [0,1]$,  $ t x + (1-t)y \in A$.
\end{definition}

%\begin{lemma}
%	Let $A$ be a subset contained in $\R^m$ or $S$, the following statements are equivalent:
%	\begin{enumerate}
%		\item $A$ is convex;
%		\item For any $x, y \in A$, $\frac{1}{2}x + \frac{1}{2}y$ also belongs to $A$;
%		\item For any $r\in \N$, $r>1$, any element $x_1, ..., x_r \in A$, and for any nonnegative numbers $t_1, ..., t_r$ such that $t_1 + ... + t_r =1$, then $\sum_{k = 1}^{r} t_k x_k \in A$.
%	\end{enumerate}
%\end{lemma}

\begin{definition}[(Strictly) Convex Functions]
	Let $X$ be a convex set in $\R^m$ and let $f: X \longrightarrow \R$ be a function. Then $f$ is called (strictly) convex if for any $x_1, x_2 \in X$ and any $t\in (0,1)$, the following inequality holds.
	\begin{flalign}
		f(tx_1 +(1-t) x_2) (<) \le tf(x_1) +(1-t) f(x_2).
	\end{flalign} 
\end{definition}

A function $f: X \longrightarrow \R$ is said to be (strictly) concave is $-f$ is (strictly) convex.
\medskip

\begin{definition}[Subdifferential]
	Recall that the subdifferential of a (convex) function $u: X \longrightarrow \R$ at $x_0 \in X$ is defined as the set:
	\begin{equation}\label{defn:subdifferential}
	\partial u(x_0) := \{ y \in \R^m| u(x) - u(x_0) \ge \langle  x- x_0,  y \rangle, \text{ for all } x \in X  \}.
	\end{equation}
	Here  $ \langle, \rangle$ denotes the Euclidean inner product. 
\end{definition}

\begin{lemma}\label{lemma:subdiff}
	This set defined in \eqref{defn:subdifferential} is nonempty for every $x_0 \in X$ if and only if $u$ is convex.
\end{lemma}

We give a proof below for a generalized version of this lemma.\medskip

For any vectors $p, w \in \R^n$, we denote $p\parallel w$ if $p$ and $w$ are parallel.\medskip

We use $\mathcal{L}^m$ to denote Lebesgue measure, which characterizes $m$-dimensional volume. A non-negative measure $\mu$ is said to be absolutely continuous with respect to  $\mathcal{L}^m$ if for every measurable set $A$, $\mathcal{L}^m(A) =0$ implies $\mu(A) =0$. This is written as $\mu \ll \mathcal{L}^m$. \medskip

Here we use $G_x=\big(\frac{\partial G}{\partial x^1}, \frac{\partial G}{\partial x^2}, ..., \frac{\partial G}{\partial x^m}\big)$, $G_y=\big(\frac{\partial G}{\partial y^1}, \frac{\partial G}{\partial y^2}, ..., \frac{\partial G}{\partial y^n}\big)$, $G_z = \frac{\partial G}{\partial z}$ to denote derivatives with respect to $x\in X\subset \R^m$ , $y \in Y \subset \R^n$, and $z\in \R$, respectively. Also, for second partial derivatives, we adopt following notations
\begin{flalign*}
G_{x,y}=
\begin{bmatrix}
\frac{\partial^2 G}{\partial x^1 \partial y^1} & \frac{\partial^2 G}{\partial x^1 \partial y^2} & ... & \frac{\partial^2 G}{\partial x^1 \partial y^n}  \\
\frac{\partial^2 G}{\partial x^2 \partial y^1} & \frac{\partial^2 G}{\partial x^2 \partial y^2} & ... & \frac{\partial^2 G}{\partial x^2 \partial y^n}  \\	
\vdots & \vdots & \ddots & \vdots \\
\frac{\partial^2 G}{\partial x^m \partial y^1} & \frac{\partial^2 G}{\partial x^m \partial y^2} & ... & \frac{\partial^2 G}{\partial x^m \partial y^n}  
\end{bmatrix},
\end{flalign*}
and $G_{x,z}=\big(\frac{\partial^2 G}{\partial x^1 \partial z}, \frac{\partial^2 G}{\partial x^2 \partial z}, ..., \frac{\partial^2 G}{\partial x^m \partial z}\big)$. 
\medskip

We say $G \in C^{1}(cl(X\times Y \times Z))$, if all the partial derivatives $\frac{\partial G}{\partial x^1}$, ...,$\frac{\partial G}{\partial x^m}$, $\frac{\partial G}{\partial y^1}$, ..., $\frac{\partial G}{\partial y^n}$, $\frac{\partial G}{\partial z}$ exist and are continuous. Also, we say $G \in C^{2}(cl(X\times Y \times Z))$, if all the partial derivatives up to second order (i.e. $\frac{\partial^2 G}{\partial \alpha \partial \beta}$, where $\alpha, \beta = x^1, ... , x^m, y^1, ..., y^n, z$) exist and are continuous. Any bijective continuous function whose inverse is also continuous, is called a homeomorphism (a.k.a.\ bicontinuous).
\medskip

\section{$G$-convex, $G$-subdifferentiability}\label{section:G-convexity}
In this section, we introduce some tools from convex analysis and the notion of $G$-convexity  (c.f. \cite{Trudinger14,Balder77,Singer97}), which is a generalization of ordinary convexity. \medskip

Let $X$, $Y$, and $Z$ be subsets of $\R^m$, $\R^n$, and $\R$ respectively. Assume $G: X \times Y \times Z \longrightarrow \R$ is any function which is strictly decreasing on its last variable. For each $(x, y) \in X\times cl(Y)$ and $u\in G(x,y, cl(Z))$, define $H(x,y,u) := z$ whenever $G(x,y,z) = u$, i.e., $H(x, y, \cdot)= G^{-1}(x,y,\cdot)$. In the context of nonlinear pricing, $G(x,y,z)$ represents the utility that consumer $x$ obtains for purchasing product $y$ at price $z$, while $H(x,y,u)$ denotes the price paid by agent $x$ for product $y$ when receiving value $u$. \medskip


From Lemma \ref{lemma:subdiff}, for any convex function $u$ on $X$ and any fixed point $x_0 \in X$, there exists $y_0 \in \partial u(x_0)$, satisfying% {$u(x) - u(x_0) \ge \langle x - x_0, y_0\rangle$}, i.e., 
\begin{equation}\label{convex_function}
	u(x) \ge  \langle x , y_0\rangle -( \langle x_0, y_0\rangle -  u(x_0)),	\text{  for all $x \in X$},
\end{equation} 
where equality holds at $x = x_0$. On the other hand, if for any $x_0\in X$, there exists $y_0$, such that \eqref{convex_function} holds for all $x\in X$, then $u$ is convex. The following definition is analogous to this property, which is a special case of $G$-convexity, when $G(x,y,z) = \langle x, y \rangle -z$. In this case, we have $H(x,y,u) = \langle x, y \rangle -u$. \medskip
	
\begin{definition}[$G$-convexity]\label{defn:GConvexity}
 A function $u\in C^0(X)$ is called {\it $G$-convex} if for each $x_0 \in X$, there exist $y_0 \in   cl(Y)$, and $z_0 \in  cl(Z)$ such that $u(x_0)=G(x_0, y_0, z_0)$, and $u(x)\ge G(x, y_0, z_0)$, for all $x\in X$.
\end{definition}
		
		%%\begin{comment}
		%%From the definition, we know if $u$ is a $G$-convex function, for any $x\in X$, there exist $y\in cl(Y)$ and $z \in cl(Z)$, such that
		%%\begin{equation}\label{2}
		%%u(x)= G(x, y, z),\ \ \   Du(x) = D_x G(x, y, z)
		%%\end{equation}
		%%
		%%Given $u(x), Du(x)$, one can solve $y, z$, according to Assumption 1. \medskip
		%%\end{comment}
		
Similarly, one can also generalize the definition of subdifferential from \eqref{convex_function}.\medskip
		
\begin{definition}[$G$-subdifferentiability]\label{defn:GSubdifferential}
	The $G$-subdifferential of a 
			%$G$-convex % 
	function $u: X \longrightarrow \R$ is a point-to-set mapping defined by
	\begin{equation*}
			\partial^G u(x):= \{ y\in  cl(Y)| u(x')\ge G(x',y, H(x,y,u(x))), \text{ for all } x'\in X\}.
	\end{equation*}
	A function $u$ is said to be {\it $G$-subdifferentiable} at $x$ if and only if $\partial^G u(x) \neq \emptyset$.\footnote{In Trudinger \cite{Trudinger14}, this point-to-set mapping $\partial^G u$ is also called $G$-$normal$ mapping; see paper for more properties related to $G$-convexity.}		
\end{definition}
			
			
			
			
In particular, if $G(x,y,z) = \langle x, y \rangle - z$, then the $G$-subdifferential coincides with the subdifferential. There are other generalizations of convexity and subdifferentiability. For instance, $h$-convexity in Carlier \cite{Carlier01}, or equivalently, $b$-convexity in Figalli-Kim-McCann \cite{FigalliKimMcCann11}, or $c$-convexity in  Gangbo-McCann \cite{GangboMcCann96}, is a special form of $G$-convexity, where $G(x,y,z)=  h(x,y) -z$, which serves an important role in the quasilinear case. For more references of convexity generalizations, see Kutateladze-Rubinov \cite{KutateladzeRubinov72}, Elster-Nehse \cite{ElsterNehse74}, Balder \cite{Balder77}, Dolecki-Kurcyusz \cite{DoleckiKurcyusz78},  Singer \cite{Singer97},  Rubinov \cite{Rubinov00a}, and Martínez-Legaz \cite{MartinezLegaz05}.\medskip
			
			
As mentioned above, it is known in convex analysis that a function is convex if and only if it is subdifferentiable everywhere. The following lemma adapts this to the notion of $G$-convexity. \medskip
			
\begin{lemma}\label{convex-subdiff0}
A function $u: X \rightarrow \R$ is $G$-convex if and only if it is $G$-subdifferentiable everywhere.
\end{lemma}
\begin{proof}[Proof]
		Assume $u$ is $G$-convex, want to show $u$ is $G$-subdifferentiable everywhere, i.e., need to prove  $\partial^G u(x_0)\neq \emptyset$ for all $x_0\in X$.
		
		Since $u$ is $G$-convex, by definition, for each $x_0$, there exists $ y_0, z_0$, such that $u(x_0) = G(x_0,y_0,z_0)$, and for all  $ x \in X$, 
		\begin{equation*}
		u(x)\ge G(x, y_0, z_0) = G(x, y_0, H(x_0,y_0,u(x_0))).
		\end{equation*}
		By the definition of $G$-subdifferentiability, $y_0 \in \partial^G u(x_0)$, i.e. $\partial^G u(x_0) \neq \emptyset$.
		
		On the other hand, assume $u$ is $G$-subdifferentiable everywhere, then for each $ x_0 \in X$,  there exists $ y_0 \in \partial^G u(x_0)$. Set $z_0:=H(x_0,y_0,u(x_0))$ so that $u(x_0) = G(x_0, y_0, z_0)$.
		
		Since $y_0\in \partial^G u(x_0)$, for all $x\in X$, we have 
		\begin{equation*}
		u(x)\ge G(x,y_0,H(x_0,y_0,u(x_0))) = G(x,y_0,z_0).
		\end{equation*}
		By definition, $u$ is $G$-convex.
\end{proof}
%\begin{proof}%[Proof of Lemma \ref{convex-subdiff0}]
%	$"\Rightarrow"$. Assume $u$ is $G$-convex, we want to show $u$ is $G$-subdifferentiable everywhere, i.e., the statement $\partial^G u(x_0)\neq \emptyset$ is true for any $ x_0\in X$.
%	
%	Since $u$ is $G$-convex, by definition, for any fixed $x_0 \in X$, there exist $y_0$ and $z_0$, such that $u(x_0) = G(x_0,y_0,z_0)$, and for any $x \in X$, $u(x)\ge G(x, y_0, z_0)$. By definition of function $H$, we know $z_0 = H(x_0,y_0,u(x_0))$. Therefore, combining the above inequality, we have $u(x) \ge G(x, y_0, H(x_0,y_0,u(x_0)))$, for any $x \in X$. By definition of $G$-subdifferentiability, it implies $y_0 \in \partial^G u(x_0)$, that is, $\partial^G u(x_0) \neq \emptyset$.
%	
%	
%	$"\Leftarrow"$. Assume $u$ is $G$-subdifferentiable everywhere. Then for any fixed $x_0 \in X$, there exists $y_0 \in \partial^G u(x_0)$. Define $z_0:=H(x_0,y_0,u(x_0))$, then by definition of function $H$, one has $u(x_0) = G(x_0, y_0, z_0)$.
%	%\sloppy
%	
%	Since $y_0\in \partial^G u(x_0)$, by definitions of $G$-subdifferentiability, we have $u(x)\ge G(x,y_0,H(x_0,y_0,u(x_0)))$\ $ = G(x,y_0,z_0)$, for any $x\in X$, where the last equality comes from the definition of $z_0$.
%	
%	By definition (of $G$-convexity), $u$ is $G$-convex.
%\end{proof}				
				
				
Using Lemma \ref{convex-subdiff0}, one can show the following result,  %Proposition \ref{incen/convex} 
which plays the role of bridge connecting incentive compatibility in the economic context with $G$-convexity and $G$-subdifferentiability in mathematical analysis, generalizing the results of Rochet \cite{Rochet87} and Carlier \cite{Carlier01}.
\medskip




\begin{proposition}[$G$-convex utilities characterize incentive compatibility]\label{incen/convex}
	Let $(y,z)$ be a pair of mappings from $X$ to $cl(Y) \times cl(Z)$.  This (product, price) pair is incentive compatible 
	if and only if $u(\cdot):=G(\cdot,y(\cdot),z(\cdot))$ is $G$-convex and $y(x)\in \partial^G u(x)$ for each $x \in X$.
\end{proposition}
					
\begin{proof}%[Proof of Proposition \ref{incen/convex}]
	$"\Rightarrow"$. Suppose $(y,z)$ is incentive compatible. For any fixed $x_0 \in X$, let $y_0 = y(x_0)$ and $z_0 = z(x_0)$. Then $u(x_0) = G(x_0, y(x_0), z(x_0)) = G(x_0, y_0, z_0)$. By incentive compatibility of the contract $(y,z)$, for any $x\in X$, one has $G(x, y(x), z(x)) \ge G(x, y(x_0), z(x_0))$. This implies $u(x) \ge G(x,y_0,z_0)$, for any $x\in X$, since $u(x)= G(x, y(x), z(x))$,  $y_0 = y(x_0)$ and $z_0 = z(x_0)$. By definition, $u$ is $G$-convex. 
	
	Since $u(x_0)=G(x_0, y_0, z_0)$, by definition of function $H$, one has $z_0 = H(x_0, y_0, u(x_0))$.  Combining with $u(x) \ge G(x, y_0, z_0)$,  for any $x\in X$, which is concluded from above, we have $u(x)\ge G(x, y_0, H(x_0, y_0, u(x_0)))$, for any $x\in X$. By definition of  $G$-subdifferentiability, one has $y_0 \in \partial^G u(x_0)$, and thus $y(x_0) = y_0 \in \partial^G u(x_0)$.
	
	$"\Leftarrow"$. Assume that $u = G(x, y(x),z(x))$ is $G$-convex, and $y(x)\in \partial^G u(x)$, for any $x\in X$. For any fixed $x \in X$, since $y(x)\in \partial^G u(x)$, for any $x'\in X$, one has 
	\begin{equation}\label{eqn_prop3.4}
	u(x')\ge G(x', y(x), H(x, y(x), u(x)))
	\end{equation} 
	Since $u(x) = G(x, y(x),z(x))$, by definition of function $H$, one has $z(x) = H(x,y(x), u(x))$. Combining with the inequality \eqref{eqn_prop3.4}, we have $u(x')\ge G(x', y(x), z(x))$. Noticing $u(x') = G(x',y(x'),$ $z(x')) $, one has $G(x',y(x'),z(x')) = u(x') \ge G(x', y(x), z(x))$.
	By definition, (y,z) is incentive compatible.
\end{proof}














\chapter{Existence: unbounded product spaces}\label{chapter:existence}

%\title{Existence in Multidimensional Screening with General Nonlinear Preferences}%without Quasilinearity}%{IMPLEMENTABILITY WITHOUT QUASILINEARITY}%\thanks{}


%%\begin{abstract}
%%	We generalize the approach of Carlier (2001) and provide an existence proof for the multidimensional screening problem with general nonlinear preferences.
%%	{We first} formulate the principal's {problem}
%%	as a maximization problem with $G$-convexity constraints, 
%%	{and then use $G$-convex analysis to prove existence.} \medskip 
%%
%%	{\it Keywords:} Principal-agent problem; Adverse selection; Bi-level optimization; Incentive-compatibility; Non-quasilinearity
%%\end{abstract}

\section{Introduction}\label{section:introduction}

	
Recently, N$\ddot{o}$ldeke-Samuelson (2015, \cite{NoldekeSamuelson15p}) provide a general existence result given that the consumer and product space are compact, by implementing a duality argument based on Galois Connections.  In this chapter, we explore existence using $G$-convex analysis, which is introduced in  Section \ref{section:G-convexity}, but with less restriction on boundedness of the product domain and without assuming the generalized single-crossing condition. As a result of lack of natural compactness, the proof of this result is quite different from~\cite{NoldekeSamuelson15p}. It should be mentioned here that the existence results from this chapter,  chapter \ref{chapter:existence_bounded}, and N$\ddot{o}$ldeke-Samuelson require no restrictions of the monopolist profit to take on a special form, which is a generalization from much of the literature.\medskip

In Section \ref{section:G-convexity}, we identify incentive compatibility with a $G$-convexity constraint. In this chapter, we will rewrite the maximization problem by converting the optimization variables from a product-price pair of mappings to a product-value pair. It can then be shown that the product-value pair converges under the $G$-convexity constraint. The existence result follows. \medskip



The remainder of this chapter is organized as follows. Section \ref{section:model} states the mathematical model and assumptions. Section \ref{section:preliminary} reformulates the monopolist's problem and prepares some propositions for the next section. In Section \ref{section:mainresult}, we state the existence theorem as well as the convergence proposition. %And Section \ref{section:futurework} proposes some directions of future work. 


 


\section{Model}\label{section:model}



{This model is a bilevel optimization. After a monopolist publishes its price menu, each agent maximizes his utility through the purchase of at most one product. Knowing only the distribution of agent types, the monopolist maximizes aggregate profits based on agents' choices, which are eventually based on the price menus.}


Suppose agents' preferences are given by some parametrized utility function  $G(x, y, z)$, where $x$ is a $M$-dimensional vector of consumer characteristics, $y$ is a $N$-dimensional vector of attributes of each product, and $z$ represents the price of each product. Denote {by} $X$ the space of agent types, by $Y$ the space of products, by $cl(Y)$ the closure of $Y$, by $Z$ the space of prices, and by $cl(Z)$ the closure of $Z$.  \medskip

The monopolist sells indivisible products to agents. Each agent buys at most one unit of product. We assume no competition, cooperation or communication between agents. For any given price menu $p: cl(Y) \rightarrow cl(Z)$, agent $x \in X$ knows his utility $G(x,y,p(y))$ for purchasing each product $y$ at price $p(y)$. It follows that every agent solves the following maximization problem 
\begin{equation}\label{eqn_optimal_product}
	u(x):=\max_{y \in cl(Y)} G(x, y, p(y)),
\end{equation}
where $u(x)$ represents the maximal utility agent $x$ can obtain, and $u: X \longrightarrow \R$ is also called the value function or indirect utility function.%, given utility function $G: X \times cl(Y) \times Z \rightarrow \R$.{\marginpar \color{blue} Jiaqi: - is this math necessary? - economists understand utility much better as a f(x,y,p(y)) — i.e. x,y,p(y) are inputs and plugging them into the function gives utility (#) output.. doesn’t need to be defined as  a mapping (which is more technical..) - generally, don’t need to define utility, profits, and cost functions as mappings (because they’re standard)—> is the mapping important to the math proof of your paper?}
~At this point, it is assumed that the maximum in \eqref{eqn_optimal_product} is attained for each agent $x$. \medskip

If agent $x$ purchases product $y$ at price $p(y)$, the monopolist would earn from this transaction a profit of $\pi(x,y,p(y))$%, given that $\pi: X \times cl(Y) \times Z \rightarrow \R$ 	
.  {For example, monopolist profit can take the form $\pi(x,y,p(y)) = p(y)-c(y)$, where %the function $c: Y\rightarrow \R$ represents manufacturing cost.} {\marginpar \color{blue} - sufficient to say that c(y) is a variable manufacturing cost function..  }
c(y) is a variable manufacturing cost function.} Summing over all agents in the distribution $d\mu(x)$, the monopolist's total profit is characterized by 
\begin{equation}\label{eqn_monopolist_integral}
	\Pi(p, y):=\int_{X} \pi(x, y(x), p(y(x))) d\mu(x),
\end{equation}
which depends on her price menu $p: cl(Y) \rightarrow cl(Z)$ and  agents' choices $y: X \rightarrow cl(Y)$.\footnote{It is worth mentioning that in some literature, the monopolist's objective is to design a product line $\tilde{Y}$ (i.e.~a subset of $cl(Y)$) and a price menu $\tilde{p}: \tilde{Y} \rightarrow \R$ that jointly maximize overall monopolist profit. Then, given $\tilde{Y}$ and $\tilde{p}$, an agent of type $x$ chooses the product $y(x)$ that solves
	\begin{equation*}
		\max_{y \in \tilde{Y}} G(x,y, \tilde{p}(y)):= u(x).
	\end{equation*}
Allowing price to take value $\bar{z}$ (which may be $+\infty$), and assuming Assumption \ref{assmp:Gregular} below, the effect of designing product line $\tilde{Y}$ and price menu $\tilde{p}: \tilde{Y}\rightarrow \R$ is equivalent to that of designing a price menu $p : cl(Y)\rightarrow (-\infty, +\infty]$, which equals $\tilde{p}$ on $\tilde{Y}$ and maps $cl(Y) \setminus \tilde{Y}$ to $\bar{z}$, such that no agents choose to purchase any product from $cl(Y) \setminus \tilde{Y}$, which is less attractive than the outside option $y_{\emptyset}$ according to Assumption~\ref{assmp:Gregular}. In this paper, we use the latter as the monopolist's objective.
 \vspace{0.1cm}
 
For any given price menu $p: cl(Y)\rightarrow (-\infty, +\infty]$, one can construct a mapping $y: X \rightarrow cl(Y)$ such that each $y(x)$ solves the maximization problem in \eqref{eqn_optimal_product}. But such mapping is not unique, for some fixed price menu, without the single-crossing type assumptions. %JIAQI- either use both commas, or can remove~
Therefore, we adopt in \eqref{eqn_monopolist_integral} the total profit as a functional of both price menu~$p$ and its corresponding mapping $y$.}\medskip

%where $x\in X \longmapsto y(x) \in cl(Y)$ denotes that product which agent $x$ chooses to buy, $p: cl(Y) \rightarrow Z$ represents her price menu and $d\mu(x)$ stands for the distribution of agents.\medskip

% DISCUSS/CLARIFY WITH XIANWEN:
%Since agents' individual characteristic is not observable by the monopolist who only knows its distribution
Since the monopolist only observes the overall distribution of agent attributes, and is unable to distinguish individual agent characteristics, the monopolist takes into account the following incentive-compatibility constraint when determining product-price pairs $(y, p(y))$, which ensures that no agent has incentive to pretend to be another agent type.\medskip 


In addition, we adopt a participation constraint in order to rule out the possibility of the monopolist charging exorbitant prices and the agents still having to make transactions despite this: each agent $x\in X$ will refuse to participate in the market if the maximum utility he can obtain %, $u(x)$, 
is less than his reservation value $u_{\emptyset}(x)$, where the function $u_{\emptyset}: X \rightarrow \R$ is given in the form $u_{\emptyset}(x): = G(x, y_{\emptyset}, z_{\emptyset})$, for some $(y_{\emptyset}, z_{\emptyset}) \in cl(Y \times Z)$, where $y_{\emptyset}$ represents the outside option, whose price equals to some fixed value $z_\nul \in \R$ beyond the monopolist's control.  \medskip

{For monopolist profit, some literature assumes $\pi(x, y_{\emptyset}, z_{\emptyset}) \ge  0$ for all $x\in X$ to ensure that the outside option is harmless to the monopolist. Here, it is not necessary to adopt such assumption for the sake of generality. \medskip}


Then the monopolist's problem can be described as follows:


\begin{equation}\label{origin_problem}
(P_1)
\begin{cases}
\sup \Pi(p,y)=\int_{X} \pi(x, y(x), p(y(x)))~ d\mu(x)\\
s.t.\ (y,p(y)) \text{~is incentive compatible%incentive compatibility
	};\\
s.t.\  G(x, y(x), p(y(x))) \ge u_{\emptyset}(x) \text{ for all } x \in X;\\
s.t.\ p \text{  is lower semicontinuous}.
\end{cases}
\end{equation}

We assume that $p$ is lower semicontinuous, without which the maximum in \eqref{eqn_optimal_product} may not be attained. However, in the equivalence form of $(P_0)$, this restriction will be encoded in $G$-convexity of the value functions, which will be shown in Proposition \ref{equiv_form}.\medskip


This paper makes the followings assumptions. {We use $C^0(X)$ to denote the space of all continuous functions on $X$, and use $C^1(X)$ to denote the space of all differentiable functions on $X$ whose derivative is continuous.}\medskip%, i.e. continuous differentiable functions.} \medskip


%We adopt technical hypotheses, following from Carlier (Assumptions \ref{assmp:Gcoordinate-monotone}-\ref{assmp:Gtech3}) and Trudinger (Assumptions \ref{assmp:Gregular} - \ref{assmp:Gdecreasing}) as below:\medskip


\begin{assumption}\label{assmp:Gregular}
	Agents' utility $G \in C^{1}(cl(X\times Y \times Z))$, where the space of agents $X$ is a bounded open convex subset in $\R^M$ with $C^1$ boundary%$\R_{+}^M$ with $C^1$ boundary {which does not intersect with the boundary of $\R_{+}^M$}
	, the space of products $Y \subset \R^N$%$Y \subset \R^N_{ +}$ needed by coordinate-monotonicity
	, and range of prices $Z=(\munderbar z,\bar z)$ with $-\infty < \munderbar z < \bar z \le +\infty$. {Assume $G(x,y,\bar{z}) := \lim_{z\longrightarrow \bar{z}} G(x,y,z) \le G(x, y_{\emptyset}, z_{\emptyset})$, for all $(x,y) \in X \times cl(Y)$; and assume this inequality is strict when $\bar{z} = +\infty$.}
\end{assumption}



Here we do not necessarily assume $X$, $Y$, and $Z$ are compact spaces; in particular, $Y$ and $Z$ are potentially unbounded %(i.e.\ we do not set \textit{a priori} bounds for product space or price).
(i.e.\ we do not set  \textit{a priori} bounds for product attributes or an \textit{a priori} upper bound for price). However, we do specify a lower bound for the price range, since the monopolist has no incentive to set price close to negative infinity. %{\color{} Here $\bar{z}$ is an uninteresting price to all the agents, for example $\bar{z} = +\infty$. }%, {as well as an upper bound for price range, since agents would purchase an outside product when price equals positive infinity.} 
\medskip



\begin{assumption}\label{assmp:Gdecreasing}
	$G(x,y,z)$ is strictly decreasing in $z$ %{ and $G_z(x,y,z)<0$ for each $(x,y,z)\in X \times Y \times Z$}, %FOR EACH
	for each $(x,y) \in cl(X \times Y)$.
\end{assumption}





This is to say that, the higher the price paid to the monopolist, the lower the utility that will be left for the agent, for any given product. %natural since agent's benefit is decreasing on the utility transferred to the monopolist, for the fixed product.
\medskip


\begin{assumption}\label{assmp:Gcoordinate-monotone}
	$G$ is coordinate-monotone
	~in $x$. That is, for each $(y,z)\in cl(Y\times Z)$, and for all $ (\alpha, \beta) \in X^2$, if $\alpha_i\ge \beta_i$ for all $ i=1,2,...,M$, then $G(\alpha,y,z)\ge G(\beta, y,z)$.
\end{assumption}


In Assumption \ref{assmp:Gcoordinate-monotone}, we assume that agent utility increases along each consumer attribute coordinate.\medskip



 {In the following, we use  $D_x G(x,y,z) := (\frac{\partial G}{\partial x_1}, \frac{\partial G}{\partial x_2}, \dots, \frac{\partial G}{\partial x_M})(x,y,z)$ to denote derivative with respect to $x$. For any vector in $\R^M$ or $\R^N$, we use $||\cdot||$ and $||\cdot||_{\alpha}$ to denote its Euclidean  $2$-norm and $\alpha$-norm ($\alpha \ge 1$), respectively. For example, for $x\in \R^M$, we have $||x|| = \sqrt{\sum_{i=1}^{M} x_i^2}$ and $||x||_{\alpha} = (\sum_{i=1}^{M} |x_i|^{\alpha})^{\frac{1}{\alpha}}$. 
 We use $H$ defined in section \ref{section:G-convexity} as the inverse of $G$ with respect to the third variable, i.e., for each $(x, y) \in X\times cl(Y)$, $H(x, y, \cdot)= G^{-1}(x,y,\cdot)$. Here, $H(x,y,u)$ represents the price paid by agent $x$ for product $y$ when receiving value $u$.}\medskip

In Rochet-Chon$\acute{e}$'s model, $H(x,y,u) = x\cdot y -u$ and $\pi(x,y,z) = z-C(y)$, for some superlinear cost function $C$. In this case, $\pi(x,y,H(x,y,u)) = x\cdot y -u -C(y)$. Since $C$ is superlinear, it is reasonable to assume the following:\medskip


\begin{assumption}\label{assmp:Gtech0}
	$\pi(x,y,H(x,y,u))$ is super-linearly decreasing in $y$. That is, there exist $\alpha \ge 1$, $a_1, a_2> 0$ and $b\in \R$, such that $\pi(x,y,H(x,y,u)) \le -a_1 ||y||_{\alpha}^{\alpha} - a_2 u +b$ for all $ (x, y, u)\in X\times cl(Y)\times \R$, or equivalently, $\pi(x,y,z) + a_2 G(x,y,z) \le -a_1 ||y||_{\alpha}^{\alpha}  +b$ for all $ (x, y, z)\in X\times cl(Y)\times \R$.
\end{assumption}


As shown in the alternative formulation, Assumption \ref{assmp:Gtech0} requires the existence of some weighted surplus which is super-linearly decreasing in product.\medskip


Assumptions \ref{assmp:Gtech1}-\ref{assmp:Gtech3} are some technical assumptions on $D_xG$.\medskip



\begin{assumption}\label{assmp:Gtech1}
	{$D_x G(x,y,z)$ is Lipschitz in $x$},
	uniformly in $(y,z)$, meaning there exists $k$ such that
	$||D_xG(x,y,z)-D_x G(x',y,z)||\le k||x-x'||$ %%FOR SOME $k$ AND 
	%FOR
	for all $(x, x',y, z)\in X^2\times cl(Y) \times cl(Z)$.
\end{assumption}

%---For $G(x,y,z) = \langle x, y \rangle -f(z)$, Assumption \ref{assmp:Gtech1} will be reduced to 

\begin{assumption}\label{assmp:Gtech2}
	$||D_x G(x,y,z)||_{1}$ increases sub-linearly in $y$. That is, there exist $ \beta \in (0, \alpha], c>0$, and $ d\in \R$, such that $||D_x G(x,y,z)||_{1}\le c||y||_{\beta}^{\beta} +d$ for all $ (x, y, z)\in X\times cl(Y) \times cl(Z)$.
\end{assumption}


\begin{assumption}\label{assmp:Gtech3}
	Coercivity of {$1$-norm} of $(D_xG)$. For all $ s>0$, there exists $r>0$, such that $\sum_{i=1}^{M} |D_{x_i}G(x,y,z)|\ge s$, for all $(x, y, z)\in X\times  cl(Y) \times cl(Z)$, whenever $||y||\ge r$.
\end{assumption}


{Allowing Assumption \ref{assmp:Gcoordinate-monotone}, {the derivatives $D_{x_i}G$ are always nonnegative; therefore,} we no longer need to take absolute values of $D_{x_i}G$ in the inequality of Assumption \ref{assmp:Gtech3}.} And then Assumption \ref{assmp:Gtech3} says that the marginal utility of agents who select the same product $y$ will increase to infinity as $||y||$ approaches infinity, uniformly for all agents and prices. {For instance, when $M = N$, utility $G(x,y,z) = \sum_{i=1}^{M} x_iy_i^2 -z $ satisfies Assumption~\ref{assmp:Gtech3}, because $\sum_{i=1}^{M} |D_{x_i}G(x,y,z)| = \sum_{i=1}^{M} D_{x_i}G(x,y,z) = \sum_{i=1}^{M} y_i^2 \rightarrow +\infty$ as $||y|| \rightarrow + \infty$. }\medskip




Assumptions~\ref{assmp:Pi1} states constraints on the continuity of principal's profit function $\pi$, integrability of participation constraint $u_{\emptyset}$, and absolutely continuity of measure $\mu$ with respect to the Lebesgue measure. 

%In Assumption , we assume principal's profit would increase as the selling price increases, for any fixed agent and product type. We also assume that profit will be limited for any fixed selling price, uniformly on agent-product type. \medskip

\begin{assumption}\label{assmp:Pi1}
	Profit function $\pi$ is continuous on $cl(X\times Y\times Z)$. The participation constraint $u_{\emptyset}$ is integrable with respect to $d \mu$, where the measure $\mu$ is absolutely continuous with respect to the Lebesgue measure and has $X$ as its support.
\end{assumption}



%%\begin{assumption}\label{assmp:Pi2}
%%	The summation of principal's profits and agents' utilities is uniformly bounded. %Uniform boundedness for the summation of principal's profits and agents' utilities. 
%%	There exists a constant $C_0$, such that $\pi(x, y,z) + G(x,y,z) \le C_0$, for all $(x, y, z) \in X \times  cl(Y) \times cl(Z)$.
%%\end{assumption}
%%
%%In Assumption \ref{assmp:Pi2}, we assume that if an agent $x$ purchases product $y$ at price $z$, then his utility plus principal's benefit will be less than some large number. Otherwise, there will be a sequence of agent-product-price triplets $\{ (x_n, y_n, z_n)\}$, such that the summation approaches infinity as $n$ goes to infinity.\medskip


For $\alpha \ge 1$, denote $L^{\alpha}(X)$ as the space of functions for which the $\alpha$-th power of the absolute value is Lebesgue integrable {with respect to the measure $d \mu$}. That is, a function $f: X\longrightarrow \R$ is in $L^{\alpha}(X)$ if and only if $\int_{X} |f|^{\alpha} d\mu <+\infty$. For instance, Assumption \ref{assmp:Pi1} implies $u_{\emptyset}\in L^{1}(X)$.\medskip








%{\bf Assumption 10.} Profit function $\pi(x,y,z)$ is strictly increasing with respect to the third variable $z$, for each $(x,y)\in X\times Y$.\medskip

%In Assumption 10, we assume principal's profit would increase as the selling price increases, for any fixed agent and product type. We also assume that profit will be limited for any fixed selling price, uniformly on agent-product type. \medskip





\bigskip



\section{Reformulation of the Monopolist's Problem}\label{section:preliminary}


The purpose of this section is to fix terminology and prepare the preliminaries for the main results of the next section. We also rewrite the monopolist's problem in Proposition \ref{equiv_form}, which is an equivalent form of \eqref{origin_problem}. \medskip


{We introduce implementability here, which is closely related to incentive-compatibility and can also be exhibited by $G$-convexity and $G$-subdifferential. %Note that in the following statements, no assumptions of single-crossing type are required.
	
\begin{definition}[implementability]
		A function $y: X \rightarrow cl(Y)$ is called \textit{implementable} if and only if there exists a function $z: X \rightarrow \R$  such that the pair $(y, z)$ is incentive compatible.
\end{definition}

\begin{remark}\label{rmk:implementability}
	Allowing Assumption \ref{assmp:Gdecreasing}, a function $y$ is implementable if and only if there exists a price menu $p: cl(Y) \rightarrow \R$ such that the pair $(y, p(y))$ is incentive compatible.
\end{remark}

\begin{proof}%[Proof of Remark \ref{rmk:implementability}]
	One direction is easier: given $p$ and $y$, define $z(\cdot):= p(y(\cdot))$. Then the conclusion follows directly.
	
	Given an incentive-compatible pair $(y, z): X \rightarrow cl(Y) \times \R$, we need to construct a price menu $p: cl(Y)\rightarrow \R$. If $y= y(x)$ for some $x\in X$, define $p(y):= z(x)$; for any other $y \in cl(Y)$, define $p(y) := \bar{z}$. \medskip
	
	We first show $p$ is well-defined. Suppose $y(x) = y(x')$ with $x\ne x'$, from incentive compatibility of $(p,y)$, we have $G(x,y(x), z(x)) \ge G(x, y(x'), z(x')) = G(x, y(x), z(x'))$. Since $G$ is strictly decreasing on its third variable, the above inequality implies $z(x) \le z(x')$. Similarly, one has $z(x) \ge z(x')$. Therefore, $z(x) = z(x')$ and thus $p$ is well-defined.
	
	The incentive compatibility of $(y, p(y))$ follows from that of $(y, z)$ and definition of $p$.
\end{proof}

As a corollary of Proposition \ref{incen/convex},  implementable functions can be characterized as $G$-subdifferential of $G$-convex functions. 


\begin{corollary}\label{cor:implementable}
	Given Assumption \ref{assmp:Gdecreasing}, a function $y: X \rightarrow cl(Y)$ is implementable if and only if there exists a $G$-convex function $u(\cdot)$ such that $y(x) \in \partial^G u(x)$ for each $x\in X$.
\end{corollary}


\begin{proof}%[Proof of Corollary \ref{cor:implementable}]
	One direction is immediately derived from the definition of implementability and Proposition \ref{incen/convex}.
	
	Suppose there exists some convex function $u$ such that $y(x) \in \partial^G u(x)$ for each X. Define $z(\cdot):= H(\cdot, y(\cdot), u(\cdot))$, then $u(x) = G(x, y(x), z(x))$.
	Proposition \ref{incen/convex} implies $(y, z)$ is incentive compatible, and thus $y$ is implementable.
\end{proof}
}

When parameterization of preferences is linear in agent types and price, Corollary \ref{cor:implementable} says that a function is implementable if and only if it is monotone increasing. In general quasilinear cases, this coincides with Proposition 1 of Carlier \cite{Carlier01}. \medskip


From the original monopolist's problem \eqref{origin_problem}, we replace product-price pair $(p,y)$ by the value-product pair $(u,y)$, using $u(\cdot) = G(\cdot, y(\cdot), p(y(\cdot)))$. %, since $G$ is strictly monotone in the third variable. 
Combining this with Proposition \ref{incen/convex}, the incentive-compatibility constraint $(y,p(y))$ is equivalent to $G$-convexity of $u(\cdot)$ and $y(x) \in \partial^G u(x)$ for all $x\in X$. Therefore, one can rewrite the monopolist's problem as follows.

\begin{proposition}\label{equiv_form}
	
	Given Assumptions \ref{assmp:Gregular} and \ref{assmp:Gdecreasing}, the monopolist's problem $(P_1)$ is equivalent to
	
	\begin{equation}\label{Principal_new_problem}
	(P_2)
	\begin{cases}
	\sup \tilde{\Pi}(u,y):=\int_{X} \pi(x, y(x), H(x,y(x), u(x))) d\mu(x)\\
	s.t.\ $u$ \text{ is } $G$\text{-convex };\\
	s.t.\ y(x) \in \partial^G u(x) \text{ and }u(x)\ge u_{\emptyset}(x) \text{ for all } x \in X.
	\end{cases}
	\end{equation}
\end{proposition}


\begin{proof} %[Proof of Proposition \ref{equiv_form}] 
	
	We need to prove both directions for equivalence of $(P_1)$ and $(P_2)$.
	
	1. For any incentive-compatible pair $(y, p(y))$, define $u(\cdot) := G(\cdot,y(\cdot), p(y(\cdot)))$. Then by Proposition \ref{incen/convex}, we have $u(\cdot)$ is $G$-convex and $y(x) \in \partial^G u(x)$ for all $x \in X$. From the participation constraint, $G(x, y(x), p(y(x))) \ge u_{\emptyset}(x)$ for all $x\in X$. This implies $u(x)\ge u_{\emptyset}(x)$ for all $x\in X$. Besides, two integrands are equal: $\pi(x, y(x), p(y(x))) = \pi(x,y(x), H(x,y(x), u(x)))$. Therefore, $(P_1) \le (P_2)$.
	
	2. On the other hand, assume $u(\cdot)$ is $G$-convex, $y(x)\in \partial^G u(x)$ and $u(x) \ge u_{\emptyset}(x)$ for all $x \in X$. From Corollary \ref{cor:implementable} and Remark \ref{rmk:implementability}, we know $y$ is implementable and there exists a price menu $p: cl(Y) \rightarrow \R$, such that the pair $(y, p(y))$ is incentive compatible, where $p(y) = H(x,y(x), u(x))$ for $y = y(x) \in y(X) :=\{ y(x) \in cl(Y) | x \in X \}$; $p(y) = \bar{z}$ for other $y\in cl(Y)$. Firstly, the mapping $p$ is well-defined, using the same argument as that in Remark \ref{rmk:implementability}. %Then by Proposition \ref{incen/convex}, $(y, p(y))$ is incentive compatible. 
	Secondly, the participation constraint holds since $G(x,y(x), p(y(x))) = u(x) \ge u_{\emptyset}(x)$ for all $x\in X$. 
	
	Thirdly, let us show this price menu $p$ is lower semicontinuous. Let $\tilde{p}$ be the restriction of $p$ on $y(X)$. Suppose that $\{y_k \} \subset y(X)$ converges $y_0 \in y(X)$ with $y_k = y(x_k)$ and $y_0 = y(x_0)$, satisfying $\lim\limits_{k \rightarrow \infty} \tilde{p}(y_k) = \liminf\limits_{y \rightarrow y_{0}} \tilde{p}(y)$.  Let $z_{k}:= \tilde{p}(y_k)$ and $z_{\infty}:=\lim\limits_{k \rightarrow \infty} z_k$. To prove lower semicontinuity of $\tilde{p}$, we need to show $\tilde{p}(y_0)\le z_{\infty}$.  Since $y_k \in \partial^G u(x_k)$, we have $u(x) \ge G(x,y_k, H(x_k, y_k, u(x_k))) = G(x, y_k, z_k)$. Taking $k\rightarrow \infty$, we have $u(x)\ge G(x, y_0, z_{\infty})$. This implies $G(x_0, y_0, \tilde{p}(y_0)) = u(x_0) \ge G(x_0, y_0, z_{\infty})$. By Assumption \ref{assmp:Gdecreasing}, we know $\tilde{p}(y_0) \le z_{\infty}$. Thus $\tilde{p}$ is lower semicontinuous. Since $p$ is an extension of $\tilde{p}$ from $y(X)$ to $cl(Y)$ as its lower semicontinuous hull, satisfying $v(y)= \bar{z}$ for all $y\in cl(Y)\setminus y(X)$, we know $p$ is also lower semicontinuous.
	
	Lastly, two integrands are equal: $\pi(x, y(x), p(y(x))) = \pi(x,y(x), H(x,y(x), u(x)))$. Therefore, $(P_1) \ge (P_2)$.
\end{proof}

\medskip
In the next section, we will show the existence result of the rewritten monopolist's problem $(P_2)$ given in \eqref{Principal_new_problem}. For preparation of the main result, we introduce the following lemma and propositions.\medskip

Proposition \ref{lemma_continuity} shows that the inverse function of $G$ is also continuous, because $G$ is continuous and monotonic on the price variable.\medskip

\begin{proposition}\label{lemma_continuity}
	Given Assumption \ref{assmp:Gregular} and Assumption \ref{assmp:Gdecreasing}, function $H$ is continuous.
\end{proposition}

\begin{proof}%[Proof of Proposition \ref{lemma_continuity}]
	(Proof by contradiction). Suppose $H$ is not continuous, then there exists a sequence ${(x_n, y_n, z_n)} \subset cl(X\times Y \times Z)$ converging to $(x, y, z)$ and $\varepsilon >0$ such that $|H(x_n, y_n, z_n) - H(x,y,z)|>\varepsilon$ for all $n\in \N$. Without loss of generality, we assume $H(x_n, y_n, z_n) - H(x,y,z)>\varepsilon$ for all $n\in \N$. Therefore, we have $H(x_n, y_n, z_n) > H(x,y,z)+\varepsilon$. By Assumption \ref{assmp:Gdecreasing}, this implies $z_n < G(x_n, y_n, H(x,y,z)+\varepsilon)$ for all $n\in \N$. Taking limit $n\rightarrow \infty$ at both sides, since $G$ is continuous from Assumption \ref{assmp:Gregular}, we have $z \le G(x, y, H(x,y,z)+\varepsilon)$. It implies $H(x,y,z) \ge H(x,y,z)+\varepsilon$. Contradiction!
\end{proof}

Given coordinate monotonicity of $G$ in the first variable, one can show that all the $G$-convex functions are nondecreasing. Therefore, the value functions are also monotonic with respect to agents' attributes.\medskip

\begin{proposition}\label{nondecreasing}
	Given Assumption \ref{assmp:Gcoordinate-monotone}, $G$-convex functions are nondecreasing in coordinates.
\end{proposition}

\begin{proof}[Proof of Proposition \ref{nondecreasing}]
	Let $u$ be any $G$-convex function, and let $\alpha$, $\beta$ be any two agent types in $X$ with $\alpha \ge \beta$. By $G$-convexity of $u$, for this $\beta$, there exist $y\in cl(Y)$ and $z \in cl(Z)$, such that $u(\beta)=G(\beta, y,z)$ and $u(x)\ge G(x, y,z)$, for any $x\in X$. Since $\alpha \ge \beta$, by Assumption \ref{assmp:Gcoordinate-monotone}, we have $G(\alpha, y,z)\ge G(\beta,y,z)$. Combining with $u(\alpha)\ge G(\alpha, y,z)$ and $u(\beta) = G(\beta,y,z)$, one has $u(\alpha) \ge u(\beta)$. Thus, $u$ is nondecreasing.
\end{proof}

Proposition \ref{Subdiff/Bdd} presents that uniform boundedness of agents' value functions on some compact subset implies uniform boundedness of corresponding agents' choices of their favorite products.\medskip

\begin{proposition}\label{Subdiff/Bdd}
	Given Assumptions \ref{assmp:Gregular}, \ref{assmp:Gdecreasing}, \ref{assmp:Gcoordinate-monotone}, \ref{assmp:Gtech3}, and let $u(\cdot)$ be a $G$-convex function on $X$, $\omega$ be a compact subset of $X$, $\delta>0, R>0$, satisfying $\omega+\delta\overline{B(0,1)}\subset X$ and $|u(x)|\le R$ for all $x\in \omega + \delta \overline{B(0,1)}$ (here, $\overline{B(0,1)}$ denotes the closed unit ball of $\R^M$). Then, there exists $T = T(\omega,\delta, R) > 0$, such that $||y||\le T$ for any $x \in \omega$ and any $y\in \partial^Gu(x)$.
\end{proposition}

\begin{proof}%[Proof of Proposition \ref{Subdiff/Bdd}]
	(Proof by contradiction).\medskip
	By Assumption \ref{assmp:Gcoordinate-monotone} and Assumption \ref{assmp:Gtech3}, for $s=\frac{4R\sqrt{M}}{\delta}$, there exists $r>0$, such that for any $(x, y, z)\in X \times  cl(Y) \times cl(Z)$, whenever $||y||\ge r$, we have $\sum\limits_{i=1}^{M}D_{x_i}G(x,y,z)\ge \frac{4R\sqrt{M}}{\delta}$.\medskip
	Assume the boundedness conclusion of this proposition is not true. Then for this $r$, there exist $ x_0 \in \omega$ and  $ y_0\in \partial^G u(x_0)$, such that $||y_0||\ge r$. Thus,
	\begin{equation}\label{eqn_coercivity}
	\sum\limits_{i=1}^{M}D_{x_i}G(x,y_0,z)\ge \frac{4R\sqrt{M}}{\delta}, \ \ \ \ \ \text{ for all } x \in X, z \in \R.
	\end{equation}
	Since $y_0 \in \partial^G u(x_0)$, by definition of $G$-subdifferential, we have $u(x)\ge G(x,y_0,H(x_0,y_0,u(x_0)))$,  for any $ x \in X$. Take $x=x_0+\delta x_{-1}$, where $x_{-1}:=(\frac{1}{\sqrt{M}}, \frac{1}{\sqrt{M}}, \cdots, \frac{1}{\sqrt{M}})$ is a unit vector in $\R^M$ with each coordinate equal to $\frac{1}{\sqrt{M}}$. Then 
	\begin{equation}\label{eqn_prop3.6}
	u(x_0+\delta x_{-1})\ge G(x_0+\delta x_{-1},y_0,H(x_0,y_0,u(x_0))).
	\end{equation}
	For any $x \in \omega+ \delta \overline{B(0,1)}$, from conditions in the proposition, we have $||u(x)||\le R$. Therefore, 
	\begin{flalign*}
	2R &\ge |u(x_0+\delta x_{-1})|+|u(x_0)|&&\\
	&\ge |u(x_0+\delta x_{-1})- u(x_0)|&& \text{(By Triangular Inequality)}\\
	&\ge u(x_0+\delta x_{-1}) - u(x_0)&& \\
	&\ge G(x_0+\delta x_{-1}, y_0, H(x_0,y_0,u(x_0)))&& \text{(By Inequality \eqref{eqn_prop3.6})  and (By definition }\\
	& -G(x_0, y_0, H(x_0, y_0, u(x_0))) && \text{of $H$, $u(x_0) = G(x_0, y_0, H(x_0, y_0, u(x_0)))$}\\
	&= \int_{0}^{1}\delta \langle x_{-1},  D_{x}G(x_0+t\delta x_{-1}, y_0, H(x_0,y_0,u(x_0)))\rangle dt&& \text{(By Fundamental Theorem of Calculus)}\\
	&= \frac{\delta}{\sqrt{M}}\int_{0}^{1} \sum\limits_{i=1}^{M}D_{x_i}G(x_0+t\delta x_{-1}, y_0, H(x_0, y_0, u(x_0))) dt&&\\
	&\ge \frac{\delta}{\sqrt{M}}\int_{0}^{1}\frac{4R\sqrt{M}}{\delta}dt&& \text{(By Inequality \eqref{eqn_coercivity})}\\
	&= \frac{\delta}{\sqrt{M}}\cdot\frac{4R\sqrt{M}}{\delta}&&\\
	&= 4R&&
	\end{flalign*}
	Contradiction!
	Thus, our assumption is wrong. The boundedness conclusion of this proposition is true. That is, there exists $T>0$, such that for any $x \in \omega$, $y \in \partial^G u(x)$, one has $||y||\le T$. In addition, here $T = T(\omega, \delta, R)$ is independent of $u$. In fact, from the above argument, we can see that $T \le r$, which does not depend on $u$.
\end{proof}



The above two propositions will also be employed in the proof of Proposition \ref{proposition:convergence}. 

\bigskip







\section{Main result}\label{section:mainresult}
In this section, we state the existence theorem, the proof of which is provided in the end of this section.


\begin{theorem}[Existence]
	Under Assumptions \ref{assmp:Gregular} - \ref{assmp:Pi1}, assume $\mu$ is equivalent to the Lebesgue measure on $X$, then the monopolist's problem $(P_2)$ admits at least one solution.
\end{theorem}

Technically, in order to demonstrate existence, we start from a sequence of value-product pairs, whose total profits have a limit that is equal to the supreme of $(P_2)$.
Then we need to show that this sequence converges, up to a subsequence, to a pair of limit mappings. Then we show this limit value-product pair satisfies the constraints of $(P_2)$; and its corresponding total payoff is better (or no worse) than those of any other admissible pairs. \medskip


In the following, we denote $W^{1,1}(X)$ as the Sobolev space of $L^1$ functions whose first derivatives exist in the weak sense and belong to $L^1(X)$. For more properties of Sobolev spaces and weak derivatives, see Evans~\cite[Chapter 5]{Evans98}. If $\omega$ is some open subset of $X$, notation $\omega \subset \subset X$ means the closure of $\omega$ is also included in $X$.\medskip

Lemma \ref{lemma1} provides sequence convergence results of convex functions, which are uniformly bounded in Sobolev spaces on open convex subsets. We state this classical result without proof, which can be found in Carlier \cite{Carlier01}.\medskip


\begin{lemma}\label{lemma1}
	Let $\{u_n\}$ be a sequence of convex functions in $X$ such that, for every open convex set $\omega \subset \subset X$, the following holds:
	\begin{equation*}
	\sup\limits_{n} ||u_n||_{W^{1,1}(\omega)} < +\infty
	\end{equation*}
	Then, there exists a function $u^*$ %.. $\bar{u}$ means indifference curve / specific utility level} 
	which is convex in $X$, a measurable subset $A$ of $X$ and a subsequence again labeled $\{u_n\}$ such that\\
	1. $\{u_n\}$ converges to $u^*$ uniformly on compact subsets of $X$;\\
	2. $\{\nabla u_n\}$ converges to $\nabla u^*$ pointwise in $A$ and $dim_{H}(X\setminus A)\le M-1$, where $dim_{H}(X\setminus A)$ is the Hausdorff dimension of $X\setminus A$.
\end{lemma}

{We extend the above convergence result to $G$-convex functions in the following proposition, which is required in the proof of the Existence Theorem, as it extracts a limit function from a converging sequence of value functions.}

\begin{proposition}\label{proposition:convergence}
	Assume Assumptions \ref{assmp:Gregular}, \ref{assmp:Gdecreasing}, \ref{assmp:Gcoordinate-monotone}, \ref{assmp:Gtech1}, \ref{assmp:Gtech3}, and let $\{u_n\}$ be a sequence of $G$-convex functions in $X$ such that for every open convex set $\omega \subset \subset X$, the following holds:
	\begin{equation*}
	\sup\limits_{n} ||u_n||_{W^{1,1}(\omega )} < +\infty
	\end{equation*}
	Then there exists a function $u^*$ which is  $G$-convex in $X$, a measurable subset $A$ of $X$, and a subsequence again labeled $\{u_n\}$ such that\\
	1. $\{u_n\}$ converges to $u^*$ uniformly on compact subsets of $X$;\\
	2. $\{\nabla u_n\}$ converges to $\nabla u^*$ pointwise in $A$ and $dim_{H}(X\setminus A)\le M-1$.
\end{proposition}

\begin{proof}%[Proof of Proposition \ref{proposition:convergence}]
	
	In this proof, we will show that the sequence of $G$-convex functions is convergent by applying results from Lemma \ref{lemma1}, then prove that the limit function is also $G$-convex. 
	Assume $\{u_n\}$ is a sequence of $G$-convex functions in $X$ such that for every open convex set $\omega \subset \subset X$, the following holds:
	\begin{equation*}
	\sup\limits_{n} ||u_n||_{W^{1,1}(\omega )} < +\infty.
	\end{equation*}
	
	{\bf Step 1:} By Assumption \ref{assmp:Gtech1}, there exists $k>0$, such that for any $(x, x')\in X^2$, $y\in cl(Y)$ and $z\in cl(Z)$, one has $||D_xG(x,y,z)-D_x G(x',y,z)||\le k||x-x'||$. Denote $G_{\lambda}(x,y,z) := G(x,y,z)+\lambda||x||^2$, where $\lambda \ge \frac{1}{2}\Lip(D_xG)$, with $\Lip(D_xG)
	:=\sup\limits_{\{(x,x',y,z)\in X\times X\times  cl(Y) \times cl(Z):~x \neq x'\}} \frac{||D_xG(x,y,z)-D_x G(x',y,z)||}{||x-x'||}$. %\le k$.
	
	Then, for any $(x, x')\in X^2$, by Cauchy–Schwarz inequality, one has 
	\begin{flalign*}
	& \langle D_xG_{\lambda}(x,y,z)-D_x G_{\lambda}(x',y,z) , x-x'\rangle &&\\
	= & ~\langle D_xG(x,y,z)-D_x G(x',y,z) , x-x'\rangle + 2\lambda ||x-x'||^2 && \text{(By Defition of $G_{\lambda}(x,y,z)$)}\\
	\ge & ~-||D_xG(x,y,z)-D_x G(x',y,z)|| ||x-x'||+ 2\lambda ||x-x'||^2 &&\text{(By Cauchy–Schwarz Inequality)}\\
	\ge & ~[2\lambda - \Lip(D_xG)]||x-x'||^2 &&\text{(By Definition of $\Lip(D_xG)$)}\\
	\ge & ~0.&&
	\end{flalign*} 
	
	Thus, $G_{\lambda}(\cdot, y, z)$ is a convex function on $X$, for any fixed $(y, z) \in cl(Y) \times cl(Z).$\medskip
	
	{\bf Step 2:}	Since $u_n$ is $G$-convex, by Lemma \ref{convex-subdiff0}, we know $$u_n(x) = \max\limits_{x'\in X, y\in \partial^G u_n(x')} G(x,y,H(x',y,u_n(x'))).$$ 
	Define $v_n(x):= u_n(x) +\lambda ||x||^2$. Then 
	\begin{flalign*}
	v_n(x) =& \max\limits_{x'\in X, y\in \partial^G u_n(x')}G(x,y,H(x',y,u_n(x'))) +\lambda ||x||^2 \\
	=& \max\limits_{x'\in X, y\in \partial^G u_n(x')}(G(x,y,H(x',y,u_n(x'))) +\lambda ||x||^2)\\
	=& \max\limits_{x'\in X, y\in \partial^G u_n(x')} G_{\lambda}(x,y,H(x',y,u_n(x'))).	
	\end{flalign*}
	
	
	Since $G_{\lambda}(\cdot,y,H(x',y,u_n(x')))$ is convex for each $(x', y)$, we have $v_n(x)$, as supremum of convex functions, is also convex, for all $n \in \N$.\medskip
	
	
	
	{\bf Step 3:}	Since $v_n:= u_n +\lambda||x||^2$ and $\sup\limits_{n}||u_n||_{W^{1,1}(\omega)} < +\infty$, one has $\sup\limits_{n}||v_n||_{W^{1,1}(\omega)} < +\infty$, for any $\omega \subset \subset X$. Hence $\{v_n\}$ satisfies all the assumptions of Lemma \ref{lemma1}. So, by conclusion of Lemma \ref{lemma1}, there exists a convex function $v^*$ in $X$ and a measurable set $A \subset X$, such that $dim (X \setminus A)\le M-1$ and up to a subsequence, $\{v_n\}$ converges to $v^*$ uniformly on compact subset of $X$ and $(\nabla v_n)$ converges to $\nabla v^*$ pointwise in A.
	
	Let $u^*(x):=v^*(x)-\lambda||x||^2$, then  $(u_n)$ converges to $u^*$ uniformly on compact subset of $X$ and $(\nabla u_n)$ converges to $\nabla u^*$ pointwise in A.\medskip
	
	{\bf Step 4:}	Finally, let us prove that $u^*$ is $G$-convex.\medskip
	Define $\Gamma(x):=\cap_{i\ge 1}\overline{\cup_{n\ge i}\partial^G u_n(x)}$, for all $x\in X$.\medskip

	
	{\bf	Step 4.1.} Claim: For any $x'\in X$, we have $\Gamma(x') \neq \emptyset$.
	
	Proof of this Claim: 
	
	{\bf Step 4.1.1.} Let us first show for any $\omega \subset\subset X$, $\sup\limits_{n}||u_n||_{L^{\infty}(\bar{\omega})}<+\infty$.
	
	If not, then there exits a sequence $\{x_n\}_{n=1}^{\infty}\subset \bar{\omega}$, such that $\limsup\limits_{n}|u_n(x_n)|=+\infty$.
	
	Since $\bar{\omega}$ is compact, there exists $\bar{x}\in \bar{\omega}$, such that, up to a subsequence, $x_n\rightarrow \bar{x}$. Again up to a subsequence, we may assume that $u_n(x_n)\rightarrow +\infty$.
	
	Since $\bar{x} \in \bar{\omega} \subset \subset X$, there exists $\delta >0$, such that $\bar{x}+\delta x_{-1} \in X$, where $x_{-1}:=(\frac{1}{\sqrt{M}}, \frac{1}{\sqrt{M}}, \cdots, \frac{1}{\sqrt{M}})$ is a unit vector in $\R^M$ with each coordinate equal to $\frac{1}{\sqrt{M}}$. For any $x>\bar{x} + \delta x_{-1}$, there exists $n_0$, such that for any $n>n_0$, we have $x>x_n$. By Proposition \ref{nondecreasing}, $u_n$ are nondecreasing, and thus
	\begin{equation}\label{eqn_integral}
	\int_{\{x\in X, x>\bar{x} +\delta x_{-1}\}} u_n(x)dx \ge m\{x\in X, x> \bar{x}+\delta x_{-1}\} u_n(x_n)\rightarrow +\infty
	\end{equation}
	where $m\{x\in X, x>\bar{x}+\delta x_{-1}\}$ denotes Lebesgue measure of $\{x\in X, x>\bar{x}+\delta x_{-1}\}$, which is positive.
	
	Therefore, we have $||u_n||_{W^{1,1}(\omega')} \ge ||u_n||_{L^{1}(\omega')} \ge \int_{\omega'} u_n(x) dx \rightarrow +\infty$. This implies $\sup\limits_{n} ||u_n||_{W^{1,1}(\omega')} = +\infty$.
	
	On the other hand, denote $\omega' := \{x\in X|~ x>\bar{x}+\delta x_{-1}\}$, then $\omega' = X \cap \{x\in \R^M|~ x> \bar{x}+\delta x_{-1} \}$. Since both $X$ and $\{x\in \R^M|~ x> \bar{x}+\delta x_{-1} \}$ are open and convex, we have $\omega'$ is also open and convex. Therefore, by assumption, we have $\sup\limits_{n} ||u_n||_{W^{1,1}(\omega')} < +\infty.$ 
	
	Contradiction! Thus, for any $\omega \subset \subset X$, we have $\sup\limits_{n}||u_n||_{L^{\infty}(\bar{\omega})}<+\infty$.
	
	{\bf Step 4.1.2.} For any fixed $x'\in X$, there exists an open set $\omega \subset \subset X$ and $\delta>0$, such that $x'\in \omega$ and $\omega + \delta \overline{B(0,1)} \subset \subset X$.
	
	From Step 4.1.1, we know  $\sup\limits_{n}||u_n||_{L^{\infty}(\omega + \delta \overline{B(0,1)})} < +\infty$. 
	So there exists $R>0$, such that for all $n\in \N$, we have $|u_n(x)|\le R$, for all $x \in \omega + \delta \overline{B(0,1)}$. Since $u_n$ are $G$-convex functions, by Proposition~\ref{Subdiff/Bdd}, there exists $T = T(\omega, \delta, R) >0$, independent of $n$, such that $||y||\le T$, for any $y \in \partial^G u_n(x')$ and any $n\in \N$. Thus, there exists a sequence $\{ y_n \}$, such that $y_n \in \partial^G u_n(x')$ and $||y_n||\le T$, for all $n\in \N$.
	
	By compactness theorem for sequence $\{ y_n \}$,  there exists $y'$, such that, up to a subsequence, $y_n \rightarrow y'$. Thus, we have $y' \in \overline{\cup_{n\ge i}\partial^G u_n(x')}$, for all $i\in \N$. It implies $y' \in \cap_{i\ge 1} \overline{\cup_{n\ge i}\partial^G u_n(x')} = \Gamma (x')$. 
	
	Therefore $\Gamma(x') \neq \emptyset$, for all $x' \in X$.\medskip
	

	
	{\bf Step 4.2.} Now for any fixed $x\in X$, and any $y\in \Gamma(x)$, by Cantor's diagonal argument, there exists $\{y_{n_k}\}_{k=1}^{\infty}$, such that $y_{n_k} \in \partial^G u_{n_k}(x)$ and $\lim\limits_{k\rightarrow \infty} y_{n_k} = y$.
	For any $k\in \N$, by definition of $G$-subdifferentiability,
	$u_{n_k}(x')\ge G(x', y_{n_k}, H(x, y_{n_k}, u_{n_k}(x)))$, for any $x' \in X$. Take limit $k \rightarrow \infty$ at both sides, we get $u^*(x') \ge G(x', y, H(x, y, u^*(x)))$, for any $x'\in X$. Here we use the fact that both functions $G$ and
	$H$ are continuous by Assumption \ref{assmp:Gregular} and Proposition \ref{lemma_continuity}.
	%since $G$ is continuous and strictly decreasing with respect to its third variable.} 
	Then by definition of $G$-subdifferentiability, the above inequality implies $y\in \partial ^G u^*(x)$. 
	
	So $\partial^G u^*(x)\neq \emptyset$, for any $x\in X$, which means $u^*$ is G-subdifferentiable everywhere. By Lemma \ref{convex-subdiff0}, $u^*$ is $G$-convex.
\end{proof}

Lastly, we show the proof of the main theorem.\medskip

\begin{proof}[Proof of the Existence Theorem]
	{\bf Step 1:} 	 Define $\Phi_u: x \longmapsto argmin_{\partial^G u(x)} -\pi(x, \cdot, H(x,\cdot,u(x)))$, then by Proposition \ref{Subdiff/Bdd}, the measurable section theorem (cf. \cite[Theorem 1.2, Chapter VIII]{EkelandTemam76}) and Lusin Theorem, one has $\Phi_u$ admits measurable selections. \medskip
	
	Let $\{(u_n, y_n)\}$ be a maximizing sequence of $(P_2)$, where maps $u_n: X\rightarrow \R$ and $y_n: X\rightarrow cl(Y)$, for all $n\in \N$. Without loss of generality, we may assume that for all $n$, $y_n(\cdot)$ is measurable and $y_n(x) \in \Phi_{u_n}(x)$, for each $x\in X$. Starting from $\{(u_n, y_n)\}$, we would find an value-product pair $(u^*, y^*)$ satisfying all the constraints in \eqref{Principal_new_problem}, and show that it is actually a maximizer.\medskip
	
	{\bf Step 2:} From Assumption \ref{assmp:Gtech0}, there exist $\alpha \ge 1$, $a_1, a_2> 0$ and $b\in \R$,  such that for each $x\in X$ and $n \in \N$,
	\begin{flalign*}
	a_1 ||y_n(x)||_{\alpha}^{\alpha} \le & -\pi(x,y_n(x),H(x, y_n(x), u_n(x))) - a_2 u_n(x) +b \\
	\le &  -\pi(x,y_n(x),H(x, y_n(x), u_n(x)))- a_2 u_{\emptyset}(x) + b,
	\end{flalign*}
	%\begin{flalign*}
	%	a ||y_n(x)||_{\alpha}^{\alpha}	 \le & G(x,y_n(x),H(x,y_n(x),u_n(x))) -b \\
	%	\le & C_0 - \pi(x,y_n(x),H(x,y_n(x),u_n(x))) -b, 
	%\end{flalign*}
	where the second inequality comes from $u_n\ge u_{\emptyset}$. Together with Assumption \ref{assmp:Pi1}, this implies $\{y_n\}$ is bounded in $L^{\alpha}(X)$.\medskip
	

	By participation constraint and Assumption \ref{assmp:Gtech0}, we know 
	\begin{equation*}
	u_{\emptyset}(x) \le u_n(x) = G(x,y_n(x),H(x,y_n(x),u_n(x))) \le \frac{1}{a_2}(b - \pi(x,y_n(x),H(x,y_n(x),u_n(x)))).
	\end{equation*}

	Together with Assumption \ref{assmp:Pi1}, we know $\{u_n\}$ is bounded in $L^1(X)$.\medskip

	By $G$-subdifferentiability, $Du_n(x) = D_x G(x, y_n(x), H(x,y_n(x),u_n(x)))$. By Assumption \ref{assmp:Gtech2}, we have $||Du_n||_{1}\le c||y_n||_{\beta}^{\beta}+d \le c(N+||y_n||_{\alpha}^{\alpha})+d$. The last inequality holds because $\beta \in (0, \alpha]$. Because $X$ is bounded and $\{y_n\}$ is bounded in $L^{\alpha}(X)$, we know $\{Du_n\}$ is bounded in $L^1(X)$.\medskip
	
	Since both $\{u_n\}$ and $\{Du_n\}$ are bounded in $L^1(X)$, one has $\{u_n\}$ is bounded in $W^{1,1}(X)$. By Proposition \ref{proposition:convergence}, there exists a $G$-convex function $u^*$ on $X$, such that, up to a subsequence, $\{u_n\}$ converges to $u^*$ in $L^1$ and uniformly on compact subset of $X$, and $\nabla u_n$ converges to $\nabla u^*$ almost everywhere.\medskip
	
	{\bf Step 3: } Denote $y^*(x)$ as a measurable selection of $\Phi_{u^*}$. Let us show $(u^*,y^*)$ is a maximizer of the principal's program $(P_2)$. \medskip
	
	{\bf Step 3.1: }By Assumption \ref{assmp:Gtech0}, for all $x$, $y_n(x)$ and $u_n(x)$,
	%there exists some constant $C_0$, such that for all $x\in X$ and $n\in \N$, we have $\pi(x,y_n(x),H(x,y_n(x), u_n(x)))+G(x,y_n(x),H(x,y_n(x), u_n(x)))\le C_0$. Therefore, 
	one has
	\begin{flalign*}
	&-\pi(x,y_n(x),H(x,y_n(x), u_n(x)))\\
	\ge & \ a_2 G(x,y_n(x),H(x,y_n(x), u_n(x))) -b \\
	=&\  a_2 u_n(x) - b \\
	\ge&\ a_2 u_{\emptyset}(x) - b.
	\end{flalign*}

 
	By Assumption \ref{assmp:Pi1}, $u_{\emptyset}$ is measurable, thus one can apply Fatou's Lemma and get
	%we have $\pi(x, y_n(x), H(x, y_n(x), u_n(x))) \le  C_0$, for all .
	
	\begin{align}\label{3}
	\begin{split}
	\sup \tilde{\Pi}(u,y) & = \limsup\limits_{n} \tilde{\Pi}(u_n, y_n) \\
	&= -\liminf\limits_{n} \int_{X} - \pi(x, y_n(x), H(x,y_n(x),u_n(x)))  ~d\mu(x)\\
	& \le - \int_{X} \liminf\limits_{n} - \pi(x, y_n(x), H(x,y_n(x),u_n(x)))~ d\mu(x). \\
	\end{split}
	\end{align}
	
	Let $\gamma(x):=\liminf\limits_{n} - \pi(x, y_n(x), H(x,y_n(x),u_n(x)))$. 
 For each $x\in X$, by extracting a subsequence of $\{y_{n} \}$, which is denoted as $\{y_{n_x}\}$, we assume $\gamma(x) = \lim\limits_{n_x} - \pi(x, y_{n_x}(x), H(x,y_{n_x}(x),u_{n_x}(x)))$. \medskip
	
	{\bf Step 3.2: } For any fixed $x \in X$, since $u_{n_x}$ are $G$-convex functions and $\{u_{n_x}\}$ is bounded in $L^1(X)$% and bounded from below by a $L^1$ function
	, by Proposition \ref{nondecreasing}, it is also bounded in $L_{loc}^{\infty}(X)$. 	Then by Proposition \ref{Subdiff/Bdd}, $\{y_{n_x}\}$ is also bounded in $L_{loc}^{\infty}(X)$ . Thus there exists a subsequence of $\{y_{n_x}(x)\}$, again denoted as $\{y_{n_x}(x)\}$, that converges. Denote $\tilde{y}$ a mapping on $X$ such that $y_{n_x}(x) \rightarrow \tilde{y}(x)$.\medskip
	
	Since $\pi$ and $H$ are continuous, we have $ \gamma(x)= - \pi(x, \tilde{y}(x), H(x,\tilde{y}(x),u^*(x)))$.\medskip
	
	For each fixed $x\in X$, since $u_{n_x}$ are $G$-convex and $y_{n_x}(x) \in \partial^G u_{n_x}(x)$, for any $x' \in X$, we have $$u_{n_x}(x')\ge G(x', y_{n_x}(x),H(x,y_{n_x}(x),u_{n_x}(x))).$$ 
	Take limit $n_x \rightarrow +\infty$ at both sides, we get $u^*(x')\ge G(x', \tilde{y}(x),H(x,\tilde{y}(x),u^*(x)))$, for any $x'\in X$. By definition of $G$-subdifferentiability, we have $\tilde{y}(x)\in \partial^Gu^*(x)$. \medskip


	{\bf Step 3.3: } By definition of $y^*$, one has $$ -\pi(x, y^*(x), H(x,y^*(x),u^*(x)))\le   -\pi(x, \tilde{y}(x), H(x,\tilde{y}(x),u^*(x))) = \gamma(x)$$.
	
	So, together with \eqref{3}, we know 
	\begin{equation}\label{minimizer}
	\sup \tilde{\Pi}(u,y) \le - \int_{X}  \gamma(x) d\mu(x) \le - \int_{X}  - \pi(x, y^*(x), H(x,y^*(x),u^*(x))) d\mu(x) = \tilde{\Pi}(u^*,y^*).
	\end{equation}
	
	
	Since $\{u_n\}$ converges to $u^*$, and $u_n(x)\ge u_{\emptyset}(x)$ for all $n\in \N$ and $x \in X$, we have $u^*(x)\ge u_{\emptyset}(x)$ for all $x \in X$. In addition, because $u^*$ is $G$-convex and $y^*(x) \in \partial^G u^*(x)$, we know $(u^*, y^*)$ satisfies all the constraints in \eqref{Principal_new_problem}. Together with \eqref{minimizer}, we proved $(u^*,y^*)$ is a solution of the principal's program.
\end{proof}



%\section{Future Work}\label{section:futurework}
%
%We have strong interest in giving an explicit solution for the non-quasilinear example on the real line and in high dimension. We also would like to investigate, among others, the conditions under which the matching map $y: X \rightarrow cl(Y)$ is continuous and/or differentiable. 







%in general whether the participation constraint is binding or not.

%For continuity of solutions and the matching map, if there exist, other characteristics of the matching will be invested.






%Another interesting topic is to discover $n$-dimentional ($n\ge 2$) explicit solutions for nonlinear utilities.\medskip





\chapter{Existence: bounded product spaces}\label{chapter:existence_bounded}

\section{Introduction}


	In this chapter, we will first state the hypotheses that will be need for this and most of the following chapters. The purpose of section \ref{section:Hypotheses} is to fix terminology for the main results of the following chapters.\medskip
	
	In section \ref{section:ExistenceBounded}, we will reformulate the principal's program in the language of $G$-convexity and $G$-subdifferentiability, state and prove the existence theorem, where the product space is bounded.  \medskip



\section{Hypotheses}\label{section:Hypotheses}

For notational convenience, we adopt the following technical hypotheses, inspired by those of 
Trudinger~\cite{Trudinger14} and Figalli-Kim-McCann \cite{FigalliKimMcCann11}. 

The following hypotheses will be relevant:  \Gone-\Gthree\ represent partial analogs of the twist, domain convexity,
and non-negative cross-curvature hypotheses from the quasilinear setting \cite{FigalliKimMcCann11} \cite{Loeper09};
\Gfour\ encodes a form of the desirability of money to each agent, while \Gfive\ quantifies the assertion that the maximum
price $\bar z$ is high enough that no agent prefers paying it for any product $y$ to the outside option.

\begin{itemize}
	\item[\Gzero] $G \in C^{1}(cl(X\times Y \times Z))$, where $X\subset \R^m, Y \subset \R^n$ are open and bounded and $Z=(\underbar z,\bar z)$ with $-\infty <\underbar z < \bar z \le +\infty$.
	
	\item[\Gone] For each $x \in X$, the map $(y,z) \in cl( Y \times Z) \longmapsto (G_x, G)(x,y,z)$ is a homeomorphism onto its range;
	
	\item[\Gtwo] its range $(cl( Y \times Z))_x := (G_x,G)(x,cl(Y \times Z)) \subset \R^{m+1}$ is convex.
\end{itemize}

For each $x_0\in X$ and $(y_0, z_0),(y_1,z_1) \in cl( Y \times Z)$, 
define $(y_t, z_t)\in cl( Y \times Z)$ such that the following equation holds:
\begin{flalign}\label{$G$-segment}
\begin{split}
(G_x, G)(x_0,y_t,z_t) = (1-t)(G_x, G)(x_0,y_0,z_0) + t (G_x, G)(x_0,y_1,z_1),\\ \text{ for each $t\in [0,1].$}
\end{split}
\end{flalign}
By \Gone \ and \Gtwo , $(y_t, z_t)$ is uniquely determined by (\ref{$G$-segment}). 
We call $t \in [0,1] \longmapsto (x_0,y_t,z_t)$ the  $G$-segment connecting $(x_0, y_0, z_0)$ and $(x_0, y_1, z_1)$.

\begin{itemize}
	\item[\Gthree] For each $x,x_0 \in X$, assume $t \in [0,1] \longmapsto G(x, y_t, z_t)$ is convex along all $G$-segments ($\ref{$G$-segment}$).
	
	\item[\Gfour]  For each $(x,y,z) \in X \times cl(Y)\times cl(Z)$, assume $G_{z}(x,y,z)<0$.  
	
	\item[\Gfive] $\pi\in C^0(cl(X\times Y \times Z))$ and $u_\nul (x) := G(x,y_\nul,z_\nul)$ for some fixed
	$(y_\nul,z_\nul) \in cl(Y \times Z)$  satisfying
	\begin{equation*}
	\ \ \ \ \ \ \ \ \ \ 		G(x,y,\bar z)  := \lim_{z \to \bar z}G(x,y,z) \le G(x,y_\nul,z_\nul) \text{ for all}\ (x,y) \in X \times cl (Y).
	\end{equation*}
	When $\bar z = +\infty$ assume this inequality is strict, and moreover that $z$ sufficiently large implies
	\begin{equation*}
	G(x,y,z) < G(x,y_\nul,z_\nul) \text{ for all}\ (x,y) \in X \times cl(Y).
	\end{equation*}
\end{itemize}

For each $u \in \R$, \Gfour \ allows us to define $H(x,y,u) := z$ 
if  $G(x,y,z) = u$, i.e. $H(x, y, \cdot)= G^{-1}(x,y,\cdot)$.




\section{Reformulation of the principal's program, Existence theorem}\label{section:ExistenceBounded}



In this section,
we'll reformulate the 
principal's program using $u$ as a proxy for the prices $v$ controlled
by the principal,  thus generalizing Carlier's approach \cite{Carlier01} to the non-quasilinear setting. Moreover, the agent's indirect utility $u$ and product selling price $v$ are $G$-dual to each other in the sense of \cite{Trudinger14}.\medskip

%\begin{definition}[$G$-convexity]\label{$G$-convexity}
%	A function $u\in C^0(X)$ is called $G$-convex if for each $x_0 \in X$, there exists $y_0 \in cl(Y)$, and $z_0 \in cl(Z)$ such that $u(x_0)=G(x_0, y_0, z_0)$, and $u(x)\ge G(x, y_0, z_0),\mbox{ for all } x\in X$.
%	\end{definition}
%	
	

		
%		While $G$-convexity acts as a generalized notion of convexity, the $G$-subdifferentiability defined in \ref{defn:GSubdifferential} generalizes the concept of differentiability/subdifferentiability. 
		
%		\begin{definition}[$G$-subdifferentiability]
%			The $G$-subdifferential of a function $u(x)$ is defined by
%			\begin{equation*}
%			\partial^G u(x):= \{ y\in cl(Y)\mid u(x')\ge G(x',y, H(x,y,u(x))), \mbox{ for all } x'\in X\}.
%			\end{equation*}
%			
%			A function $u$ is said to be $G$-subdifferentiable at $x$ if and only if $\partial^G u(x) \neq \emptyset$.
%			\end{definition}
%			
%			In \cite{Trudinger14}, this point-to-set map $\partial^G u$ is also called $G$-$normal$ mapping. For more properties related to $G$-convexity, see \cite{Trudinger14}.

{\color{blue}			
%			Lemma \ref{convex-subdiff} shows an equivalent relationship between $G$-convexity and $G$-subdifferentiability, a special case of which is that a function is convex if and only if it is subdifferentiable on (the interior of) its domain. 
%			
%			\begin{lemma}[$G$-subdifferentiability characterizes $G$-convexity]
%				\label{convex-subdiff}
%				Allowing hypothesis \Gfour, a function $u: X \longrightarrow \R$ is $G$-convex if and only if it is $G$-subdifferentiable everywhere.
%				\end{lemma}
			
}
					
	We now show each $G$-convex function defined in Definition \ref{defn:GConvexity} can be achieved by 
	some price menu $v$,  and conversely each price menu yields a $G$-convex indirect utility \cite{Trudinger14}.
	We require either \Gfive\ or \eqref{repulsed}, which asserts
	all agents are repelled by the maximum price,  and insensitive to which contract they receive at that price.\medskip
					
					
\begin{proposition}[Duality between prices and indirect utilities]\label{Prop:Gtransform}
	Assume \Gzero\ and \Gfour. {\rm (a)} If 
	\begin{equation}\label{repulsed}
	\begin{split}
		G(x,y,\bar{z}) := \lim_{z \to \bar z}G(x,y,z) = \inf_{(\tilde{y}, \tilde{z})\in cl(Y\times Z)} G(x, \tilde{y}, \tilde{z}), \\
		\text{ for all } (x,y)\in X\times cl(Y),
	\end{split}
	\end{equation} 
	then a function $u \in C^0(X)$ is $G$-convex if and only if there exist a lower semicontinuous 
	$v: cl(Y) \longrightarrow cl(Z)$ such that $u(x) = \max_{y\in cl(Y)} G(x,y,v(y))$.
	{\rm (b)} If instead of \eqref{repulsed} we assume \Gfive,
	then $u_\nul \le u \in C^0(X)$ is $G$-convex if and only if there exists a lower semicontinuous function $v: cl(Y) \longrightarrow cl(Z)$ with $v(y_\nul) \le z_\nul$ such that $u(x) = \max_{y\in cl(Y)} G(x,y,v(y))$.
\end{proposition}
						
\begin{proof}
	1. Suppose $u$ is $G$-convex. Then for any agent type $x_0\in X$, there exists a product and price
	$(y_0,z_0) \in cl(Y \times Z)$, such that $u(x_0)=G(x_0,y_0,z_0)$ and $u(x)\ge G(x,y_0,z_0)$, for all $x\in X$. 
							
	Let $A:=\cup_{x\in X} \partial^{G} u(x)$ denote the corresponding set of products.          
	For $y_0\in A$, define $v(y_0) = z_0$, where $z_0\in cl(Z)$ and $x_0\in X$ satisfy $u(x_0)=G(x_0,y_0,z_0)$ and $ u(x)\ge G(x, y_0,z_0)$ for all $x\in X$.
	We shall shortly show this makes $v:A \longrightarrow cl(Z)$ (i) well-defined and (ii) lower semicontinuous. Taking (i)  for granted,  our construction yields
	\begin{equation}\label{restricted1}
	u(x)  =
		\max_{y \in A} G(x,y,v(y)) \qquad \forall x \in X.
	\end{equation}
							
	(i) Now for $y_0 \in A$, suppose there exist $(x_0, z_0), (x_1, z_1)\in X\times cl(Z)$ with 
	$z_0 \ne z_1$, such that $u(x_i) = G(x_i, y_0, z_i)$ and $u(x) \ge G(x, y_0, z_i)$ for all $x\in X$ and $i=0,1$.  Without loss of generality, assume $z_0< z_1$. By \Gfour, we know $u(x_1) = G(x_1, y_0, z_1)<G(x_1, y_0, z_0)$, contradicting $u(x)\ge G(x,y_0,z_0)$, for all $x\in X$.   
	Having shown $v:A \longrightarrow cl(Z)$ is well-defined, we now show it is lower semicontinuous.
							
							
	(ii) Suppose $\{y_{k}\}\subset A$ converges to $y_0 \in A$ and 
		$z_\infty:= \lim\limits_{k \to \infty} v(y_k) = \liminf\limits_{y \rightarrow y_0}v(y)$.  We need to show 
		$v(y_0) \le z_\infty$.  Letting $z_k := v(y_k)$ for each $k$,  there exists $x_k \in X$ such that 
		\begin{equation}\label{sequence3.6}
			u(x) \ge G( x, y_k,z_k) \qquad {\forall} x \in X \text{ and}\ k =0,1,2,\ldots,
		\end{equation}
	with equality holding at $x=x_k$.  In case (b) we deduce $z_\infty < \infty$ from 
		\begin{equation*}
			G(x_k,y_k,z_k) =u(x_k)  \ge G(x_k,y_\nul,z_\nul)	
		\end{equation*}
	and \Gfive.
	Taking $k \to \infty$,  \Gzero\ (or \eqref{repulsed}  in case (a) when $z_\infty=+\infty$)
	implies
	\begin{equation}\label{limit3.6}
		u(x) \ge G(x,y_0,z_\infty) \qquad {\forall} x \in X.
	\end{equation}
	Applying \Gfour\ to $G(x_0,y_0,z_0)=u(x_0)\ge G(x_0,y_0,z_\infty)$ yields 
	the desired semicontinuity: $z_0 \le z_\infty$.
							
	{(iii)} We extend $v$ from $A$ to $cl(Y)$ by taking its lower semicontinuous hull;
			this does not change the values of $v$ on $A$, but satisfies $v(y_0) := \bar z$  on $y_0 \notin cl(A)$.
			We now 	show this choice of price menu $v$ yields \eqref{1}. 
			Recall for each $x \in X$,  there exists $(y_0,z_0) \in cl(Y \times Z)$ such that
				\begin{equation*}
				u(x) = G(x,y_0,z_0) \ge (u_\nul(x) := G(x,y_\nul,z_\nul) \ge) \sup_{y \in cl(Y)\setminus cl(A)} G(x,y,v(y)),
				\end{equation*}
			in view of \eqref{repulsed} (or \Gfive), and the fact that $v(y) = \bar z$ for each $y$ outside $cl(A)$.
			Thus to establish \eqref{1},  we need only show that \eqref{restricted1} remains true when
			the domain of the maximum is enlarged from $A$ to $cl(A)$.
			Since we have chosen the {\em largest} lower semicontinuous extension of $v$ outside of $A$,
			each $y_0 \in cl(A) \setminus A$ is approximated by a sequence  $\{y_{k}\}\subset A$
			for which $z_k := v(y_k)$ converges to $z_\infty :=v(y_0)$. As before, \eqref{sequence3.6} holds
			and implies \eqref{limit3.6}, showing \eqref{restricted1} indeed remains true when
			the domain of the maximum is enlarged from $A$ to $cl(A)$, and establishing \eqref{1}.
			Finally, if $v(y_\nul) > z_\nul$ in case (b) then \Gfour\ yields $u(x) \ge u_\nul(x) > G(x,y_\nul,v(y_\nul))$, and
			we may redefine $v(y_\nul):=z_\nul$ without violating either \eqref{1} or the lower semicontinuity of $v$.\medskip
							
	2. Conversely, suppose there exist a lower semicontinuous function $v: cl(Y)\longrightarrow cl(Z)$, such that $u(x)=\max_{y\in {cl(Y)}} G(x, y,v(y))$. Then for any $x_0\in X$, there exists $y_0 \in cl(Y)$, such that $u(x_0) = G(x_0, y_0, v(y_0))$. Let $z_0:= v(y_0)$, then $u(x_0)= G(x_0, y_0, z_0)$, and for all $x\in X$, $u(x)\ge G(x, y_0, z_0)$. By definition, $u$ is $G$-convex.   If $v(y_\nul) \le z_\nul$ then $u(\cdot) \ge G(\cdot ,y_\nul,v(y_\nul)) \ge u_\nul(\cdot)$ 
	by \eqref{1} and \Gfour.
\end{proof}
							
							
\begin{remark}[Optimal agent strategies] Assume \Gzero\ and \Gfour.
	When $\bar z< \infty$,  lower semicontinuity of $v:cl(Y) \longrightarrow cl(Z)$ is enough to ensure the maximum 
	\eqref{1} is attained.  However, when $\bar z=+\infty$ we can reach the same conclusion either by assuming the limit \eqref{repulsed} converges uniformly with respect to  $y \in cl(Y)$,  or else by assuming $v(y_\nul) \le z_\nul$ and \Gfive.
\end{remark}
								
								
\begin{proof}
	For any fixed $x \in X$ let $u(x) = \sup\limits_{y\in cl(Y)} G(x,y,v(y))$. We will show that the maximum is attained. Since $cl(Y)$ is compact, suppose $\{y_{k}\}\subset cl(Y)$ converges to $y_0 \in cl(Y)$,
	$z_\infty:= \limsup\limits_{k \to \infty} v(y_k)$ and $u(x) = \lim\limits_{k\to \infty} G(x, y_k,  v(y_k))$. By extracting subsequence of $\{y_k\}$ and relabelling, without loss of generality, assume $\lim\limits_{k\to \infty} v(y_k) = z_{\infty}$. 
								
	1.  If $z_{\infty}< \bar{z}$ then lower semicontinuity of $v$ yields $v(y_0)\le z_{\infty} < +\infty$. By \Gfour, one has
		\begin{equation}
		\begin{split}
			G(x, y_0, v(y_0)) \ge & ~G(x, y_0, z_{\infty}) = \lim\limits_{k\to \infty} G(x, y_k,  v(y_k))\\
			= & ~u(x) = \sup\limits_{y\in cl(Y)} G(x,y,v(y)).
		\end{split}
		\end{equation}
	Therefore, the maximum is attained by $y_0$.\medskip
									
	2. If $z_{\infty} = \bar{z}$ then $\lim\limits_{k\to \infty} v(y_k) = \bar{z} = + \infty$.
									
		2.1. By assuming the limit \eqref{repulsed} converges uniformly with respect to  $y \in cl(Y)$, we have 
			\begin{equation*}
			\begin{split}
				\inf_{(\tilde{y}, \tilde{z})\in cl(Y\times Z)} G(x, \tilde{y}, \tilde{z}) = &~ G(x,y_0,\bar{z}) = \lim_{k \to \infty} G(x,y_k, v(y_k)) \\
				=&~ u(x) = \sup \limits_{y\in cl(Y)} G(x,y,v(y)).
			\end{split}
			\end{equation*}
		In this case, the maximum is attained by $y_0$. 
									
		2.2. By assuming \Gfive, for sufficient large $k$, we have $G(x, y_k, v(y_k)) < G(x, y_{\emptyset}, z_{\emptyset})$. Taking $k \to \infty$, by $v(y_{\emptyset}) \le z_{\emptyset}$  and \Gfour, one has
			\begin{equation*}
			\begin{split}
				\sup\limits_{y\in cl(Y)} G(x,y,v(y)) = &~u(x) = \lim_{k \to \infty} G(x,y_k, v(y_k)) \\
				\le &~ G(x, y_{\emptyset}, z_{\emptyset}) \le G(x, y_{\emptyset}, v(y_{\emptyset})).
			\end{split}
			\end{equation*}
	Thus, the maximum is attained by $y_{\emptyset}$.
\end{proof}
									
									
	From the definition of $G$-convexity, we know if $u$ is a $G$-convex function, for any $x \in X$ where $u$ happens
	to be differentiable,  denoted $x\in \dom Du$, there exists $y\in cl(Y)$ and $z\in cl(Z)$ such that
	\begin{equation}\label{EqnInverse}
	u(x)= G(x, y, z),\ \ \   Du(x) = D_x G(x, y, z).
	\end{equation}
	Conversely, when (\ref{EqnInverse}) holds, one can identify $(y, z) \in cl( Y \times Z)$ in terms of $u(x)$ and $Du(x)$, according to Condition \Gone. We denote it as 
	\begin{equation*}
	\bar{y}_G (x,u(x),Du(x)) := (y_G, z_G)(x,u(x),Du(x)),
	\end{equation*} 
	and drop the subscript
	$G$ when it is clear from context. 
	Under our hypotheses, $\bar y_G$ is a continuous function 
	{ on the relevant domain of 
		definition.\footnote{Namely $(id_X, G, G_x)(\{(x,y,z) \in cl(X \times Y \times Z) \mid G(x,y,z) \ge G(x,y_\nul,z_\nul)\})$.}}
	It will often prove convenient to augment the types $x$ and $y$ with an extra real variable;
	here and later we use the notation $\bar x \in \R^{m+1}$ and $\bar y \in \R^{n+1}$ to signify this augmentation.
	In addition, the set $X\setminus \dom Du$ has Lebesgue measure zero, which will be shown in the proof of Theorem \ref{Thm:Existence}.	\medskip
									
									
									
									
									
The following proposition not only reformulates the principal's problem, but manifests the existence of maximizer(s). 
Besides chapter \ref{chapter:existence} 
%of this thesis (Zhang~\cite{Zhang}) 
which relaxes relative compactness of the domain, for other existence results guaranteeing this supremum is attained in the non-quasilinear setting, 
see N\" oldeke-Samuelson \cite{NoldekeSamuelson15p} who require mere continuity of the direct utility $G$.\medskip
									

\begin{theorem}[Reformulating the principal's program using the agents' indirect utilities]\label{Thm:Existence}
	Assume hypotheses \Gzero-\Gone ~and \Gfour-\Gfive, 
	$\bar z < +\infty$  and $\mu \ll \mathcal{L}^m$. Setting  $$\tilde{\Pi}(u,y)=\int_{X} \pi(x, y(x), H(x,y(x), u(x))) d\mu(x),$$
	the principal's problem $(P_0)$ is equivalent to
		\begin{equation*}
			(P_3)
			\begin{cases}
				\max \tilde{\Pi}(u,y) \\
				\text{\rm among }
				$G$\text{\rm-convex  $u(x) \ge u_\nul(x)$ with }
				y(x) \in \partial^G u(x)\ \text{\rm for all } x \in X.
			\end{cases}
		\end{equation*}
	This maximum is attained. Moreover, $u$ determines $y(x)$ uniquely for a.e. $x \in X$.
\end{theorem}
										
\begin{proof}
	1. Proposition \ref{Prop:Gtransform} encodes a bijective correspondence between
lower semicontinuous price menus $v:cl(Y) \longrightarrow cl(Z)$ with $v(y_\nul) \le z_\nul$
and $G$-convex indirect utilities $u \ge u_\nul$; it also shows \eqref{1} is attained.
Fix a $G$-convex $u \ge u_\nul$ and the corresponding price menu
$v$.  For each $x \in X$ let $y(x)$ denote the point achieving the maximum \eqref{1}, so that 
$u(x) = G(x,y(x),z(x))$ with $z(x):=v(y(x)) = H(x,y(x),u(x))$ and $\Pi(v,y) = \tilde \Pi(u, y)$.
From \eqref{1} we see
\begin{equation}\label{TIEGsub}
	G(\cdot, y(\cdot),v\circ y(\cdot)) = u(\cdot) \ge G(\cdot, y(x),H(x,y(x),u(x))),
\end{equation}
so that $y(x) \in \partial^Gu(x)$.  Apart from the measurability established below,
Proposition \ref{incen/convex} asserts incentive compatibility of $(y,v \circ y)$,
while $u \ge u_\nul$ shows individual rationality, so $(P_3)\le(P_0)$.  \medskip
											
	2. The reverse inequality begins with a lower semicontinuous price menu $v:cl(Y) \longrightarrow cl(Z)$ with 
$v(y_\nul) \le z_\nul$ and an incentive compatible, individually rational map $(y,v \circ y)$ on $X$. 
Proposition \ref{incen/convex}
then asserts $G$-convexity of $u(\cdot) :=G(\cdot, y(\cdot),v(y(\cdot)))$ and that $y(x) \in \partial^Gu(x)$ for each 
$x \in X$.  Choosing $\cdot = x$ in the corresponding inequality \eqref{TIEGsub}
produces equality, whence \Gfour\ implies
$v(y(x)) = H(x,y(x),u(x))$ and $\Pi(v,y) = \tilde \Pi(u, y)$.  Since $u \ge u_\nul$ follows from individual 
rationality, we have established equivalence of $(P_3)$ to $(P_0)$.  Let us now argue the supremum $(P_3)$ is attained.\medskip
							
							
	3. Let us first show  $\pi(x, y(x), H(x,y(x), u(x)))$ is measurable on $X$ for all $G$-convex $u$ and $y(x) \in \partial^G u(x)$.
											
	By \Gzero, we know $G$ is Lipschitz, i.e., there exists $L>0$, such that $|G(x_1, y_1, z_1)-G(x_2, y_2, z_2)|< L ||(x_1-x_2,y_1-y_2, z_1-z_2)||$, for all $(x_1,y_1,z_1),$ $(x_2, y_2, z_2) \in cl(X\times Y\times Z)$.
	Since $u$ is $G$-convex, for any $x_1, x_2 \in X$, there exist $(y_1, z_1), (y_2, z_2) \in cl( Y \times Z)$, such that $u(x_i) = G(x_i,y_i,z_i)$, for $i=1,2$. Therefore, 
	\begin{equation*}
	\begin{split}
	&u(x_1)-u(x_2) \ge G(x_1, y_2, z_2) - G(x_2, y_2, z_2) > -L ||x_1-x_2||;\\
	& 	u(x_1)-u(x_2) \le G(x_1, y_1, z_1) - G(x_2, y_1, z_1) < L ||x_1-x_2||.
	\end{split}
	\end{equation*}
	That is to say, $u$ is also Lipschitz with Lipschitz constant $L$. By Rademacher's theorem and $\mu \ll \mathcal{L}^m$, we have $\mu(X\setminus \dom Du) =\mathcal{L}^m(X\setminus \dom Du) =0$.  Moreover, since $u$ is continuous, $\frac{\partial u(x)}{\partial x_j} = \lim\limits_{h\rightarrow 0} \frac{u(x+he_j)-u(x)}{h}$ is measurable on $\dom Du$, for $j=1,2,..., m$, where $e_j=(0,...0, 1, 0, ...,0)$ is the unit vector in $\R^m$ with $j$-th coordinate nonzero. Thus, $Du$ is also Borel on $\dom Du$.
	
	Since $y(x) \in \partial^G u(x) $, for all $x\in \dom Du$, we have
	\begin{equation}\label{EqnInverse2}
	\begin{split}
		u(x) &= G(x, y(x), H(x, y(x), u(x))), \\
		Du(x) &= D_xG(x, y(x), H(x, y(x), u(x))).
	\end{split}
	\end{equation}
	By \Gone, there exists a continuous function $\yG$, such that 
	\begin{equation*}
		y(x) = \yG(x, u(x), Du(x)).
	\end{equation*} 
	Thus $y(x)$ is Borel on $\dom Du$, which implies $\pi(x, y(x), H(x,y(x), u(x)))$ is measurable on $X$, given $\pi \in C^0(cl(X\times Y\times Z))$ and $\mu \ll \mathcal{L}^m$. Here we use the fact that $H$ is also continuous since $G$ is continuous and strictly decreasing with respect	to its third variable.\medskip
											
											
	4. To show the supremum is attained,
	let $\{u_k\}_{k\in \NN}$ be a sequence of %\linebreak
	$G$-convex functions, $u_k(x) \ge u_{\emptyset}(x)$ and  $y_k(x) \in \partial^G u_k(x)$ for any $x\in X$ and $k \in \NN$,  such that $\lim_{k\rightarrow \infty} \tilde{\Pi}(u_k, y_k) = \sup \tilde{\Pi}(u,y)$, among all feasible $(u,y)$. Below we construct a feasible pair $(u_{\infty}, y_{\infty})$ 
	attaining the maximum.
											
	4.1. Claim: There exists $M>0$, such that $|u(x)|<M$, for any $G$-convex $u$ and any $x \in X$. Thus $\{u_k\}_{k \in \NN}$ is uniformly bounded.
											
		Proof: Since $u$ is $G$-convex, for any $x\in X$, there exists $(y, z)\in cl( Y \times Z) $, such that $u(x) = G(x,y,z)$. Notice that $G$ is bounded, since $G$ is continuous on a compact set. Thus, there exists $M>0$, such that $|u(x)| =| G(x,y,z)|<M$ is also bounded.
											
	4.2. From part 1, we know $\{u_k\}_{k\in \NN}$ are uniformly Lipschitz with Lipschitz constant $L$, thus $\{u_k\}_{k\in \NN}$ are uniformly equicontinuous.
											
	4.3. By Arzel\`a-Ascoli theorem, there exists a subsequence of $\{u_k\}_{k\in \NN}$, again denoted as $\{u_k\}_{k\in \NN}$, and $u_{\infty} : X \longrightarrow \R$ such that $\{u_k\}_{k\in \NN}$  converges uniformly to $u_{\infty}$ on $X$.
											
	4.4. Claim: $u_{\infty}$ is also Lipschitz.
											
		Proof: For any $\varepsilon >0$, any $x_1, x_2 \in X$, since $\{u_k\}_{k\in \NN}$ converges to $u_{\infty}$ uniformly, there exist $K>0$, such that for any $k >K$, we have $|u_k(x_i) -u_{\infty}(x_i)|<\varepsilon$, for $i=1,2$. Therefore, 
			\begin{equation*}
			\begin{split}
			&~|u_{\infty}(x_1) -u_{\infty}(x_2)| \\
			\le & ~|u_k(x_1)-u_{\infty}(x_1)| + |u_k(x_2)-u_{\infty}(x_2)| + |u_k(x_1) -u_k(x_2)| \\
			< & ~2\varepsilon + L ||x_1-x_2||.
			\end{split}
			\end{equation*}
		Since the above inequality is true for all $\varepsilon >0$, thus $u_{\infty}$ is also Lipschitz.
											
	4.5. For any $x\in X$, since $u_k(x)\ge u_{\emptyset}(x)$ and $\lim\limits_{k\rightarrow \infty} u_{k}(x) = u_{\infty}(x)$, we have $u_{\infty}(x)\ge u_{\emptyset}(x)$. Therefore, $u_{\infty}$ satisfies the participation constraint.
											
	4.6. For any fixed $x\in X$, since $\{y_k(x)\}_{k \in \NN} \subset cl(Y)$ which is compact, there exists a subsequence $\{y_{k_l}(x)\}_{l\in \NN}$ which converges. Define $y_{\infty}(x):= \lim\limits_{l\rightarrow \infty} y_{k_l}(x) \in cl(Y)$.
	For each $l\in \NN$, because  $y_{k_l}(x)\in \partial^G u_{k_l} (x)$, by definition, we have $u_{k_l}(x_0)\ge G(x_0, y_{k_l}(x), H(x, y_{k_l}(x), u_{k_l}(x)))$, for any $x_0\in X$. This implies, for all $x_0\in X$, we have
		\begin{equation*}
		\begin{split}
			u_{\infty}(x_0) = \lim\limits_{l \rightarrow \infty} u_{k_l}(x_0) &~\ge \lim_{l\rightarrow \infty}  G(x_0, y_{k_l}(x), H(x, y_{k_l}(x), u_{k_l}(x)))\\
			&~\ge G(x_0, y_{\infty}(x), H(x, y_{\infty}(x), u_{\infty}(x))).
		\end{split}
		\end{equation*} 
	Thus, $y_{\infty}(x) \in \partial^G u_{\infty}(x)$. 
											
	Therefore, $\partial^G u_{\infty}(x) \ne \emptyset$, for any $x\in X$. By Lemma \ref{convex-subdiff0}, this implies $u_{\infty}$ is $G$-convex.
											
	At this point, we have found a feasible pair $(u_{\infty}, y_{\infty})$, satisfying all the constraints in $(P_3)$.
											
	4.7. Claim: For any $x\in \dom Du_{\infty}$, the sequence $\{y_k(x)\}_{k\in \NN} \subset cl(Y)$ converges to $y_{\infty}(x)$.
											
		Proof: Since $u_{\infty}$ is Lipschitz, by Rademacher's theorem, $u_{\infty}$ is differentiable almost everywhere in $X$, i.e. $\mu(X\setminus \dom Du_{\infty}) = \mathcal{L}^m(X\setminus \dom Du_{\infty}) =0$.
											
		For any $x\in \dom Du_{\infty}$ and any $\tilde{y}\in \partial^G u_{\infty}(x)$, we have 
			\begin{equation*}
				\tilde{y}(x) = \yG(x, u_{\infty}(x), Du_{\infty}(x)),
			\end{equation*}
		according to equation $(\ref{EqnInverse2})$ and hypothesis \Gone. This implies $ \partial^G u_{\infty}(x)$ is a singleton for each $x\in \dom Du_{\infty}$, i.e. $\partial^G u_{\infty}(x) =\{y_{\infty}(x)\}$.
											
		For any $x\in \dom Du_{\infty}$, by similar argument to that above in part 4.6, we can show that any (other) accumulation points of $\{y_k(x)\}_{k\in \NN}$ are elements in the set $\partial^G u_{\infty}(x)=\{y_{\infty}(x)\}$, i.e. the sequence $\{y_k(x)\}_{k\in \NN}$ converges to $y_{\infty}(x)$.
											
	4.8. Finally, since $\mu \ll \mathcal{L}^m$, by Fatou's lemma, we have 
		\begin{flalign*}
			\tilde{\Pi}(u_{\infty}, y_{\infty}) = \int_{X} \pi(x, y_{\infty}(x), H(x,y_{\infty}(x), u_{\infty}(x))) d\mu(x)  
		\end{flalign*}
		\begin{flalign*}
			\hspace{2.65cm}	&=\int_{X} \limsup\limits_{k\rightarrow \infty}\pi(x, y_k(x), H(x,y_k(x), u_k(x))) d\mu(x) 
			\\
			&\ge \limsup\limits_{k\rightarrow \infty} \int_{X} \pi(x, y_k(x), H(x,y_k(x), u_k(x))) d\mu(x) 
			\\
			& = \lim\limits_{k\rightarrow \infty} \tilde{\Pi}(u_k, y_k)\\
			&= \sup \tilde{\Pi}(u,y),
		\end{flalign*}
	among all feasible (u,y). Thus, the supremum is attained.
\end{proof}
											
\begin{remark}[More Singular measures]
	If $G \in C^2$ (uniformly in $z \in Z$) the same conclusions extend to $\mu$ which need not be absolutely
	continuous with respect to Lebesgue,  provide $\mu$ vanishes on all hypersurfaces parameterized 
	locally as a difference of convex functions \cite{FigalliKimMcCann11} \cite{Gigli11},  essentially because $G$-convexity then implies semiconvexity of $u$.  On the other hand,  apart from its final sentence,  
	the proposition extends to all probability measures $\mu$ if $G$ is merely continuous, according to 
	N\"oldeke-Samuelson \cite{NoldekeSamuelson15p}.  Our argument is simpler than
	theirs on one point however:  
	Borel measurability of $y(x)$ on $\dom Du$ follows automatically from $(G0)-(G1)$;
	in the absence of these extra hypotheses,  they are required to make a measurable selection from among each
	agent's preferred products to define $y(x)$.
\end{remark}
												
\begin{remark}[Tie-breaking rules for singular measures] 
	When an agent $x$ finds more than one product which maximize his utility, in order to reduce the ambiguity, it is convenient to assume the principal has satisfactory persuasion to  convince the agent to choose one of those products which maximize the principal's profit. 
	According to equation (\ref{EqnInverse}) and condition $(G1)$, this scenario would occur only for $x\in X \setminus \dom Du $, which has Lebesgue measure zero. Thus this convention has no effect for absolutely continuous measures, 
	but can be used as in Figalli-Kim-McCann \cite{FigalliKimMcCann11} to extend our result to singular measures.
\end{remark}
													
													
													
													
													


\chapter{Convexity}\label{chapter:convexity}


%%\begin{abstract}
%%A monopolist wishes to maximize her profits by finding an optimal price policy. 
%%After she announces a menu of products and prices, each agent $x$ will choose to buy that product $y(x)$ which maximizes his own utility, if positive. 
%%The principal's profits are the sum of the net earnings produced by each product sold.  
%%These are determined by the costs of production and the distribution of products sold, which in turn are based on the distribution of anonymous agents and the choices they make in response to the principal's price menu. 
%%In this paper, we provide a necessary and sufficient condition for the convexity or concavity of the principal's (bilevel) optimization problem, assuming each agent's disutility is a strictly increasing but not necessarily affine (i.e.\ quasilinear) function of the price paid. 
%%Concavity when present, makes the problem more amenable to computational and theoretical analysis; it is key to obtaining uniqueness and stability results for the principal's strategy in particular. 
%%Even in the quasilinear case, our analysis goes beyond previous work by addressing convexity as well as concavity, by establishing conditions which are not only sufficient but necessary, and by requiring fewer hypotheses on the agents' preferences.
%%\end{abstract}



\section{Introduction}\label{secttion:intro}

In this chapter, we will show concavity and uniqueness results of the principal's problem, under the settings in section \ref{section:Hypotheses}.
\medskip
%Most of hypotheses we need in this chapter is listed in section \ref{section:Hypotheses}. 

In section \ref{section:concavity}, we will first rewrite the principal's problem as \eqref{eqn:principalFinal}, then state the equivalent condition to convexity of the functional domain $\mathcal{U}_\nul$. Then we will show a variety of necessary and sufficient conditions for concavity (and convexity) of the principal's problem, and the resulting uniqueness of her optimal strategy.  

\medskip


%
%Section \ref{section:4thorder} gives a differential criterion for the crucial hypotheses --- \Gthree\ of the next section --- clarifying its relation to that of Figalli, Kim \& McCann \cite{FigalliKimMcCann11}, and the Ma-Trudinger-Wang criteria for regularity of optimal maps \cite{MaTrudingerWang05} which inspired it. 


In section \ref{section:privateInformation},  we assume the monopolist's utility does not depend on the agent's private information, which in certain circumstances allows us to provide a necessary and sufficient condition for concavity of her profit functional.
\medskip



\medskip


\section{Concavity and Convexity Results}\label{section:concavity}
The advantage of the reformulation from Section \ref{section:ExistenceBounded} is to make the principal's objective $\pmb \Pi$ 
depend on a scalar function $u$ instead of a vector field $y$.
By \Gone, the optimal choice $y(x)$ of   Lebesgue almost every agent $x\in X$
is uniquely determined by $u$.  Recall that $\bar{y}_G(x, u(x), Du(x))$ is the unique solution $(y,z)$ of the system ($\ref{EqnInverse}$), for any $x\in \dom Du$. 
Then the principal's problem $(P_3)$ can be rewritten as maximizing a functional 
depending only on the agents' indirect utility $u$:
\begin{equation}\label{eqn:principalFinal}
	(P_4) \ \ \ \ \ \  \max\limits_{ u\ge u_\nul \atop \text{$u$ is $G$-convex}} \pmb\Pi(u) := \max\limits_{u \ge u_\nul \atop \text{$u$ is $G$-convex}} \int_X \pi(x, \bar{y}_G(x, u(x), Du(x)))  d \mu(x) .
\end{equation}


Define $\mathcal{U}:=\{u:X\longrightarrow \R \mid  u \text{ is } G\text{-convex}\}$
and $\mathcal{U}_{\emptyset}:=\{u \in \mathcal{U}\mid  u\ge u_{\emptyset}\}$. 
Then the problem becomes to maximize $\pmb\Pi$ on $\mathcal{U}_\nul$. 
In this section, we give conditions under which the function space $\mathcal{U}_\nul$ 
is convex and the functional $\pmb\Pi$ is concave, often strictly.
Uniqueness and stability of the principal's maximizing strategy follow from strict concavity as in \cite{FigalliKimMcCann11}. We also provide conditions under which $\pmb\Pi$ is convex. In this situation, the maximizers of $\pmb\Pi$ may not be unique, but are attained at extreme points of 
$\mathcal{U}_{\emptyset}$. (Recall that $u \in \mathcal{U}$ is called {\em extreme} if 
$u$ does not lie at the midpoint of any segment in $\mathcal{U}$.)
\medskip


\begin{theorem}[$G$-convex functions form a convex set]\label{convex set}
	If $G: cl(X\times Y\times Z) \longrightarrow \R$ satisfies \Gzero-\Gtwo, then \Gthree\ becomes necessary and sufficient for the convexity of the set $\mathcal{U}$.
\end{theorem}
\begin{proof}
	Assuming \Gzero-\Gtwo, for any $u_0, u_1 \in \mathcal{U}$, define $u_t(x) := (1-t)u_0(x)+tu_1(x)$, $t\in (0,1)$. We want to show $u_t$ is $G$-convex as well, for each $t\in (0,1)$. 
	
	For any fixed $x_0 \in X$, since $u_0, u_1$ are $G$-convex, there exist $(y_0,z_0),$ $ (y_1, z_1) \in cl( Y \times Z)$, such that $u_0(x_0)= G(x_0, y_0,z_0)$, $u_1(x_0)=G(x_0,y_1,z_1)$, $u_0(x)\ge G(x,y_0,z_0)$ and $u_1(x)\ge G(x,y_1,z_1)$, for all $ x\in X$. 
	
	Denote $(x_0, y_t, z_t)$ the $G$-segment connecting $(x_0, y_0, z_0)$ and $(x_0, y_1, z_1)$. Then $u_t(x_0) = (1-t)u_0(x_0)+tu_1(x_0)=(1-t) G(x_0, y_0,z_0)+ tG(x_0,y_1,z_1) = G(x_0,y_t, z_t)$, where the last equality comes from ($\ref{$G$-segment}$). 
	
	In order to prove $u_t$ is $G$-convex, it remains to show $u_t(x)\ge  G(x,y_t,z_t)$, for all $x\in X$. 
	
	By \Gthree, $G(x, y_t, z_t)$ is convex in $t$, i.e., $G(x,y_t,z_t)\le (1-t)G(x, y_0,z_0)+tG(x, y_1, z_1)$. So, $u_t(x)=$ $  (1-t)u_0(x)+tu_1(x) \ge (1-t) G(x,y_0,z_0)+tG(x, y_1, z_1)\ge G(x, y_t, z_t)$, for each $x \in X$. By definition, $u_t$ is $G$-convex, i.e., $u_t\in \mathcal{U}$, for all $t \in (0,1)$. Thus, $\mathcal{U}$ is convex.
	
	Conversely, assume $\mathcal{U}$ is convex. For any fixed $x_0 \in X$, $(y_t,z_t) \in cl( Y \times Z)$ with $(x_0, y_t, z_t)$ being a $G$-segment, we would like to show $G(x,y_t, z_t) \le (1-t)G(x,y_0,z_0)+tG(x,y_1, z_1)$, for any $x\in X$.
	

	
	Define $u_i(x) := G(x, y_i, z_i)$, for $i=0,1$. Then by definition of $G$-convexity, $u_0, u_1 \in \mathcal{U}$. Denote $u_t:= (1-t)u_0+tu_1$, for all $t\in (0,1)$. Since $\mathcal{U}$ is a convex set, $u_t$ is also $G$-convex. For this $x_0$ and each $t\in (0,1)$, there exists $(\tilde{y}_t, \tilde{z}_t) \in cl( Y \times Z)$, such that $u_t(x)\ge G(x, \tilde{y}_t, \tilde{z}_t)$, for all $x\in X$, and equality holds at $x_0$. Thus, $Du_t(x_0) = D_x G(x_0, \tilde{y}_t, \tilde{z}_t)$.
	
	Since $(x_0, y_t, z_t)$ is a $G$-segment, from $(\ref{$G$-segment})$, we know $D_x G(x_0, y_t, z_t) = (1-t) D_x G(x_0, y_0, z_0) +t D_x G(x_0, y_1, z_1) = (1-t) Du_0(x_0) +t Du_1(x_0) = Du_t(x_0)$. Thus, by \Gone,  $(\tilde{y}_t , \tilde{z_t}) = (y_t, z_t)$, for each $t\in (0,1)$. Therefore, $(1-t)G(x,y_0,z_0)+tG(x,y_1, z_1) = u_t \ge G(x, \tilde{y}_t, \tilde{z}_t) = G(x, y_t, z_t)$, for all $x\in X$, i.e., $G(x, y_t, z_t)$ is convex in $t$ along any $G$-segment $(x_0, y_t, z_t)$.
\end{proof}



The following theorem provides a sufficient and necessary condition for the functional $\pmb \Pi(u)$ to be concave. It reveals the relationship between linear interpolations on the function space $\mathcal{U}$ and G-segments on the underlying type space $cl( Y \times Z)$.\medskip

\begin{theorem}[Concavity of the principal's objective]\label{maintheorem}
	If $G$ and $\pi: cl(X\times Y\times Z) \longrightarrow \R$ satisfy \Gzero-\Gfive, the following statements are equivalent:
	
	$(i)$ $t\in[0,1] \longmapsto \pi(x, y_t ,z_t)$ is concave along all G-segments $(x, y_t, z_t)$;
	
	$(ii)$ $\pmb \Pi(u)$ is concave in $\mathcal{U}$ for all $\mu\ll \mathcal{L}^m$. 
\end{theorem}

\begin{proof}
	$(i)\Rightarrow (ii).$ For any $u_0, u_1 \in \mathcal{U}$, $t\in (0,1)$, define $u_t = (1-t)u_0+tu_1$. We want to prove $\pmb \Pi(u_t) \ge (1-t) \pmb \Pi(u_0) + t\pmb \Pi(u_1)$, for any $\mu\ll \mathcal{L}^m$.
	
	Equations $(\ref{EqnInverse})$ implies that there exist $y_0, y_1: \dom Du \longrightarrow cl(Y)$ and $z_0, z_1: \dom Du \longrightarrow cl(Z)$ such that
	\begin{flalign}\label{Eqn:u_01}
	\begin{split}
	(G_x, G) (x, y_0(x), z_0(x)) &= (Du_0, u_0)(x),\\
	(G_x, G) (x, y_1(x), z_1(x)) &= (Du_1, u_1)(x).
	\end{split}
	\end{flalign}
	For each $x\in \dom Du$, $(y_0(x), z_0(x)), (y_1(x), z_1(x)) \in cl( Y \times Z)$, let $t\in [0,1] \longmapsto (x, y_t(x), z_t(x))$ be the $G$-segment connecting $(x, y_0(x), z_0(x))$ and $(x, y_1(x), z_1(x))$. Combining $(\ref{Eqn:u_01})$ and  $(\ref{$G$-segment})$, we have 
	\begin{equation}\label{EqnG-segments}
	(G_x, G) (x, y_t(x), z_t(x)) = (Du_t, u_t)(x).
	\end{equation}
	Thus, by concavity of $\pi$ on $G$-segments, for every $t \in [0,1]$,
	\begin{align*}
	\pmb \Pi (u_t)&= \int_X \pi(x, y_t(x), z_t(x))  d \mu (x)\\
	&\ge \int_X (1-t)\pi(x, y_0(x), z_0(x)) +t\pi(x, y_1(x), z_1(x))    d\mu (x)\\
	&=(1-t)\pmb \Pi (u_0)+ t \pmb \Pi(u_1).
	\end{align*}
	Thus, $\pmb \Pi $ is concave in $\mathcal{U}$.
	
	$(ii)\Rightarrow (i).$ To derive a contradiction, assume $(i)$ fails.  Then there exists a $G$-segment $(x_0, y_t(x_0), z_t(x_0))$ and $t_0\in (0,1)$ such that 
	\begin{equation*}
		\pi(x_0, y_{t_0}(x_0), z_{t_0}(x_0)) <  (1-t_0)\pi(x_0, y_0(x_0),z_0(x_0))+ t_0 \pi(x_0, y_1(x_0), z_1(x_0)).
	\end{equation*}
	Let $u_0(x) := G(x, y_0(x_0), z_0(x_0))$, $u_1(x):= G(x, y_1(x_0), z_1(x_0))$ and $u_{t_0} = (1-t_0)u_0 +t_0 u _1$. Then $u_0, u_1, u_{t_0} \in \mathcal{U}$.  From $(\ref{Eqn:u_01})$ we know, $y_i(x)\equiv y_i(x_0)$, $z_i(x)\equiv z_i(x_0)$, for $i=0,1$.  Let $t\in [0,1] \longmapsto (x, y_t(x), z_t(x))$ be the $G$-segment connecting $(x, y_0(x), z_0(x))$ and $(x, y_1(x), z_1(x))$. And combining $(\ref{EqnInverse})$ and $(\ref{$G$-segment})$, we have
	\begin{flalign*}
	(G_x, G) (x, y_0(x_0), z_0(x_0)) &= (Du_0, u_0)(x),\\
	(G_x, G) (x, y_1(x_0), z_1(x_0)) &= (Du_1, u_1)(x).\\
	(G_x, G) (x, y_{t_0}(x), z_{t_0}(x)) &= (Du_{t_0}, u_{t_0})(x).
	\end{flalign*}
	
	Since $\pi$, $y_{t_0}$ and $z_{t_0}$ are continuous, there exists $\varepsilon >0$, such that for all $x\in B_{\varepsilon}(x_0)$, one has
	\begin{flalign*}
	\pi(x, y_{t_0}(x), z_{t_0}(x)) < (1-t_0)\pi(x, y_0(x_0),z_0(x_0))  + t_0 \pi(x, y_1(x_0), z_1(x_0)).
	\end{flalign*} 
	Here we use $B_{\varepsilon}(x_0)$ denote the open ball in $\R^m$ centered at $x_0$ with radius $\varepsilon$.
	Take $d\mu = d\mathcal{L}^m
	\mid _{B_{\varepsilon} (x_0)}/{\mathcal{L}^m} 
	(B_{\varepsilon}(x_0))$ to be uniform measure on $B_\varepsilon(x_0)$. 
	Thus, 
	\begin{align*}
	\pmb \Pi (u_{t_0}) &= \int_X \pi(x, y_{t_0}(x), z_{t_0}(x)) d\mu(x)\\
	&<  \int_X    (1-t_0)\pi(x, y_0(x_0),z_0(x_0))  + t_0 \pi(x, y_1(x_0), z_1(x_0))     d\mu(x)\\
	&=(1-t_0)\pmb \Pi(u_0)+ t_0 \pmb \Pi(u_1).
	\end{align*}
	This contradicts the concavity of $\pmb \Pi$.
\end{proof}

A similar proof shows the following result. Corollary $\ref{Cor:concave}$ implies that concavity of the principal's profit is equivalent to concavity of principal's utility along qualified $G$-segments. Moreover, Theorem~\ref{convex set} and Corollary $\ref{Cor:concave}$ together imply that the principal's profit $\pmb \Pi$ is a concave functional on a convex space, under assumptions \Gzero-\Gfive, $\mu\ll \mathcal{L}^m$, and $(i)'$ below. 
\medskip

\begin{corollary}\label{Cor:concave}
	If $G$ and $\pi$ satisfy \Gzero-\Gfive,  the following are equivalent:
	
	$(i)'$ $t\in[0,1] \longmapsto \pi(x, y_t(x) ,z_t(x))$ is concave  along all G-segments $(x, y_t(x), z_t(x))$ whose 
	endpoints satisfy $\min\{G(x, y_0(x), z_0(x)),G(x,y_1(x),$
	$z_1(x))\} \ge u_{\emptyset}(x)$;
	
	$(ii)'$ $\pmb \Pi(u)$ is concave in $\mathcal{U}_{\emptyset}$  for all $\mu\ll \mathcal{L}^m$. 
\end{corollary}




To obtain uniqueness and stability of optimizers requires a stronger form of convexity.
Recall that a function $f$ defined on a convex subset of a normed space 
is said to be strictly convex if $f((1-t)x + ty)>(1-t) f(x) + tf(y)$ 
whenever $0<t<1$ and $x\ne y$. It is said to be (2-)uniformly concave, if there exists    
$\lambda>0$, such that for any $x, y$ in the domain of $f$ and $t\in [0,1]$, the following inequality holds.
\begin{flalign*}
f((1-t)x+ty) -(1-t)f(x) - t f(y) \ge t(1-t)\lambda||x-y||^2.
\end{flalign*}
For such strengthenings,  it is necessary to view indirect utilities $u \in \mathcal U$ 
as equivalence classes
of functions which differ only on sets of $\mu$ measure zero.  More precisely, it is natural to adopt 
the Sobolev norm
\begin{equation*}
	\|u\|^2_{W^{1,2}(X,d\mu)} := \int_X( |u|^2 + |Du|^2) d\mu(x)	
\end{equation*}
on $\mathcal{U}$ and $\mathcal{U}_{\emptyset}$. We then have the following results:
\medskip

\begin{corollary}\label{Cor:strictconcave}
	Let $\pi$ and $G$ satisfy \Gzero-\Gfive. If 
	
	$(iii)$  $t\in[0,1] \longmapsto \pi(x, y_t ,z_t)$ is strictly concave along all G-segments $(x, y_t, z_t)$, then  
	
	$(iv)$ $\pmb \Pi(u)$ is strictly concave in $\mathcal{U} \subset W^{1,2}(X,d\mu)$  for all $\mu\ll \mathcal{L}^m$. If 	
	
	$(iii)'$ $t\in[0,1] \longmapsto \pi(x, y_t(x) ,z_t(x))$ is strictly concave along all G-segments $(x, y_t(x), z_t(x))$ whose 
	endpoints satisfy $\min\{G(x, y_0(x), z_0(x)),$ $G(x,$ $y_1(x), z_1(x))\} \ge u_{\emptyset}(x)$, then 
	
	$(iv)'$ $\pmb \Pi(u)$ is strictly concave in $\mathcal{U}_{\emptyset} \subset W^{1,2}(X,d\mu)$ for all $\mu\ll \mathcal{L}^m$. 
\end{corollary}

In addition, Theorem $\ref{convex set}$ and Corollary $\ref{Cor:strictconcave}$ together imply strict concavity of principal's profit on a convex space, which guarantees a unique solution to the monopolist's problem.\medskip


Define $\bar{G}(\bar{x}, \bar{y})=\bar{G}(x,x_0, y,z) := x_0 G(x, y,z)$, where $\bar{x}=(x, x_0)$, $\bar{y}=(y,z)$ and $x_0\in X_0$, where $X_0 \subset (-\infty, 0)$ is an open bounded interval containing $-1$. {Hereafter, except in chapter \ref{chapter:analytic_representation}, we use $x_0$ to denote a number in $X_0$.} For further applications, we need the following non-degeneracy assumption. \medskip

\begin{itemize}
	\item[\Gsix] $G\in C^2(cl(X\times Y \times Z)
	)$, and $D_{\bar{x},\bar{y}}(\bar{G})(x,-1,y,z)$ has full rank, for each $(x,y,z)\in cl(X\times Y\times Z)$. 
\end{itemize}

Since \Gone\ implies $m \ge n$,  full rank means $D_{\bar{x},\bar{y}}(\bar{G})(x,-1,y,z)$ has rank $n+1$.
\medskip


\begin{theorem}[Uniform concavity of the principal's objective]\label{maintheorem2}
	Assume $G\in  C^{2}(cl(X\times Y\times Z))$ satisfies \Gzero-{\Gsix}. In case $\bar{z}=+\infty$, assume the homeomorphisms of \Gone\ are uniformly bi-Lipschitz. Then the following statements are equivalent:
	
	$(v)$ Uniform concavity of $\pi$ along G-segments, i.e., there exists $\lambda>0$, for any $G$-segment $(x, y_t, z_t)$, and any $t\in [0,1]$, 
	\begin{flalign}\label{uniformconvavity}
	\begin{split}
	&\pi(x, y_t(x), z_t(x)) - (1-t)\pi(x, y_0(x), z_0(x)) - t\pi(x, y_1(x), z_1(x)) \\
	\ge & t(1-t)\lambda||(y_1(x)-y_0(x),z_1(x)-z_0(x))||^2_{\R^{n+1}}
	\end{split}
	\end{flalign} 
	
	$(vi)$ $\pmb \Pi(u)$ is uniformly concave in $\mathcal{U} \subset W^{1,2}(X,d\mu)$,  uniformly for all $\mu\ll \mathcal{L}^m$. 
\end{theorem}

\begin{proof}
	$(v)\Rightarrow (vi).$ With the same notation as last proof, we want to prove there exists  
	$\tilde{\lambda}>0$, such that $\pmb \Pi(u_t) - (1-t) \pmb \Pi(u_0) - t\pmb \Pi(u_1)\ge t(1-t)\tilde{\lambda}||u_1-u_0||^2_{W^{1,2}(X,d\mu)}$, for any $\mu\ll \mathcal{L}^m$, $u_0, u_1 \in \mathcal{U}$ and $t\in (0,1)$. 
	
	Similar to last proof, we have (\ref{Eqn:u_01}) and (\ref{EqnG-segments}). Denote $\Lip(G_x,G)$ the uniform Lipschitz constant of the map $(x,y,z)\in X\times Y\times Z \longmapsto (G_x,G)(x,y,z)$.

	Thus by uniform concavity of $\pi$ on $G$-segments, there exists  
	$\lambda>0$, such that for every $t \in [0,1]$,
	\begin{align*}
	&\pmb \Pi (u_t)-(1-t)\pmb \Pi (u_0)- t \pmb \Pi(u_1)\\
	&= \int_X \pi(x, y_t(x), z_t(x)) - (1-t)\pi(x, y_0(x), z_0(x)) - t\pi(x, y_1(x), z_1(x)) d \mu (x)\\
	\end{align*}
	\begin{align*}
	&\ge \int_{X} t(1-t)\lambda||(y_1(x)-y_0(x),z_1(x)-z_0(x))||^2_{\R^{n+1}} d\mu (x)\\
	&\ge \int_{X} t(1-t)\lambda||(Du_1(x)-Du_0(x),u_1(x)-u_0(x))||^2_{\R^{n+1}}/{\Lip}^2(G_x,G) d\mu (x)\\
	&= t(1-t)\frac{\lambda}{\Lip^2(G_x,G)} ||u_1-u_0||_{W^{1,2}(X,d\mu)}.
	\end{align*}
	Thus, $\pmb \Pi $ is uniformly concave in $\mathcal{U}$, with $\tilde{\lambda} = \frac{\lambda}{\Lip^2(G_x,G)} >0$.
	
	$(vi)\Rightarrow (v).$ To derive a contradiction, assume $(v)$ fails.  Then for any $\lambda>0$, there exists a $G$-segment $(x^0, y_t(x^0), z_t(x^0))$, and some $\tau \in (0,1)$, such that $\pi(x^0, y_{\tau}(x^0), z_{\tau}(x^0)) -  (1-\tau)\pi(x^0, y_{0}(x^0),z_{0}(x^0))-\tau \pi(x^0, y_{1}(x^0), z_{1}(x^0)) < \tau (1-\tau)\lambda ||(y_{1}(x^0)-y_{0}(x^0), z_{1}(x^0)-z_{0}(x^0))||^2_{\R^{n+1}}$.
	
	Take $u_0(x) := G(x, y_{0}(x^0), z_{0}(x^0))$, $u_1(x):= G(x, y_{1}(x^0), z_{1}(x^0))$ and for $t\in (0,1)$, assign $u_{t} := (1-t)u_0 +t u _1$. Then $ u_{t} \in \mathcal{U}$, for $t \in [0,1]$.  From $(\ref{Eqn:u_01})$ we know, $y_i(x)\equiv y_i(x^0)$, $z_i(x)\equiv z_i(x^0)$, for $i=0,1$.  Let $t\in [0,1] \longmapsto (x, y_t(x), z_t(x))$ be the $G$-segment connecting $(x, y_0(x), z_0(x))$ and $(x, y_1(x), z_1(x))$. And combining $(\ref{EqnInverse})$ and $(\ref{$G$-segment})$, we have
	\begin{flalign*}
	(G_x, G) (x, y_{0}(x^0), z_{0}(x^0)) &= (Du_0, u_0)(x),\\
	(G_x, G) (x, y_{1}(x^0), z_{1}(x^0)) &= (Du_1, u_1)(x).\\
	(G_x, G) (x, y_{t}(x), z_{t}(x)) &= (Du_{t}, u_{t})(x).
	\end{flalign*}
	
	Since $\pi$, $y_{\tau}$ and $z_{\tau}$ are continuous, there exists $\varepsilon >0$, such that for all  $x\in B_{\varepsilon}(x^0)$,
	\begin{flalign*}
	&\pi(x, y_{\tau}(x), z_{\tau}(x)) -  (1-\tau)\pi(x, y_{0}(x^0),z_{0}(x^0))-\tau \pi(x, y_{1}(x^0), z_{1}(x^0)) \\
	&< \tau (1-\tau)\lambda||(y_{1}(x^0)-y_{0}(x^0), z_{1}(x^0)-z_{0}(x^0))||^2_{\R^{n+1}}.
	\end{flalign*} 
	Here we use $B_{\varepsilon}(x^0)$ denote the open ball in $\R^m$ centered at $x^0$ with radius $\varepsilon$.
	Take $d\mu = d\mathcal{L}^m
	\mid _{B_{\varepsilon} (x^0)}/\mathcal{L}^m
	(B_{\varepsilon}(x^0))$ to be uniform measure on $B_\varepsilon(x^0)$. 
	By \Gsix,   the map $\bar{y}_G:(x,p,q)\longmapsto(y,z)$, which solves equation (\ref{EqnInverse}), is uniformly Lipschitz on $X\times \R \times\R^m$. Denote  $\Lip(\bar{y}_G)$ its Lipschitz constant.
	
	Thus for such $\tau$, $u_0$, $u_1$ and $\mu$, we have
	\begin{flalign*}
	&\pmb \Pi (u_{\tau})- (1-\tau)\pmb \Pi(u_0)- \tau \pmb \Pi(u_1)\\
		&= \int_X \pi(x, y_{\tau}(x), z_{\tau}(x)) -  (1-\tau)\pi(x, y_{0}(x^0),z_{0}(x^0)) -\tau \pi(x, y_{1}(x^0), z_{1}(x^0)) d\mu(x)
	\end{flalign*}
	\begin{flalign*}		
	&<  \int_X    \tau(1-\tau)\lambda ||(y_{1}-y_{0}, z_{1}-z_{0})||^2_{\R^{n+1}} d\mu(x)\\
	&\le \tau(1-\tau)\lambda {\Lip}^2(\bar{y}_G)||u_1-u_0||^2_{W^{1,2}(X,d\mu)}.\hspace{6cm}
	\end{flalign*}
	This contradicts the uniform concavity of $\pmb \Pi$.
\end{proof}

A similar argument implies the following equivalence. Theorem $\ref{convex set}$ and Corollary $\ref{Cor:concave2}$ together imply that the principal's profit $\pmb \Pi$ is a uniformly concave functional on a convex space, under assumptions \Gzero-\Gsix, $\mu\ll \mathcal{L}^m$, and $(v)'$.  



\begin{corollary}\label{Cor:concave2}
	Under the same assumptions as in Theorem \ref{maintheorem2},  the following are equivalent:
	
	$(v)'$ Uniform concavity of $\pi$  (in the sense of equation (\ref{uniformconvavity})) along G-segments $(x, y_t(x), z_t(x))$ whose 
	endpoints satisfy $\min\{G(x, y_0(x), z_0(x)), G(x,$ $y_1(x),z_1(x))\} \ge u_{\emptyset}(x)$;
	
	$(vi)'$ $\pmb \Pi(u)$ is uniformly concave  in $\mathcal{U}_{\emptyset} \subset W^{1,2}(X,d\mu)$   uniformly
	for all $\mu\ll \mathcal{L}^m$. 
\end{corollary}

The preceding concavity results also have convexity analogs. Unlike strict concavity, strict convexity does not
imply uniqueness of the principal's profit-maximizing strategy, though it suggests it should only be attained at 
extreme points of the strategy space $\mathcal U$,  where extreme point needs to be interpreted appropriately.
\medskip


\begin{remark}[Convexity of principal's objective]\label{remarkmaintheorem1}
	If $\pi$ and $G$ satisfy \Gzero-\Gfive,  the equivalences 
	$(i) \Leftrightarrow (ii)$ and $(i)' \Leftrightarrow (ii)'$ and implications $(iii) \Rightarrow (iv)$ and $(iii)' \Rightarrow (iv)'$ remain true when all occurences of concavity are replaced by convexity.
	Similarly,  the equivalences $(v) \Leftrightarrow (vi)$ and $(v)' \Leftrightarrow (vi)'$ remain true when both
	occurences of uniform concavity are replaced by uniform convexity in Theorem \ref{maintheorem2}. 
\end{remark}
\medskip





Assuming \Gsix, we denote $(\bar{G}_{\bar{x}, \bar{y}})^{-1}$ the left inverse of $D_{\bar{x},\bar{y}}(\bar{G})(x,x_0,y,z)$.
We will use Einstein notation for simplifying expressions including summations of vectors, matrices, and general tensors for higher order derivatives. There are essentially three rules of Einstein summation notation, namely: 1. repeated indices are implicitly summed over; 2. each index can appear at most twice in any term; 3.~both sides of an equation must contain the same non-repeated indices. For example, $a_{ij}v_i =\sum_{i}a_{ij}v_i$, $a_{ij}b^{kj}v_k=\sum_{j}\sum_{k}a_{ij}b^{kj}v_k$. We also use comma to separate subscripts: the subscripts before comma represent derivatives with respect to first variable and those after comma represent derivatives with respect to second variable. For instance, for $b=b(x,y)$, $b_{,kl}$ represents second derivatives with respect of $y$ only. And for $G=G(x,y,z)$, where $z\in \R$, $G_{i,jz}$ denotes third order derivatives with respect of $x$, $y$ and $z$, instead of using another comma to separate subscripts corresponding to $y$ and $z$. Starting from now, for subscripts, we use $i,k,j,l, \alpha, \beta$ denoting integers  from either $\{1,...,m\}$ or $\{ 1,..., n\}$, and $\bar{i},\bar{k},\bar{j},\bar{l}$ denoting augmented indices from $\{1,...,m+1\}$ or $ \{1, ..., n+1\}$. For instance, $\pi_{i,}$ denotes first order derivative with respect to $x$ only, $\pi_{,\bar{k}\bar{j}}$ represents Hessian matrix with respect to $\bar{y}$ only, and $\bar{G}_{\bar{i},\bar{k}\bar{j}}$ denotes a third order derivative tensor which can be viewed as taking $\bar{x}$-derivative of $\bar{G}_{,\bar{k}\bar{j}}$.
\medskip

The following remark reformulates concavity of $\pi$ on $G$-segments using non-positive definiteness of a matrix. This equivalent form provides a simple method to verify concavity condition stated in Theorem $\ref{maintheorem}$. We will apply this matrix form to establish Corollary $\ref{bar{G}^*-Concavity}$ and Example $\ref{general example1} - \ref{general example3}$.\medskip

\begin{lemma}[Characterizing concavity of principal's profit in the smooth case]\label{LemmaProfitConcavity}
	When $G \in C^3(cl(X\times Y \times Z))$ satisfies \Gzero-\Gsix \ and  $\pi \in C^2(cl(X\times Y \times Z))$, then differentiating $\pi$ along an arbitrary $G$-segment $t \in[0,1] \longrightarrow (x,y_t,z_t)$ yields
	\begin{equation}\label{pi second}
	\frac{d^2}{dt^2} \pi(x, y_t, z_t) = (\pi_{,\bar{k}\bar{j}}- \pi_{,\bar{l}} \bar{G}^{\bar i,\bar l}\bar{G}_{\bar{i},\bar{k}\bar{j}}) \dot {\bar y}^{\bar k} \dot {\bar y}^{\bar j}
	\end{equation}
	where $\bar{G}^{\bar i,\bar l}$ denotes the left inverse of the matrix $\bar{G}_{\bar{i}, \bar{k} }$
	and $\dot {\bar y}^{\bar k} = (\frac{d}{dt})\bar y^{\bar k}_t$.
	Thus $(i)$ in Theorem $\ref{maintheorem}$  is equivalent to non-positive  definiteness of the quadratic form 
	$\pi_{,\bar{k}\bar{j}}- \pi_{,\bar{l}} \bar{G}^{\bar i,\bar l}\bar{G}_{\bar{i},\bar{k}\bar{j}}$
	on $T_{\bar y}  (Y\times Z) = \R^{n+1}$, { for each  $(x, \bar y) \in X \times Y \times Z$.} Similarly, Theorem \ref{maintheorem2} $(v)$ is equivalent to uniform negative definiteness of the same form.
\end{lemma}	

\begin{proof}%[Proof of Lemma \ref{LemmaProfitConcavity}]
	For any G-segments $(x, y_t, z_t)$ satisfying equation (\ref{EqnG-segments}) and $\pi \in C^2(cl(X\times Y \times Z)
	)$,
	$t\in[0,1] \longmapsto \pi(x, y_t ,z_t)$ is concave [uniformly concave] if and only if $\frac{d^2}{dt^2} \pi(x, y_t, z_t)\le  0$ $[\le -\lambda ||(\dot{y}_t,\dot{z}_t)||^2_{\R^{n+1}}<0]$, for all $t \in [0,1]$.
	
	On the one hand, since $\frac{d}{dt}\pi(x, y_t, z_t) = \pi_{,\bar{k}} \dot{\bar y}^{\bar{k}}$, taking another derivative with respect of $t$ gives 
	\begin{equation}\label{EqnSecondDiffProfit}
	\frac{d^2}{dt^2} \pi(x, y_t, z_t) = 
	\pi_{,\bar{k}\bar{j}}\dot{\bar y}^{\bar{k}} \dot{\bar y}^{\bar{j}}+ \pi_{,\bar{l}}\ddot {\bar y}^{\bar{l}}.
	\end{equation}
	On the other hand, taking second derivative with respect of t at both sides of equation (\ref{EqnG-segments}), which is equivalent to $\bar{G}_{\bar{i},}(x, x_0, y_t(x), z_t(x)) = (x_0Du_t, u_t)(x)$, for some fixed $x_0\in X_0$, implies
	\begin{equation}\label{EqnSecondDiffbar{G}}
	\bar{G}_{\bar{i},\bar{k}\bar{j}}\dot{\bar y}^{\bar{k}} \dot {\bar y}^{\bar{j}}+\bar{G}_{\bar{i},\bar{k}} \ddot{\bar y}^{\bar{k}} =0 
	\end{equation}
	
	Combining equations (\ref{EqnSecondDiffProfit}) with (\ref{EqnSecondDiffbar{G}}) yields
	\eqref{pi second}.
	For $x \in X$, there is a $G$-segment with any given tangent direction through $\bar y = (y,z) \in Y \times Z$.
	Thus, the non-positivity  of $\frac{d^2}{dt^2} \pi(x, y_t, z_t)$ along all G-segments $(x, y_t, z_t)$ is equivalent to non-positive  definiteness of the matrix $(\pi_{,\bar{k}\bar{j}}- \pi_{,\bar{l}} \bar{G}^{\bar i,\bar l}\bar{G}_{\bar{i},\bar{k}\bar{j}})$ on $T_{\bar y} (Y\times Z)=\R^{n+1}$.
	
	In addition, the uniform concavity of $\pi(x, y_t, z_t)$ along all G-segments $(x, y_t, z_t)$ is equivalent to uniform negative definiteness of $(\pi_{,\bar{k}\bar{j}}- \pi_{,\bar{l}} \bar{G}^{\bar i,\bar l}\bar{G}_{\bar{i},\bar{k}\bar{j}})$ on $\R^{n+1}$.
\end{proof}









\section{Concavity of principal's objective when her utility does not depend directly on agents' private types:  
	A sharper, more local result} 
\label{section:privateInformation}


In this section, we reveal a necessary and sufficient condition  for the concavity of principal's maximization problem, not for some specific examples in chapter \ref{chapter:examples}, but for many other private-value circumstances, where principal's utility only directly depends on the products sold and their selling prices, but not the buyer's type. 
\medskip


	Before we state the results, we need the following definition, which is a generalized Legendre transform 
	(see Moreau \cite{Moreau70}, Kutateladze-Rubinov \cite{KutateladzeRubinov72}, Elster-Nehse \cite{ElsterNehse74}, Balder \cite{Balder77}, Dolecki-Kurcyusz \cite{DoleckiKurcyusz78}, Gangbo-McCann\cite{GangboMcCann96}, Singer\cite{Singer97}, Rubinov\cite{Rubinov00a, Rubinov00b}, and Mart\'inez-Legaz \cite{MartinezLegaz05} 
	for more references).  
	\medskip
	
	
	
	\begin{definition}[$\bar{G}$-concavity, $\bar{G}^*$-concavity]\label{(-bar{G})-convexity}
		A function $\phi: cl(X \times X_0) \longrightarrow \R$ is called $\bar{G}$-concave if $\phi = (\phi^{\bar{G}^*})^{\bar{G}}$ and a function $\psi: cl( Y \times Z) \longrightarrow \R$ is called $\bar{G}^*$-concave if $\psi = (\psi^{\bar{G}})^{\bar{G}^*}$, where 
		\begin{equation}\label{(-bar{G})-transform}
		\begin{split}
		\psi^{\bar{G}}(\bar{x})&~=\min_{\bar{y} \in cl( Y \times Z)} \bar{G}(\bar{x},\bar{y}) - \psi(\bar{y}),\\ \text{ and } \phi^{\bar{G}^*}(\bar{y})&~= \min_{\bar{x} \in cl(X \times X_0)} \bar{G}(\bar{x}, \bar{y}) - \phi(\bar{x}).
		\end{split}
		\end{equation}
		We say $\psi$ is strictly $\bar{G}^*$-concave, if in addition $\psi^{\bar G} \in C^1(X\times X_0)$. 
	\end{definition}
	
	Note that,  apart from an overall sign and the extra variables,
	Definition \ref{(-bar{G})-convexity} coincides with a quasilinear version $G(\bar{x},\bar{y},z) = \bar{G}(\bar{x},\bar{y})-z$ of
	Definition \ref{defn:GConvexity}.
	\medskip
	
	
	The following corollary characterizes the concavity of principal's profit when her utility on one hand is not influenced by the agents' identity, and on the other hand has adequate generality to  encompass a
	tangled nonlinear relationship between products and selling prices. It generalizes the convexity result in Figalli-Kim-McCann \cite{FigalliKimMcCann11}, where $G(x,y,z) = b(x,y)-z$ and $\pi(x,y,z) = z-a(y)$. 
	\medskip
	
	
	\begin{corollary}[Concavity of principal's objective with her utility not depending on agents' types]\label{bar{G}^*-Concavity}
		If $G \in C^3(cl(X\times Y \times Z)
		)$ satisfies \Gzero-\Gsix,  $\pi \in C^2( cl( Y \times Z)
		)$ is $\bar{G}^*$-concave and $\mu\ll \mathcal{L}^m$, then $\pmb \Pi$ is concave. 
	\end{corollary}
	
	\begin{proof}%[Proof of Corollary \ref{bar{G}^*-Concavity}]
		According to Lemma \ref{LemmaProfitConcavity}, for concavity, we only need to show non-positive definiteness of $(\pi_{\bar{k}\bar{j}}- \pi_{\bar{l}} \bar{G}^{\bar i,\bar l}\bar{G}_{\bar{i},\bar{k}\bar{j}})$ on $\R^{n+1}$, i.e., for any $\bar{x} = (x,x_0) \in X \times X_0$,  $\bar{y} \in Y\times Z$ and $\xi \in \R^{n+1} $, $\big(\pi_{\bar{k}\bar{j}}(\bar{y})- \pi_{\bar{l}}(\bar{y}) \bar{G}^{\bar i,\bar l}(\bar{x}, \bar{y})\bar{G}_{\bar{i},\bar{k}\bar{j}}(\bar{x}, \bar{y})\big)\xi^{\bar{k}}\xi^{\bar{j}} \le 0$.
		
		For any fixed $\bar{x} = (x, x_0) \in X \times X_0$, $\bar{y} \in Y\times Z$, $\xi \in \R^{n+1}$, there exist $\delta >0$ and $t \in (-\delta, \delta) \longmapsto \bar{y}_t \in Y\times Z$, such that $\bar{y}_t|_{t=0} =\bar{y}$,  $\dot{\bar{y}}|_{t=0} = \xi$ and $\frac{d^2}{dt^2} \bar{G}_{\bar{i}, }(\bar{x}, \bar{y}_t) = 0$. Thus, 
		\begin{equation}\label{EqnSecondDiffbar{G}2}
		0 = \frac{d^2}{dt^2}\bigg|_{t=0}\bar{G}_{\bar{i},}(\bar{x}, \bar{y}_t) = \bar{G}_{\bar{i}, \bar{k}\bar{j}}(\bar{x}, \bar{y}) \xi^{\bar{k}}\xi^{\bar{j}} + \bar{G}_{\bar{i}, \bar{k}}(\bar{x}, \bar{y}) (\ddot{\bar{y}}_t)^{\bar{k}}\Big|_{t=0}.
		\end{equation}
		Since $\pi$ is $\bar{G}^*$-concave, we have $\pi(\bar{y}) =\min_{\tilde{x} \in cl(X \times X_0)} \bar{G}(\tilde{x}, \bar{y}) - \phi(\tilde{x})$, for some $\bar{G}$-concave function $\phi$. Since $cl(X \times X_0)$ is compact, for this $\bar{y}$, there exists ${\bar{x}}^* = ({x}^*, {x_0}^*) \in cl(X \times X_0)$, such that 
		$\pi_{\bar{l}}(\bar{y}) = \bar{G}_{,\bar{l}}({\bar{x}}^*, \bar{y})$ for each $\bar{l} = 1, 2,..., n+1$ and $\pi_{\bar{k}\bar{j}}(\bar{y})\xi^{\bar{k}}\xi^{\bar{j}} \le \bar{G}_{,\bar{k}\bar{j}}({\bar{x}^*}, \bar{y})\xi^{\bar{k}}\xi^{\bar{j}}$  for each $\xi \in \R^{n+1}$. 
		Combined with (\ref{EqnSecondDiffbar{G}2}) this yields
		\begin{align*}\label{Eqn:proofcor}
		\begin{split}
		&\big(\pi_{\bar{k}\bar{j}}(\bar{y})- \pi_{\bar{l}}(\bar{y}) \bar{G}^{\bar i,\bar l}(\bar{x}, \bar{y})\bar{G}_{\bar{i},\bar{k}\bar{j}}(\bar{x}, \bar{y})\big)\xi^{\bar{k}}\xi^{\bar{j}} \\
		&\le
		\big(\bar{G}_{,\bar{k}\bar{j}}({\bar{x}}^*, \bar{y})- \bar{G}_{,\bar{l}}({\bar{x}}^*, \bar{y})\bar{G}^{\bar i,\bar l}(\bar{x}, \bar{y})\bar{G}_{\bar{i},\bar{k}\bar{j}}(\bar{x}, \bar{y})\big)\xi^{\bar{k}}\xi^{\bar{j}} \\
		&= \bar{G}_{,\bar{k}\bar{j}}({\bar{x}}^*, \bar{y})\xi^{\bar{k}}\xi^{\bar{j}}+ \bar{G}_{,\bar{l}}({\bar{x}}^*, \bar{y})\cdot (\ddot{\bar{y}}_t)^{\bar{l}}\big|_{t=0}\\
		\end{split}
		\end{align*}
		\begin{align*}
		\begin{split}
		& = \frac{d^2}{dt^2}\bigg|_{t=0} \bar{G}({\bar{x}}^*, \bar{y}_t) \hspace{5.33cm}\\
		& = {x_0}^* \cdot\frac{d^2}{dt^2}\bigg|_{t=0} G({x}^*, \bar{y}_t) \\
		&\le 0.
		\end{split}
		\end{align*}
		The last inequality comes from ${x_0}^* \le 0$ and \Gthree.
	\end{proof}
	
The following proposition shows a version of sufficient and necessary condition to concavity in corollary \ref{bar{G}^*-Concavity}.\medskip


\begin{proposition}[Concavity of principal's objective when her payoff is independent of agents' types]\label{bar{G}^*-Concavity2}
	Suppose $G \in C^3(cl(X\times Y \times Z)
	)$ satisfies \Gzero-\Gsix,  $\pi \in C^2( cl( Y \times Z)
	)$, and assume there exists a set $J\subset cl(X)$ such that for each $\bar{y}\in Y\times Z$, $ 0\in ( \pi_{\bar{y}}+G_{\bar{y}})(cl(J), \bar{y})$.  Then the following statements are equivalent:
	
	\begin{enumerate}[(i)]
		\item \text{\rm local ${\bar G}^*$-concavity of} $\pi$: i.e. $\pi_{\bar{y}\bar{y}}(\bar{y}) + G_{\bar{y}\bar{y}}(x, \bar{y})$ is non-positive definite whenever
		$(x, \bar{y}) \in cl(J)\times Y \times Z$ satisfies $\pi_{\bar{y}}(\bar{y})+ G_{\bar{y}}(x, \bar{y})=0$;
		\item $\pmb \Pi$ is concave on $\mathcal{U}$ for all $\mu\ll \mathcal{L}^m$.
	\end{enumerate}
\end{proposition}


\begin{remark}
	The sufficient condition, i.e., existence of $J \subset cl(X)$(, such that for each  $\bar{y}\in Y\times Z$, $ 0\in ( \pi_{\bar{y}}+G_{\bar{y}})(cl(J), \bar{y})$), make the statement more general than taking some specific subset of $cl(X)$ instead. In particular, if $J=cl(X)$, this condition is equivalent to: for each $\bar{y} \in Y\times Z$, there exists $x \in cl(X)$, such that $( \pi_{\bar{y}}+G_{\bar{y}})(x, \bar{y}) = 0$. One of its economic interpretations is that for each product-price type, there exists a customer type, such that his marginal disutility, the gradient with respect to product type (e.g., quality, quantity, etc.) and price type, coincides with the marginal utility of the monopolist. 
\end{remark}

\begin{proof}[Proof of Proposition \ref{bar{G}^*-Concavity2}]
	$(i)\Rightarrow (ii).$ Similar to the proof of Corollary \ref{bar{G}^*-Concavity}, we only need to show non-positive definiteness of $(\pi_{\bar{k}\bar{j}}- \pi_{\bar{l}} \bar{G}^{\bar i,\bar l}\bar{G}_{\bar{i},\bar{k}\bar{j}})$, i.e., for any $\bar{x} = (x,x_0) \in X \times X_0$,  $\bar{y} \in Y\times Z$ and $\xi \in \R^{n+1} $, $\big(\pi_{\bar{k}\bar{j}}(\bar{y})- \pi_{\bar{l}}(\bar{y}) \bar{G}^{\bar i,\bar l}(\bar{x}, \bar{y})\bar{G}_{\bar{i},\bar{k}\bar{j}}(\bar{x}, \bar{y})\big)\xi^{\bar{k}}\xi^{\bar{j}} \le 0$.
	
	For any fixed $\bar{x} = (x, x_0) \in X \times X_0$, $\bar{y} \in Y\times Z$, $\xi \in \R^{n+1}$, there exist $\delta >0$ and a curve $t \in (-\delta, \delta) \longmapsto \bar{y}_t \in Y\times Z$, such that $\bar{y}_t|_{t=0} =\bar{y}$,  $\dot{\bar{y}}_t|_{t=0} = \xi$ and $\frac{d^2}{dt^2} \bar{G}_{\bar{i}, }(\bar{x}, \bar{y}_t) = 0$. Thus, 
	\begin{equation}\label{EqnSecondDiffbar{G}3}
	0 = \frac{d^2}{dt^2}\bigg|_{t=0}\bar{G}_{\bar{i},}(\bar{x}, \bar{y}_t) = \bar{G}_{\bar{i}, \bar{k}\bar{j}}(\bar{x}, \bar{y}) \xi^{\bar{k}}\xi^{\bar{j}} + \bar{G}_{\bar{i}, \bar{k}}(\bar{x}, \bar{y})\cdot (\ddot{\bar{y}}_t)^{\bar{k}}\big|_{t=0}
	\end{equation}
	For this $\bar{y}$, since $0\in (\pi_{\bar{y}}+G_{\bar{y}})(cl(J), \bar{y})$, there exists $x^*\in cl(J)$, such that $( \pi_{\bar{y}}+G_{\bar{y}})(x^*, \bar{y})=0$. By property $(i)$, one has $(\pi_{\bar{y}\bar{y}}(\bar{y}) + G_{\bar{y}\bar{y}}(x^*, \bar{y}))\xi^{\bar{k}}\xi^{\bar{j}} \le 0$.
	Let $\bar{x}^* =(x^*,-1)$, then $\pi_{\bar{l}}(\bar{y}) = \bar{G}_{,\bar{l}}({\bar{x}}^*, \bar{y})$  and $\pi_{\bar{k}\bar{j}}(\bar{y})\xi^{\bar{k}}\xi^{\bar{j}} \le \bar{G}_{,\bar{k}\bar{j}}({\bar{x}^*}, \bar{y})\xi^{\bar{k}}\xi^{\bar{j}}$, for each $\bar{l} = 1, 2,..., n+1$. Thus, combining (\ref{EqnSecondDiffbar{G}3}) and \Gthree, we have 
	\begin{flalign}\label{Eqn:proofcor2}
	\begin{split}
		&\big(\pi_{\bar{k}\bar{j}}(\bar{y})- \pi_{\bar{l}}(\bar{y}) \bar{G}^{\bar i,\bar l}(\bar{x}, \bar{y})\bar{G}_{\bar{i},\bar{k}\bar{j}}(\bar{x}, \bar{y})\big)\xi^{\bar{k}}\xi^{\bar{j}} \\
		&\le
		\big(\bar{G}_{,\bar{k}\bar{j}}({\bar{x}}^*, \bar{y})- \bar{G}_{,\bar{l}}({\bar{x}}^*, \bar{y})\bar{G}^{\bar i,\bar l}(\bar{x}, \bar{y})\bar{G}_{\bar{i},\bar{k}\bar{j}}(\bar{x}, \bar{y})\big)\xi^{\bar{k}}\xi^{\bar{j}} \\
		&= \bar{G}_{,\bar{k}\bar{j}}({\bar{x}}^*, \bar{y})\xi^{\bar{k}}\xi^{\bar{j}}+ \bar{G}_{,\bar{l}}({\bar{x}}^*, \bar{y})\cdot (\ddot{\bar{y}}_t)^{\bar{l}}\big|_{t=0}\\
		& = \frac{d^2}{dt^2}\bigg|_{t=0} \bar{G}({\bar{x}}^*, \bar{y}_t)\\
		& = - \frac{d^2}{dt^2}\bigg|_{t=0} G({x}^*, \bar{y}_t)\\
		&\le 0.
	\end{split}
	\end{flalign}
	
	
	$(ii)\Rightarrow (i).$  
	For any $(x,\bar{y}) \in cl(J)\times {Y} \times {Z}$, satisfying $\pi_{\bar{y}}(\bar{y}) +  G_{\bar{y}}(x, \bar{y})=0$, we would like to show  $(\pi_{\bar{k}\bar{j}}(\bar{y}) + {G}_{,\bar{k}\bar{j}}({x}, \bar{y}))\xi^{\bar{k}}\xi^{\bar{j}} \le 0$, for any $\xi\in \R^{n+1}$. Let $\bar{x} = (x, -1)$, there exist $\delta >0$ and a curve $t \in (-\delta, \delta) \longmapsto \bar{y}_t \in Y\times Z$, such that $\bar{y}_t|_{t=0} =\bar{y}$,  $\dot{\bar{y}}_t|_{t=0} = \xi$ and $\frac{d^2}{dt^2} \bar{G}_{\bar{i}, }(\bar{x}, \bar{y}_t) = 0$. Thus, equation \eqref{EqnSecondDiffbar{G}3} holds.
	
	
	
	Since $\pmb \Pi$ is concave, by Theorem $\ref{maintheorem}$ and Lemma  $\ref{LemmaProfitConcavity}$ as well as equation \eqref{EqnSecondDiffbar{G}3}, we have 
	\begin{flalign*}
	0 \ge& \big(\pi_{\bar{k}\bar{j}}(\bar{y})- \pi_{\bar{l}}(\bar{y}) \bar{G}^{\bar i,\bar l}(\bar{x}, \bar{y})\bar{G}_{\bar{i},\bar{k}\bar{j}}(\bar{x}, \bar{y})\big)\xi^{\bar{k}}\xi^{\bar{j}} \\
	=& \big(\pi_{\bar{k}\bar{j}}(\bar{y}) - \bar{G}_{,\bar{k}\bar{j}}({\bar{x}}, \bar{y}) + \bar{G}_{,\bar{k}\bar{j}}({\bar{x}}, \bar{y})-  \bar{G}_{,\bar{l}}({\bar{x}}, \bar{y})\bar{G}^{\bar i,\bar l}(\bar{x}, \bar{y})\bar{G}_{\bar{i},\bar{k}\bar{j}}(\bar{x}, \bar{y})\big)\xi^{\bar{k}}\xi^{\bar{j}} \\
	=& \big(\pi_{\bar{k}\bar{j}}(\bar{y}) - \bar{G}_{,\bar{k}\bar{j}}({\bar{x}}, \bar{y})\big)\xi^{\bar{k}}\xi^{\bar{j}} + \frac{d^2}{dt^2}\bigg|_{t=0} \bar{G}(\bar{x}, \bar{y}_t)\\
	=&\big(\pi_{\bar{k}\bar{j}}(\bar{y}) - \bar{G}_{,\bar{k}\bar{j}}({\bar{x}}, \bar{y})\big)\xi^{\bar{k}}\xi^{\bar{j}} \\
	=& (\pi_{\bar{k}\bar{j}}(\bar{y}) + {G}_{,\bar{k}\bar{j}}({x}, \bar{y}))\xi^{\bar{k}}\xi^{\bar{j}},
	\end{flalign*}
	which completes the proof.
\end{proof}


The following remark provides an equivalent condition for  uniform concavity of principal's maximization problem. Its proof is very similar to that of the above proposition.
\medskip

\begin{remark}\label{R:B2}
	In addition to the hypotheses of Proposition \ref{bar{G}^*-Concavity2}, when
	$\bar{z}=+\infty$ assume the homeomorphisms of \Gone\ are uniformly bi-Lipschitz. Then
	the following statements are equivalent:
	\begin{enumerate}[(i)]
		\item $\pi_{\bar{y}\bar{y}}(\bar{y}) + G_{\bar{y}\bar{y}}(x, \bar{y})$ is  uniformly negative definite 
		for all $(x, \bar{y}) \in cl(J)\times Y \times Z$ such that $\pi_{\bar{y}}(\bar{y}) + G_{\bar{y}}(x, \bar{y})=0$;
		\item $\pmb \Pi$ is uniformly concave  on $\mathcal{U} \subset W^{1,2}(X,d\mu)$, uniformly for all $\mu\ll \mathcal{L}^m$.
	\end{enumerate}
\end{remark}



When $m=n$, $G(x,y,z)= b(x,y)-z \in C^3(cl(X\times Y \times Z)
)$ satisfies \Gzero-\Geight,  and $\pi(y,z)=z-a(y) \in C^2( cl( Y \times Z)
)$, then Corollary \ref{bar{G}^*-Concavity} shows $b^*$-convexity of $a$ is a sufficient condition for
concavity of $\pmb \Pi$  for all $\mu\ll \mathcal{L}^m$. One may wonder under what hypotheses it would become a necessary condition as well. From Theorem A.1 in \cite{KimMcCann10}, under the same assumptions as above, the manufacturing cost $a$ is $b^*$-convex if and only if it satisfies the following local $b^*$-convexity hypothesis: $D^2a(y)\ge D^2_{yy}b(x,y)$ whenever $Da(y) = D_{y}b(x,y)$. 
\medskip

Combined with Proposition \ref{bar{G}^*-Concavity2}, we have the following corollary.
\medskip

\begin{corollary}
	Adopting the terminology of Figalli-Kim-McCann \cite{FigalliKimMcCann11},  i.e. (B0)-(B3), $G(x,y,z) = b(x,y)- z \in C^3(cl(X\times Y \times Z)$ and 
	$\pi(x,y,z) = z -a(y) \in C^2( cl( Y \times Z)$, then $a(y)$ is $b^*$-convex if  and only if $\mathbf \Pi$ is concave on $\mathcal U$ and for every $y \in Y$, there exists $x \in cl(X)$ such that $Da(y) = D_y b(x,y)$.
\end{corollary}

\begin{proof}
	Assume $a$ is $b^*$-convex, by definition, there exists a function  $a^*: cl(X) \rightarrow \R$, such that for any $y\in Y$, $a(y) = \max_{x\in cl(X)} b(x,y) - a^*(x)$. Therefore, for any $y_0 \in Y$, there exists $x_0 \in cl(X)$, such that $a(y) \ge b(x_0,y) - a^*(x_0)$ for all $y \in Y$, with equality holds at $y = y_0$. This implies, $Da(y_0) = D_y b(x_0, y_0)$. Taking $J=cl(X)$ and applying Proposition \ref{bar{G}^*-Concavity2}, we have concavity of $\mathbf \Pi$, since local $b^*$-convexity of $a$ is automatically satisfied by a $b^*$-convex function $a$. 
	
	On the other hand, assuming $\mathbf \Pi$ is concave on $\mathcal U$ and for every $y \in Y$, there exists $x \in cl(X)$ such that $Da(y) = D_y b(x,y)$, Proposition \ref{bar{G}^*-Concavity2} implies local $b^*$-convexity of $a$. Together with Theorem A.1 in \cite{KimMcCann10}, we know $a$ is $b^*$-convex.
\end{proof}




































\chapter{Analytic representation of condition \Gthree}\label{chapter:analytic_representation}



\section{A fourth-order differential re-expression of condition \Gthree}\label{section:4thorder}


%Lastly, We introduce a theorem expressing our crucial hypothesis \Gthree \  in a fourth-order differential form, which is %the main result in Appendix \ref{A:4thorder}.

In our convexity argument, hypothesis \Gthree\ plays a crucial role. In this chapter, we 
localize this hypothesis using differential calculus. Inspired by and strongly connected with Trudinger's theory of generalized prescribed Jacobian equations, this form is analogous to the non-negative cross-curvature condition (B3) 
of Figalli-Kim-McCann \cite{FigalliKimMcCann11}, a fourth order condition in the spirit of the Ma-Trudinger-Wang \cite{MaTrudingerWang05}. For another formulation, see e.g.\ \cite{GuillenKitagawa15}.
\medskip

%The resulting expression
%shows it to be a direct analog of the non-negative cross-curvature (B3) from \cite{FigalliKimMcCann11},
%which in turn was inspired by \cite{MaTrudingerWang05}. 

Apart from the assumptions of Section \ref{section:Hypotheses}, we shall need the non-degeneracy
condition assumed in Section \ref{section:concavity}:
\begin{itemize}
	\item[\Gsix] $G\in C^2(cl(X\times Y \times Z)
	)$, and $D_{\bar{x},\bar{y}}(\bar{G})(x,-1,y,z)$ has full rank, for each $(x,y,z)\in cl(X\times Y\times Z)$. 
\end{itemize}
For this and the next chapter only, we assume the dimensions of spaces $X$ and $Y$ are equal, i.e. $m=n$,
so that the matrix mentioned in \Gsix\ is square.
We shall also need to extend the twist and convex range hypotheses \Gone\ and \Gtwo\  
to the function $H$ in place of $G$. This is equivalent to assuming:

\begin{itemize}
	\item[\Gseven] For each $(y, z)\in cl( Y \times Z)$ the map $x \in X \longmapsto \frac{G_y}{G_z}(\cdot, y,z)$ is one-to-one;
	\item[\Geight] its range $X_{(y,z)} := \frac{G_y}{G_z}(X,y,z) \subset \R^n$ is convex.
\end{itemize}

We can now state:
\medskip

\begin{theorem}\label{theorem:4thorder}
	Assume \Gzero-\Gtwo\ and \Gfour-\Geight. If, in addition, $G\in C^4(cl(X\times Y \times Z)
	)$,  then the following statements are equivalent:
	\begin{enumerate}[(i)]
		\item \Gthree.
		
		\item 	 For any given  $x_0, x_1\in X$, any curve $(y_t, z_t) \in cl( Y \times Z)$ connecting $(y_0,z_0)$ and $(y_1, z_1)$, we have 
		\begin{equation*}
		\frac{\partial^2}{\partial s^2 }\Biggl(\frac1{G_z(x_s, y_t, z_t)}\frac{\partial^2}{\partial t^2} G(x_s,y_t,z_t) \Biggr)\Bigg|_{t=t_0}\le 0,
		\end{equation*}
		whenever  { $s\in [0,1] \longmapsto \frac{G_y}{G_z}(x_s, y_{t_0}, z_{t_0})$} forms an affinely parametrized line segment for some $t_0 \in [0,1]$.
		
		
		\item For any given curve $x_s\in X$ connecting $x_0$ and $x_1$,  any $(y_0, z_0),  (y_1, z_1) \in cl( Y \times Z) $, we have 
		\begin{equation*}
		\frac{\partial^2}{\partial s^2 }\Biggl(\frac{1}{G_z(x_s, y_t, z_t)}\frac{\partial^2}{\partial t^2} G(x_s,y_t,z_t) \Biggr)\Bigg|_{s=s_0}\le 0,
		\end{equation*}
		whenever $t\in [0,1] \longmapsto (G_x, G)(x_{s_0}, y_t, z_t)$  forms an affinely parametrized line segment for some $s_0\in [0,1]$.
		
	\end{enumerate} 
\end{theorem}

The proof of this theorem and its embellishments are represented in the following section:



\section{Proof and variations on Theorem \ref{theorem:4thorder}}\label{A:4thorder}




\begin{proof}[Proof of Theorem \ref{theorem:4thorder}]
	$\mathrm{(i)}\Rightarrow \mathrm{(ii)}.$ Suppose for some $t_0 \in [0,1]$, $s \in [0,1] \longmapsto \frac{G_y}{G_z}(x_s, y_{t_0}, z_{t_0})$ forms an affinely parametrized line segment.
	
	For any fixed $s_0\in [0,1]$, consider $x_{s_0}\in X$, there is a $G$-segment $(x_{s_0}, y_t^{s_0},$ $ z_t^{s_0})$ passing through $(x_{s_0}, y_{t_0}, z_{t_0})$ at $t=t_0$ with the same tangent vector as $(x_{s_0}, y_t, z_t)$ at $t=t_0$, i.e., there exists
	another curve $(y_t^{s_0}, z_t^{s_0}) \in cl( Y \times Z)$, such that $(y_t^{s_0},z_t^{s_0})\mid _{t=t_0} = (y_t,z_t)\mid _{t=t_0}$,  $(\dot{y_t}^{s_0},\dot{z_t}^{s_0})\mid _{t=t_0} \parallel (\dot{y}_t,\dot{z}_t)\mid _{t=t_0}$, and $(G_x, G)(x_{s_0},y_t^{s_0},z_t^{s_0}) = (1-t)(G_x, G)(x_{s_0},y_0^{s_0},z_0^{s_0})+t (G_x, G)(x_{s_0},y_1^{s_0},z_1^{s_0})$. 
	
	Computing the fourth mixed derivative yields
	\begin{flalign*}
	&\frac{\partial^2}{\partial s^2 }\Biggl(\frac{1}{G_z(x_s, y_t, z_t)}\frac{\partial^2}{\partial t^2} G(x_s,y_t,z_t) \Biggr)\\
	=~&\frac{\partial^2}{\partial s^2}\Biggl(\frac{1}{G_z}\Biggr) \frac{\partial^2}{\partial t^2} G + 2 \frac{\partial}{\partial s} \Biggl(\frac{1}{G_z}\Biggr) \frac{\partial^3}{\partial s \partial t^2} G + \frac{1}{G_z} \frac{\partial^4}{\partial s^2 \partial t^2} G  \\
	=~& [-(G_z)^{-2}G_{i,z} \ddot{x_s}^{i} - (G_z)^{-2} G_{ij,z}\dot{x_s}^i \dot{x_s}^j+ 2(G_z)^{-3} G_{i,z} G_{j,z} \dot{x_s}^i \dot{x_s}^j]\cdot [G_{,k}\ddot{y_t}^{k}+G_z\ddot{z}_t+G_{,kl}\dot{y_t}^k \dot{y_t}^l\\
	&\hspace{0.5cm} + 2G_{,kz} \dot{y_t}^k \dot{z}_t +G_{zz} (\dot{z}_t)^2]\\
	&+ 2[-(G_z)^{-2}G_{i,z}\dot{x_s}^i]\cdot[G_{j,k}\dot{x_s}^j\ddot{y_t}^{k}+G_{j,z}\dot{x_s}^j\ddot{z}_t+G_{j,kl}\dot{x_s}^j\dot{y_t}^k \dot{y_t}^l + 2G_{j,kz}\dot{x_s}^j \dot{y_t}^k \dot{z}_t +G_{j,zz} \dot{x_s}^j(\dot{z}_t)^2]\\
	&+(G_z)^{-1}[G_{i,k}\ddot{x_s}^i\ddot{y_t}^{k} +G_{ij,k}\dot{x_s}^i\dot{x_s}^j\ddot{y_t}^{k} +G_{i,z}\ddot{x_s}^i\ddot{z}_t +G_{ij,z}\dot{x_s}^i\dot{x_s}^j\ddot{z}_t  +G_{i,kl}\ddot{x_s}^i\dot{y_t}^k \dot{y_t}^l +G_{ij,kl}\dot{x_s}^i\dot{x_s}^j\dot{y_t}^k \dot{y_t}^l \\
	&\hspace{0.5cm} +2G_{i,kz}\ddot{x_s}^i\dot{y_t}^k\dot{z}_t 
	 +2G_{ij,kz}\dot{x_s}^i\dot{x_s}^j\dot{y_t}^k\dot{z}_t +G_{i,zz}\ddot{x_s}^i(\dot{z}_t)^2 +G_{ij,zz}\dot{x_s}^i\dot{x_s}^j(\dot{z}_t)^2]\\ 
	=~& [((G_z)^{-1}G_{i,k}-(G_z)^{-2}G_{i,z}G_{,k})\ddot{x_s}^i +((G_z)^{-1}G_{ij,k}- (G_z)^{-2} G_{,k}G_{ij,z} -2(G_z)^{-2}G_{i,z}G_{j,k} \\
		& \hspace{0.5cm}   +2(G_z)^{-3}G_{,k}G_{i,z} G_{j,z}) \dot{x_s}^i \dot{x_s}^j]\ddot{y_t}^k \\
		&+ [(G_z)^{-1}G_{i,kl}-(G_z)^{-2}G_{i,z}G_{,kl}]\ddot{x_s}^i\dot{y_t}^k \dot{y_t}^l \\
		&+ [(G_z)^{-1}G_{ij,kl}-(G_z)^{-2} G_{ij,z}G_{,kl}+ 2(G_z)^{-3} G_{i,z} G_{j,z}G_{,kl} -2(G_z)^{-2}G_{i,z}G_{j,kl}]\dot{x_s}^i \dot{x_s}^j\dot{y_t}^k \dot{y_t}^l\\
		&+[2(G_z)^{-1}G_{i,kz}-2(G_z)^{-2}G_{i,z}G_{,kz}] \ddot{x_s}^{i}\dot{y_t}^k\dot{z}_t\\
		&+[2(G_z)^{-1}G_{ij,kz}-2(G_z)^{-2} G_{ij,z}G_{,kz}+ 4(G_z)^{-3}G_{i,z}G_{j,z}G_{,kz}-4(G_z)^{-2}G_{i,z}G_{j,kz}]\dot{x_s}^i\dot{x_s}^j\dot{y_t}^k\dot{z}_t\\
			&+[(G_z)^{-1}G_{i,zz}-(G_z)^{-2}G_{i,z}G_{zz}] \ddot{x_s}^{i}(\dot{z}_t)^2 \\
			&+[(G_z)^{-1}G_{ij,zz} -(G_z)^{-2} G_{ij,z}G_{zz} +2(G_z)^{-3} G_{i,z} G_{j,z}G_{zz} -2(G_z)^{-2}G_{i,z}G_{j,zz}]\dot{x_s}^i \dot{x_s}^j(\dot{z}_t)^2.
	\end{flalign*}

	
	The coefficient of $\ddot{y_t}^k$ vanishes when this expression is evaluated at $t =  t_0$, due to the assumption that $s \in [0,1] \longmapsto \frac{G_y}{G_z}(x_s, y_{t_0}, z_{t_0})$ forms an affinely parametrized line segment, which implies
	\begin{flalign*}
	0=~& \frac{\partial^2}{\partial s^2} \frac{G_y}{G_z}(x_s, y_{t_0}, z_{t_0})\\
	=~&[((G_z)^{-1}G_{i,y}-(G_z)^{-2}G_{i,z}G_{,y})\ddot{x}^i +((G_z)^{-1}G_{ij,y}- (G_z)^{-2} G_{,y}G_{ij,z} \\
	&\hspace{0.5cm}-2(G_z)^{-2}G_{i,z}G_{j,y}  +2(G_z)^{-3}G_{,y}G_{i,z} G_{j,z}) \dot{x}^i \dot{x}^j]
	\end{flalign*}
	for all  $s\in [0,1]$.
	
	Since $(\dot{y_t}^{s_0},\dot{z_t}^{s_0})|_{t=t_0} \parallel (\dot{y}_t,\dot{z}_t)|_{t=t_0}$, there exists some constant $C_1>0$, such that $(\dot{y}_t,\dot{z}_t)|_{t=t_0} =C_1(\dot{y_t}^{s_0},\dot{z_t}^{s_0})|_{t=t_0} $. Moreover, since  $ (y_t,z_t)\mid _{t=t_0}=(y_t^{s_0},z_t^{s_0})\mid _{t=t_0} $, we have
	\begin{flalign*}
	&\frac{\partial^2}{\partial s^2 }\Biggl(\frac{1}{G_z(x_s, y_t, z_t)}\frac{\partial^2}{\partial t^2} G(x_s,y_t,z_t) \Biggr) \Bigg|_{t=t_0} \\
	=~& C_1^2	\frac{\partial^2}{\partial s^2 }\Biggl(\frac{1}{G_z(x_s, y_t^{s_0}, z_t^{s_0})}\frac{\partial^2}{\partial t^2} G(x_s,y_t^{s_0},z_t^{s_0}) \Biggr) \Bigg|_{t=t_0} .
	\end{flalign*}
	Denote $g(s):=\frac{\partial^2}{\partial t^2}|_{t=t_0} G(x_s,y_t^{s_0},z_t^{s_0})$, for $s\in[0,1]$. Since $(x_{s_0},y_t^{s_0},z_t^{s_0})$ is a $G$-segment, by \Gthree, we have $g(s) \ge 0$, for all $s \in [0,1]$.
	By definition of $(y_t^{s_0},z_t^{s_0})$, it is clear that $g(s_0) =0$.  The first- and second-order
	conditions for an interior minimum then give $g'(s_0) =0 \le g''(s_0)$; (in fact $g'(s_0)=0$ also follows
	directly from the definition of a $G$-segment).
	
	By the assumption \Gfour,  we have $G_z <0$, thus, 
	\begin{flalign*}
	~&\frac{\partial^2}{\partial s^2 }\Biggl(\frac{1}{G_z(x_s, y_t, z_t)}\frac{\partial^2}{\partial t^2} G(x_s,y_t,z_t) \Biggr) \Bigg|_{(s,t)=(s_0,t_0)}\\
	=~& C_1^2 	\frac{\partial^2}{\partial s^2 }\Biggl(\frac{1}{G_z(x_s, y_t^{s_0}, z_t^{s_0})}\frac{\partial^2}{\partial t^2} G(x_s,y_t^{s_0},z_t^{s_0}) \Biggr) \Bigg|_{(s,t)=(s_0,t_0)} \\
	=~&C_1^2\frac{\partial^2}{\partial s^2}\Bigg|_{(s,t)=(s_0,t_0)} \Biggl(\frac{1}{G_z}(x_s,y_t^{s_0},z_t^{s_0})\Biggr) g(s_0)\\
	~&+ 2C_1^2 \frac{\partial}{\partial s}\Bigg|_{(s,t)=(s_0,t_0)} \Biggl(\frac{1}{G_z(x_s,y_t^{s_0},z_t^{s_0})}\Biggr) g'(s_0) + \frac{C_1^2}{G_z(x_s,y_t^{s_0},z_t^{s_0})} g''(s_0)  \\
	\le ~& 0.
	\end{flalign*}
	
	$\mathrm{(ii)}\Rightarrow \mathrm{(i)}.$ For any fixed $x_0 \in X$ and $G$-segment $(x_0, y_t, z_t)$,  we need to show $\frac{\partial^2}{\partial t^2}G(x_1, y_t, z_t) \ge 0$,  for all $t \in [0,1]$ and $x_1 \in X$.
	
	For any fixed $t_0 \in [0,1]$ and $x_1 \in X$, define $x_s$ as the solution $\hat{x}$ to the equation
	\begin{equation}
	\frac{G_y}{G_z}(\hat{x}, y_{t_0},z_{t_0}) = (1-s) \frac{G_y}{G_z}(x_0, y_{t_0},z_{t_0}) +s \frac{G_y}{G_z}(x_1, y_{t_0},z_{t_0}).
	\end{equation}
	
	By \Gseven\ and \Geight, $x_s$ is uniquely determined for each $s\in (0,1)$. In addition, $x_0$ and $x_1$ satisfy the above equation for $s =0$ and $s=1$, respectively.  
	
	Define $g(s):=
	\frac{1}{G_z(x_s,y_t,z_t)}\frac{\partial^2}{\partial t^2}G(x_s, y_t, z_t)
	\Big|_{t=t_0}$ for $s\in [0,1]$.
	
	Then $g(0) =0 = g'(0)$ from the two conditions defining a $G$-segment. 
	
	In our setting,  $s\in [0,1] \longmapsto \frac{G_y}{G_z}(x_s, y_{t_0}, z_{t_0})$ forms an affinely parametrized line segment, thus
	$0\ge \frac{\partial^2}{\partial s^2 }\Bigl(\frac{1}{G_z(x_s, y_t, z_t)}\frac{\partial^2}{\partial t^2} G(x_s,y_t,z_t) \Bigr)\Big|_{t=t_0} = g''(s)$ for all $ s \in [0,1]$ by hypothesis $\mathrm{(ii)}$.
	
	Hence $g$ is concave in $[0,1]$, and $g(0)=0$ is a critical point, thus $g(1)\le 0$. Since $G_z<0$ this implies $\frac{\partial^2}{\partial t^2}\Big|_{t=t_0}G(x_1, y_t, z_t) \ge 0$ for any $t_0 \in [0,1]$ and $x_1 \in X$,
	as desired.
	\medskip
	
	$\mathrm{(i)}\Rightarrow \mathrm{(iii)}.$
	For any fixed $s_0\in [0,1]$, suppose $t\in [0,1] \longmapsto (G_x, G)(x_{s_0}, y_t,$ $ z_t)$  forms an affinely parametrized line segment. For any fixed $t_0 \in [0,1]$, define $g(s):=\Big(\frac{1}{G_z(x_s,y_t,z_t)}\frac{\partial^2}{\partial t^2}G(x_s, y_t, z_t)\Big)\Big|_{t=t_0}$, for all $s \in [0,1]$. By \Gthree-\Gfour, we know $g(s)\le 0$, for all $s\in [0,1]$. By the definition of $(y_t, z_t)$, we have $g(s_0)=g'(s_0) =0$. Thus $g''(s_0)\le 0$.
	\medskip
	
	$\mathrm{(iii)}\Rightarrow \mathrm{(i)}.$ For any fixed $x_0 \in X$, suppose $(x_0, y_t^{0}, z_t^{0})$ is a $G$-segment, then we need to show $\frac{\partial^2}{\partial t^2}G(x_1, y_t^{0}, z_t^{0}) \ge 0$,  for all $t \in [0,1], x_1 \in X$.
	
	For any fixed $t_0 \in [0,1]$, $x_1 \in X$, define $x_s$ as the solution $\hat{x}$ of equation
	\begin{equation*}
	\frac{G_y}{G_z}(\hat{x}, y_{t_0}^{0},z_{t_0}^{0}) = (1-s) \frac{G_y}{G_z}(x_0, y_{t_0}^{0},z_{t_0}^{0}) +s \frac{G_y}{G_z}(x_1, y_{t_0}^{0},z_{t_0}^{0}).
	\end{equation*}
	
	By \Gseven\ and \Geight, $x_s$ is uniquely determined for each $s\in (0,1)$. In addition, $x_0$ and $x_1$ satisfy the above equation for $s =0$ and $s=1$ respectively. 
	
	Define $g(s):=\frac{1}{G_z(x_s,y_t^{0},z_t^{0})}\frac{\partial^2}{\partial t^2}G(x_s, y_t^{0}, z_t^{0})\Big|_{t=t_0}$, for $s\in [0,1]$.
	
	Then $g(0) =g'(0)  =0 $ by the two conditions defining a $G$-segment. 
	
	
	For any fixed $s_0 \in [0,1]$, there is a $G$-segment $(x_{s_0}, y_t^{s_0}, z_t^{s_0})$ passing through $(x_{s_0}, y_{t_0}^{0}, z_{t_0}^{0})$ at $t=t_0$ with the same tangent vector as $(x_{s_0}, y_t^{0}, z_t^{0})$ at $t=t_0$, i.e., there exists 
	another curve $(y_t^{s_0}, z_t^{s_0}) \in cl( Y \times Z)$ and some constant $C_2>0$, such that $(y_t^{s_0},z_t^{s_0})\mid _{t=t_0} = (y_t^{0},z_t^{0})\mid _{t=t_0}$,  $(\dot{y_t}^{s_0},\dot{z_t}^{s_0})\mid _{t=t_0} = \frac{1}{C_2} (\dot{y}_t^{0},\dot{z}_t^{0})\mid _{t=t_0}  $, and $(G_x, G)(x_{s_0},y_t^{s_0},z_t^{s_0}) = (1-t)(G_x, G)(x_{s_0},y_0^{s_0},z_0^{s_0})+t (G_x, G)(x_{s_0},y_1^{s_0},z_1^{s_0})$. 
	
	Computing the mixed fourth derivative yields
	\begin{flalign*}
	&\frac{\partial^2}{\partial s^2 }\Biggl(\frac{1}{G_z(x_s, y_t^{0}, z_t^{0})}\frac{\partial^2}{\partial t^2} G(x_s,y_t^{0},z_t^{0}) \Biggr) \Bigg|_{(s,t)=(s_0, t_0)} \\
	=~& C_2^2	\frac{\partial^2}{\partial s^2 }\Biggl(\frac{1}{G_z(x_s, y_t^{s_0}, z_t^{s_0})}\frac{\partial^2}{\partial t^2} G(x_s,y_t^{s_0},z_t^{s_0}) \Biggr) \Bigg|_{(s,t)=(s_0, t_0)} ,
	\end{flalign*}
	where the equality is derived from the condition that $s \in [0,1] \longmapsto \frac{G_y}{G_z}(x_s, y_{t_0},$ $ z_{t_0})$ forms an affinely parametrized line segment,  $(y_t^{s_0},z_t^{s_0})\mid _{t=t_0} = (y_t,z_t)\mid _{t=t_0}$ and $ (\dot{y}_t^{0},\dot{z}_t^{0})\mid _{t=t_0} = C_2(\dot{y_t}^{s_0},\dot{z_t}^{s_0})\mid _{t=t_0} $. Moreover, the latter expression is non-positive by assumption $\mathrm{(iii)}$.
	
	Thus $g''(s_0)\le 0$ for all $s_0 \in [0,1]$. Since $g$ is concave in $[0,1]$, and $g(0)=0$ is a critical point, we have $g(1)\le 0$. Thus $G_z<0$ implies $\frac{\partial^2}{\partial t^2}\Big|_{t=t_0}G(x_1, y_t^{0}, z_t^{0}) \ge 0$ for all $t_0 \in [0,1]$ and $x_1 \in X$, as desired. 
\end{proof}


For strictly concavity of the profit functional, one might need a strict version of hypothesis \Gthree:
\begin{itemize}
\item[\Gthree$_{s}$] For each $x,x_0 \in X$ and $x\ne x_0$, assume $t \in [0,1] \longmapsto G(x, y_t, z_t)$ is strictly convex along all $G$-segments $(x_0, y_t, z_t)$ defined in (\ref{$G$-segment}).
\end{itemize}


\begin{remark}\label{(C5)_s and (C5)_u}
	Strict inequality in $\mathrm{(ii)}$ [or $\mathrm{(iii)}$] implies \Gthree$_{s}$ but the reverse is not necessarily true, i.e. \Gthree$_{s}$ is intermediate in strength between \Gthree\ and strict inequality version of $\mathrm{(ii)}$ [or $\mathrm{(iii)}$]. Besides, strict inequality versions of $\mathrm{(ii)}$ and $\mathrm{(iii)}$ are equivalent, and denoted by $\Gthree_{u}$.
	
	Note inequality \eqref{foc} below and its strict and uniform versions \Gthree$_{s}$ and \Gthree$_{u}$
	precisely generalize of the analogous hypotheses $(B3)$, $(B3)_{s}$ and $(B3)_{u}$ from the quasilinear case in
	\cite{FigalliKimMcCann11}.
\end{remark}
\begin{proof}
	We only show strict inequality of $\mathrm{(ii)}$ implies that of $\mathrm{(iii)}$ here since the other direction is similar.
	
	For any fixed $s_0\in [0,1]$, suppose $t\in [0,1] \longmapsto (G_x, G)(x_{s_0}, y_t, z_t)$  forms an affinely parametrized line segment. For any fixed $t_0 \in [0,1]$, define ${x}_s^{t_0}$ as a solution to the equation
	$\frac{G_y}{G_z}({x}_{s}^{t_0}, y_{t_0},z_{t_0})$ $= (1-s) \frac{G_y}{G_z}({x}_{0}^{t_0}, y_{t_0},z_{t_0}) +s \frac{G_y}{G_z}({x}_{1}^{t_0}, y_{t_0},z_{t_0})$, with initial conditions ${x}_s^{t_0} |_{s=s_0}= x_{s_0}$ and $\dot{x}_s^{t_0}|_{s=s_0} = C_1 $ $\cdot \dot{x}_s|_{s=s_0}$, for some constant $C_1 >0$.	
	Thus, by strict inequality of $\mathrm{(ii)}$, we have 
	\begin{flalign*}
	0>~&\frac{\partial^2}{\partial s^2 }\Big(\frac1{G_z(x_s^{t_0}, y_t, z_t)}\frac{\partial^2}{\partial t^2} G(x_s^{t_0},y_t,z_t) \Big)\Big|_{(s,t)=(s_0, t_0)}\hspace{1.38cm}\\
	=~& \frac{\partial^2}{\partial s^2 }\Big(\frac1{G_z(x_s^{t_0}, y_t, z_t)}\Big)\frac{\partial^2}{\partial t^2} G(x_s^{t_0},y_t,z_t) \Big|_{(s,t)=(s_0, t_0)}\\
	~&+ \frac{\partial}{\partial s }\Big(\frac1{G_z(x_s^{t_0}, y_t, z_t)}\Big)\frac{\partial^3}{\partial s \partial t^2} G(x_s^{t_0},y_t,z_t) \Big|_{(s,t)=(s_0, t_0)}\\
	~&+ \frac1{G_z(x_s^{t_0}, y_t, z_t)}\frac{\partial^4}{\partial s^2 \partial t^2} G(x_s^{t_0},y_t,z_t) \Big|_{(s,t)=(s_0, t_0)}\\
	=~& 
	-\frac{G_{x,z}(x_s^{t_0},y_t,z_t)}{G_z^2(x_s^{t_0}, y_t, z_t)}\frac{\partial^2}{\partial t^2} G_x(x_s^{t_0},y_t,z_t) (\dot{x}_s^{t_0})^2 \Big|_{(s,t)=(s_0, t_0)}\\
	~&+ \frac1{G_z(x_s^{t_0}, y_t, z_t)}\frac{\partial^2}{\partial t^2} G_{xx}(x_s^{t_0},y_t,z_t)(\dot{x}_s^{t_0})^2  \Big|_{(s,t)=(s_0, t_0)}\\
	=~&C_1^2\Big[-\frac{G_{x,z}(x_{s},y_t,z_t)}{G_z^2(x_{s}, y_t, z_t)}\frac{\partial^2}{\partial t^2} G_x(x_{s},y_t,z_t) (\dot{x}_s)^2 \Big|_{(s,t)=(s_0, t_0)}\\
	~&+ \frac1{G_z(x_{s}, y_t, z_t)}\frac{\partial^2}{\partial t^2} G_{xx}(x_{s},y_t,z_t)(\dot{x}_s)^2  \Big|_{(s,t)=(s_0, t_0)}\Big]\\
	=~&C_1^2\frac{\partial^2}{\partial s^2 }\Big(\frac1{G_z(x_s, y_t, z_t)}\frac{\partial^2}{\partial t^2} G(x_s,y_t,z_t) \Big)\Big|_{(s,t)=(s_0, t_0)}.
	\end{flalign*}
	Here we use the initial condition ${x}_s^{t_0} |_{s=s_0}= x_{s_0}$ and $\dot{x}_s^{t_0}|_{s=s_0} = C_1 \dot{x}_s|_{s=s_0}$.  Besides, since $(x_{s_0}, y_t, z_t)$ forms a $G$-segment, therefore we have 
	\begin{flalign*}
	\frac{\partial^2}{\partial t^2} G(x_s^{t_0},y_t,z_t) \Big|_{(s,t)=(s_0, t_0)}=\frac{\partial^2}{\partial t^2} G(x_s,y_t,z_t) \Big|_{(s,t)=(s_0, t_0)}=0,
	\end{flalign*}
	\begin{flalign*}
	\text{ and }\ \  \frac{\partial^2}{\partial t^2} G_x(x_s^{t_0},y_t,z_t) \Big|_{(s,t)=(s_0, t_0)}=\frac{\partial^2}{\partial t^2} G_x(x_s,y_t,z_t) \Big|_{(s,t)=(s_0, t_0)}=0.
	\end{flalign*}
	
	From the above inequality and $C_1 >0$, one has 
	\begin{equation*}
	\frac{\partial^2}{\partial s^2 }\Big(\frac1{G_z(x_s, y_t, z_t)}\frac{\partial^2}{\partial t^2} G(x_s,y_t,z_t) \Big)\Big|_{(s,t)=(s_0, t_0)}<0,
	\end{equation*} 
	whenever $\dot{x_s}|_{s=s_0}$ and $(\dot{y_t},\dot{z}_t)|_{t=t_0}$ are nonzero. Since this inequality holds for each fixed $t_0 \in [0,1]$, the strict version of  $\mathrm{(iii)}$ is proved.
\end{proof}

Combining $\mathrm{(ii)}$ and $\mathrm{(iii)}$, one can conclude they are also equivalent to the following statement $\mathrm{(iv)}$.


\begin{corollary}\label{FourthOrder4}
	Assuming \Gzero-\Gtwo, \Gfour-\Geight\ and $G\in C^4(cl(X\times Y \times Z)
	)$,  then \Gthree\ is equivalent to the following statement:\\
	\begin{enumerate}[(i)]
	\item[$\mathrm{(iv)}$] For any given curve $x_s\in X$ connecting $x_0$ and $x_1$,  and any curve $(y_t, z_t) \in cl( Y \times Z)$ connecting $(y_0,z_0)$ and $(y_1, z_1)$, we have 
	\begin{equation}\label{foc}
	\frac{\partial^2}{\partial s^2 }\Biggl(\frac{1}{G_z(x_s, y_t, z_t)}\frac{\partial^2}{\partial t^2} G(x_s,y_t,z_t) \Biggr)\Bigg|_{(s,t) = (s_0,t_0)}\le 0,
	\end{equation}
	whenever either of the two curves $t\in [0,1] \longmapsto (G_x, G)(x_{s_0}, y_t, z_t)$ and  $s\in [0,1] \longmapsto \frac{G_y}{G_z}(x_s, y_{t_0}, z_{t_0})$ forms an affinely parametrized line segment.
	\end{enumerate}
\end{corollary}	






\chapter{Geometric re-expression of \Gthree}\label{chapter:geometry}

\section{Introduction}

In optimal transportation theory, Ma-Trudinger-Wang \cite{MaTrudingerWang05} in 2005 gave sufficient conditions on a transportation cost to guarantee smoothness of the optimal transportation map, while Leoper \cite{Loeper09} showed these conditions are also necessary. In 2010, Kim-McCann \cite{KimMcCann10} expressed them via non-negativity of the sectional curvature of certain null-planes in a novel but natural pseudo-Riemannian geometry which was induced by the cost function on some product space.\medskip

In chapter \ref{chapter:analytic_representation}, we've shown \Gthree~is in fact a fourth order condition in the spirit of the Ma-Trudinger-Wang condition. Inspired by Kim-McCann \cite{KimMcCann10}, we will show in this chapter a geometric representation of \Gthree, which is non-negativity of the sectional curvature in a pseudo-Riemannian geometry induced by the utility $\barG$, up to the additional variable, on the product space $X\times \R \times Y \times \R$.\medskip

\section{Settings}\label{section:geometric_setting}

In this section, we will define the pseudo-Riemannian metric $g$ and calculate the Christoffel symbols $\Gamma$ and the curvature tensor $R$. \medskip

Let $\barG (x,w, y,z) = w G(x,y,z)$, $\barx = (x,w)$, $\bary = (y,z)$, and $\delta(\barx, \bary, \barx_{0}, \bary_{0}) = -\barG (\barx, \bary) -\barG(\barx_0, \bary_0) +\barG(\barx, \bary_0) + \barG(\barx_0, \bary)$.\medskip

For some fixed $(\barx_0, \bary_0)$, one can view the $\delta$ defined above as a function of variable $\Xi =(\barx, \bary)$ on the space $M = X \times \R \times Y \times \R$, with the following first order derivatives.
\begin{equation}
	\nabla_i \delta (\barx, \bary, \barx_0, \bary_0) = \begin{cases}
	-\barG_{i,}(\barx, \bary) + \barG_{i,}(\barx, \bary_0) , & i\le n+1\\
	-\barG_{,\bar{i}}(\barx, \bary) + \barG_{,\bar{i}}(\barx_0, \bary), & i> n+1 \\
	\end{cases}
\end{equation}
Here we adopt this notation $\bar{i}:= i-(n+1)$ for $n+1 <i\le 2(n+1)$. For notation convention, we use comma to separate subscripts of $\barG$, which correspond to the derivatives with respect to the variables $\barx$ in spaces $X \times \R$ (before comma) and $\bary$ in $Y \times \R$ (after comma), respectively.\medskip

If $G\in C^2$, for each fixed $(\barx_0, \bary_0)$, since the first derivative of $\delta$ vanishes at $(\barx, \bary) = (\barx_0, \bary_0)$,  
the second order derivatives of $\delta$ at $\Xi_0 = (\barx_0, \bary_0)$ is well-defined and can be written as follows.
\begin{equation}\label{pseudo_metric}
	\nabla_{ij} \delta (\barx, \bary, \barx_0, \bary_0){\big|}_{(\barx, \bary) = (\barx_0, \bary_0)}
	=\begin{cases}
	-\barG_{i,\bar{j}}(\barx_0, \bary_0),& i\le n+1 < j\\
	-\barG_{j, \bar{i}}(\barx_0, \bary_0),& i>n+1 \ge j\\
	0,& \text{ otherwise}
	\end{cases}
\end{equation}  

Let $T_{\Xi_0}M$ denote the tangent space to $M$ at ${\Xi_0}$. Define the pseudo-Riemannian metric $ g_{\Xi_0}: T_{\Xi_0}M\times T_{\Xi_0}M \rightarrow \R$ at $\Xi_0$ on $M$ to be the above $2(n+1) \times 2(n+1)$ symmetric matrix.\medskip

One can calculate the Christoffel symbols using the following formula with Einstein summation convention and $g^{ml}$ being the inverse matrix of $g$:
\begin{equation}\label{Gamma}
	\Gamma_{ij}^m = \frac{1}{2} g^{ml}(\frac{\partial g_{il}}{\partial \Xi_j} + \frac{\partial g_{jl}}{\partial \Xi_i} - \frac{\partial g_{ij}}{\partial \Xi_l}).
\end{equation}

Here is a calculation of each component.
\begin{equation}
g_{il;j}:=	\frac{\partial g_{il}}{\partial \Xi_j} = \begin{cases}
-	\barG_{i, \bar{j}\bar{l}}, & i \le n+1 < j, l; \\
-	\barG_{ij, \bar{l}}, & i,j \le n+1 <l; \\
-	\barG_{l, \bar{i}\bar{j}}, & l \le n+1 <i,j; \\
-	\barG_{jl, \bar{i}}, & j,l \le  n+1<i ; \\
	0 & \text{ otherwise}.
	\end{cases}
\end{equation}

Similarly, one has
\begin{equation}
g_{lj;i}:=\frac{\partial g_{lj}}{\partial \Xi_i} = \begin{cases}
-\barG_{j, \bar{i}\bar{l}}, & j \le n+1 < i, l; \\
-\barG_{ij, \bar{l}}, & i,j \le n+1 <l; \\
-\barG_{l, \bar{i}\bar{j}}, & l \le n+1 < i,j; \\
-\barG_{il, \bar{j}}, &  i,l \le n+1 <j ; \\
0 & \text{ otherwise};
\end{cases}
\end{equation}

\begin{equation}
g_{ij;l}:=\frac{\partial g_{ij}}{\partial \Xi_l} = \begin{cases}
-\barG_{i, \bar{j}\bar{l}}, & i \le n+1 < j, l; \\
-\barG_{il, \bar{j}}, & i,l \le n+1 <j; \\
-\barG_{j, \bar{i}\bar{l}}, & j \le n+1 < i, l ; \\
-\barG_{jl, \bar{i}}, &  j,l \le n+1 < i; \\
0 & \text{ otherwise}.
\end{cases}
\end{equation}


Therefore, putting together these three terms, one has
\begin{equation}
	\frac{\partial g_{il}}{\partial \Xi_j} +\frac{\partial g_{lj}}{\partial \Xi_i}
	-\frac{\partial g_{ij}}{\partial \Xi_l}=
	\begin{cases}
-	2\barG_{ij, \bar{l}}, & i,j \le n+1 < l; \\
-	2\barG_{l, \bar{i}\bar{j}}, & i,j >n+1 \ge l; \\
	0, & \text{otherwise}.
	\end{cases}
\end{equation}

Denote $\barG^{m, \bar{l}}$ as the inverse matrix of $\barG_{k, \bar{l}}$. Since $g^{ml} g_{lk} = g_{k}^{m}$, one has\\

\begin{equation}
	g^{ml} = \begin{cases}
-	\barG^{m, \bar{l}}, & m \le n+1< l; \\
-	\barG^{l, \bar{m}}, & l \le n+1< m; \\
	0, & \text{ otherwise}.
	\end{cases}
\end{equation}

Therefore, the Christoffel symbols could be represented as follows.
\begin{flalign}\label{Gamma_2}
\begin{split}
	\Gamma_{ij}^{m} &= \frac{1}{2} g^{ml}(g_{il;j}+g_{lj;i}-g_{ij;l})\\
	&=\begin{cases}
	\barG^{m, \bar{l}} \barG_{ij, \bar{l}}, & i,j,m \le n+1 <l; \\
	\barG^{l, \bar{m}} \barG_{l, \bar{i}\bar{j}}, & i,j,m > n+1 \ge l; \\
	0, & \text{ otherwise}.
	\end{cases}
\end{split}	
\end{flalign}

Then one can calculate the pseudo Riemannian curvature tensor $R$.
\begin{flalign}
	R_{ijkl} &= g_{im}R^{m}_{jkl}\\
		  &= -g_{im}[\frac{\partial }{\partial \Xi_l} \Gamma_{jk}^{m} -\frac{\partial }{\partial \Xi_k} \Gamma_{jl}^m + \Gamma_{l\alpha}^{m}\Gamma_{jk}^{\alpha} - \Gamma_{k\alpha}^{m} \Gamma_{jl}^{\alpha} ]\\
		  &= \begin{cases}\label{curvature_tensor}
		  -\barG_{il,\alpha} \barG^{\beta, \alpha} \barG_{\beta, \bar{j}\bar{k}} + \barG_{il, \bar{j}\bar{k}}, & i,l \le n+1 <k,j; \\
		  \barG_{ik,\alpha} \barG^{\beta, \alpha} \barG_{\beta, \bar{j}\bar{l}} - \barG_{ik, \bar{j}\bar{l}}, & i,k \le n+1 <l,j; \\
		  -\barG_{jk,\alpha} \barG^{\beta, \alpha} \barG_{\beta, \bar{i}\bar{l}} + \barG_{jk, \bar{i}\bar{l}}, & j,k \le n+1 <l,i; \\
		  \barG_{jl,\alpha} \barG^{\beta, \alpha} \barG_{\beta, \bar{i}\bar{k}} - \barG_{jl, \bar{i}\bar{k}}, & j,l \le n+1 <k,i; \\
		  0, & \text{ otherwise}.
		  \end{cases}
\end{flalign}






\section{G-segments are geodesics}


\begin{definition}[G-segment with the notion of additional variable]\label{DefGsegment}
		For each $\barx_0 =(x_0,  w_0)\in X \times \R$, $\bary_0, \bary_1 \in cl( Y \times Z)$ with $w_0 \ne 0 %\color{red}w_0 >0?
		$, 
		define $\bary_t\in cl( Y \times Z)$ such that the following equation holds:
		\begin{flalign}\label{$G$-segment2}
		\begin{split}
		D_{\barx}\barG(\barx_0,\bary_t) = (1-t)D_{\barx}\barG(\barx_0, \bary_0) + t D_{\barx}\barG(\barx_0, \bary_1),\\ \text{ for each $t\in [0,1].$}
		\end{split}
		\end{flalign}
		By \Gone \ and \Gtwo , $\bary_t$ is uniquely determined by (\ref{$G$-segment2}). 
		We call $t \in [0,1] \longmapsto (\barx_0,\bary_t)$ the  $G$-segment connecting $(\barx_0, \bary_0)$ and $(\barx_0, \bary_1)$ on $M$.
\end{definition}

For any continuous, piecewise continuously differentiable curves $\gamma: [0,1] \longrightarrow M$, let $E(\cdot)$ denote the energy functional:
\begin{flalign}
	E(\gamma) = \frac{1}{2} \int_{0}^1 g_{\gamma(t)}(\dot{\gamma}(t), \dot{\gamma}(t)) dt.
\end{flalign}

Then the Euler-Lagrange equations of motion for the functional $E$ are given by 
\begin{flalign}\label{geodesicEqn}
	\frac{d^2 \Xi^{m}}{dt^2} + \Gamma_{ij}^{m} \frac{d\Xi^{i}}{dt}\frac{d\Xi^{j}}{dt} = 0,
\end{flalign}

where $\Gamma_{ij}^{m}$ is the Christoffel symbol define in \eqref{Gamma}. The above equality \eqref{geodesicEqn} is the so-called geodesic equation.

\begin{proposition}[G-segments are geodesics]
	Assume \Gone\ and \Gtwo. For any G-segment $t\in [0,1] \longmapsto (\barx_0, \bary_t)$, defined in Definition \ref{DefGsegment}, connecting $(\barx_0, \bary_0)$ and $(\barx_0, \bary_1)$ on $M$, it satisfies the geodesic equation \eqref{geodesicEqn}.
\end{proposition}

\begin{proof}
	Since G-segment $t\in [0,1] \longmapsto (\barx_0, \bary_t)$ satisfies \eqref{$G$-segment2}, the following equations hold:
	\begin{flalign}
	\label{$G$-segment3.1}	0 =\ & \partial_{t}^2 D_{\barx}\barG(\barx_0,\bary_t)\\
	\label{$G$-segment3.2}	=\ & D_{\barx^{k}\bary^{m}}\barG\cdot\ddot{\bary}_{t}^m +  D_{\barx^{k}\bary^{i}\bary^{j}}\barG\cdot\dot{\bary}_{t}^{i}\ \dot{\bary}_{t}^{j}.
	\end{flalign}
	This implies 
	\begin{flalign}
		\ddot{\bary}_{t}^m + [(D_{\barx^{k}\bary^{m}}\barG)^{-1}]^{k,m} \cdot D_{\barx^{k}\bary^{i}\bary^{j}}\barG \cdot \dot{\bary}_{t}^{i}\ \dot{\bary}_{t}^{j} =0.
	\end{flalign}
	Rewrite the above equation in terms of variable $\Xi$, then one has
	\begin{flalign}
		\ddot{\Xi}_t^m + \barG^{k, \bar{m}} \cdot \barG_{k, \bar{i} \bar{j}} \cdot \dot{\Xi}_t^{i}\ \dot{\Xi}_{t}^{j} = 0, \text{ where } k \le n+1 < i,j,m.
	\end{flalign}
	Combining with \eqref{Gamma_2}, this implies the geodesic equation \eqref{geodesicEqn}.
\end{proof}

\section{\Gthree\ is a non-negative sectional curvature condition}

Recall that in concavity arguments in Chapter \ref{chapter:convexity}, condition \Gthree\ plays the most important role. In Section \ref{section:4thorder}, we introduced the fourth-order differential re-expression of \Gthree. One may also wonder what is the geometric meaning of the \Gthree\ condition. In this section, we are going to show the geometric re-expression of \Gthree.

\begin{itemize}
	\item[\Gthree] For any $\barx=(x, w)$ with $w>0$, assume $\partial_t^2 \barG(\barx, \bary_t) \ge 0$, whenever there exists $\barx_0 =(x_0, 1)$ such that $\partial_t^2 D_{\barx}\barG(\barx_0, \bary_t) = 0$. 
\end{itemize}

For the pseudo Riemannian metric tensor $g$ defined in \eqref{pseudo_metric} and any two tangent vectors $P, Q\in T_{\Xi_0}M$, define the unnormalized sectional curvature at the point $\Xi_0 \in M$ as
\begin{flalign}
	\sec_{\Xi_0}^{(M, g)}P \wedge Q 
	:= R_{ijkl}(\Xi_{0})\cdot P^{i}\cdot P^{l} \cdot Q^{j} \cdot Q^{k},
\end{flalign}
where $R$ is the curvature tensor shown in \eqref{curvature_tensor}.
\medskip


The following theorem describes equivalent expressions of the \Gthree\ condition. Part $\mathrm{(i)}$ and $\mathrm{(iii)}$ are taken from Theorem \ref{theorem:4thorder}. While keeping the ordering number of statements from Chapter \ref{chapter:analytic_representation} (Theorem \ref{theorem:4thorder} and Corollary \ref{FourthOrder4}),  part $\mathrm{(v)}$ is a variation of $\mathrm{(iii)}$ by rewriting \Gthree\ condition with the notion of the additional variable, and part $\mathrm{(vi)}$ is the non-negative sectional curvature condition on the manifold $M$ defined in section \ref{section:geometric_setting}.


\begin{theorem}\label{prop:4thorder2}
	Assume \Gzero-\Gtwo\ and \Gfour-\Geight. If, in addition, $G\in C^4(cl(X\times Y \times Z)
	)$,  then the following statements are equivalent:
	\begin{enumerate}[(i)]
		\item \Gthree.
				
		\item[(iii)] For any given curve $x_s\in X$ connecting $x_0$ and $x_1$,  any $(y_0, z_0),  (y_1, z_1) \in cl( Y \times Z) $, we have 
		\begin{equation}
		\frac{\partial^2}{\partial s^2 }\Biggl(\frac{1}{G_z(x_s, y_t, z_t)}\frac{\partial^2}{\partial t^2} G(x_s,y_t,z_t) \Biggr)\Bigg|_{s=s_0}\le 0,
		\end{equation}
		whenever $t\in [0,1] \longmapsto (G_x, G)(x_{s_0}, y_t, z_t)$  forms an affinely parametrized line segment for some $s_0\in [0,1]$.
		
	\item[(v)] Let $s\in [0,1] \longmapsto w_s \in (0, \infty)$
	, for any given curve $x_s\in X$ connecting $x_0$ and $x_1$, any $\bary_0,  \bary_1 \in cl( Y \times Z) $, we have 
	\begin{equation}\label{barG-4thorder}
	\frac{\partial^4}{\partial s^2 \partial t^2} \Bigg|_{s=s_0} \barG(\barx_s,\bary_t)  \ge 0,
	\end{equation}
	whenever $t\in [0,1] \longmapsto (\barx_{s_0}, \bary_t)$  forms a G-segment for some $s_0\in [0,1]$.
	
%	\item Let $s\in [0,1] \longmapsto w_s \in (0, \infty)$
%		, for any given curve $x_s\in X$ connecting $x_0$ and $x_1$, any $\bary_0,  \bary_1 \in cl( Y \times Z) $, we have 
%		\begin{equation}\label{Curvature_sectional}
%		\sec_{(\barx_{s_0}, \bary_{t})}^{(M, g)}(p\oplus 0)\wedge (0\oplus q) \ge 0,
%		\end{equation}
%		whenever $t\in [0,1] \longmapsto (\barx_{s_0}, \bary_t)$  forms a G-segment for some $s_0\in [0,1]$, where $p=\dot{\barx}_{s_0}$ and $q = \dot{\bary}_t$.

	
	\item[(vi)] For any vectors $P = p \oplus 0$, $Q = 0 \oplus q \in \R^{2n+2}$, and $\Xi_{\nul} = (x_{\nul}, w_{\nul}, \bary_{\nul}) \in M$ with $w_{\nul} \in (0, \infty)$, the sectional curvature satisfies
		\begin{equation}\label{Curvature_sectional2}
		\sec_{\Xi_{\nul}}^{(M, g)} P \wedge Q \ge 0.
		\end{equation}
	\end{enumerate} 
\end{theorem}

\begin{proof}
	$\mathrm{(i)}$ and $\mathrm{(iii)}$ are equivalent from Theorem \ref{theorem:4thorder}. Only need to show the equivalences of $\mathrm{(iii)}$ and $\mathrm{(v)}$, $\mathrm{(v)}$ and $\mathrm{(vi)}$.
	
	$\mathrm{(iii)}\Rightarrow \mathrm{(v)}.$ Let $x_0$, $x_1$ be any two points on $X$, $(y_0, z_0)$,  $(y_1, z_1)$ be any two points on $cl( Y \times Z)$, $w_s$ be any curve on $(0, \infty)$, $x_s\in X$ be any curve connecting $x_0$ and $x_1$, and $(y_t, z_t) \in cl(Y\times Z)$ be any curve connecting $(y_0, z_0)$ and $(y_1, z_1)$. Suppose there exists $s_0\in [0,1]$, such that $t\in [0,1] \longmapsto (\barx_{s_0}, \bary_t)$  forms a G-segment. Then from \eqref{$G$-segment3.1}, we know $t\in [0,1] \longmapsto (G_x, G)(x_{s_0}, y_t, z_t)$ forms an affinely parametrized line segment, i.e., 
	\begin{flalign}
	\label{$G$-segment4.1}	0 &\ = \frac{\partial^2}{\partial t^2} G_x(x_{s_0}, y_t, z_t)\\
	\label{$G$-segment4.2}	0 &\ = \frac{\partial^2}{\partial t^2} G(x_{s_0}, y_t, z_t)
	\end{flalign}
	Therefore, one has
	\begin{flalign}
		0\ge &\	\frac{\partial^2}{\partial s^2 }\Biggl(\frac{1}{G_z(x_s, y_t, z_t)}\frac{\partial^2}{\partial t^2} G(x_s,y_t,z_t) \Biggr)\Bigg|_{s=s_0}\\
		= &\ \frac{1}{G_z(x_{s_0}, y_t, z_t)}  \frac{\partial^4}{\partial s^2 \partial t^2} \Bigg|_{s=s_0} G(x_s,y_t,z_t) + \frac{\partial^2}{\partial s^2} \Bigg|_{s=s_0} \frac{1}{G_z}(x_s, y_t, z_t) \cdot \frac{\partial^2}{\partial t^2} G(x_{s_0}, y_t, z_t)\\
		&\ +2 \frac{\partial }{\partial s}\Bigg|_{s=s_0}\frac{1}{G_z(x_s, y_t, z_t)} \cdot \frac{\partial^2}{ \partial t^2}  G_x(x_{s_0}, y_t, z_t)\cdot \dot{x}_s\Bigg|_{s=s_0}\\
		= &\ \frac{1}{G_z(x_{s_0}, y_t, z_t)}  \frac{\partial^4}{\partial s^2 \partial t^2} \Bigg|_{s=s_0} G(x_s,y_t,z_t),
	\end{flalign}
	Notice that $G_z <0$ because of \Gfour. Thus, the above inequality is equivalent to 
	\begin{flalign}\label{4thorderEqn2}
		\frac{\partial^4}{\partial s^2 \partial t^2} \Bigg|_{s=s_0} G(x_s,y_t,z_t) \ge 0.
	\end{flalign}
	On the other hand, for the same curves described above, since $\barG(\barx_s, \bary_t) = \barG(x_s, w_s, y_t, z_t) = w_s G(x_s,y_t,z_t)$, applying \eqref{$G$-segment4.1} and \eqref{$G$-segment4.2}, one has
	\begin{flalign}
		\frac{\partial}{\partial s} \barG(\barx_s, \bary_t) &\ = \dot{w}_s G(x_s, y_t, z_t) + w_s \frac{\partial}{\partial s} G(x_s, y_t, z_t)\\
		\frac{\partial^2}{\partial s^2} \barG(\barx_s, \bary_t) &\ = \ddot{w}_s G(x_s, y_t, z_t) + 2 \dot{w}_s \frac{\partial}{\partial s} G(x_s, y_t, z_t) + w_s \frac{\partial^2}{\partial s^2} G(x_s, y_t, z_t)\\
		\frac{\partial^4}{\partial s^2 \partial t^2} \Bigg|_{s=s_0} \barG(\barx_s, \bary_t) &\ = \ddot{w}_{s_0} \frac{\partial^2}{\partial t^2} G(x_{s_0}, y_t, z_t) + 2 \dot{w}_{s_0} \dot{x}_{s_0} \frac{\partial^2}{\partial t^2}  G_x(x_{s_0}, y_t, z_t) + w_{s_0} \frac{\partial^4}{\partial s^2 \partial t^2}\Bigg|_{s=s_0} G(x_s, y_t, z_t)\\
		&\ = w_{s_0} \frac{\partial^4}{\partial s^2 \partial t^2}\Bigg|_{s=s_0} G(x_s, y_t, z_t).
	\end{flalign}
	
	Since $w_{s_0}$ is positive, the equation \eqref{4thorderEqn2} is equivalent to 
	
	\begin{flalign}\label{4thorderEqn3}
	\frac{\partial^4}{\partial s^2 \partial t^2} \Bigg|_{s=s_0} \barG(\barx_s, \bary_t) \ge 0.
	\end{flalign}
	
	$\mathrm{(v)}\Rightarrow \mathrm{(iii)}.$ Let $x_0$, $x_1$ be any two points on $X$, $(y_0, z_0)$,  $(y_1, z_1)$ be any two points on $cl( Y \times Z)$,  $x_s\in X$ be any curve connecting $x_0$ and $x_1$, and $(y_t, z_t) \in cl(Y\times Z)$ be any curve connecting $(y_0, z_0)$ and $(y_1, z_1)$, satisfying that  $t\in [0,1] \longmapsto (G_x, G)(x_{s_0}, y_t, z_t)$ forms an affinely parametrized line segment. Let $w_s \equiv 1$, for all $s =\in [0,1]$. Then, by definition, $t\in [0,1] \longmapsto (\barx_{s_0}, \bary_t)$  forms a G-segment. From $\mathrm{(v)}$, one has
	\begin{flalign}
	\frac{\partial^4}{\partial s^2 \partial t^2} \Bigg|_{s=s_0} \barG(\barx_s,\bary_t)  \ge 0.
	\end{flalign}
	
	By the similar computations as in the first part and $G_z <0$ because of \Gfour, one has 
	\begin{flalign}
		\frac{\partial^2}{\partial s^2 }\Biggl(\frac{1}{G_z(x_s, y_t, z_t)}\frac{\partial^2}{\partial t^2} G(x_s,y_t,z_t) \Biggr)\Bigg|_{s=s_0}	= 	&\ 	\frac{1}{G_z(x_{s_0}, y_t, z_t)}  \frac{\partial^4}{\partial s^2 \partial t^2} \Bigg|_{s=s_0} G(x_s,y_t,z_t) \\
	=	&\ 
	\frac{1}{G_z(x_{s_0}, y_t, z_t)} \frac{\partial^4}{\partial s^2 \partial t^2} \Bigg|_{s=s_0} \barG(\barx_s,\bary_t) \le 0.
	\end{flalign}
	
	
	$\mathrm{(v)}\Rightarrow \mathrm{(vi)}.$ 
	Let $\barx_s$ be any curve on $ X \times (0, \infty)$ with $\barx_{s}|_{s=0} = (x_{\nul}, w_{\nul})$ and $\dot{\barx}_{s}|_{s=0} = p$, $\bary_t$ be any curve on $cl( Y \times Z)$, satisfying $\bary_t|_{t=0} = \bary_{\nul}$, $\dot{\bary}_t|_{t=0} = q$ and the following equation
	\begin{flalign}\label{$G$-segment4.3}
		\partial_t^2 D_{\barx}\barG(\barx_{0}, \bary_t) =0.
	\end{flalign}
	That is, the curve $t\in [0,1] \longmapsto (\barx_{0}, \bary_t)$  forms a G-segment. Thus by $\mathrm{(v)}$ we know 
	\begin{flalign}\label{4thorderEqn4}
			\frac{\partial^4}{\partial s^2 \partial t^2} \Bigg|_{s=0} \barG(\barx_s,\bary_t)  \ge 0.
	\end{flalign}
	On the other hand, from \eqref{$G$-segment4.3} we know
	\begin{flalign}
		 \barG_{i,j}(\barx_{0}, \bary_t) \cdot \ddot{\bary}_t^{j} + 
		 \barG_{i,kl} (\barx_{0}, \bary_t) \cdot \dot{\bary}_t^{k} \cdot \dot{\bary}_t^{l}=0.
	\end{flalign}
	
	This implies 
	\begin{flalign}
		 \ddot{\bary}_t^{j} = - 
		 \barG^{i,j}(\barx_{0}, \bary_t) \cdot \barG_{i,kl} (\barx_{0}, \bary_t) \cdot \dot{\bary}_t^{k} \cdot \dot{\bary}_t^{l}.
	\end{flalign}
	
	Thus
	
	\begin{flalign}
		&\frac{\partial^4}{\partial s^2 \partial t^2} \barG(\barx_s, \bary_t)\Bigg|_{s=0} \\
		=\ & \frac{\partial^2}{\partial t^2} [\barG_{i,}(\barx_s, \bary_t)\cdot \ddot{\barx}_s^{i} + \barG_{il,}(\barx_s, \bary_t)\cdot \dot{\barx}_s^{i}\cdot\dot{\barx}_s^{l}]\Bigg|_{s=0}\\
		=\ &  \frac{\partial^2}{\partial t^2} \barG_{il,}(\barx_{0}, \bary_t)\cdot \dot{\barx}_{0}^{i}\cdot\dot{\barx}_{0}^{l}\\
		=\ & \barG_{il,j}(\barx_{0}, \bary_t)\cdot \dot{\barx}_{0}^{i}\cdot\dot{\barx}_{0}^{l} \cdot \ddot{\bary}_t^{j} + \barG_{il,jk}(\barx_{0}, \bary_t)\cdot \dot{\barx}_{0}^{i}\cdot\dot{\barx}_{0}^{l} \cdot \dot{\bary}_t^{j} \cdot \dot{\bary}_t^{k}\\
	\label{4-tensor1}	=\ & [- \barG_{il,\alpha}(\barx_{0}, \bary_t)\cdot \barG^{\beta,\alpha}(\barx_{0}, \bary_t) \cdot \barG_{\beta,jk} (\barx_{0}, \bary_t) + \barG_{il,jk}(\barx_{0}, \bary_t)]\cdot \dot{\barx}_{0}^{i}\cdot\dot{\barx}_{0}^{l} \cdot\dot{\bary}_t^{j} \cdot \dot{\bary}_t^{k},
	\end{flalign} 
	where $i,l, j, k, \alpha, \beta = 1,2,..., n+1.$
	
	Denote $\Xi_{t} := (\barx_{0}, \bary_t)$. Since $P = \dot{\barx}_{0} \oplus 0$, one can rewrite \eqref{4-tensor1} as 
	\begin{flalign}
	&	\sum_{i,l =1}^{n+1}\sum_{j,k = n+2}^{2n+2} \sum_{\alpha, \beta = 1}^{n+1}
	[- \barG_{il,\alpha}(\Xi_{t})\cdot \barG^{\beta,\alpha}(\Xi_{t}) \cdot \barG_{\beta,\bar{j}\bar{k}} (\Xi_{t}) + \barG_{il,\bar{j}\bar{k}}(\Xi_{t})]\cdot P^{i}\cdot P^{l} \cdot\dot{\Xi}_t^{j} \cdot \dot{\Xi}_t^{k} \\
	= \ & \sum_{i,l =1}^{n+1}\sum_{j,k = n+2}^{2n+2} R_{ijkl}(\Xi_{t})\cdot P^{i}\cdot P^{l} \cdot\dot{\Xi}_t^{j} \cdot \dot{\Xi}_t^{k} \\
	= \ &  \sum_{i,l, j, k = 1}^{2n+2}
	R_{ijkl}(\Xi_{t})\cdot P^{i}\cdot P^{l} \cdot\dot{\Xi}_t^{j} \cdot \dot{\Xi}_t^{k} \\
	= \ & \sec_{\Xi_{t}}^{(M, g)}P \wedge \dot{\Xi}_t.
	\end{flalign}
	Therefore, equation \eqref{4thorderEqn4} implies 
	\begin{flalign}
		\sec_{\Xi_{t}}^{(M, g)}P \wedge \dot{\Xi}_t \ge 0.
	\end{flalign}
	In particular, for $t=0$, since $Q = \dot{\Xi}_0 \in \R^{2n+2}$, 
	\begin{flalign}
	\sec_{\Xi_{0}}^{(M, g)}P \wedge Q \ge 0.
	\end{flalign}
	
	
	$\mathrm{(vi)}\Rightarrow \mathrm{(v)}.$ Let $x_0$, $x_1$ be any two points on $X$, $(y_0, z_0)$,  $(y_1, z_1)$ be any two points on $cl( Y \times Z)$, $w_s$ be any curve on $(0, \infty)$, $x_s\in X$ be any curve connecting $x_0$ and $x_1$, and $(y_t, z_t) \in cl(Y\times Z)$ be any curve connecting $(y_0, z_0)$ and $(y_1, z_1)$. Suppose there exists $s_0\in [0,1]$, such that $t\in [0,1] \longmapsto (\barx_{s_0}, \bary_t)$  forms a G-segment. Then from \eqref{$G$-segment3.1}, we know 
	
	
	
		\begin{flalign}
		\barG_{i,j}(\barx_{s_0}, \bary_t) \cdot \ddot{\bary}_t^{j} + 
		\barG_{i,kl} (\barx_{s_0}, \bary_t) \cdot \dot{\bary}_t^{k} \cdot \dot{\bary}_t^{l}=0.\\
		\ddot{\bary}_t^{j} = - 
		\barG^{i,j}(\barx_{s_0}, \bary_t) \cdot \barG_{i,kl} (\barx_{s_0}, \bary_t) \cdot \dot{\bary}_t^{k} \cdot \dot{\bary}_t^{l}.
		\end{flalign}
		
		Denote $\Xi_{t} := (\barx_{s_0}, \bary_t)$. For any fixed $t_0 \in [0,1]$, let $P = \dot{\barx}_{s_0} \oplus 0$, $Q = \dot{\Xi}_{t_0} = 0 \oplus \dot{\bary}_{t_0} \in \R^{2n+2}$. Thus from part $\mathrm{(vi)}$, 
		\begin{flalign}
			 \sec_{\Xi_{t_0}}^{(M, g)}P \wedge Q
			 \ge  0.
		\end{flalign}
		On the other hand, with similar calculations as above, one has
		\begin{flalign}
		&\frac{\partial^4}{\partial s^2 \partial t^2} \barG(\barx_s, \bary_t)\Bigg|_{s=s_0, t = t_0} \\
		=\ & [- \barG_{il,\alpha}(\barx_{s_0}, \bary_{t_0})\cdot \barG^{\beta,\alpha}(\barx_{s_0}, \bary_{t_0}) \cdot \barG_{\beta,jk} (\barx_{s_0}, \bary_{t_0}) + \barG_{il,jk}(\barx_{s_0}, \bary_{t_0})]\cdot \dot{\barx}_{s_0}^{i}\cdot\dot{\barx}_{s_0}^{l} \cdot\dot{\bary}_{t_0}^{j} \cdot \dot{\bary}_{t_0}^{k}\\
		= \ & \sec_{\Xi_{t_0}}^{(M, g)}P \wedge Q.
		\end{flalign}
	Therefore, \eqref{barG-4thorder} holds for all $t \in [0,1]$.
\end{proof}



\begin{remark}
	Strict inequality versions of $\mathrm{(v)}$ and $\mathrm{(vi)}$ in Theorem \ref{prop:4thorder2} are equivalent to strict inequality of $\mathrm{(iii)}$, and thus $\Gthree_{u}$.
\end{remark}

%
%\begin{remark}
%	Theorem \ref{prop:4thorder2} holds for $w \in \R$ whenever the equation \eqref{barG-4thorder} is replaced by the following:
%	\begin{equation}\label{barG-4thorder2}
%	w_0\frac{\partial^4}{\partial s^2 \partial t^2} \Bigg|_{s=s_0} \barG(\barx_s,\bary_t)  \ge 0,
%	\end{equation}
%	\color{red} need to change the definition of G-segment in Definition \ref{DefGsegment}. And also check the proof again. 
%\end{remark}


\chapter{Examples}\label{chapter:examples}




\section{Several examples for the quasilinear case with explicit solutions}
In 2011, Figalli-Kim-McCann \cite{FigalliKimMcCann11} provided a non-negative curvature condition (B3) which is equivalent to concavity of the maximization functional $\pmb \Pi$ or convexity of the functional $\mathcal{L}$ defined below, under some other constraints. One may wonder whether this curvature condition (B3) is equivalent to uniqueness of the functionals as well. According to Loeper \cite{Loeper09}, (B3) is satisfied if and only if the Riemannian sectional curvature is non-negative.
%, some part of the thesis aims to investigate uniqueness without concavity on the hyperbolic spaces with constant negative curvatures. 
This section shows a negative answer, via uniqueness examples on the hyperbolic spaces with constant negative curvatures%negative curvature spaces
, where (B3) is violated. \medskip

Let $D$ be a disk with a small radius $\bar{r}$ on $\HH^n$($n\ge 2$), and consider spaces $X= Y = D$, utility $G(x,y,z)=-\frac{1}{2}d_H^2(x,y) -z$, and profit $\pi(x,y,z) = z$ (i.e., $\pi(x,y,z) = z -a(y)$ with $a \equiv 0$), where
\begin{equation}\label{distanceh}
\begin{split}
&d_H(x,y)= R\cosh^{-1}(\frac{x_0y_0-x_1y_1-\cdots -x_ny_n}{R^2})= R\cosh^{-1}[\cosh \frac{r}{R}\cosh \frac{s}{R}-\sinh \frac{r}{R}\sinh\frac{s}{R} M],\\
&M=\sum_{i=1}^{n-1}\cos\theta_i \cos\varphi_i\Big(\prod_{j=1}^{i-1} \sin\theta_j\sin\varphi_j\Big)+ \prod_{j=1}^{n-1} \sin\theta_j\sin\varphi_j,\\
&\text{ here } x=(x_0,x_1,\dots, x_n),~y=(y_0,y_1,\dots, y_n) \in D,\\& x_0 = R\cosh \frac{r}{R},~ x_i = R\sinh \frac{r}{R}\cos\theta_i \Pi_{j=1}^{i-1}\sin \theta_j, ~\forall i=1,2,...,n-1, ~x_n = R\sinh \frac{r}{R} \Pi_{j=1}^{n-1}\sin \theta_j;\\
&y_0 = R\cosh \frac{s}{R},~ y_i = R\sinh \frac{s}{R}\cos\varphi_i \Pi_{j=1}^{i-1}\sin \varphi_j, ~\forall i=1,2,...,n-1,~ y_n = R\sinh \frac{s}{R} \Pi_{j=1}^{n-1}\sin \varphi_j.
\end{split}
\end{equation} 

Here $\prod$ means the product notation. In order to distinguish it from the profit functional $\pmb \Pi$, equivalently, in this section, we minimize $\mathcal{L} := - \pmb \Pi$. We take $\mu$ be uniform measure on this hyperbolic disk $D$, and participation constraint $u_{\emptyset}(x)= -\frac{1}{2}|x|^2$.\medskip

Then the monopolist problem becomes

\begin{flalign}
	\max\limits_{ u\ge u_\nul \atop \text{$u$ is $G$-convex}} \pmb\Pi(u) &~=\max\limits_{u \ge u_\nul \atop \text{$u$ is $G$-convex}} \int_X \pi(x, \bar{y}_G(x, u(x), Du(x)))  d \mu(x)  \\
	&~= \max\limits_{u \ge u_\nul \atop \text{$u$ is $G$-convex}} \int_X -\frac{1}{2} d_H^2(x, y_G(x, u(x), Du(x))) - u(x)  d \mu(x).\\
	&~= -\min\limits_{u \ge u_\nul \atop \text{$u$ is $G$-convex}} \int_X \frac{1}{2} d_H^2(x, y_G(x, u(x), Du(x))) + u(x)  d \mu(x).\\
(P_5)\hspace{3cm}	&~= -\min\limits_{u \ge u_\nul \atop \text{$u$ is $G$-convex}} \mathcal{L}(u).
\end{flalign}

\begin{lemma}\label{dislemma}
	Let $x(t)$ be any curve on $D$, with $|\dot{x}(t_0)|=1$, $y$ be any point on $D$. If $D_td_H(x(t),y)|_{t=t_0}$ exists, then $|D_td_H(x(t),y)|_{t=t_0}|\le 1$. 
\end{lemma}
\begin{proof}
	By triangular inequality, we have
%	\begin{equation*}
%	\begin{split}
\begin{flalign*}
		|D_td_H(x(t),y)|_{t=t_0}| =& \bigg|\lim\limits_{t\rightarrow t_0}\frac{d_H(x(t),y)-d_H(x(t_0),y)}{t-t_0}\bigg|= \lim\limits_{t\rightarrow t_0}\bigg|\frac{d_H(x(t),y)-d_H(x(t_0),y)}{t-t_0}\bigg|\\
		\le&  \lim\limits_{t\rightarrow t_0^+}\frac{d_H(x(t),x(t_0))}{t-t_0} = |\dot{x}(t_0)|=1.
\end{flalign*}
%	\end{split}
%	\end{equation*}
\end{proof}

\begin{proposition}\label{disprop}
	Since  $\Big|\frac{\partial x(r, \theta_1,...,\theta_{n-1})}{\partial r}\Big|= \lim\limits_{s\rightarrow 0}\frac{d( x(r, \theta_1,...,\theta_{n-1}),  x(r+s, \theta_1,...,\theta_{n-1}))}{s}=\lim\limits_{s\rightarrow 0}\frac{s}{s} = 1$,
	by Lemma (\ref{dislemma}), we have $ |D_r d_H(x(r, \theta_1,...,\theta_{n-1}), y)|\le 1, \forall x, y \in D.$
\end{proposition}
\begin{theorem}
	The program $(P_5)$ has an unique minimizer on $D$. And 
	\begin{equation*}
	\argmin\limits_{\substack{u\ge -\frac{1}{2}r^2 \\ u \text{ is radially symmetric} \\ u \in C^1(D)}} \mathcal{L}(u)= \argmin\limits_{\substack{u\ge -\frac{1}{2}r^2 \\ u \text{ is radially symmetric} \\ u \text{ is $G$-convex}}} \mathcal{L}(u)=\argmin\limits_{\substack{u\ge -\frac{1}{2}|x|^2 \\ u \text{ is $G$-convex}}} \mathcal{L}(u) =: \bar{u}(r)
	\end{equation*}
	where
	\begin{equation*}
	\bar{u}(r) = 
	\begin{cases}
	-\frac{1}{2} r^2 & ,0\le r\le \tilde{r} \\
	\parbox[t]{.8\textwidth}{$\int_{\tilde{r}}^{r}\sinh^{1-n}(\frac{t}{R})\int_{0}^{t}\sinh^{n-1}(\frac{\sigma}{R})d\sigma dt - \int_{0}^{\bar{r}}\sinh^{n-1}(\frac{\sigma }{R})d \sigma  \int_{\tilde{r}}^{r}\sinh^{1-n}(\frac{t}{R}) d t -\frac{1}{2}(\tilde{r})^2 $} & ,\tilde{r} < r \le \bar{r} \\
	\end{cases}.
	\end{equation*}
	Here $\tilde{r}$ satisfies 
	\begin{equation*}
	\int_{\bar{r}}^{\tilde{r}}\sinh^{n-1}(\frac{\sigma }{R}) d\sigma  +\tilde{r}\sinh^{n-1}(\frac{\tilde{r}}{R})=0.
	\end{equation*}
\end{theorem}

\begin{proof}
	{\bf Step 1:} Firstly, find the minimizer of $\mathcal{L}(u)$ for all $u$ satisfying $u(r)\ge -\frac{1}{2}r^2$ and $u$ is radially symmetric. Assume $\bar{u} \in C^2$ piecewisely on $D$ is such a minimizer. For each agent $x$, $\bar{u}(x)= sup_{y} -\frac{1}{2}d_H^2(x,y)-v(y)$, one can find the optimal $y_G(x, \bar{u}(x), D\bar{u}(x))$ via
	\begin{equation}\label{usym}
	\begin{split}
	D_r\bar{u}(x) = D_r(-\frac{1}{2} d_H^2(x, y_G(x, \bar{u}(x), D\bar{u}(x))))\\
	D_{\theta_i}\bar{u}(x) = D_{\theta_i}(-\frac{1}{2} d_H^2(x, y_G(x, \bar{u}(x), D\bar{u}(x))))
	\end{split}
	\end{equation}
	From the above equations, we can see that $y_G(x, \bar{u}(x), D\bar{u}(x))$ could be uniquely determined by $x$ and $D\bar{u}(x)$. In this section, we use $y_G(x, D\bar{u}(x))$ to denote $y_G(x, \bar{u}(x), D\bar{u}(x))$.	\\
	Since $\bar{u}$ is radially symmetric, thus $D_{\theta_i} \bar{u}(x) =0 $, for all $i=1,2,...,n-1$, and $D_r\bar{u}(x)= \bar{u}'(r)$. From (\ref{usym}), one can derive $\theta_i = \varphi_i$, for all $i=1,2,...,n-1$, and $d_H(x, y_G(x, D\bar{u}(x))) = |r-s|$, $D_rd_H(x,y_G(x, D\bar{u}(x)))=\frac{\sin\frac{r-s}{R}}{|\sin\frac{r-s}{R}|}= sign(r-s) $, for $x, y_G(x, D\bar{u}(x)) \in D$ with polar coordinates introduced in (\ref{distanceh}).\\
	 Again from (\ref{usym}), $\bar{u}'(r)+|r-s|\cdot sign(r-s)=0$ implies $s=r+\bar{u}'(r)$, and $(\bar{u}'(r))^2 =(s-r)^2 = d_H^2(x, y_G(x, D\bar{u}(x)))$. Notice here magnitude of $y_G(x, D\bar{u}(x))$ should be non-negative, so we have constraint $r+\bar{u}'(r) \ge 0$, which will be used later. \\
	After calculating $y_G(x, D\bar{u}(x))$, one can compute
	\begin{equation*}
	\begin{split}
	\mathcal{L}(\bar{u})=& \int_D \frac{1}{2} d_H^2(x,y_G(x, D\bar{u}(x)))+\bar{u}(x) d\mu(x)\\
	=&\int_{0}^{\bar{r}}\int_{0}^{\pi}\cdots\int_{0}^{\pi}\int_{0}^{2\pi}[\frac{(\bar{u}'(r))^2}{2}+\bar{u}(r)]R^{n-1}(\sinh\frac{r}{R})^{n-1}(\sin \theta_1)^{n-2}\cdots (\sin\theta_{n-2}) d\theta_{n-1}\cdots d\theta_1 dr\\	
	=&C_0\int_{0}^{\bar{r}}[\frac{(\bar{u}'(r))^2}{2}+\bar{u}(r)] (\sinh\frac{r}{R})^{n-1} dr, \\
	&\text{ where }	C_0=\int_{0}^{\pi}\cdots\int_{0}^{\pi}\int_{0}^{2\pi}R^{n-1}(\sin \theta_1)^{n-2}\cdots (\sin\theta_{n-2}) d\theta_{n-1}\cdots d\theta_1 \text{ is a positive constant.} \\
	\end{split}
	\end{equation*}
	Since $\bar{u}(r)\ge -\frac{1}{2}r^2$, let $A\in [0,\bar{r}]$, such that $A\times[0,\pi]^{n-2}\times[0,2\pi] = \{(r,\theta_1, ...,\theta_{n-1})| \bar{u}(r) = -\frac{1}{2}r^2\}$. Define $B =[0,\bar{r}]\setminus A$, so $\bar{u}(r)> -\frac{1}{2}r^2$ on $B$. Since $\bar{u}$ is a minimizer of $\mathcal{L}(u)$, for all radially symmetric functions $w$ satisfying $w=0$ on A, one has
	\begin{equation*}
	\begin{split}
	0=\frac{\partial \mathcal{L}(\bar{u}+\varepsilon w)}{\partial \varepsilon}\bigg|_{\varepsilon=0} &= C_0\int_B [\bar{u}'(r)w'(r)+w(r)](\sinh\frac{r}{R})^{n-1}dr\\
	&=C_0 \int_B w[(\sinh\frac{r}{R})^{n-1}-\bar{u}''(r)(\sinh\frac{r}{R})^{n-1}- \frac{n-1}{R}\bar{u}'(r)(\sinh\frac{r}{R})^{n-2}\cosh\frac{r}{R}]dr\\& \ \ \  +C_0 \bar{u}'(r)w(r)(\sinh\frac{r}{R})^{n-1}\bigg|_{\partial B}\\
	\end{split}
	\end{equation*}
	By fundamental lemma of Calculus of Variations, together with the inequality we derived from non-negativity of magnitude of $y_G(x, D\bar{u}(x))$, we have following constraints for $\bar{u}$:
	\begin{equation*}
	(ODE) \begin{cases} 
	r+\bar{u}'(r) \ge 0, & (1)\mbox{ on } B \\ 
	\bar{u}''(r)+\frac{n-1}{R}\cdot \bar{u}'(r)(\coth\frac{r}{R}) -1=0, & (2)\mbox{ on } Int(B) \\
	w(r)\bar{u}'(r)(\sinh\frac{r}{R})^{n-1}|_{\partial B} =0,&(3) \mbox{ for all } w\in \{w \mbox{ is radially symetric }| w= 0 \mbox{ on } A\}\\
	\end{cases}
	\end{equation*}
	The equation (ODE)(2) implies,  $\forall r \in B$,
	\begin{equation*}
	\bar{u}(r) = \int_{0}^{r}(\sinh\frac{t}{R})^{1-n} \int_{0}^{t} (\sinh\frac{\sigma }{R})^{n-1} d\sigma  dt + C_1 \int_{0}^{r}(\sinh\frac{t}{R})^{1-n} dt +C_2
	\end{equation*}
	Taking derivative of $\bar{u}$, 
	\begin{align*}
	\bar{u}'(r) = (\sinh\frac{r}{R})^{1-n} [\int_{0}^{r} (\sinh\frac{\sigma }{R})^{n-1} d\sigma  +C_1],\\
	\bar{u}''(r) = 1-\frac{n-1}{R}\cdot \frac{\cosh \frac{r}{R}}{\sinh^n(\frac{r}{R})}[\int_{0}^{r} (\sinh\frac{\sigma }{R})^{n-1} d\sigma +C_1].
	\end{align*}
	%    \begin{equation*}
	%    	\begin{split}
	%    	\bar{u}'(r) = (\sinh\frac{r}{R})^{1-n} [\int_{0}^{r} (\sinh\frac{\sigma }{R})^{n-1} d\sigma  +C_1],\\
	%    	\bar{u}''(r) = 1-\frac{n-1}{R}\cdot \frac{\cosh \frac{r}{R}}{\sinh^n(\frac{r}{R})}[\int_{0}^{r} (\sinh\frac{\sigma }{R})^{n-1} d\sigma +C_1].\\
	%    	\end{split}
	%    \end{equation*}
	Consider the sign of $C_1$, there are two cases:
	\begin{itemize}
		\item[1).]  If $C_1 \ge 0,$ then $\bar{u}'(r)\ge 0$, which implies $\bar{u}(r)$ is increasing on $B$.
		\item[2).] If $C_1<0$, then $\bar{u}''(r)\ge 1-\frac{n-1}{R}\cdot \frac{\cosh \frac{r}{R}}{\sinh^n(\frac{r}{R})}\int_{0}^{r} (\sinh\frac{\sigma }{R})^{n-1} d\sigma $. Define 
		$$h_1(r):= \frac{R\sinh^n(\frac{r}{R})}{(n-1)\cosh(\frac{r}{R})} - \int_{0}^{r} (\sinh\frac{\sigma }{R})^{n-1} d\sigma.$$ 
		Then $h_1(0)=0, h_1'(r) = \frac{\sinh^{n-1}(\frac{r}{R})}{(n-1)\cosh^2(\frac{r}{R})}\ge 0, \forall r\in [0, \bar{r}].$ Thus, $h_1(r) \ge 0 , \forall r\in [0, \bar{r}]$, which implies, $u''(r)\ge 0$, i.e., $\bar{u}$ is convex on $B$.
	\end{itemize}
	In either case, $A$ is path-connected, since one cannot join two points on $u_{\emptyset}(r) = -\frac{1}{2}r^2$ by either increasing or convex curve above the graph of $u_{\emptyset}$.\\
	Assume $A=[\alpha_1, \alpha_2] \neq [0, \bar{r}]$. For the relative position of $A$ and $B$, considering $A \cup B =[0,\bar{r}]$, there are three cases:
	\begin{itemize}
		\item [1).] If $\alpha_1>0, \alpha_2<\bar{r}$, assume $B = B_1 \cup B_2$ with $B_1 = [0, \alpha_1)$ and $B_2 = (\alpha_2, \bar{r}]$. Let $	\bar{u}'(r) = (\sinh\frac{r}{R})^{1-n} [\int_{0}^{r} (\sinh\frac{\sigma }{R})^{n-1} d\sigma  +C_{1_0}]$ on $B_1$ and 	$\bar{u}'(r) = (\sinh\frac{r}{R})^{1-n} [\int_{0}^{r} (\sinh\frac{\sigma }{R})^{n-1} d\sigma  + C_{1_{\bar{r}}}]$ on $B_2$.
		Then by (ODE)(3), one has $\bar{u}'(r)(\sinh\frac{r}{R})^{n-1}|_{r=0}^{\bar{r}} =0$, i.e. $[\int_{0}^{\bar{r}} (\sinh\frac{\sigma }{R})^{n-1} d\sigma  +C_{1_{\bar{r}}}] - C_{1_0} =0$, which implies 
		\begin{flalign}\label{eqn:constant_C_1}
			C_{1_{\bar{r}}} = C_{1_0} -  \int_{0}^{\bar{r}} (\sinh\frac{\sigma }{R})^{n-1} d\sigma .
		\end{flalign}
		Since $\bar{u}(r) \ge u_{\emptyset} (r)$ on $B_1$ with equality holds at $r= \alpha_1$, we know $\bar{u}$ could not be increasing on $B_1$. Therefore, by the above discussion, we know $C_{1_0} <0$ and $C_{1_{\bar{r}}}<0$ by \eqref{eqn:constant_C_1}. Thus $\bar{u}$ is convex on $B$. In particular,  $\bar{u}$ is convex and decreasing on $B_1$. Let $\tilde{u} = \bar{u}$ on $A \cup B_2$, and $\tilde{u} = u_{\emptyset}$ on $B_1$. Then for any $r\in B_1$, $\tilde{u}(r) - \bar{u}(r) \le 0$ and $0\ge  \tilde{u}'(r)> u_{\emptyset}'(\alpha_1) \ge \bar{u}'(\alpha_1) \ge \bar{u}'(r) $. Thus,
		\begin{flalign}
			\mathcal{L}(\tilde{u}) - \mathcal{L}(\bar{u}) = C_0 \int_{0}^{\alpha_1} \Big\{[\frac{(\tilde{u}'(r))^2}{2} -\frac{(\bar{u}'(r))^2}{2}]+[\tilde{u}-\bar{u}(r)]\Big\} (\sinh\frac{r}{R})^{n-1} dr < 0.
		\end{flalign}
		Therefore, this case is reduced to the following one where $\alpha_1 =0$.
	
		\item[2).] 	If $\alpha_1=0, \alpha_2\neq\bar{r}$, then by (ODE)(3), one has $\bar{u}'(r)(\sinh\frac{r}{R})^{n-1}|_{r=\bar{r}} =0$, which implies $\bar{u}'(\bar{r}) =0$. Thus, $C_1 = -\int_{0}^{\bar{r}} (\sinh\frac{\sigma }{R})^{n-1} d\sigma $. In this case, $\bar{u}$ is convex and decreasing on $B$.
		
		\item [3).] If $\alpha_1\neq0,\alpha_2=\bar{r}$, then by (ODE)(3), one has $\bar{u}'(r)(\sinh\frac{r}{R})^{n-1}|_{r=0} =0$, i.e. $[\int_{0}^{r} (\sinh\frac{\sigma }{R})^{n-1} d\sigma  +C_1]|_{r=0} =0$, which implies $C_1 = 0$. In this case, $\bar{u}$ is increasing on $B$. Notice that $\bar{u}(r)\ge -\frac{1}{2} r^2$ with equality holds at $r=\alpha_1$. Contradiction!
	\end{itemize}

	Summing up all the possible cases above, we know there exist $\alpha \in [0, \bar{r}]$, such that $A=[0,\alpha], B=(\alpha, \bar{r}]$. By (ODE)(1), we have on B, $\int_{\bar{r}}^{r}\sinh^{n-1}(\frac{\sigma}{R}) d\sigma +r\sinh^{n-1}(\frac{r}{R})\ge 0$. Define 
	$$ h_2(r)=\int_{\bar{r}}^{r}\sinh^{n-1}(\frac{\sigma}{R}) d\sigma +r\sinh^{n-1}(\frac{r}{R}).$$ 
	Then $h_2'(r) = 2\sinh^{n-1}(\frac{r}{R}) +(n-1)\sinh^{n-2}(\frac{r}{R})(\cosh\frac{r}{R})\frac{r}{R}>0$, which means $h_2$ is strictly increasing. Notice that $h_2(\bar{r})>0$, and $h_2(0)<0$. Thus there is a unique solution of $h_2(r)=0$ in $[0, \bar{r}]$, denote it $\tilde{r}$. Then (ODE)(1) implies, $\alpha\ge \tilde{r}$, where $\tilde{r}$ satisfies
	\begin{equation}
	\int_{\bar{r}}^{\tilde{r}}\sinh^{n-1}(\frac{\sigma}{R}) d\sigma +\tilde{r}\sinh^{n-1}(\frac{\tilde{r}}{R})=0.
	\end{equation}
	Since $\bar{u}$ is continuous, at $r=\alpha$, one have 
	\begin{equation*}
	-\frac{1}{2} \alpha^2 = \int_{0}^{\alpha}(\sinh\frac{t}{R})^{1-n} \int_{0}^{t} (\sinh\frac{\sigma }{R})^{n-1} d\sigma  dt  -\int_{0}^{\bar{r}} (\sinh\frac{\sigma }{R})^{n-1} d\sigma \int_{0}^{\alpha}(\sinh\frac{t}{R})^{1-n} dt +C_2,
	\end{equation*}
	which implies 
	\begin{equation*}
	C_2=-\frac{1}{2} \alpha^2 - \int_{0}^{\alpha}(\sinh\frac{t}{R})^{1-n} \int_{0}^{t} (\sinh\frac{\sigma }{R})^{n-1} d\sigma  dt  +\int_{0}^{\bar{r}} (\sinh\frac{\sigma }{R})^{n-1} d\sigma \int_{0}^{\alpha}(\sinh\frac{t}{R})^{1-n} dt 
	\end{equation*}
	For any given $\alpha \in [\tilde{r},\bar{r}]$, denote 
	\begin{equation*}
	\bar{u}_{\alpha}(r) =
	\begin{cases}
	-\frac{1}{2} r^2, &0\le r\le \alpha\\
	\int_{\alpha}^{r}(\sinh\frac{t}{R})^{1-n} \int_{0}^{t} (\sinh\frac{\sigma }{R})^{n-1} d\sigma  dt -\int_{0}^{\bar{r}} (\sinh\frac{\sigma }{R})^{n-1} d\sigma\int_{\alpha}^{r}(\sinh\frac{t}{R})^{1-n} dt -\frac{1}{2} \alpha^2  , & \alpha <r\le \bar{r}\\
	\end{cases}
	\end{equation*}
	Define $h_3(\alpha) = \mathcal{L}(\bar{u}_{\alpha})$, $\forall \alpha\in [\tilde{r}, \bar{r}]$. \\
	Then $h_3'(\alpha) = \frac{C_0}{2}\sinh^{1-n}(\frac{\alpha}{R})[\alpha\sinh^{n-1}(\frac{\alpha}{R})-\int_{\alpha}^{\bar{r}} \sinh^{n-1}(\frac{\sigma}{R}) d\sigma]^2 \ge 0$, for all $\alpha \in [\tilde{r}, \bar{r}].$ \\
	Define $h_4(\alpha):=\alpha \sinh^{n-1}(\frac{\alpha}{R})-\int_{\alpha}^{\bar{r}} \sinh^{n-1}(\frac{\sigma}{R}) d\sigma$. \\
	Then $h_4'(\alpha)= 2\sinh^{n-1}(\frac{\alpha}{R}) +\frac{n-1}{R}\alpha  \sinh^{n-2}(\frac{\alpha}{R}) \cosh(\frac{\alpha}{R}) >0$  on $[\tilde{r}, \bar{r}]$ and $h_4(\tilde{r})=0$, which implies $h_4(\alpha)> 0$, for $\alpha \in (\tilde{r}, \bar{r}]$. \\
	Thus, $h_3'(\alpha)> 0$, for all $\alpha \in (\tilde{r}, \bar{r}]$, which implies
	\begin{equation*}
	\min\limits_{\alpha\in [\tilde{r}, \bar{r}]} \mathcal{L}(\bar{u}_{\alpha}) = \min\limits_{\alpha\in [\tilde{r}, \bar{r}]} h_3({\alpha}) = h_3(\tilde{r}) = \mathcal{L}(\bar{u}_{\tilde{r}}).
	\end{equation*}
	Therefore, $\bar{u}(r) = \bar{u}_{\tilde{r}}(r)$. And
	\begin{equation*}
	\bar{u}(r) =
	\begin{cases}
	-\frac{1}{2} r^2 & ,0\le r\le \tilde{r} \\
	\parbox[t]{.75\textwidth}{$\int_{\tilde{r}}^{r}\sinh^{1-n}(\frac{t}{R})\int_{0}^{t}\sinh^{n-1}(\frac{\sigma}{R})d\sigma dt - \int_{0}^{\bar{r}}\sinh^{n-1}(\frac{\sigma }{R})d \sigma  \int_{\tilde{r}}^{r}\sinh^{1-n}(\frac{t}{R}) d t -\frac{1}{2}(\tilde{r})^2 $} & ,\tilde{r} < r \le \bar{r} \\
	\end{cases},
	\end{equation*}
	\begin{equation*}
	\bar{u}'(r) =
	\begin{cases}
	- r & ,0\le r\le \tilde{r} \\
	\parbox[t]{.75\textwidth}{$\sinh^{1-n}(\frac{r}{R})\int_{\bar{r}}^r\sinh^{n-1}(\frac{\sigma}{R})d\sigma   $} & ,\tilde{r} < r \le \bar{r} \\
	\end{cases},
	\end{equation*}
	\begin{equation*}
	\bar{u}''(r) =
	\begin{cases}
	- 1 & ,0\le r\le \tilde{r} \\
	\parbox[t]{.75\textwidth}{$1-\frac{(n-1)\cosh(\frac{r}{R})\int_{\bar{r}}^r\sinh^{n-1}(\frac{\sigma}{R})d\sigma}{R\sinh^n(\frac{r}{R})}   $} & ,\tilde{r} < r \le \bar{r} \\
	\end{cases}.
	\end{equation*}
	Since $\partial_{-}\bar{u}'(\tilde{r})= \partial_{+}\bar{u}'(\tilde{r})= -\tilde{r}$, thus $\bar{u}'(\tilde{r})$ exists, and $\bar{u}'(\tilde{r})=-\tilde{r}$.\\
	We now show the above $\bar{u}(r)$ is indeed the (global) minimizer, i.e.,
	\begin{equation*}
	\bar{u}(r)=\argmin\limits_{\substack{u\ge -\frac{1}{2}r^2 \\ u \text{ is radially symmetric}\\ u \in C^1(D)}} \mathcal{L}(u).
	\end{equation*}
	Let $w$ be any symmetric $C^1$ function such that $w\ge 0 $ on $A\times[0,\pi]^{n-2}\times[0,2\pi]$, then 
	\begin{equation*}
	\begin{split}
	&\ \mathcal{L}(\bar{u}+w)-\mathcal{L}(\bar{u})\\
	= &\ C_0 \int_{0}^{\bar{r}} [\frac{1}{2}(\bar{u}'(r)+w'(r))^2-\frac{1}{2}(\bar{u}'(r))^2+w]\sinh^{n-1}(\frac{r}{R}) dr\\
	\ge &\ C_0\int_{0}^{\bar{r}} [\bar{u}'(r)w'(r)+w]\sinh^{n-1}(\frac{r}{R}) dr\\
	=&\ C_0\int_{0}^{\tilde{r}} [-rw'(r)+w]\sinh^{n-1}(\frac{r}{R}) dr + C_0\int_{\tilde{r}}^{\bar{r}} [\bar{u}'(r)w'(r)+w]\sinh^{n-1}(\frac{r}{R}) dr\\
	=&\ C_0\int_{0}^{\tilde{r}} w[2\sinh^{n-1}(\frac{r}{R})+\frac{n-1}{R}r\sinh^{n-2}(\frac{r}{R})\cosh(\frac{r}{R})]dr-C_0rw(r)\sinh^{n-1}(\frac{r}{R})\bigg|_0^{\tilde{r}}\\
	&\ \ + C_0\int_{\tilde{r}}^{\bar{r}} w[(1-\bar{u}''(r))\sinh^{n-1}(\frac{r}{R})-\frac{n-1}{R}\bar{u}'(r)\sinh^{n-2}(\frac{r}{R})\cosh(\frac{r}{R})]dr\\
	&\ \ +C_0\bar{u}'(r)w(r)\sinh^{n-1}(\frac{r}{R})\bigg|_{\tilde{r}}^{\bar{r}}\\
	\ge &-C_0 r w(r)\sinh^{n-1}(\frac{r}{R})\bigg|_0^{\tilde{r}}+C_0\bar{u}'(r)w(r)\sinh^{n-1}(\frac{r}{R})\bigg|_{\tilde{r}}^{\bar{r}}\ \ \ \ \\ %(\text{the first integrand is non-negative and the second one is zero by (ODE)(2)})\\
	=& \ 0.\\
	\end{split}
	\end{equation*}
	In addition, if $\mathcal{L}(\bar{u}+w)-\mathcal{L}(\bar{u})=0$, from above inequalities and since $w \in C^1$, we have $w'(r)=0$ on $D$, i.e., $w=C_3$, for some non-negative constant $C_3$. Then $0= \mathcal{L}(\bar{u}+w)-\mathcal{L}(\bar{u})\ge  C_0\int_{0}^{\bar{r}} [\bar{u}'(r)w'(r)+w]\sinh^{n-1}(\frac{r}{R}) dr =  C_0C_3\int_{0}^{\bar{r}} \sinh^{n-1}(\frac{r}{R}) dr\ge 0$. Since both $C_0$ and $\int_{0}^{\bar{r}} \sinh^{n-1}(\frac{r}{R}) dr$ are positive, we have $C_3 = 0$, i.e. $w\equiv 0 $ on $D$. Therefore, $\bar{u}$ is the unique minimizer.\\
	
	{\bf Step 2:} To check that $\bar{u}(x)$ is $G$-convex, it is equivalent to prove $\bar{u}(x)$ is $b$-convex in the sense of~\cite{FigalliKimMcCann11}, or equivalently $-\bar{u}(x)$ is $-b$-concave in the sense of Definition \ref{(-bar{G})-convexity}, where $b(x,y):=-\frac{1}{2}d_H^2(x,y)$. That is, we need to show $\bar{u}(x) = -((-\bar{u})^{(-b)^*})^{(-b)}(x)$, for all $x \in D$. Denote $\psi(y) = -(-\bar{u})^{(-b)^*}(y)$, and $\phi(x) = -(-\psi)^{(-b)}(x)$. Then it is equivalent to show $\bar{u}(x) = \phi(x)$, where
	\begin{flalign*}
		\phi(x) &~= \sup\limits_{y} b(x,y)%-\frac{1}{2}d_H^2(x,y) 
		- \psi(y), \\
		\text{and } \psi(y) &~= \sup\limits_{x} b(x,y) %-\frac{1}{2}d_H^2(x,y)
		- \bar{u}(x).
	\end{flalign*}
	By definition, we have
	\begin{equation*}
	\begin{split}
	\psi(y)=&\sup\limits_{x} -\frac{1}{2}d_H^2(x,y) - \bar{u}(x)\\
	=&\sup\limits_{r} -\frac{1}{2}(r-s)^2 - \bar{u}(r)\ \ \ (\text{ because } |M|\le 1, \text{ and } M=1 \text{ when } \theta_i= \varphi_i, \forall i =1,2,..., n-1).\\
	\end{split}
	\end{equation*}
	For each $s\in [0,\bar{r}]$, define $h_5^s(r):= -\frac{1}{2}(r-s)^2 - \bar{u}(r)$. Then 
	\begin{equation*}
	(h_5^s)'(r) =
	\begin{cases}
	s & ,0\le r\le \tilde{r} \\
	s-r-\sinh^{1-n}(\frac{r}{R})\int_{\bar{r}}^r\sinh^{n-1}(\frac{\sigma}{R})d\sigma    & ,\tilde{r} < r \le \bar{r} \\
	\end{cases},
	\end{equation*}
	\begin{equation*}
	(h_5^s)''(r)= 
	\begin{cases}
	0 & ,0\le r\le \tilde{r} \\
	-1-\bar{u}''(r)<-1    & ,\tilde{r} < r \le \bar{r} \\
	\end{cases}.
	\end{equation*}
	Thus, $h_5^s(r)$ is concave on $[0,\bar{r}]$ and strictly concave on $[\tilde{r}, \bar{r}]$. Since $\bar{u}'$ is continuous, thus $(h_5^s)'(r)$ is continuous. Notice $(h_5^s)'(\tilde{r})=s>0$ and $(h_5^s)'(\bar{r})<0$. Therefore, for each $s\in [0,\bar{r}]$, $(h_5^s)'(\beta) =0$ has exactly one solution on $[0, \bar{r}]$, which is located on $[\tilde{r}, \bar{r}]$ and takes the maximum value of $h_5^s$.\\
	Let $h_6(s)$ be the unique solution of $(h_5^s)'(\beta) =0$, for all $s\in [0,\bar{r}]$.\\
	Then $h_6(s)\in [\tilde{r}, \bar{r}]$, and $h_6(0) = \tilde{r}$, $h_6(\bar{r})= \bar{r}$. \\
	Since $(h_5^s)'(s)>0 = (h_5^s)'(h_6(s))$ and $(h_5^s)'$ is strictly decreasing on $[\tilde{r}, \bar{r}]$, we have $h_6(s)>s$.\\ 
	For any $0\le s_1<s_2\le \bar{r}$, $ (h_5^{s_2})'(h_6(s_2)) =0 =  (h_5^{s_1})'(h_6(s_1)) < (h_5^{s_2})'(h_6(s_1))$, thus $h_6(s_1)<h_6(s_2)$, i.e., $h_6$ is strictly increasing. By Implicit Function Theorem, one has $h_6 \in C^1$. Thus, $h_6' >0$.
	Here, denote
	\begin{equation*}
	\bar{u}(r)= 
	\begin{cases}
	u_1(r) & ,0\le r\le \tilde{r} \\
	u_2(r) & ,\tilde{r} < r \le \bar{r} \\
	\end{cases}.
	\end{equation*}
	Then $\psi(y)=\sup\limits_{r}h_5^s(r) = h_5^s(h_6(s)) = -\frac{1}{2}(h_6(s)-s)^2 - u_2(h_6(s))$.
	\begin{equation*}
	\begin{split}
	\phi(x) =&\sup\limits_{y} -\frac{1}{2}d_H^2(x,y) - \psi(y)\\
	=&\sup\limits_{s} -\frac{1}{2}(r-s)^2+ \frac{1}{2}(h_6(s)-s)^2 + u_2(h_6(s)) \ \ \\ %(\text{ because } |M|\le 1, \text{ and } M=1 \text{ when } \theta_i= \varphi_i, \forall i =1,2,..., n-1.)\\
	\end{split}
	\end{equation*}
	For each $r\in [0,\bar{r}]$, define $h_7^r(s):=-\frac{1}{2}(r-s)^2+ \frac{1}{2}(h_6(s)-s)^2 + u_2(h_6(s))$. Then
	\begin{align*}
	(h_7^r)'(s)=r-h_6(s),\\
	(h_7^r)''(s)=-(h_6)'(s) < 0.\\
	\end{align*}
	Plugging in $s=0,\bar{r}$, we have $(h_7^r)'(0)=r-\tilde{r}$, $(h_7^r)'(\bar{r})=r-\bar{r}\le 0$. Therefore,
	\begin{equation*}
	\begin{split}
	&1). \mbox{ For } r \in [0, \tilde{r}], (h_7^r)'(s)< (h_7^r)'(0) \le 0, \text{ then we have } \phi(x) =\sup\limits_{s} (h_7^r)(s) = (h_7^r)(0)=-\frac{1}{2}r^2 = u_1(r);\\
	&2).  \mbox{ For } r \in [ \tilde{r}, \bar{r}], \mbox{since} (h_7^r)'(0) \ge 0, (h_7^r)'(\bar{r}) \le 0, \text{ then we have } \phi(x) =\sup\limits_{s} (h_7^r)(s) = (h_7^r)((h_6^{-1})(r))= u_2(r).\\
	\end{split}
	\end{equation*}
	Thus, $\phi(x) = \bar{u}(x)$, which implies $\bar{u}$ is $G$-convex. So,
	\begin{equation*}
	\bar{u}(r)= \argmin\limits_{\substack{u\ge -\frac{1}{2}r^2 \\ u \text{ is radially symmetric} \\ u \text{ is $G$-convex}}} \mathcal{L}(u).
	\end{equation*}\\
	
	{\bf Step 3:} We are going to show
	\begin{equation*}
	\bar{u}=\argmin\limits_{\substack{u\ge -\frac{1}{2}|x|^2 \\ u \text{ is $G$-convex}}} \mathcal{L}(u) .
	\end{equation*}

	Suppose $u$ is any $G$-convex function (not necessarily radially symmetric), then there exists a function $v$, such that $u(x)=\max\limits_{y} b(x,y)-v(y)$. Thus, 
	$$D_ru(x) = D_rb(x,y_G(x,Du(x))) = -d_H(x,y_G(x,Du(x)))\cdot D_rd_H(x,y_G(x,Du(x))),$$ 
	where $y_G(x, Du(x))=\arg\max\limits_{y} b(x,y)-v(y)$. By Proposition ($\ref{disprop}$), we have
	\begin{equation*}
	-b(x,y_G(x,Du(x))) = \frac{1}{2}d_H^2(x,y_G(x,Du(x))) =\frac{|D_r u(x)|^2}{2|D_rd_H(x,y_G(x,Du(x)))|^2} \ge \frac{|D_r u(x)|^2}{2}.  
	\end{equation*}
	For any $w$ on $D$, such that $\bar{u}+w$ is $G$-convex, and $w\ge 0$ on $A\times[0,\pi]^{n-2}\times[0, 2\pi]$, we have
	\begin{equation*}
	\begin{split}
	&\mathcal{L}(\bar{u}+w)-\mathcal{L}(\bar{u}) \\
	=&\int_D -b(x, y_G(x,D\bar{u}(x)+Dw(x)))+b(x,y_G(x, D\bar{u}(x)))+w(x)d\mu(x)\\
	\ge& \int_D\frac{1}{2}|D_r\bar{u}+D_rw|^2-\frac{1}{2}|\bar{u}'(r)|^2 +w(x) d\mu(x)\\
	\ge& \int_D [\bar{u}'(r)\cdot D_rw+w]R^{n-1}\sinh^{n-1}(\frac{r}{R})\sin^{n-2}\theta_1\cdots \sin\theta_{n-2} dr d\theta_1 \cdots d\theta_{n-1} \\
	=&\int_{A\times [0,\pi]^{n-2}\times[0,2\pi]} \Big[-r\cdot\sinh^{n-1}(\frac{r}{R})\cdot D_rw+w\cdot \sinh^{n-1}(\frac{r}{R})\Big]R^{n-1}\sin^{n-2}\theta_1\cdots \sin\theta_{n-2} dr d\theta_1 \cdots d\theta_{n-1}\\
	+&\int_{B\times [0,\pi]^{n-2}\times[0,2\pi]} \Big[\int_{\bar{r}}^r\sinh^{n-1}(\frac{t}{R}) dt\cdot D_rw+w\cdot \sinh^{n-1}(\frac{r}{R})\Big]R^{n-1}\sin^{n-2}\theta_1\cdots \sin\theta_{n-2} dr d\theta_1 \cdots d\theta_{n-1}\\
	=&\int_{ [0,\pi]^{n-2}\times[0,2\pi]} \Big\{\int_{0}^{\tilde{r}}w\Big[2\sinh^{n-1}(\frac{r}{R})+\frac{n-1}{R}r\sinh^{n-2}(\frac{r}{R})\cosh(\frac{r}{R})\Big] dr-\Big[r\sinh^{n-1}(\frac{r}{R})\cdot w\Big]\bigg|_{r=0}^{\tilde{r}}\Big\}R^{n-1}\cdot\\
	&\sin^{n-2}\theta_1\cdots \sin\theta_{n-2}  d\theta_1 \cdots d\theta_{n-1}\\
	+&\int_{ [0,\pi]^{n-2}\times[0,2\pi]} \Big[\int_{\bar{r}}^{r}\sinh^{n-1}(\frac{t}{R})dt \cdot w\Big] \bigg|_{r=\tilde{r}}^{\bar{r}}\cdot R^{n-1}\sin^{n-2}\theta_1\cdots \sin\theta_{n-2} d\theta_1 \cdots d\theta_{n-1}\\
	\ge &\int_{[0,\pi]^{n-2}\times [0, 2\pi]}\Big[-\tilde{r}\sinh^{n-1}(\frac{\tilde{r}}{R}) -\int_{\bar{r}}^{\tilde{r}}\sinh^{n-1}(\frac{t}{R})dt\Big] \cdot w(\tilde{r}, \theta_1,...,\theta_{n-1})\cdot R^{n-1}\sin^{n-2}\theta_1\cdots \sin\theta_{n-2} d\theta_1 \cdots d\theta_{n-1}\\
	=& \ \ 0. \\
	\end{split}
	\end{equation*}
	Thus 
	\begin{equation*}
	\bar{u}\in \argmin\limits_{\substack{u\ge -\frac{1}{2}|x|^2 \\ u \text{ is $G$-convex}}} \mathcal{L}(u) .
	\end{equation*}
	If, in addition,  $\mathcal{L}(\bar{u}+w)-\mathcal{L}(\bar{u})=0$, then the above equalities must be equality. Thus $D_rw(x)=0$, a.e. $x \in D$. Since both $\bar{u}+w$ and $\bar{u}$ are $G$-convex, $w \in C^{1,1}$, thus $D_r w(x)\equiv 0, \forall x\in D$. So, one can write $w(x)= w(\theta_1, ..., \theta_{n-1})$. Since for $x\in A\times [0,\pi]^{n-1}\times[0,2\pi]$, $w(x)\ge 0$, we have $w\ge 0$ on $D$. Then from the above inequalities, we get 
	\begin{equation*}
	0=\mathcal{L}(\bar{u}+w)-\mathcal{L}(\bar{u})\ge \int_D w(x)R^{n-1}\sinh^{n-1}(\frac{r}{R})\sin^{n-2}\theta_1\cdots \sin\theta_{n-2} dr d\theta_1 \cdots d\theta_{n-1}\ge 0
	\end{equation*} 
	Thus, $ w(x)R^{n-1}\sinh^{n-1}(\frac{r}{R})\sin^{n-2}\theta_1\cdots \sin\theta_{n-2} =0$, for a.e. $ x \in D$, which implies $w \equiv 0 $ on $D$. So,  $\bar{u}$ is the unique minimizer, i.e. 
	\begin{equation*}
	\bar{u}= \argmin\limits_{\substack{u\ge -\frac{1}{2}|x|^2 \\ u \text{ is $G$-convex}}} \mathcal{L}(u).
	\end{equation*}




\end{proof}



	\begin{remark}
	We also have uniqueness results with different explicit solutions on $\mathbf{S}^n$ and $\R^n$, where the uniqueness is also ensured by Figalli-Kim-McCann\cite{FigalliKimMcCann11}. Moreover, the solutions on $\mathbf{S}^n$, $\HH^n$ converge to those on $\R^n$, as curvatures go to 0. \\
	The unique minimizer of the principal-agent problem on $\R^n$ is given by
	\begin{flalign*}
			\bar{u}_{\R^n}(r) = 
			\begin{cases}
			-\frac{1}{2} r^2 & ,0\le r\le \tilde{r}_{\R^n} \\
			\parbox[t]{.6\textwidth}{$\frac{(\bar{r})^n}{n(n-2)}r^{2-n} + \frac{r^2}{2n} - \frac{(\bar{r})^2}{2(n-2)(n+1)^{\frac{2-n}{n}}} $} & ,\tilde{r}_{\R^n} < r \le \bar{r}
			\end{cases},
	\end{flalign*}
		where $\tilde{r}_{\R^n} = \frac{\bar{r}}{(n+1)^{\frac{1}{n}}} $. 
	
	And the unique minimizer on  $\mathbf{S}^n$ is given by
		\begin{flalign*}
		\bar{u}_{\mathbf{S}^n}(r) = 
		\begin{cases}
		-\frac{1}{2} r^2 & ,0\le r\le \tilde{r}_{\mathbf{S}^n} \\
		\parbox[t]{.75\textwidth}{$\int_{\tilde{r}_{\mathbf{S}^n}}^{r}\sin^{1-n}(\frac{t}{R})\int_{0}^{t}\sin^{n-1}(\frac{\sigma}{R})d\sigma dt - \int_{0}^{\bar{r}}\sin^{n-1}(\frac{\sigma }{R})d \sigma  \int_{\tilde{r}_{\mathbf{S}^n}}^{r}\sin^{1-n}(\frac{t}{R}) d t -\frac{(\tilde{r}_{\mathbf{S}^n})^2}{2} $} & ,\tilde{r}_{\mathbf{S}^n} < r \le \bar{r}
		\end{cases},
		\end{flalign*}
	where $\tilde{r}_{\mathbf{S}^n}$ satisfies 
	\begin{equation*}
		\int_{\bar{r}}^{\tilde{r}_{\mathbf{S}^n}}\sin^{n-1}(\frac{\sigma }{R}) d\sigma  +\tilde{r}_{\mathbf{S}^n}\sin^{n-1}(\frac{\tilde{r}_{\mathbf{S}^n}}{R})=0.
		\end{equation*}
	And the proofs are similar to that of Theorem \ref{disprop}.

\end{remark}


\section{Convexity results on several examples for the non-quasilinear case}



We  close with several examples,  which are established by computing two derivatives
of $\pi(x,y_t,z_t)$ along an arbitrary $G$-segment $t\in [0,1] \longmapsto (x,y_t,z_t)$.
These computations are tedious but straightforward.\medskip

For specific non-quasilinear agent preferences,  we use the explicit expression in Lemma \ref{LemmaProfitConcavity} for the 
desired second derivative to establish the following examples,  which assume the principal is indifferent to
whom she transacts business with and that her preferences depend linearly on payments.  
These examples give conditions under which the principal's program inherits concavity or convexity 
from the agents' price sensitivity.
Although the resulting conditions appear complicated,  they illustrate
the subtle interplay between the preferences of agent and principal for products in the first example,
and between the preferences of the agents for products as opposed to prices in the second. \medskip



\begin{example}[Nonlinear yet homogeneous sensitivity of agents to prices]
	\label{general example1}
	Take $\pi(x, y, z) =z- a(y)$, $G(x, y, z) = b(x,y)-f(z)$, satisfying \Gzero-\Gsix,  $G \in C^3(cl(X\times Y \times Z)
	)$, $\pi \in C^2(cl(X\times Y \times Z)
	)$, and assume $\bar{z}<+\infty$.
	
	1. If $f(z)$ is convex [respectively concave] in $cl(Z)$, then $\pmb \Pi(u)$ is concave [respectively convex] for all $\mu\ll \mathcal{L}^m$ if and only if there exists $\varepsilon \ge 0$ such that each $(x,y,z) \in X \times Y\times Z$ and $\xi \in \R^{n}$ satisfy 
	\begin{equation}\label{robert1}
	\begin{split}
	\pm \Bigg\{a_{kj}(y)-\frac{b_{,kj}(x,y)}{f'(z)}+\Big(\frac{b_{,l}(x,y)}{f'(z)}- a_l(y)\Big)b^{i,l}(x,y) b_{i,kj}(x,y)\Bigg\} \xi^{k}\xi^{j} \\
	\ge  \varepsilon \mid \xi\mid ^2.
	\end{split}
	\end{equation}
	
	2. In addition, $\pmb \Pi(u)$ is uniformly concave [respectively uniformly convex] on $W^{1,2}(X,d\mu)$  uniformly for all $\mu\ll \mathcal{L}^m$ if and only if $\pm f''> 0$ and  \eqref{robert1} holds with $\varepsilon >0$.
\end{example}

\begin{proof}%[Proof of Example \ref{general example1}]
	From Lemma \ref{LemmaProfitConcavity}, $\pmb \Pi(u)$ is concave for all $\mu\ll \mathcal{L}^m$ if and only if $(\pi_{,\bar{k}\bar{j}}- \pi_{,\bar{l}} \bar{G}^{\bar i,\bar l}\bar{G}_{\bar{i},\bar{k}\bar{j}})\big|_{x_0=-1}$ is non-positive definite, and uniformly concave uniformly for all $\mu\ll \mathcal{L}^m$ if and only if this matrix is uniform negative definite.
	
	In this example, we have $\pi(x,y,z)= z-a(y)$, $\bar{G}(x,x_0, y,z) = x_{0} G(x,y,z)$ $ = x_{0}(b(x,y)-f(z))$. Thus, 
	\begin{flalign*}
	\pi_{,\bar{k}\bar{j}}= \begin{pmatrix}
	-a_{kj} & \mathbf{0}\\
	\mathbf{0} & 0\\
	\end{pmatrix}, \ \ 
	\pi_{,\bar{l}}= (-a_{l}, 1), \ \ 
	\bar{G}_{\bar{i},\bar{l}}\big|_{x_0=-1} = \begin{pmatrix}
	-b_{i,l} & \mathbf{0}\\
	b_{,l} & -f'(z)\\
	\end{pmatrix}.\\	    
	\end{flalign*}
	
	By \Gfour, $f'(z) >0$ for all $z\in cl(Z)$. By \Gsix, since $\bar{G}_{\bar{i},\bar{l}}\big|_{x_0=-1}$ has the full rank, the matrix $(b_{i,l})$ also has its full rank. Taking $b^{i,l}$ as its left inverse, we have
	\begin{flalign*}
	\bar{G}^{\bar i,\bar l}\big|_{x_0=-1} = \begin{pmatrix}
	-b^{i,l} & \mathbf{0}\\
	-\frac{b_{,l}b^{i,l}}{f'(z)} & \frac{1}{-f'(z)}\\
	\end{pmatrix}, \ \ 
	\bar{G}_{\bar{i},\bar{k}\bar{j}}\big|_{x_0=-1} =\begin{pmatrix}
	-b_{i,k\bar{j}}& \mathbf{0}  \\
	b_{,k\bar{j}} & (-f'(z))_{\bar{j}}\\
	\end{pmatrix}.
	\end{flalign*}
	Therefore, 
	\begin{flalign*}
	&(\pi_{,\bar{k}\bar{j}} - \pi_{,\bar{l}}\bar{G}^{\bar i,\bar l} \bar{G}_{\bar{i},\bar{k}\bar{j}})\big|_{x_0=-1} \\
	&=\begin{pmatrix}
	-a_{kj} &\mathbf{0} \\
	\mathbf{0} &  0\\
	\end{pmatrix} - \begin{pmatrix}
	(-a_{l}b^{i,l}+\frac{b_{,l}}{f'(z)}b^{i,l}) b_{i,k\bar{j}}-\frac{b_{,k\bar{j}}}{f'(z)},\frac{(f'(z))_{\bar{j}}}{f'(z)}
	\end{pmatrix}\\
	&= - \begin{pmatrix}
	a_{kj}+(-a_{l}+\frac{b_{,l}}{f'(z)})b^{i,l} b_{i,kj}-\frac{b_{,kj}}{f'(z)} & \mathbf{0}\\
	\mathbf{0} & \frac{f''(z)}{f'(z)}\\
	\end{pmatrix}.
	\end{flalign*}
	
	
	Since \Gfour\ and $f$ is convex, we have $f'(z) >0$ and $f''(z)\ge 0$, for all $z\in cl(Z)$. Thus, $\pi_{,\bar{k}\bar{j}}-\pi_{,\bar{l}}\bar{G}^{\bar i,\bar l}\bar{G}_{\bar{i},\bar{k}\bar{j}}$ is non-positive definite if and only if $a_{kj}+(-a_{l}+\frac{b_{,l}}{f'(z)})b^{i,l} b_{i,kj}-\frac{b_{,kj}}{f'(z)}$ is non-negative definite, i.e., 
	there exist $\varepsilon \ge 0$ such that each $(x,y,z) \in X \times Y\times Z$ and $\xi \in \R^{n}$ satisfy 
	\begin{equation*} \Bigg\{a_{kj}(y)-\frac{b_{,kj}(x,y)}{f'(z)}+\Big(\frac{b_{,l}(x,y)}{f'(z)}- a_l(y)\Big)b^{i,l}(x,y) b_{i,kj}(x,y)\Bigg\} \xi^{k}\xi^{j} \ge  \varepsilon \mid \xi\mid ^2.
	\end{equation*}
	
	In addition, $\pi_{,\bar{k}\bar{j}}-\pi_{,\bar{l}}\bar{G}^{\bar i,\bar l}\bar{G}_{\bar{i},\bar{k}\bar{j}}$ is uniform negative definite if and only if $f''>0$ and $\varepsilon>0$, which is equivalent to that $\pmb \Pi(u)$ is uniformly concave uniformly for all $\mu\ll \mathcal{L}^m$. Similarly, one can show equivalent conditions for $\pmb \Pi(u)$ being convex or uniformly convex.
\end{proof}

Although 
the next two examples are not completely general, they have the following economic interpretation. The same selling price impacts utility differently for different types of agents. In other words, it models the situation where agents have different sensitivities to the same price.  In Example $\ref{general example2}$,  the principal's utility is linear and depends exclusively on her revenue, which is a simple special case of Example $\ref{general example3}$.
\medskip

\begin{example}[Inhomogeneous sensitivity of agents to prices, zero cost]\label{general example2}
	Take $\pi(x, y, z) =z$,  $G(x,y,z)$ $= b(x,y)-f(x,z)$,  satisfying \Gzero-\Gsix,  $G \in C^3(cl(X\times Y \times Z)
	)$, $\pi \in C^2(cl(X\times Y \times Z)
	)$, and assume $\bar{z}<+\infty$. Suppose $D_{x,y}b(x,y)$ has full rank for each $(x,y) \in X\times Y$, and denote its left inverse $b^{i,l}(x,y).$
	
	1. If $(x,y,z)\longmapsto h(x,y,z):=f(x,z)-b_{,l}(x,y)b^{i,l}(x,y)f_{i,}(x,z)$ is strictly increasing and convex [respectively concave] with respect to $z$, then $\pmb \Pi(u)$ is concave [respectively convex]  for all $\mu\ll \mathcal{L}^m$ if and only if there exists $\varepsilon \ge 0$ such that each $(x,y) \in X \times Y$ and $\xi \in \R^{n}$ satisfy
	\begin{equation}\label{robert2}
	\pm \Big\{-b_{,kj}(x,y)+b_{,l}(x,y)b^{i,l}(x,y) b_{i,kj}(x,y)\Big\} \xi^{k}\xi^{j} \ge  \varepsilon \mid \xi\mid ^2.
	\end{equation}
	
	2. In addition, $\pmb \Pi(u)$ is uniformly concave [respectively uniformly convex] on $W^{1,2}(X,d\mu)$ uniformly for all $\mu\ll \mathcal{L}^m$ if and only  if $\pm h_{zz}> 0$ and \eqref{robert2} holds with $\varepsilon >0$.
	
\end{example}


\begin{example}[Inhomogeneous sensitivity of agents to prices]\label{general example3}
	Take $\pi(x, y,$ $ z) =z-a(y)$,  $G(x,y,z)= b(x,y)-f(x,z)$,  satisfying \Gzero-\Gsix,  $G \in C^3(cl(X\times Y \times Z)
	)$, $\pi \in C^2(cl(X\times Y \times Z)
	)$, and assume $\bar{z}<+\infty$. Suppose $D_{x,y}b(x,y)$ has full rank for each $(x,y) \in X\times Y$, and $1- (f_{z})^{-1}b_{,\beta}b^{\alpha,\beta}f_{\alpha,z} \ne 0$, for all $(x, y,z) \in X\times Y\times Z$.
	
	1. If $(x,y,z)\longmapsto h(x,y,z):=a_{l}b^{i,l}f_{i,zz}+\frac{(a_{\beta}b^{\alpha,\beta}f_{\alpha,z}-1)(b_{,l}b^{i,l}f_{i,zz}-f_{zz})}{f_{z} -b_{,\beta}b^{\alpha,\beta}f_{\alpha,z}} \ge 0 ~[\le 0]$ , then $\pmb \Pi(u)$ is concave  [respectively convex] for all $\mu\ll \mathcal{L}^m$ if and only if there exists $\varepsilon \ge 0$ such that each $(x,y,z) \in X \times Y\times Z$ and $\xi \in \R^{n}$ satisfy 
	\begin{equation}\label{robert3}
	\begin{split}
	\pm \Bigg\{&~a_{kj} -a_{l}b^{i,l}b_{i,kj}\\
	&+\frac{1-a_{\beta}b^{\alpha,\beta}f_{\alpha,z}}
	{1- (f_{z})^{-1}b_{,\beta}b^{\alpha,\beta}f_{\alpha,z}}
	\Big(-\frac{b_{,kj}}{f_z}+\frac{b_{,l}}{f_z}b^{i,l} b_{i,kj}\Big) \Bigg\}\xi^{k}\xi^{j} 
	\ge  \varepsilon \mid \xi\mid ^2.
	\end{split}
	\end{equation}
	
	2. If in addition, $\pmb \Pi(u)$ is uniformly concave [respectively uniformly convex] on $W^{1,2}(X,d\mu)$ uniformly for all $\mu\ll \mathcal{L}^m$ if and only if $\pm h>0$ and  \eqref{robert3} holds with $\varepsilon>0$.
\end{example}

\begin{proof}%[Proof of Example \ref{general example3}]
	Similar to the proof of Example \ref{general example1},  $\pmb \Pi(u)$ is concave for all $\mu\ll \mathcal{L}^m$ if and only if $(\pi_{,\bar{k}\bar{j}}-\pi_{,\bar{l}}\bar{G}^{\bar i,\bar l}\bar{G}_{\bar{i},\bar{k}\bar{j}})$ is non-positive definite, and  uniformly concave uniformly for all $\mu\ll \mathcal{L}^m$ if and only if this tensor is uniform negative definite.
	
	Since $D_{x,y}b(x,y)$ has full rank for each $(x,y) \in X\times Y$, and for all $(x, y,z) \in X\times Y\times Z$, $1- (f_{z})^{-1}b_{,\beta}b^{\alpha,\beta}f_{\alpha,z} \ne 0$, for $\pi(x,y,z) = z-a(y)$, $\bar{G}(x,x_0, y, z) = x_0(b(x,y)-f(x,z))$, we have 
	\begin{flalign*}
	&-(\pi_{,\bar{k}\bar{j}} - \pi_{,\bar{l}}\bar{G}^{\bar i,\bar l} \bar{G}_{\bar{i},\bar{k}\bar{j}}) \\
	& = \begin{pmatrix}
	\begin{split}
	&a_{kj} -a_{l}b^{i,l}b_{i,kj} +\frac{(a_{\beta}b^{\alpha,\beta}f_{\alpha,z}-1)(b_{,kj}-b_{,l}b^{i,l} b_{i,kj})}{f_{z} -b_{,\beta}b^{\alpha,\beta}f_{\alpha,z}}
	\end{split} & \begin{split}
	\mathbf{0}
	\end{split}\vspace{0.5cm}\\
	\begin{split}
	\mathbf{0}
	\end{split} & \begin{split}
	h(x,y,z)	
	\end{split}\\
	\end{pmatrix},
	\end{flalign*}
	where $h(x,y,z) = a_{l}b^{i,l}f_{i,zz}+\frac{(a_{\beta}b^{\alpha,\beta}f_{\alpha,z}-1)(b_{,l}b^{i,l}f_{i,zz}-f_{zz})}{f_{z} -b_{,\beta}b^{\alpha,\beta}f_{\alpha,z}}$. Since $h(x,y,z)\ge 0$, then $(\pi_{,\bar{k}\bar{j}}-\pi_{,\bar{l}}\bar{G}^{\bar i,\bar l}\bar{G}_{\bar{i},\bar{k}\bar{j}})$ is non-positive definite if and only if there exist $\varepsilon \ge 0$ such that each $(x,y,z) \in X \times Y\times Z$ and $\xi \in \R^{n}$ satisfy 
	\begin{equation*}
	\Bigg\{a_{kj} -a_{l}b^{i,l}b_{i,kj}+\frac{(a_{\beta}b^{\alpha,\beta}f_{\alpha,z}-1)(b_{,kj}-b_{,l}b^{i,l} b_{i,kj})}{f_{z} -b_{,\beta}b^{\alpha,\beta}f_{\alpha,z}}\Bigg\} \xi^{k}\xi^{j} \ge  \varepsilon \mid \xi\mid ^2.
	\end{equation*} 
	
	In addition, $\pmb \Pi(u)$ is uniformly concave uniformly for all $\mu\ll \mathcal{L}^m$ if and only if $h>0$ and $\varepsilon>0$.
\end{proof}



Example \ref{general example 4} asserts the concavity of monopolist's maximization in the zero-sum setting,
where the agent's utilities are relatively general but the principal's profit is extremely special. In addition, more non-quasilinear examples could be discovered by applying Lemma $\ref{LemmaProfitConcavity}$.\medskip



\begin{example}[Zero sum transactions]\label{general example 4}
	Take $\pi(x, y, z) = -G(x,y,z)$, satisfying \Gzero-\Gfive\ and $\mu\ll \mathcal{L}^m$, which means the monopolist's profit in each transaction coincides exactly with the 
	agent's loss. From $(\ref{$G$-segment})$, since $G$ is linear on $G$-segments, we know $\pmb \Pi(u)$ is linear.
\end{example}			






%% This adds a line for the Bibliography in the Table of Contents.
\addcontentsline{toc}{chapter}{Bibliography}
%% *** Set the bibliography style. ***
%% (change according to your preference/requirements)
\bibliographystyle{plain}
%% *** Set the bibliography file. ***
%% ("thesis.bib" by default; change as needed)
\bibliography{thesis}

%% *** NOTE ***
%% If you don't use bibliography files, comment out the previous line
%% and use \begin{thebibliography}...\end{thebibliography}.  (In that
%% case, you should probably put the bibliography in a separate file and
%% `\include' or `\input' it here).

\end{document}
